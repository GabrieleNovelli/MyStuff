\documentclass[12pt,a4paper]{book}

\usepackage[italian]{babel}
\usepackage{newlfont}
\usepackage{color}
\usepackage{float}
\usepackage{frontespizio}
\usepackage{amsmath,amssymb}
\usepackage{amsthm}
\usepackage{geometry}
\usepackage{tikz}
\usepackage{biblatex}
\usepackage{csquotes}
\usepackage{pgfplots}
\usepackage{hyperref}
\usepackage{amssymb}
\usepackage{comment}
\usepackage[compat=1.0.0]{tikz-feynman}
\usepackage{tikz-cd}
\usepackage[utf8]{inputenc}
\usepackage[T1]{fontenc}
\usepackage{geometry}
\usepackage{multicol}
\usepackage{pgfpages}
\usepackage[print,1 to1]{booklet}
\usepackage{truncate}
\usepackage{fancyhdr}

% Configurazione della pagina per il libretto
\geometry{centering,bindingoffset=1cm,top=6.5cm,bottom=6.5cm,left=4.5cm,right=4.5cm}


\usepackage{titlesec}
\titleformat{\section}[block]{\bfseries\centering\fontsize{10}{12}\selectfont}{\thesection}{1em}{}


\hypersetup{
	colorlinks=true,
	linkcolor=blue,
	filecolor=magenta,      
	urlcolor=cyan,
	pdftitle={Overleaf Example},
	pdfpagemode=FullScreen,
}

\theoremstyle{definition}
\newtheorem{Def}{Definizione}[chapter]

\theoremstyle{Theorem}
\newtheorem{Theo}[Def]{Teorema}
\newtheorem{Prop}[Def]{Proposizione}

\newtheorem{Lm}[Def]{Lemma}

\theoremstyle{definition}
\newtheorem{Ex}[Def]{Esempio}

\theoremstyle{definition}
\newtheorem{Lem}[Def]{Lemma:}

\theoremstyle{definition}
\newtheorem{Obs}[Def]{Osservazione:}
\begin{document}
	\setpdftargetpages
	Ciao Noe;
	Questo libricino è stato partorito in seguito alla conversazione che abbiamo avuto da brilli. Spero che i contenuti di queste pagine ti possano far ricordare il mio nome per tutta la tua vita, ovunque essa ti porti. Sei stata la personalità che mi ha più segnato in ambito accademico ed il rapporto che sono riuscito a costruire con te è uno dei più belli che possiedo. Prego con tutto il mio cuore che la nostra amicizia possa continuare in eterno. Fisica non sarà più la stessa cosa (casa) senza di te. \\
	\\
	Mi sono impegnato incredibilmente per cercare di rendere comprensibili gli argomenti trattati qua dentro. Spero che i miei sforzi non siano stati vani.\\
	\\
	Con amore;\\\\
	Gabriele Novelli
	\chapter{Il temperamento equabile}
	\subsection{Le basi}
	Per capire a fondo il temperamento equabile, è necessario comprendere le basi di come funziona l'orecchio umano. In generale, un suono "puro" o "nota", corrisponde ad un'onda sonora, caratterizzato da una frequenza $\omega$ fissata.\\
	\\
	Immaginiamo ora due suoni con frequenza $\omega_1,\omega_2$. Per nostra natura, riconosceremo due suoni come uguali se le loro frequenze stanno tra loro in rapporti di numeri interi: $\omega_1=k\omega_2$, $k\in \mathbb{N}$. Per fare un esempio concreto, chiamando $La$ (o $A$) la nota corrispondente a $440 Hz$, il nostro cervello riconoscerà come $La$ anche le note con frequenza $880 Hz,1320 Hz$ e così via. Essendo percepite come uguali (ma ad altezze diverse), due note le cui frequenze stanno tra loro come numeri interi risulteranno decisamente consonanti all'orecchio umano, se suonate in successione.
	\begin{Def}
		Chiamiamo intervallo tra due note le differenze tra le loro frequenze.
	\end{Def}
	\begin{Def}
		Chiamiamo intervalli di Ottave Giuste le differenze di frequenze $\omega_1-\omega_2$ tra due note le cui frequenze differiscono per numeri interi: ${\omega_1\over\omega_2}\in \mathbb{N}$.
	\end{Def}
	Dunque diciamo che $880Hz$ è un'Ottava Giusta sopra $440 Hz$, ma anche che $1320 Hz$ è due Ottave Giuste sopra $440 Hz$, e così via. Notare che ciò che rimane costante è il rapporto tra le frequenze, non la differenza: $300Hz$ è un'Ottava giusta sopra $150 Hz$, e $880Hz$ è un'ottava giusta sopra $440 Hz$, ma $300Hz-150Hz\neq 880Hz-440Hz$.\\
	\\
	L'ottava è particolarmente piacevole all'ascolto. Ma vi è un altro intervallo estremamente piacevole: la quinta giusta.
	\begin{Def}
		Chiamiamo intervallo di quinta Giusta la differenza di frequenza $\omega_1-\omega_2$ tra due note le cui frequenze differiscono per $3\over 2$: ${\omega_1\over\omega_2}={3\over 2}$.
	\end{Def}
	Nel caso del $La$ (o A) a $440 Hz$, la Quinta Giusta sarà a $440\times {3\over 2}Hz=660 Hz$.\\
	\\
	Essendo la Quinta Giusta e l'Ottava Giusta tra le consonanze migliori al nostro orecchio, vorremmo costruire uno strumento musicale in grado di suonare entrambi questi suoni. Mostreremo ora come ciò sia semplicemente impossibile.\\
	\\
	Lo spettro delle frequenze è ovviamente continuo, ergo, esistono un'infinità di suoni. Tuttavia, ogni strumento ha un'estensione finita. Dunque dovremo selezionare solo un numero finito di frequenze in $\mathbb{R}^+$ da includere in esso. Prendiamo come esempio un pianoforte. Esso ha 88 tasti: saremo costretti a selezionare solo 88 suoni tra l'infinità continua delle frequenze.
	\begin{Def}
		Chiamiamo scala una qualsiasi suddivisione di un'Ottava Giusta, selezionando un numero finito di frequenze all'interno di essa.
	\end{Def}
	Supponiamo allora di voler costruire proprio un pianoforte, utilizzando come nota di riferimento il $La$ a $440 Hz$ ed un'estensione di un'Ottava Giusta. Il nostro approccio sarà quello di assegnare i nostri tasti a frequenze all'interno di una determinata ottava, in quanto eccedendo al di fuori di essa ritroveremmo "le stesse note", nel senso che il nostro orecchio le sentirebbe come uguali.\\
	Dunque il primo tasto del pianoforte corrisponderà a $440 Hz$ e l'ultimo ad $880 Hz$. Tra uno dei tasti intermedi inseriremo $660 Hz$, dato che è la quinta giusta. Ora il nostro pianoforte ha 3 tasti, tutti consonanti tra loro. Vorremmo aggiungerne di più, ma come? Beh, potremmo semplicemente procedere per quinte ed ottave giuste. \\
	Dunque, una quinta giusta sopra $660 Hz$ è $990Hz$, che è fuori dal nostro range $(880<990)$. Tuttavia, possiamo includere questo intervallo nel nostro range scendendo di un'Ottava Giusta: ${990 Hz\over 2}=495 Hz$, che è perfettamente nel range voluto. Possiamo ripetere questo ragionamento ancora ed ancora: saliamo di una Quinta giusta ed occasionalmente scendiamo di un'Ottava Giusta. \\
	Dunque, una Quinta Giusta sopra $495 Hz$ è $495Hz \times {3\over 2}=742.5Hz$ che è ancora dentro l'Ottava. Una Quinta sopra $742.5 Hz$ è $742.5 Hz\times {3\over 2}=1113.75 Hz$, che è superiore a $880 Hz$; scendiamo allora di un'Ottava Giusta: ${1113.75 Hz\over 2}=556.875$ e così via.\\
	\\
	Applicando questo procedimento iterativamente, riusciamo ad includere nella nostra ottava molti più tasti. Tuttavia, vorremmo non doverne includere infiniti (il nostro pianoforte non è infinito e vorremmo avere più di una singola ottava di estensione). Ci fermeremo allora quando, salendo di una Quinta o scendendo di un'Ottava, ritroveremo $880Hz$.\\
	Dimostriamo ora che questo non accade mai.
	\\
	 \begin{Prop}
	 	Partendo da una frequenza fissata $\omega$, non è possibile ritrovare un'ottava $2\omega$ semplicemente salendo di Quinte o scendendo di Ottave.
	 \end{Prop}
	 \begin{proof}
	 	Sia $\omega$ la nostra frequenza di partenza, sia $n$ il numero di volte in cui saliamo di una quinta e $m$ il numero di volte in cui scendiamo di una Ottava. Allora vogliamo trovare due indici $n,m$ tali per cui:
	 	$$\omega=\bigg({3\over 2}\bigg)^n\bigg({1\over 2}\bigg)^m=2\omega$$
	 	con $n,m\in \mathbb{N}$. Questa equazione ci restituisce:
	 	$$3^n=2^{n+m+1}$$
	 	Vediamo chiaramente che questa equazione non è risolta da alcuna terna di indici dato che $3^n$ è un numero dispari mentre $2^{n+m+1}$ è pari.
	 \end{proof}
	 Questo significa che saremmo costretti ad utilizzare un numero infinito di tasti per comporre una singola ottava. Chiaramente, questa strada non è quella giusta.
	 \subsection{L'intonazione Giusta}
	 Cambiamo totalmente approccio. L'idea è quella di includere più intervalli rispetto a quelli di Quinta giusta ed Ottava Giusta. Dividiamo allora l'ottava in 7 intervalli, corrispondenti ai seguenti rapporti di frequenze:
	 $$1,{9\over8},{5\over4},{4\over3},{3\over 2},{5\over3},{15\over8}$$
	 Questo significa che se la prima nota della scala ha frequenza $\omega$, la seconda avrà $9\omega\over8$, la terza ${5\omega\over 4}$ e così via.\\
	 Per fare un esempio, se prendiamo come frequenza di base $440 Hz$, corrispondente ad un $La$ (o A), avremo:\\
	 \begin{center}
	 	\begin{tikzpicture} [scale=0.8,every node/.style={font=\small}]
	 		\draw (0,0)--(13,0);
	 		\node at (1,1) {440 Hz};
	 		\node at (2.5,1) {495 Hz};
	 		\node at (4.3,1) {550 Hz};
	 		\node at (6.5,1) {586.$\bar{6}$ Hz};
	 		\node at (8.3,1) {660 Hz};
	 		\node at (10.3,1) {$733.\bar{3}$ Hz};
	 		\node at (12,1) {825 Hz};
	 		
	 		\node at (1,0) {\textbullet};
	 		\node at (2.5,0) {\textbullet};
	 		\node at (4.3,0) {\textbullet};
	 		\node at (6.5,0) {\textbullet};
	 		\node at (8.3,0) {\textbullet};
	 		\node at (10.3,0) {\textbullet};
	 		\node at (12,0) {\textbullet};
	 	\end{tikzpicture}
	 \end{center}
	 Tutti questi intervalli risultano abbastanza consonanti tra di loro e possono essere usati per costruire un pianoforte: scegliendo il $La$  come nota di base, ad ogni tasto corrisponderà una di queste frequenze.
	 Il problema che sorge con questa suddivisione è la trasposizione: supponiamo di non voler suonare partendo dal $La$, a $440 Hz$, ma dalla nota corrispondente a $550 Hz$, che chiameremo $\bar{Do}$. Il problema è chiaro: la seconda nota nella scala di $\bar{Do}$ dovrebbe stare in rapporto a $550 Hz$ per $9\over 8$, ovvero $550Hz\times {9\over 8}=618.75Hz$. Tuttavia, nella nostra scala, la nota successiva a $550 Hz$ è $586 Hz$, molto lontana da quella voluta. Questo è un disastro: servirebbe ri-accordare ogni volta lo strumento, dipendentemente dalla nota che si sceglie come riferimento.\\
	 Insomma, la suddivisione che abbiamo costruito con l'intonazione giusta è strutturata sul $La$, ed in questo senso è una scala di $La$. Tuttavia, la scala intonata giusta di $\bar{Do}$ (e di qualsiasi altra nota) non contiene le stesse frequenze.\\
	 \\ 
	 Il problema dell'intonazione giusta è principalmente che tra due note consecutive non vi è sempre lo stesso rapporto di frequenze: per fare un esempio, il rapporto tra la seconda e la terza nota è ${9\over 8}\div {5\over 4}={9\over 10}$ mentre tra la terza e la quarta è ${5\over 4}\div {4\over 3}= {15\over 16}$.
	 \subsection{Il temperamento equabile}
	 L'intonazione giusta non è un'idea stupida, ma preclude la trasposizione. Noi vorremmo invece dividere la nostra ottava in un numero di intervalli che ci permetta di trasporre, ovvero di ottenere lo stesso rapporto di frequenze tra due note adiacenti. Vedremo come questo sia possibile, a patto di approssimare gli intervalli tra le note.\\
	 \\
	 Notiamo innanzitutto che prendendo una qualsiasi frequenza $\omega$, salendo di $12$ Quinte giuste e scendendo di $7$ Ottave Giuste, otterremo:
	 $$\bigg({3\over 2}\bigg)^{12}\times \bigg({1\over 2}\bigg)^7=1.0136$$
	 Ovvero ritorneremo "quasi" alla frequenza di partenza. Dunque, seguendo questa coincidenza matematica, dividiamo la nostra ottava in $12$ intervalli, tutti uguali. In particolare, questi saranno:
	 $$1,2^{1\over12},2^{2\over12},2^{3\over12},2^{4\over12},2^{5\over12},2^{6\over12},2^{7\over12},2^{8\over12},2^{9\over12},2^{10\over12},2^{11\over12},2^{12\over12}=2$$
	 Dove ogni numero indica ovviamente il rapporto della sue frequenza rispetto alla fondamentale.\\
	 Notiamo subito che nonostante la suddivisione sia in 12, abbiamo inserito 13 intervalli. Questo poichè il tredicesimo coincide con il dodicesimo, in quanto è un multiplo intero della frequenza iniziale.\\
	 Chiamiamo la scala ottenuta seguendo questa prescrizione scala equabile.
	 Nel caso del $La$ (o A), avremo:
	 \begin{center}
	 	\begin{tikzpicture} [scale=0.76,every node/.style={font=\small}]
	 		\draw (0,0)--(15,0);
	 		\node at (1.2,1) {440 Hz};
	 		\node at (2.4,-1) {466.1638 Hz};
	 		\node at (3.6,0.5) {493.8833 Hz};
	 		\node at (4.8,-0.5) {523.2511 Hz};
	 		\node at (6,1) {554.3653 Hz};
	 		\node at (7.2,-1) {587.3295 Hz};
	 		\node at (8.4,0.5) {622.2540 Hz};
	 		\node at (9.6,-0.5) {659.2551 Hz};
	 		\node at (10.8,1) {698.4564 Hz};
	 		\node at (12,-1) {739.9888 Hz};
	 		\node at (13.2,0.5) {783.9909 Hz};
	 		\node at (14.4,-0.5) {830.6094 Hz};
	 		
	 		\draw (0,0)--(15,0);
	 		\node at (1.2,0) {\textbullet};
	 		\node at (2.4,0) {\textbullet};
	 		\node at (3.6,0) {\textbullet};
	 		\node at (4.8,0) {\textbullet};
	 		\node at (6,0) {\textbullet};
	 		\node at (7.2,0) {\textbullet};
	 		\node at (8.4,0) {\textbullet};
	 		\node at (9.6,0) {\textbullet};
	 		\node at (10.8,0) {\textbullet};
	 		\node at (12,0) {\textbullet};
	 		\node at (13.2,0) {\textbullet};
	 		\node at (14.4,0) {\textbullet};
	 	\end{tikzpicture}
	 \end{center} 
	 Il problema della trasposizione è risolto: se volessimo costruire la scala equabile partendo da un'altra nota; ad esempio dalla terza nota della scala corrispondente a $493.8833 Hz$, le successive note starebbero tra loro negli stessi rapporti di frequenze di quelle elencate sopra e coinciderebbero con esse.\\
	 \\
	 Cerchiamo di chiarire ancora meglio: prendiamo come nota di riferimento sempre il $La$, che da ora in avanti chiameremo $A$. Assegniamo ora alle frequenze temperate equabilmente i seguenti nomi:
	 \begin{center}
	 	\begin{tabular}{||c|c||}
	 		\hline 
	 		Frequenza & Nome\\
	 		\hline
	 		440Hz&A\\
	 		\hline 
	 		466.1638 Hz&$A\sharp$=$B\flat$\\
	 		\hline 
	 		493.8833 Hz&B\\
	 		\hline
	 		523.2511 Hz&C\\
	 		\hline
	 		554.3653 Hz&$C\sharp$=$D\flat$\\
	 		\hline 
	 		587.3295 Hz&D\\
	 		\hline
	 		622.2540 Hz&$D\sharp=E\flat$\\
	 		\hline
	 		659.2551 Hz&E\\
	 		\hline
	 		698.4564 Hz&F\\
	 		\hline
	 		739.9888 Hz&$F\sharp=G\flat$\\
	 		\hline
	 		783.9909 Hz&G\\
	 		\hline 
	 		830.6094 Hz&$G\sharp=A\flat$\\
	 		\hline 880 Hz &A\\
	 		\hline
	 	\end{tabular}
	 \end{center}
	 Possiamo allora riscrivere la scala sostituendo i nomi alle frequenze come:
	 \begin{center}
	 		\resizebox{13 cm}{!}{
	 	\begin{tabular}{|c|c|c|c|c|c|c|c|c|c|c|c|c|}
	 		\hline
	 		$A$&$A\sharp=B\flat$&$B$&$C$&$C\sharp=D\flat$&$D$&$D\sharp=E\flat$&$E$&$F$&$F\sharp=G\flat$&$G$&$G\sharp$&$A$ \\
	 		\hline
	 	\end{tabular}
	 }
	 \end{center}
	 Notare che a $440Hz$ ed a $880Hz$ abbiamo associato lo stesso nome, in quanto l'orecchio le sente come lo stesso suono ma a due altezze diverse. Inoltre, la nota sopra $880Hz$, secondo il temperamento equabile, sarà alla frequenza $932.328 Hz=880 Hz\times 2^{1\over 12}$, che è esattamente il doppio di $466.1638 Hz=A\sharp=B\flat$ e per ciò avrà lo stesso nome. Questo ragionamento vale anche per tutte quante le note successive, che allora avranno lo stesso nome delle loro predecessore.\\
	 \\
	 E' evidente che il temperamento equabile ci ha permesso di costruire una scala: una suddivisione dell'ottava in 12 intervalli in rapporti costanti di frequenze. Questa struttura si ripete di ottava in ottava, mantenendo la stessa struttura: ogni nota è in rapporto di frequenza con le adiacenti di esattamente $2^{1\over 12}$.
	 Da ora in poi chiameremo la scala equabile con il nome di scala cromatica. 
	 \begin{Def}
	 	Chiamiamo semitono la l'intervallo tra due note che corrisponde ad un rapporto di $2^{1\over12}$, ovvero due note adiacenti. Chiamiamo \textbf{tono} l'intervallo tra due note che corrisponde ad un rapporto di $2^{2\over 12}$ (due semitoni).
	 \end{Def} 
	 Il temperamento equabile ha risolto il problema della trasposizione, ma al prezzo di rovinare la Quinta Giusta. In particolare, ricordiamo che questa corrisponde ad un rapporto di $3\over 2$. Nel caso della scala cromatica, la settima nota della scala sta in un rapporto di $2^{7\over 12}\simeq 1.4983$ con la fondamentale. E' facile accogersi che questo numero è sorprendentemente vicino a $3\over 2$. Dunque, nel caso del temperamento equabile, la Quinta non è più Giusta... ma quasi. Il prezzo da pagare per ottenere una suddivisione della scala che ci permetta di trasporre, è quello di approssimare gli intervalli giusti.\\
	 \\
	 Cerchiamo ora di dare un po' di nomenclatura. Ripetiamo la struttura della scala cromatica o equabile:
	 $$1,2^{1\over12},2^{2\over12},2^{3\over12},2^{4\over12},2^{5\over12},2^{6\over12},2^{7\over12},2^{8\over12},2^{9\over12},2^{10\over12},2^{11\over12},2^{12\over12}=2$$
	 Questi sono dodici intervalli o gradi. Diamo ora i seguenti nomi:
	 \begin{center}
	 		\resizebox{13 cm}{!}{
	 	\begin{tabular}{||c|c|c||} 
	 		\hline
	 		Intervallo & Nome & Distanza in semitoni\\ [0.5ex] 
	 		\hline\hline
	 		1 & Unisono o Fondamentale &0\\ 
	 		\hline
	 		$2^{1\over12}$ & Seconda minore& 1 semitono \\
	 		\hline
	 		$2^{2\over12}$ & Seconda maggiore& 2 semitoni= 1 tono\\
	 		\hline
	 		$2^{3\over12}$ & Terza minore&3 semitoni = 1 tono e mezzo\\
	 		\hline
	 		$2^{4\over12}$ & Terza maggiore & 4 semitoni = 2 toni \\ [1ex] 
	 		\hline
	 		$2^{5\over12}$ & Quarta & 5 semitoni = 2 toni e mezzo \\
	 		\hline
	 		$2^{6\over12}$ & Quarta aumentata o quinta diminuita &6 semitoni = 3 toni\\
	 		\hline
	 		$2^{7\over12}$ & Quinta&7 semitoni = 3 toni e mezzo\\
	 		\hline
	 		$2^{8\over12}$ & Quinta aumentata o Sesta minore (diminuita) & 8 semitoni = 4 toni \\ [1ex] 
	 		\hline
	 		$2^{9\over12}$ & Sesta & 9 semitoni = 4 toni e mezzo \\
	 		\hline
	 		$2^{10\over12}$ & Sesta aumentata o Settima minore & 10 semitoni = 5 toni\\
	 		\hline
	 		$2^{11\over12}$ & Settima maggiore &11 semitoni = 5 toni e mezzo\\
	 		\hline
	 		$2$ & Ottava &12 semitoni = 6 toni \\  
	 		\hline
	 	\end{tabular}
	 }
	 \end{center}
	  Nel caso della scala di $A$, avremo:
	  \begin{center}
	  	\begin{tabular}{||c|c||} 
	  		\hline
	  		Intervallo & Nome \\ [0.5ex] 
	  		\hline\hline
	  		A & Unisono o Fondamentale \\ 
	  		\hline
	  		$A\sharp=B\flat$ & Seconda minore \\
	  		\hline
	  		$B$ & Seconda maggiore\\
	  		\hline
	  		$C$ & Terza minore\\
	  		\hline
	  		$C\sharp=D\flat$ & Terza maggiore \\ [1ex] 
	  		\hline
	  		$D$ & Quarta \\
	  		\hline
	  		$D\sharp=E\flat$ & Quarta aumentata o quinta diminuita\\
	  		\hline
	  		$E$ & Quinta\\
	  		\hline
	  		$F$ & Quinta aumentata o Sesta minore (diminuita) \\ [1ex] 
	  		\hline
	  		$F\sharp=G\flat$ & Sesta \\
	  		\hline
	  		$G$ & Sesta aumentata o Settima minore\\
	  		\hline
	  		$G\sharp=A\flat$ & Settima maggiore\\
	  		\hline
	  		$A$ & Ottava \\  
	  		\hline
	  	\end{tabular}
	  \end{center}
	  Nel caso della scala di avremo:
	  	  \begin{center}
	  	  	
	  	\begin{tabular}{||c|c||} 
	  		 
	  		\hline
	  		Intervallo & Nome \\ [0.5ex] 
	  		\hline\hline
	  		C & Unisono o Fondamentale \\ 
	  		\hline
	  		$C\sharp=D\flat$ & Seconda minore \\
	  		\hline
	  		$D$ & Seconda maggiore\\
	  		\hline
	  		$D\sharp=E\flat$ & Terza minore\\
	  		\hline
	  		$E$ & Terza maggiore \\ [1ex] 
	  		\hline
	  		$F$ & Quarta \\
	  		\hline
	  		$F\sharp=G\flat$ & Quarta aumentata o Quinta diminuita\\
	  		\hline
	  		$G$ & Quinta\\
	  		\hline
	  		$G\sharp=A\flat$ & Quinta aumentata o Sesta minore (diminuita) \\  
	  		\hline
	  		$A$ & Sesta \\
	  		\hline
	  		$A\sharp=B\flat$ & Sesta aumentata o Settima minore\\
	  		\hline
	  		$B$ & Settima maggiore\\
	  		\hline
	  		$C$ & Ottava \\  
	  		\hline
	  	\end{tabular}
	  \end{center}
	  E così via.\\
	  \\
	  In ultimo, diamo dei nomi alle note: 
	   \begin{center}
	  	\begin{tabular}{||c|c||} 
	  		\hline
	  		Nota & Nome \\ [0.5ex] 
	  		\hline\hline
	  		C & Do\\ 
	  		\hline
	  		$C\sharp=D\flat$ & Do diesis o Re bemolle \\
	  		\hline
	  		$D$ & Re\\
	  		\hline
	  		$D\sharp=E\flat$ & Re diesis o Mi bemolle\\
	  		\hline
	  		$E$ & Mi \\ [1ex] 
	  		\hline
	  		$F$ & Fa \\
	  		\hline
	  		$F\sharp=G\flat$ & Fa diesis o Sol bemolle\\
	  		\hline
	  		$G$ & Sol\\
	  		\hline
	  		$G\sharp=A\flat$ & Sol diesis o La bemolle \\  
	  		\hline
	  		$A$ & La \\
	  		\hline
	  		$A\sharp=B\flat$ & La diesis o Si bemolle\\
	  		\hline
	  		$B$ & Si\\
	  		\hline
	  		$C$ & Do \\  
	  		\hline
	  	\end{tabular}
	  \end{center}
	  Nonostante possa sembrar più conveniente utilizzare i nomi canonici Do Re Mi ecc, noi useremo la notazione alfabetica, in quanto è molto più conveniente.
	 \chapter{La scala maggiore}
	 La scala maggiore è una scala di 7 note costruita sulla scala cromatica. La struttura della scala maggiore è la seguente:
	 $$\hbox{Tono-Tono-Semitono-Tono-Tono-Tono-Semitono}$$ 
	 Ricordiamo che due note adiacenti distano un semitono, mentre due semitoni compongono un tono. Vediamo un esempio:
	 \begin{Ex}
	 	Vogliamo costruire la scala maggiore di $A$, a partire da tutte le note della scala cromatica:
	 	\begin{center}
	 			\resizebox{11 cm}{!}{
	 		\begin{tabular}{|c|c|c|c|c|c|c|c|c|c|c|c|c|}
	 			\hline
	 			$A$&$A\sharp=B\flat$&$B$&$C$&$C\sharp=D\flat$&$D$&$D\sharp=E\flat$&$E$&$F$&$F\sharp=G\flat$&$G$&$G\sharp$&$A$ \\
	 			\hline
	 		\end{tabular}
	 	}
	 	\end{center}
	 	La struttura soprastante chiede le seguenti note:
	 	$$A,B,C\sharp,D,E,F\sharp,G\sharp,A$$
	 	Questa è la scala maggiore di $A$.\\
	 	\\
	 	Abbiamo appena detto che la scala maggiore ha 7 note, ma qui ce ne sono scritte 8...
	 	Questo perché l'ultima nota coincide con la prima, quindi viene omessa dalla scala. Noi la includiamo per chiarezza (o per confondere le idee).\\
	 		 \end{Ex}
	 		 Consideriamo un altro esempio:
	 \begin{Ex}
	 	Vogliamo costruire la scala maggiore di $C$, a partire da tutte le note della scala cromatica:
	 		 	\begin{center}
	 		 			\resizebox{12 cm}{!}{
	 		 		\begin{tabular}{|c|c|c|c|c|c|c|c|c|c|c|c|c|}
	 		 			\hline
	 		 			$A$&$A\sharp=B\flat$&$B$&$C$&$C\sharp=D\flat$&$D$&$D\sharp=E\flat$&$E$&$F$&$F\sharp=G\flat$&$G$&$G\sharp$&$A$ \\
	 		 			\hline
	 		 	\end{tabular}}
	 		 \end{center}
	 		 La struttura soprastante chiede le seguenti note:
	 		 	$$C,D,E,F,G,A,B,C$$
	 		 Questa è la scala maggiore di $C$.\\
	 		 \\
	 		 Cerchiamo di costruirla passo passo assieme: la nota di partenza è $C$. Vogliamo seguire la struttura
	 		 $$\hbox{Tono-Tono-Semitono-Tono-Tono-Tono-Semitono}$$
	 		 Dunque, la nota che dista 1 tono da $C$ è $D$: questa sarà la nostra seconda nota della scala. Vogliamo poi avanzare di ancora un tono: la nota che dista 1 tono da $D$ (o 2 toni da $C$) è $E$; questa sarà la terza nota della scala. Vogliamo ora avanzare di un semitono, dunque prenderemo $F$. Poi avanziamo di 1 tono e selezioniamo $G$, poi $A$, poi... poi è finita la scala cromatica?... NO! La nota successiva ad $A$ è sempre $B$, semplicemente di un'ottava più in alto rispetto alla $B$ che lo precedeva! Prendiamo allora come settima nota $B$ ed infine di nuovo $C$. 
	 		 \end{Ex}
	 		 Si invita il lettore a costruire le scale maggiori di tutte e 12 le note.\\
	 		 \\
	 		 Ora che la struttura della scala maggiore è chiara, cerchiamo di studiarne le peculiarità. Possiamo far ciò prendendo come esempio una a caso tra le 12 scale, in quanto, grazie al temperamento equabile, la struttura è la stessa per tutte. Dunque tutti i risultati che troveremo per una nota, varranno per tutte le altre.\\
	 		 \\
	 		 Consideriamo la scala maggiore di $C=Do$. Essa è composta da:
	 		 $$C,D,E,F,G,A,B,C$$
	 		 Dunque, essa contiene: una seconda maggiore ($D$), una terza maggiore ($E$), una quarta ($F$), una quinta ($G$), una sesta maggiore ($A$), una settima maggiore $B$ (ed un'ottava $C$ grazie graziella e grazie al cazzo).\\
	 		 Il lettore munito di pianoforte potrà suonare queste note una dopo l'altra ed accorgersi di come questa scala risulti particolarmente felice all'orecchio. Ciò è dovuto principalmente alla terza maggiore. Approfondiremo questo in seguito.\\
	 		 \\
	 		 Tuttavia, è facile anche accorgersi che ogni nota della scala maggiore paia voler "risolvere" alla fondamentale, che nel nostro caso è $C$. Non è difficile accorgersi di questo: suonando una dopo l'altra senza un ordine preciso le note della scala maggiore di $C$, è facile vedere come la nota $C$ suoni meravigliosamente rispetto alle altre. Suoneranno particolarmente bene le seguenti combinazioni:
	 		 $$C\rightarrow G\rightarrow C;C\rightarrow D\rightarrow G\rightarrow C$$
	 		 O ancora:
	 		 $$C\rightarrow E\rightarrow G\rightarrow C;B\rightarrow C$$
	 		 Insomma, la sensazione che si prova è che la fondamentale della scala sia la nota più importante della scala all'orecchio; quella che suona meglio e che conferisce un senso di completezza maggiore.
	 		 \subsection{La scala minore}
	 		 Avendo capito la struttura della scala maggiore, non è difficile comprendere quella della scala minore. La struttura che chiediamo ora è:
	 		 $$\hbox{Tono-Semitono-Tono-Tono-Semitono-Tono-Tono}$$
	 		 Vediamo un esempio:
	 		 \begin{Ex}
	 		 	La scala minore di $C$ è costruita a partire da quella cromatica, ovvero da queste note:
	 		 	\begin{center}
	 		 			\resizebox{10 cm}{!}{
	 		 		\begin{tabular}{|c|c|c|c|c|c|c|c|c|c|c|c|c|}
	 		 			\hline
	 		 			$A$&$A\sharp=B\flat$&$B$&$C$&$C\sharp=D\flat$&$D$&$D\sharp=E\flat$&$E$&$F$&$F\sharp=G\flat$&$G$&$G\sharp$&$A$ \\
	 		 			\hline
	 		 		\end{tabular}
	 		 	}
	 		 	\end{center}
	 		 	La nostra prescrizione ci impone:
	 		 	$$\hbox{Tono-Semitono-Tono-Tono-Semitono-Tono-Tono}$$
	 		 	Allora la seconda nota della scala sarà quella che dista un tono da $C$, ovvero $D$. La terza disterà un semitono da $D$ e sarà quindi $E\flat$. La quarta disterà un tono da $E\flat$ e quindi sarà $F$, poi $G$, $A\flat$ e $B\flat$. Otterremo:
	 		 	$$C,D,E\flat,F,G,A\flat,B\flat,C$$
	 		 	Notiamo, comparando la scala minore di $C$ con quella maggiore di $C$, che le differenze sono nella terza, nella sesta e nella settima. In particolare, la scala minore ha questi tre intervalli abbassati di un semitono.
	 		 \end{Ex}
	 		 Vediamo un altro esempio:
	 		 \begin{Ex}
	 		 	Costruiamo la scala minore di $D$. Come al solito, dovremo scegliere 7 tra le 12 note cromatiche:
	 		 	\begin{center}
	 		 			\resizebox{11 cm}{!}{
	 		 		\begin{tabular}{|c|c|c|c|c|c|c|c|c|c|c|c|c|}
	 		 			\hline
	 		 			$A$&$A\sharp=B\flat$&$B$&$C$&$C\sharp=D\flat$&$D$&$D\sharp=E\flat$&$E$&$F$&$F\sharp=G\flat$&$G$&$G\sharp$&$A$ \\
	 		 			\hline
	 		 		\end{tabular}}
	 		 	\end{center}
	 		 	Seguendo la solita prescrizione:
	 		 		$$\hbox{Tono-Semitono-Tono-Tono-Semitono-Tono-Tono}$$
	 		 		Se la prima nota è $D$, la seconda disterà un tono da $D$ e sarà quindi $E$. La terza nota della scala disterà mezzo tono da $E$ e quindi sarà $F$. La quarta nota invece disterà un tono da $F$ e sarà $G$. Avremo poi $A$, $B\flat$, $C$. Dunque la scala minore di $D$ sarà:
	 		 		$$D,E,F,G,A,B\flat,C,D$$
	 		 \end{Ex}
	 		 Notiamo una curiosa coincidenza: la scala minore di $A$ contiene le seguenti note: 
	 		 $$A,B,C,D,E,F,G,A$$
	 		 Queste note sono ESATTAMENTE le stesse che compongono la scala maggiore di $C$... In particolare, la scala minore di $A$ può essere ottenuta a partire da quella maggiore di $C$, ma prendendo come nota di partenza il sesto grado della scala, ovvero $A$.\\
	 		 Cerchiamo di spiegare meglio: la scala maggiore di $C$ è
	 		$$C,D,E,F,G,A,B,C$$
	 		Se prendiamo queste note, ma partiamo invece che da $C$, dal sesto grado, ovvero $A$, otteniamo:
	 		$$A,B,C,D;E,F,G,A$$ che è la scala minore di $A$!!!
	 		Insomma, la verità è che a livello strutturale non c'è nessuna differenza tra le scale minori e quelle maggiori: sono le stesse scale ma viste da punti diversi!\\
	 		\\
	 		Questa costruzione vale ovviamente per tutte quante le note. Vediamolo con qualche esempio:
	 		\begin{Ex}
	 			La scala maggiore di $F$ (il lettore è invitato a costruirla) è
	 			$$F,G,A,B\flat,C,D,E$$
	 			Il sesto grado di questa scala è $D$. Riscrivendo le note a partire da $D$ otteniamo:
	 			$$D,E,F,G,A,B\flat,C,D$$
	 			che è la scala minore di $D$
	 		\end{Ex}
	 		E'allora evidente il seguente teorema:
	 		\begin{Theo}
	 			Vi è una corrispondenza biunivoca tra scale minori e maggiori.
	 		\end{Theo}
	 		In generale la scala minore è decisamente più triste della scala maggiore. Questo alone di tristezza è da imputare principalmente alla terza minore: abbassare la terza in generale conferisce più malinconia alla scala. Tutto questo sarà più chiaro studiando l'armonizzazione delle scale.
	 		\begin{Obs}
	 			Il lettore attento avrà sicuramente notato un sottile problema con la nostra costruzione: se per ogni scala maggiore esiste una scala minore, e la fondamentale della scala maggiore è la nota verso cui tutte le altre tendono, come fa la fondamentale della corrispettiva scala minore ad essere diversa da quella della scala maggiore?\\
	 			Mi spiego peggio: prendiamo la scala maggiore di $C$:
	 			$$C,D,E,F,G,A,B,C$$
	 			In questa scala abbiamo detto che tutte le note vogliono risolvere (tendono musicalmente) alla fondamentale $C$, nel senso che il senso di appagamento più grande per il nostro orecchio all'interno di questa scala è dato da $C$.\\
	 			La scala minore di $A$ contiene esattamente le stesse note:
	 			$$A,B,C,D,E,F,G,A$$
	 			Tuttavia, la fondamentale qua è $A$... ma questo allora significa che tutto tende ad $A$?...Si...più o meno.\\
	 			La risposta a questa domanda è molto complicata ed ha a che fare con un bel po' di armonia e di soggettività, ma può essere banalmente semplificata nella seguente massima:
	 			$$\hbox{\textit{Se suono più spesso $C$ allora la fondamentale è $C$, ed il mio orecchio sente la scala}}$$ $$\hbox{\textit{come se fosse la scala maggiore di $C$. Se suono più spesso $A$}}$$
	 			$$\hbox{\textit{la fondamentale è $A$ e il mio orecchio sente la scala come fosse la scala minore di $A$}}$$
	 		\end{Obs}
	 		\chapter[l'armonizzazione]{L'armonizzazione delle scale maggiori e minori}
	 		Armonizzare una scala significa costruire degli accordi su di una scala.
	 		\begin{Def}
	 			Un accordo è un insieme di più note suonate insieme.
	 		\end{Def}
	 		Saremo interessati ad accordi di $3$ note in quanto essi sono i più utilizzati e spesso e volentieri i più belli.\\
	 		\\
	 		Partiamo dall'armonizzazione della scala maggiore.
	 		Prendiamo come esempio la scala maggiore di $C$:
	 		$$C,D,E,F,G,A,B$$
	 		Vi sono tanti possibili accordi di 3 note, ma i più usati sono quelli maggiori e quelli minori. Un accordo maggiore è costituito dalle 3 seguenti note: 
	 		\begin{enumerate}
	 			\item La fondamentale dell'accordo;\\
	 			\item La terza maggiore della fondamentale;\\
	 			\item La quinta della fondamentale;\\
	 		\end{enumerate}
	 		Nel caso di $C$, l'accordo maggiore di $C$ si indica con un abuso di notazione con $C$... ed è composto dalle note: $C,E,G$. Noi per chiarire lo chiameremo $CMaj$. Questo vale anche per tutte le altre note: $DMaj, E\flat Maj,G\sharp MAj...$ ecc ecc.
	 		Un accordo minore è invece costituito dalle seguenti 3 note:
	 		\begin{enumerate}
	 			\item La fondamentale dell'accordo;\\
	 			\item La terza minore della fondamentale;\\
	 			\item La quinta della fondamentale;\\
	 		\end{enumerate}
	 		Nel caso di $C$, l'accordo minore di $C$ si indica con $Cm$... ed è composto dalle note: $C,E\flat,G$.
	 		Notiamo che la differenza tra un accordo maggiore e minore è solamente nella terza. Tuttavia, suonando questi due accordi ci si accorge subito di come quello maggiore risulti notevolmente più gioioso, mentre quello minore più malinconico.
	 		\begin{Ex}
	 			Costruiamo un po' di accordi maggiori e minori sulla scala cromatica, per prendere familiarità:\\
	 			\\
	 			L'accordo di $E$ maggiore avrà la fondamentale ($E$), la terza maggiore $(G\sharp)$ e la quinta ($B$). Sarà allora
	 			$$EMaj=E,G\sharp,B$$
	 			L'accordo di $E$ minore avrà la fondamentale ($E$), la terza minore $(G)$ e la quinta ($B$). Sarà allora 
	 			$$Em=E,G,B$$
	 			E' chiaro che per passare da un accordo maggiore ad uno minore basta abbassare la terza di un semitono.\\
	 			\\
	 			Facciamone ora uno difficile: L'accordo di $F\sharp$ maggiore ha: la fondamentale $F\sharp$, la terza maggiore $A\sharp$ e la quinta $C\sharp$ e sarà quindi:
	 			$$F\sharp Maj=F\sharp,A\sharp,C\sharp$$
	 			L'accordo di $F\sharp$ minore ha: la fondamentale $F\sharp$, la terza minore $A$ e la quinta $C\sharp$ e sarà quindi:
	 			$$F\sharp m=F\sharp,A,C\sharp$$
	 		\end{Ex}
	 		Ora che abbiamo compreso come si costruiscono gli accordi maggiori e minori, armonizziamo la scala maggiore. L'idea è quella di prendere ogni nota e costruirvici un accordo sopra. Dunque, ad ogni nota della scala affiancheremo una terza (maggiore o minore) ed una quinta, rimanendo nella scala. Facciamo un esempio:
	 		\begin{Ex}
	 			La scala maggiore di $C$ è:
	 			$$C,D,E,F,G,A,B,C$$
	 			Su ogni nota di questa scala costruiamo un accordo. Avremo un accordo di $C$, uno di $D$, uno di $E$ e così via.\\
	 			\\
	 			Partiamo da $C$. Rimanendo all'interno della scala, $C$ ha una terza maggiore $E$ ed una quinta $G$. Allora l'accordo sarà $CEG$ ovvero $CMaj$.\\
	 			Prendiamo ora la seconda nota, $D$. La sua terza è, rimanendo nella scala, $F$, che è una terza minore per $D$. Avremo poi una quinta di $D$, ovvero $A$. Allora l'accordo che costruiremo su $D$ sarà $Dm=DFA$.\\
	 			Prendiamo poi la terza nota: $E$. La sua terza all'interno della scala è $G$, che è una terza minore per $E$. Avremo poi una quinta $B$ (sempre per $E$). Allora l'accordo sarà $Em=EGB$ minore. Avanzando con questo metodo, costruiamo i seguenti accordi:
	 			\begin{center}
	 				\begin{tabular}{||c|c||}
	 					\hline
	 					Nome Nota & Accordo\\
	 					\hline
	 					$C$ & $CMaj=CEG$\\
	 					\hline
	 					$D$ & $Dm=DFA$\\
	 					\hline
	 					$E$ & $Em=EGB$\\
	 					\hline
	 					$F$ & $FMaj=FAC$\\
	 					\hline
	 					$G$ & $GMaj=GBD$\\
	 					\hline
	 					$A$ & $Am=ACE$\\
	 					\hline
	 					$B$ & $Bm^{\flat5}=BDF$\\
	 					\hline
	 				\end{tabular}
	 			\end{center}
	 			Notiamo subito una peculiarità: se proviamo a costruire l'accordo sull'ultimo grado della scala, otteniamo qualche cosa di inusuale: la nota $B$ ha una terza minore: $D$, ed una quinta diminuita $F$. Tutti gli altri accordi hanno una quinta "giusta", mentre questo no. Chiamiamo questo accordo $$Bm^{\flat5}$$
	 			Indicando con $m$ la terza minore e con $\flat5$ la quinta abbassata di un semitono. Chiamiamo questo accordo diminuito per semplicità.
	 		\end{Ex}
	 		\begin{Obs}
	 			La costruzione che abbiamo operato sulla scala di $C$ è estendibile a tutte le altre scale grazie al temperamento equabile! In particolare, il primo grado della scala avrà un accordo maggiore, il secondo ed il terzo minori, il quarto ed il quinto maggiori, il sesto ancora minore ed il settimo diminuito.\\
	 			\\
	 			Il lettore potrà verificare ciò che scrivo armonizzando tutte e 12 le scale. Riportiamo qua un paio di esempi:
	 		\end{Obs}
	 		\begin{Ex}
	 			Per la scala di $E$ avremo:
	 				\begin{center}
	 				\begin{tabular}{||c|c||}
	 					\hline
	 					Nome Nota & Accordo\\
	 					\hline
	 					$E$ & $EMaj$\\
	 					\hline
	 					$F\sharp$ & $F\sharp m$\\
	 					\hline
	 					$G\sharp$ & $G\sharp m$\\
	 					\hline
	 					$A$ & $AMaj$\\
	 					\hline
	 					$B$ & $BMaj$\\
	 					\hline
	 					$C\sharp$ & $C\sharp m$\\
	 					\hline
	 					$D\sharp$ & $D\sharp m^{\flat5}$\\
	 					\hline
	 				\end{tabular}
	 			\end{center}
	 		\end{Ex}
	 		\begin{Ex}
	 			Per la scala di $B$ avremo:
	 				Per la scala di $E$ avremo:
	 			\begin{center}
	 				\begin{tabular}{||c|c||}
	 					\hline
	 					Nome Nota & Accordo\\
	 					\hline
	 					$B$ & $BMaj$\\
	 					\hline
	 					$C\sharp$ & $C\sharp m$\\
	 					\hline
	 					$D\sharp$ & $D\sharp m$\\
	 					\hline
	 					$E$ & $EMaj$\\
	 					\hline
	 					$F\sharp$ & $F\sharp Maj$\\
	 					\hline
	 					$G\sharp$ & $G\sharp m$\\
	 					\hline
	 					$A\sharp$ & $A\sharp m^{\flat5}$\\
	 					\hline
	 				\end{tabular}
	 			\end{center}
	 		\end{Ex}
	 		La struttura generale sarà la seguente:
	 		\begin{center}
	 			\begin{tabular}{||c|c||}
	 				\hline
	 				Nome Nota & Accordo\\
	 				\hline
	 				$Fondamentale$ & $Maj$\\
	 				\hline
	 				$Seconda maggiore$ & $m$\\
	 				\hline
	 				$Terza maggiore$ & $m$\\
	 				\hline
	 				$Quarta$ & $Maj$\\
	 				\hline
	 				$Quinta$ & $Maj$\\
	 				\hline
	 				$Sesta$ & $m$\\
	 				\hline
	 				$Settima maggiore$ & $m^{\flat5}$\\
	 				\hline
	 			\end{tabular}
	 		\end{center}
	 		Possiamo ora armonizzare la scala minore. L'idea è quella ri ripetere il procedimento fatto per la scala maggiore: prendere ogni nota ed associarvici un accordo costruendolo all'interno della scala con terze e quinte. Tuttavia, ricordiamo che ad ogni scala maggiore è associata una scala minore! Ma allora, l'armonizzazione della scala minore sarà la stessa di quella della scala maggiore, semplicemente partendo dal sesto grado.
	 		\begin{center}
	 			\begin{tabular}{||c|c||}
	 				\hline
	 				Nome Nota & Accordo\\
	 				\hline
	 				$Fondamentale$ & $m$\\
	 				\hline
	 				$Seconda maggiore$ & $m^{\flat5}$\\
	 				\hline
	 				$Terza minore$ & $Maj$\\
	 				\hline
	 				$Quarta$ & $m$\\
	 				\hline
	 				$Quinta$ & $m$\\
	 				\hline
	 				$Sesta minore$ & $Maj$\\
	 				\hline
	 				$Settima minore$ & $Maj$\\
	 				\hline
	 			\end{tabular}
	 		\end{center}
	 		\begin{Ex}
	 			La scala minore di $A$ può essere armonizzata prendendo quella di $C$ e riscrivendola a partire dal sesto grado:
	 			\begin{center}
	 				\begin{tabular}{||c|c||}
	 					\hline
	 					Nome Nota & Accordo\\
	 					\hline
	 					$A$ & $Am$\\
	 					\hline
	 					$B$ & $Bm^{\flat5}$\\
	 					\hline
	 					$C$ & $CMaj$\\
	 					\hline
	 					$D$ & $Dm$\\
	 					\hline
	 					$E$ & $Em$\\
	 					\hline
	 					$F$ & $FMaj$\\
	 					\hline
	 					$G$ & $GMaj$\\
	 					\hline
	 				\end{tabular}
	 			\end{center}
	 		\end{Ex}
\end{document}
