\documentclass[12pt,a4paper]{report}

\usepackage[italian]{babel}
\usepackage{newlfont}
\usepackage{color}
\usepackage{float}
\usepackage{frontespizio}
\usepackage{amsmath,amssymb}
\usepackage{amsthm}
\usepackage{geometry}
\usepackage{tikz}
\usepackage{biblatex}
\usepackage{csquotes}
\usepackage{pgfplots}
\usepackage{hyperref}
\usepackage{amssymb}
\usepackage{comment}
\usepackage[compat=1.0.0]{tikz-feynman}
\usepackage{tikz-cd}
\usepackage{mathtools}

\hypersetup{
	colorlinks=true,
	linkcolor=blue,
	filecolor=magenta,      
	urlcolor=cyan,
	pdftitle={Overleaf Example},
	pdfpagemode=FullScreen,
}

\textwidth=450pt\oddsidemargin=0pt
\geometry{a4paper, top=3cm, bottom=3cm, left=3cm, right=3cm, % heightrounded, bindingoffset=5mm 
}
\theoremstyle{definition}
\newtheorem{Def}{Definizione}[chapter]

\theoremstyle{Theorem}
\newtheorem{Theo}[Def]{Teorema}
\newtheorem{Prop}[Def]{Proposizione}

\newtheorem{Lm}[Def]{Lemma}

\theoremstyle{definition}
\newtheorem{Ex}[Def]{Esempio}

\theoremstyle{definition}
\newtheorem{Lem}[Def]{Lemma:}

\theoremstyle{definition}
\newtheorem{Obs}[Def]{Osservazione:}
\begin{document}
	\chapter{The underlying necessary algebra}
	\section{Sequences of vector spaces}
		Consider a finite set of vector spaces indexed like $V_i$. Consider also a set of linear maps $f_i:V_i\rightarrow V_{i+1}$. We call a \textbf{sequence} of vector spaces simply a chained relation:
		$$...\xrightarrow{f_{i-1}} V_i\xrightarrow{f_i}V_{i+1}\xrightarrow{f_{i+1}}V_{i+2}\xrightarrow{f_{i+2}}...$$
		An \textbf{exact sequence} of vector spaces is a sequence in which the image of each map exactly matches the kernel of the following map. This means that in the above sequence we would have:
		$$Im(f_i)=Ker(f_{i+1})$$
		This means that each element sent by $f_i$ from $V_i$ to $V_{i+i}$ will be sent to $0$ by $f_{i+1}$. Namely: $f_{i+i}(f_i)(v_{V_i})=0_{V_{i+1}}$.\\
		\\
		A common sequence is the \textbf{short exact sequence}, which has only five terms and is in the form:
		$$0\rightarrow A\xrightarrow{f}B\xrightarrow{g}C\rightarrow0$$
		The 0 on the left means that $f$ is injective, since the only linear image of $0$ is 0 and so $Ker(f)=0_A$ is trivial. The 0 on the right instead means that $g$ is surjective. In fact the $Ker$ of the following map, which corresponds tot he image of $g$, is all of the space since every element is sent linearly to 0. Note also that since $Im(f)=Ker(g)$, all of the elements sent into $B$ by $f$ from $A$ are mapped to 0 in $C$.
	\chapter{Smooth Manifolds}
	\chapter{Metric Manifolds}
	\section{Riemaniann metric}
	\begin{Def}\label{Def_4.1}
		Consider a manifold $M$ and it's tangent space $T_pM$ at any point. We call Riemaniann metric the assignment of an inner product $\langle,\rangle_p$ on the tangent space. This means that:
		\begin{itemize}
			\item $\langle,\rangle$ is bilinear;
			\item $\langle,\rangle$ is symmetric;
			\item $\langle,\rangle$ is positive definite.
		\end{itemize}
		Moreover, we require that the innser product is smooth. This means that if $X,Y\in\mathfrak{X}(M)$ then $p\rightarrow\langle X_p,Y_p\rangle_p$ is a smooth function of $M$.\\
		We call a Riemaniann Manifold a manifold with a Riemaniann metric.
	\end{Def}
	\chapter{Bundles}
	\section{Fiber Bundles}
	\begin{Def} \label{Def_5.1}
		Let $E,M,F$ be topological spaces and let $M$ be connected. Let $\pi:E\rightarrow M$ be a map with the following properties:
		\begin{itemize}
			\item $\pi$ is continuous and surjective;
			\item per every point $x_M\in M$ there existes an open set $U\subset M$ such that there exists an homeomorphism $\varphi:\pi^{-1}(U)\rightarrow U\times F$ called local trivialization, which makes the follwing diagram commute:
		\end{itemize}
		\[
		\begin{tikzcd}
			\pi^{-1}(U) \arrow{r}{\varphi} \arrow[swap]{dr}{\pi} & U\times F \arrow{d}{Proj}\\
			& U 
		\end{tikzcd}
		\]
		Where $Proj: U\times F\rightarrow U$ is the standard projection. $Proj(x_M,x_F)=x_M$
		We call $(E,M,\pi,F)$ a fiber bundle, $M$ base space, $E$ total space and $F$ fiber.
	\end{Def}
	\begin{center}
		\includegraphics[scale=0.3]{bundle_cilinder.pdf}
	\end{center}
	Basically, this definition tells us that locally we can see $E$ as homeomoprhic to the product space $M\times F$. Now we make some examples:
	\begin{Ex}[Product Bundle]\label{Ex_5.1}
		Let $M$ be a connected topological space and $F$ a fiber. We set $E=M\times F$ and construct a projection in the usual obvious way:\\ $\pi:M\times F\rightarrow M$ like $\pi(x_M,x_F)=x_M$. This map is obviusly continuous and surjective since it is a projection. Now we just need to construct local trivializations: we set
		$$\varphi:\pi^{-1}(U)\rightarrow U\times F$$
		such that $\varphi(x_M,x_F)=(x_M,x_F)$ the identity map.
	\end{Ex}
	\begin{Ex}[\textbf{Cilinder}] \label{Ex_5.2}
		An example of the product bundle is the cilinder. It can be constructed as a fiber bundle over the circle $S^1=\{z\in \mathbb{C}|z=e^{i\theta}, \theta\in[0,2\pi]\}$. We call $Cil=S^1\times [0,1]$ the cilinder and cosntruct the product bundle just like we did in example \ref{Ex_5.1}. We define the projection as: $\pi:Cil\rightarrow S^1$ that sends $\pi(z,t)\rightarrow z$. This is a projection so it is continuous and surjective.\\
		As for the local trivializations, we can take any subset of the circle $U$ and trivialize the bundle with the identity:
		$\varphi:\pi^{-1}(U)\rightarrow U\times [0,1]=\mathbb{I}_d$.\\
		In this case the fiber of the cilinder is a line. So each point has associaated a line and the totality of those lines composes the cilinder.
		\begin{center}
			\includegraphics[scale=0.3]{bundle_generic.pdf}
	\end{center}
	\end{Ex}
	\begin{Ex}[\textbf{Möbius strip}]\label{Ex_5.3}
		This is an example of a non trivial bundle.
	\end{Ex}
	\begin{Def}\label{Def_5.2}
		We say that a fiber bundle $(E,M,\pi,F)$ is a vector bundle if $M$ and $E$ are manifolds, $F$ and $\pi^{-1}(p)=F_p\in E$ are real finite-dimensional vector spaces; and the trivializing functions $\varphi:\pi^{-1}(U)\rightarrow U\times F$ are such that the induced map $\pi^{-1}(p)\rightarrow\{p\}\times F$ is a linear isomorphism.
	\end{Def}
	\begin{Def}\label{Def_5.3}
		A vector bundle is said to be trivial if $E$ is isomorphic to $M\times F$ the product bundle.
	\end{Def}
	It clearly follows that the product bundle is trivial and so it is called the trivial bundle also.
	\begin{Def}\label{Def_5.4}
		A vector bundle is said to be smooth if $\pi$ is smooth and the local trivializations are diffeomorphisms.
	\end{Def}
	\section{The tangent bundle}
		Let M be a manifold. At any point $p\in M$ we can define the tangent space $T_pM\simeq \mathbb{R}^n$. We define the tangent bundle as $TM=\bigsqcup_{p\in M}T_pM=\bigcup_p\in M \{p\}\times T_pM$ the disjoint union of all the tangent spaces. Now we want to endow the tangent bundle with a manifold structure.\\
		We have a projection $\pi:TM\rightarrow M$ obviously defined as $\pi(p,v_p)=p$ where $v_p\in T_pM$ and it is clearly surjective and continuous due to it being a projection
		\subsection{The topology of the tangent bundle}
		We want to construct a map that lets $TM$ have an induced topology from the manifold M.\\
		\\
		Let $(U,\phi)$ be a chart on $M$ with $\phi:U\rightarrow \mathbb{R}^n$. Then, any vector in $p\in U$ can be written in local coordiantes: $v_p=v^i_p\partial_i|_p$. At $p$, the functions $v^i_p$ are functions of $v$ since they are already evaluated in $p$.\\
		Now we construct the following mapping:
		$$\Phi:TU\rightarrow \phi(U)\times \mathbb{R}^n; \hbox{ like: }
		\Phi(p,v_p)=(x^i_pe_i,v^i_p\partial_i|_p)$$
		This mapping is 1-1 and surjective so it is a bijection inside $U$. Moreover, it is continuous. Thus, being an homeomorphism, $\Phi$ insuces a topology on $TM$.
		\begin{Theo} \label{Theo_5.1}
			The tangent bundle $TM$ has a second countable and Haussdorf topology
		\end{Theo}
		\begin{proof}
			Non ho voglia
		\end{proof}
		\subsection{The smooth structure}
		Consider an Atlas $\mathfrak{A}_M=\{U_\alpha,\phi_\alpha\}$ for the manifold $M$. We want to show that the induced collection $\mathfrak{A}_{TM}=\{TU_\alpha,\Phi_\alpha\}$ is an atlas on $TM$, knowing that $TM=\bigcup TU_\alpha$ (obviously) and $\Phi_\alpha:TU_\alpha\rightarrow U_\alpha\times \mathbb{R}^n$ is the map constructed above.\\
		\\
		Now, as we have seen, the maps $\Phi_\alpha$ are homeomorphisms between open sets of $TM$ and $\mathbb{R}^{2n}$. It remains to check the compatibility between two overlapping charts.\\
		\\
		Consider two overlapping charts $\Phi_1,\Phi_2$ on $TU_1,TU_2$ open sets of the tangent bundle, with $TU_1\cap TU_2\neq \emptyset$. Then on the two charts $(U_1,\phi_1)$, $(U_2,\phi_2)$ we can express tangent vectors in two different coordiante basis:
		$$v=a^i{\partial\over \partial{x^i}}=b^j{\partial \over \partial y^j}$$
		Clearly at any point:
		$$a^i=b^j{\partial x^i\over\partial y^j} \hbox{ and } b^j=a^i{\partial y^j\over\partial x^i}$$
		Now it only remains to show that the composition $\Phi_1\circ \Phi_2^{-1}$ is a diffeomorphism.\\
		By definition:
		$$\Phi_1\circ \Phi_2^{-1}:\phi_2(U_2\cap U_1)\times \mathbb{R}^n\rightarrow \phi_1(U_2\cap U_1)\times \mathbb{R}^n$$
		This map, being a composition of heomeomorphisms, is still an homeomorphism. Now, taking a point $p\in U_2\cap U_1$ and a tangent vector $v_p$ we wxplicitly write:
		$$\Phi_1\circ \Phi_2^{-1}(x^i_p,a^i_p)=(y^j_p,b^j_p)$$
		where we have shortened the notation:
		$$(x^i,a^i)=(x^1_p,...,x^n_p,a^1_p,...,a^n_p)$$
		However, we can make the following substitution:
		$$(y^j_p,b^j_p)=\bigg((\phi_2\circ \phi_1^{-1})(\phi_1(p)),a^i_p{\partial y^j\over\partial x^i}\bigg|_p\bigg)=\bigg((\phi_2\circ \phi_1^{-1})(\phi_1(p)),a^i_p{\partial (\phi_2\circ \phi_1^{-1})^j\over\partial r^i}\bigg|_p(\phi_1(p))\bigg)$$
		Due to $\phi_2\circ \phi_1^{-1}$ being a diffeomorphism, $\Phi_1\circ \Phi_2^{-1}$ is also a diffeomorphism.\\
		This completes the proof.\\
		\\
		Thus, the Tangent bundle $TM$ of a manifold of dimension $n$ is also a manifold, but of dimension $2n$, with an atlas given by:
		$$\mathfrak{A}_{TM}=\{TU_\alpha,\Phi_\alpha\} \hbox{ where } \Phi_\alpha:TU_\alpha\rightarrow U_\alpha\times \mathbb{R}^n$$
		\subsection{The smooth vector bundle structure of the tangent bundle}
		Consider a manifold $M$ of dimension $n$ and it's tangent bundle $TM$ with the induced $2n$-manifold structure from $M$. We have a projection $\pi:TM\rightarrow M$ defined as $\pi(p,v_p)=p$ which is continuous and surjective. 
		\\
		Before constructing the local trivializations, we need to show that this projection is smooth. This is done via charts. Consider a chart $(U,\phi)$ on $M$ and the corresponding $(TU,\Phi:TU\rightarrow \phi(U)\times \mathbb{R}^n)$. Then we can look at the composition:
		$$\phi\circ \pi\circ \Phi^{-1}:\phi(U)\times \mathbb{R}^n\rightarrow \mathbb{R}^n \hbox{ such that }$$ 
		$$\phi(\pi(\Phi^{-1}(x^i,a^i)))=\phi(\pi(p,v_p))=\phi(p)=x^i$$
		This is clearly smooth and so $\pi$ is to, since $\phi$ and $\Phi$ are smooth and homeomorphisms.\\
		\\
		Now we need to construct local trivializations. We wish for an homeomorphism $\varphi:\pi^{-1}(U)\rightarrow U\times F$ where the fiber at each point is the tangent space $T_pM$. We also wish for this homeomorphism to be a diffeomorphism and to reduce on the fiber to a linear isomorphism.\\
		\\
		Given any chart on $TM$ like $(TU,\Phi)$, we take as our candidate the map: 
		$$\varphi:(\phi^{-1}\times \mathbb{I}_{\mathbb{R}^n})\circ \Phi:TU\rightarrow U\times \mathbb{R}^n$$
		In particular we have the following succession:
		$$\varphi:TU\xrightarrow{\text{$\Phi$}} \phi(U)\times \mathbb{R}^n\xrightarrow{\text{$\phi^{-1}\times \mathbb{I}_{\mathbb{R}^n}$}}U\times \mathbb{R}^n$$
		This map is clearly an homeomorphism and a diffeomorphism since the charts on manifolds are both and so is the identity. As for the linear isomorphism reduction, consider the map:
		$\varphi_p$ where we fix the first argument $p$. The induced map is the follwoing:
		$$\varphi_p(v_p)=\varphi(p,v_p)=(p,v^i_p\partial_i|_p)$$
		This is clearly a linear isomorphism between $\mathbb{R}^n$ and itself (we have the idenity composed with $\Phi$, which reduces to an isomorphism).
	\section{Sections and frames}
	\begin{Def}\label{Def_5.5}
		Let $(E,M,\pi,F)$ be a vector bundle. We say that a map \\$s:U\subset M\rightarrow E$ is a section if $\pi\circ s=\mathbb{I}_d$. \\
		Let $(U_M,\phi_M)$ be a chart on $M$, $\phi_M:U_M\rightarrow \mathbb{R}^n$ and $(U_E,\phi_E)$ be a chart on $E$, $\phi_E:U_E\rightarrow \mathbb{R}^k$. Then we say that $s$ is a smooth section if $\phi_E\circ s\circ \phi^{-1}_M:\mathbb{R}^n\rightarrow \mathbb{R}^k$ is a smooth function.\\
		We denote the space of all smooth sections $\Gamma(E)$.
		\\
		We say that a section is global if it is defined over the entire manifold.
	\end{Def}
	\begin{Obs}\label{Obs_1.1}
		The set $\Gamma(E)$ is clearly a vector space and it is endowed with the following operations:
		\begin{itemize}
			\item $(s+t)(p)=s(p)+t(p)$;
			\item $st(p)=s(p)t(p)$.
		\end{itemize}
	\end{Obs}
	\begin{Ex}\label{Ex_5.4}
		Consider the Cilinder $Cil=S^1\times [0,1]$ as a fiber bundle. A generic section is a map $s:S^1\rightarrow Cil$ such that $s(z)=(z,t)$. So to each subset of the circle it associates a "line" in the cilinder.
	\end{Ex}
	\begin{Ex}
		Consider the tangent bundle $TM$ of a manifold $M$. A section is a map that associates to each point a tangent vector. $s:M\rightarrow TM$. Note that every vector field is a section of the tangent bundle, since it does the same thing: it takes a point and associates a vector to it. Thus, smooth vector fields are smooth sections of the tangent bundle and vice-versa. We call the space $\Gamma(TM)=\mathfrak{X}(M)$.
	\end{Ex}
	\begin{Def}\label{Def_5.6}
		Let $(E,M,\pi,F)$ be a vector bundle. We define a \textbf{frame} on an open set $U\subset M$ as a collection of sections $\{s_i\}$ such that $\{s_i(p)\}$ form a basis for $E$ at $p$, which means a base of the fiber $F$ at p. 
	\end{Def}
	\begin{Obs}
		The fact that we can express sections in a particular base is due to the fact that the vector bundle structure ensures that the fiber in each point has the structure of a vector field. This implies that each expansion of a section will be $s=s^ie_i$; where $e_i:M\rightarrow E$ are the elements of the base. Since sections are maps that take points and give back couples of points$\times$fiber, the elements $s^i$ in the expansion will be functions: $s^i:M\rightarrow \mathbb{R}$. In general, those functions can be discontinuous or not smooth.\\
		Moreover, since $s^i$ are functions of points, we can apply vector fields to them!
	\end{Obs}
	\begin{Prop}
		A smooth vector bundle is trivial if and only if it has a smooth frame in each open set.
	\end{Prop}
	\begin{proof}
		Let $E$ be trivial. This means there exists a diffeomorphic trivialization $\varphi:E\rightarrow M\times \mathbb{R}^k$. Let now $\{e_i\}$ be a base for $\mathbb{R}^k$. Then, $\{(p,e_i)\}$ form a base of $E$ due to the presence of the isomorphism, which sends a base into a base. This map is smooth in an obvious way.\\
		\\
		Now we check the contrary: suppose we have a smooth frame on a generic open subset $U\subset M$.
	\end{proof}
	\section{Connections}
	\begin{Def}\label{Def_5.7}
		Let $(E,M,\pi,F)$ be a vector bundle. We call \textbf{connection} a map $\nabla:\mathfrak{X}(M)\times\Gamma(E)\rightarrow\Gamma(E)$ such that:
		\begin{itemize}
			\item $\nabla$ is $\mathfrak{F}$-linear in $\mathfrak{X}(M)$ and $\mathbb{R}$ linear in $\Gamma(E)$: $\nabla_{fX}(\lambda s)=\lambda f\nabla_X s$ for $f:M\rightarrow \mathbb{R}, s\in\Gamma(E),\lambda\in\mathbb{R}$;
			\item $\nabla$ respects Leibniz: $\nabla_X (fs)=(Xf)s+f\nabla_X s$.
		\end{itemize}
		If $E=TM$ we say that the connection is affine.
	\end{Def}
	Note that since $X$ is a smooth vector field on $M$, we have that $Xf=df(X)$ is the differential of $f$ applied to $X$.
	\begin{Def}\label{Def_5.8}
		We say that a section $s$ is flat if $\nabla_X s=0$ for every $X\in \mathfrak{X}(M)$.
	\end{Def}
	\subsection{On the existence of connections}
	We are now going to show how to construct a connection. In particular, we will see that a connection always exists on a manifold with a smooth vector bundle.
	\begin{Prop}
		One can always construct a connection on the trivial vector bundle of a manifold.
	\end{Prop}
	\begin{proof}
		Consider a trivial vector bundle $(E,M,\pi,F\simeq \mathbb{R}^k)$. By definition, this bundle is isomorphic to $M\times \mathbb{R}^k$ through an isomorphism:
		$$\phi:E\rightarrow M\times\mathbb{R}^k$$
		Let now $\{e_i\}$ be a base for $\mathbb{R}^k$ and, since the bundle is trivial, we have a global frame given by the family of sections
		$$s_i:M\rightarrow E;\hbox{ such that } s_i(p)=(p,e_i)$$
		Those, by definition of frame, $s_i(p)$ are a base for $M\times \mathbb{R}^k$ at the point $p$. From this it immediatly follows that $\phi^{-1}\circ s_i$ is a frame for $E$. Since every element of $E$ can be described by this base, which is global and so holds for any point, also any section will be decomposed as:
		$$s=h^i(\phi^{-1}\circ s_i); \hbox{ where $h^i$ are functions of the points }$$
		Now, we define the connection such that the sections $\phi^{-1}\circ s_i$ are flat:
		$$\nabla_X(s)=(Xh^i)((\phi^{-1}\circ s_i))$$
		Thuis clearly respects Leibniz and the linearity.
	\end{proof}
	The idea now is to use this construction on the trivial bundle to build a connection on a bundle which can be locally but not globally trivialized. In order to do so we need a lemma:
	\begin{Prop}
		If $\nabla_i$ is a collection of connections, then $t^i\nabla_i$ is still a connection, provided that $\sum t^i=1$.
	\end{Prop}
	\begin{proof}
		We just need to check when the properties of the resulting connections are satisfied. The linearity persists with the sum. The problem lies inside the Leibniz rule.
		Consider $t_i\nabla^i_X (fs)=$ where the index $i$ identifies the family of connections. Then, since each one of them satisfies Leibniz on its own, we must have:
		$$t_i\nabla^i_X (fs)=\sum t_i(Xf)s+f(t_i\nabla^i_X s)$$ 
		and the only way for the first term to respect Leibniz is to add up to 1.
	\end{proof}
	\begin{Theo}
		Any smooth vector bundle over a manifold has a connection
	\end{Theo}
	The idea is to look at the bundle locally and then build the connection as a sum of the "trivial ones". Consider trivializing open set $U\in M$ and the corresponding trivializing map $\varphi:\pi^{-1}(U)\rightarrow U\times F$. Inside this set the bundle is trivial and so it has an induced connection $\nabla_\alpha$. Now, consider a family of open sets $\{U_\alpha\}$ that covers the manifold and a partition of unity $\{\rho_\alpha\}$ that identifies them (it can always be found for a manifold due to it being second countable). We are going to have a family of induced connections on each $U_\alpha$, which we are going to index as $\nabla^\alpha$.\\
	We define $\nabla_X s=\rho_\alpha\nabla_X^\alpha s$. Since on each subset the coefficients add up to 1. Thus, this is a connection.
	\subsection{The Riemaniann structure induced on the tangent bundle}
	We are now interested in finding relations between the covariant derivative and the metric tensor. Suppose to have a metric manifold with $\langle,\rangle$. We want to use the relations between $M$ and $TM$ to induce a metric on $TM$. This is done via a very similar procedure with respect to connections:
	\begin{Prop}
		It is always possible to induce a Riemaniann metric on the trivial bundle.
	\end{Prop}
	\begin{proof}
		Let $E$ be the trivial bundle on $M$. By definition, there is an isomorphism $\phi:E\rightarrow M\times \mathbb{R}^k$. This induces the following metric tensor: given $u,v\in E_p$ fiber of $E$ at $p$. the fiber map $\phi_p:E_p\rightarrow \mathbb{R}^k$ is a linear isomorphism, so we set:
		$$\langle u,v\rangle=\langle\phi_p(u),\phi_p(v) \rangle$$
		This is obviously a Riemaniann metric since the starting one is.
	\end{proof}
	\begin{Obs}
		Note that, if we choose an orthonormal base of $\mathbb{R}^k$ and through the isomorphism, we construct a base of sections, this base will be also orthonormal for the induced scalar product due to it's definition. This construction is possible only because on each fiber $E$ has the structure of a vector space.
	\end{Obs}
	Now, we can use the existence of the metric on a trivialiging subset to construct a general metric on the whole manifold. This procedure makes use of the particion of unity.
	\begin{Theo}
		On every smooth vector bundle on a Riemaniann manifold there is a Riemaniann metric.
	\end{Theo}
	\begin{proof}
		Consider a trivializing open cover $\{U_\alpha,\Phi:U_\alpha\rightarrow \phi(U_\alpha)\times \mathbb{R}^k\}$. On each of those subsets $U_\alpha$ there exists a Riemaniann metric $\langle,\rangle_\alpha$, induced by the one on the manifold. Consider now a partition of unity $\{\rho^\alpha\}$ subordinate to the open cover $U_\alpha$.\\
		Define the following general metric on the whole manifold:
		$$\langle,\rangle=\rho^\alpha\langle,\rangle_\alpha$$
		This is still a Riemaniann metric since is the sum of Riemaniann metrics and on each point it becomes a finite sum by the properties of the partition of unity of a manifold.
	\end{proof}
	\subsection{Metric connections}
	\begin{Def}\label{Def_5.9}
		We say that a connection $\nabla$ on a Riemaniann bundle $E$ is compatible with the metric, or that it is a metric connection, if:
		$$X\langle s,t\rangle =\langle \nabla_X s,t\rangle+\langle s,\nabla_X t\rangle$$
	\end{Def}
	\begin{Prop}
		The connection induced on the trivial bundle is always compatible with the metric.
	\end{Prop}
	\begin{proof}
		Let $E$ be the trivial bundle on $M$ and let $\phi$ be the homeomorphism-isomorphism $\phi:E\rightarrow M\times \mathbb{R}^k$.
		Let $\{e_i\}$ be an orthonormal base for $\mathbb{R}^k$, then, through the isomorphism, we can construct a global frame for $E$: we define the sections $s_i:M\rightarrow M\times \mathbb{R}^k$ which give rise to a global frame for $M\times \mathbb{R}^k$. Then, a global frame for $E$ will be $\phi^{-1}\circ s_i:M\rightarrow E$.
		Thus, each section of $E$ will be expressed as: 
		$$s=a^i(\phi^{-1}\circ s_i)$$
		Take now any two sections of $E$ like $s=a^i(\phi^{-1}\circ s_i)$ $t=b^j(\phi^{-1}\circ s_j)$ and a smooth vector field $X\in\mathfrak{X}(M)$. By construction we know that for any two sections the induced metric si $\langle s,t\rangle=\langle \phi_p(s),\phi_p(t)\rangle$. Thus, we evalue:
		$$X\langle s,t\rangle=X\langle a^i(\phi^{-1}\circ s_i),b^j(\phi^{-1}\circ s_j)\rangle =X\langle a^i s_i,b^js_j\rangle $$
		By orthonormality of the base of $\mathbb{R}^k$, we have:
		$$X\langle s,t\rangle=\sum_i (Xa^i)b^i+a^i(Xb^i)$$
		But recall that we had seen that the induced connection on the trivial bundle was: 
		$$\nabla_X s=(Xa^i)(\phi^{-1}\circ s_i)$$
		So that:
		$$X\langle s,t\rangle=\langle \nabla_Xs,t\rangle+\langle s,\nabla_Xt\rangle$$
	\end{proof}
	\begin{Theo}
		On any Riemaniann bundle there is a connection compatible with the metric.
	\end{Theo}
	\begin{proof}
		Consider
	\end{proof}
	\section{Connections and matrices}
	Recall that a connection on a smoothe vector bundle is a map
	$$\nabla:\mathfrak{X}(M)\times \Gamma(E)\rightarrow\Gamma(E)$$
	That respects linearity and Leibniz. Thus, consider an open set $U\in M$ and the corresponding trivialization for $E$. We can find a frame in this set, indexed with $\{s_i:M\rightarrow E\}$. This means that we can write any section as a linear combination:
	$s=h^is_i$. Thus we find:
	$$\nabla_Xs=(Xh^i)s_i+h^i\nabla_Xs_i$$
	Since the resulto of the connection applied to a section must also be a section, it means that $\nabla_Xs_i$ must be a section for every $i$. Thus, it can be expanded in the same base:
	$$\nabla_X s_i=\omega_i^j(X)s_j$$
	By $\mathfrak{F}$-linearity of $\nabla$, we have that also $\omega$ is $\mathfrak{F}$-linear and so it is a 1-form on $U\in M$. In fact $\omega$ can be seen as a map at any point:
	$$\omega_p:\mathfrak{X}(M)\rightarrow\mathbb{R}$$ 
	where the subscript $p$ means we are evaluating it at a point $p$.
	We are going to call $\omega$ \textbf{connection form} and $\omega^j_i$ \textbf{connection matrix}.
	\begin{Theo}
		Let $(E,M,\pi,F)$ be a Riemaniann bundle and $\nabla$ a connection on it. Then:
		\begin{itemize}
			\item If $\nabla$ is compatible with the metric then its connection matrix $\omega^i_j$ relative to any orthonormal frame $\{e_i\}$ over any trivializing open set $U\subset M$ is skew-symmetric.
			\item If any point $p\in M$ has a trivializing open set $U$ such that we can find an orthonormal frame for which $\omega^i_j$ is skwe-symmetric, then the connection is compatible with the metric.
		\end{itemize}
	\end{Theo}
	\begin{proof}
		Suppose that $\nabla$ is compatible with the metric. Take any trivializing open set $U\subset M$ and an orthonormal frame  defined over it $\{e_i\}$. Then:
		$$X\langle e_i,e_j\rangle=0=\langle\nabla_X e_i,e_j\rangle+\langle e_i,\nabla_Xe_j\rangle=$$
		$$=\langle\omega^k_i(X)e_k,e_j\rangle+\langle e_i,\omega^\rho_j(X)e_\rho\rangle=\omega^j_i(X)+\omega^i_j(X)=0$$
		Suppose now that for any point $p\in M$ we can find a frame $\{e_i\}$ in which the connectrion matrix is skew-symmetric. Then we have that, given two sections $s=s^ie_i$, $t=t^ie_i$ we can write:
		$$X\langle s,t\rangle=X\langle s^ie_i,t^je_j\rangle=X(s^it^i)=(Xs^i)t^i+s^i(Xt^i)$$
		On the other hand:
		$$\langle \nabla_Xs^ie_i,t^je_j\rangle+\langle s^ie_i,\nabla_Xt^je_j\rangle=$$
		$$(Xs^i)t^j+s^i(Xt^j)+s^it^j[\langle \omega^k_ie_k,e_j\rangle+\langle e_i,\omega^k_je_k\rangle]$$
		By the skew-symmetry of $\omega$ and the orthonormality of the base elements, we get that the last terms vanish and so:
		$$X\langle s,t\rangle=\langle\nabla_Xs,t\rangle+\langle s,\nabla_Xt\rangle$$
	\end{proof}
	\section{The covariant derivative}
	A vector field is defined as a map $X:M\rightarrow TM$, a section of the tangent bundle. Consider now any curve $\gamma:[a,b]\in\mathbb{R}\rightarrow M$. We can see a vector field along a curve as a map that assigns to each point a tangent vector. Thus:
	$$X:[a,b]\rightarrow \bigsqcup_{p\in \gamma(t)} T_pM=TM|_{\gamma(t)}$$
	Since vector fields are sections of the tangent bundle, we call the space of smooth vector fields along the curve $\gamma$ in the usual way: $\Gamma(TM|_{\gamma(t)})$.
	The idea now is to associate to any affine connection on the manifold $\nabla$ a derivative, which we are going to call covariant derivative.
	\begin{Prop}
		Let $\nabla$ be an affine connection on a manifold $M$ and $\gamma:[a,b]\rightarrow M$ a smooth curve. Then there exists a unique map called coavriant derivative:
		$${D\over dt}:\Gamma(TM|_{\gamma(t)})\rightarrow\Gamma(TM|_{\gamma(t)})$$
		such that it respects the following properties:
		\begin{itemize}
			\item $\forall X\in \Gamma(TM|_{\gamma(t)})$ the covariant derivative is $\mathbb{R}-linear$;
			\item the covariant derivative satisfies the Leibniz rule:
			$${D(fX)\over dt}={df\over dt}X+f{DX\over dt}$$;
			\item if $\mathbb{X}$ is obtained by restricting $X$ to $\gamma$, then we have:
			$${D\mathbb{X}\over dt}=\nabla_{\gamma'(t)}X$$
		\end{itemize}
	\end{Prop}
	\begin{proof}
		Suoppose that the covariant derivative exists. Then, consider a trivializing open set $U\in M$ for $TM$ and an orthonormal frame $\{e_i\}$ of sections on it. Since smmooth vector fields are smooth sections of the tangent bundle, we can write:
		$$V(t)=v^i(t)e_i$$
		Then, by the 3 properties, we have:
		$${DV(t)\over dt}={dv^i(t)\over dt}e_{i,\gamma'(t)}+v^i\nabla_{\gamma'(t)}e_i$$
		Thus, supposing that the covariant derivative exists, this formula proves its uniqueness.\\
		\\
		As for the existence, we can define the covariant derivative like the above formulation. Naively, this definition satisfies all properties. Moreover, the covariant derivative is independent, by uniqueness, of the frame.
	\end{proof}
	\begin{Prop}
		If $\nabla$ is an affine connection compatible with the metric, then given a smooth curve $\gamma$ and two smooth vector fields $X,Y$, we get:
		$${d\over dt}\langle X,Y\rangle=\langle {DX\over dt},Y\rangle+\langle X,{DY\over dt}\rangle$$
	\end{Prop}
	\begin{proof}
		The proof is just a corollary of the definition of covariant derivative. Let us take a trivializing open set $U$ of $TM$ and an orthonormal frame. Then the vector fields will be written as:
		$$X=x^ie_i \hbox{ and } Y=y^ie_i$$
		Now we just substitute those expansions in the above formulas and see thay coincide:
		$${d\over dt}\langle X,Y\rangle={d\over dt}(x^iy^i)={dx^i\over dt}y^i+x^i{dy^i\over dt}$$
		As for the second one:
		$$\langle {DX\over dt},Y\rangle+\langle X,{DY\over dt}\rangle=
		\langle {dx^i\over dt}e_i+x^i\nabla_{\gamma'(t)}e_i,y^je_j\rangle+\langle x^ie_i,{dy^j\over dt}e_j+y^j\nabla_{\gamma'(t)}e_j\rangle=$$
		$$=\langle {dx^i\over dt}e_i,y^je_j\rangle+\langle x^i\nabla_{\gamma'(t)}e_i,y^je_j\rangle+\langle x^ie_i,{dy^j\over dt}e_j\rangle+\langle x^ie_i,y^j\nabla_{\gamma'(t)}e_j\rangle$$
		Now, the two terms containing the connection can be rweritten, since the connection is supoposed tobe compatible with the metric:
		$$\langle x^i\nabla_{\gamma'(t)}e_i,y^je_j\rangle+\langle x^ie_i,y^j\nabla_{\gamma'(t)}e_j\rangle=x^iy^j\gamma'(t)\langle e_i,e_j\rangle=x^iy^j\gamma'(t)\delta^i_j=0$$
		In fact, applying a vector fdield to a constant function on the manifold gives 0.
		Thus, we get the equality.
	\end{proof}
	\begin{Obs}
		Note that we have defined the covariant derivatie only on an affine connection, which means a connection on the tangent bundle $TM$ of a manifold $\nabla:\mathfrak{X}(M)\times\mathfrak{X}(M)\rightarrow\mathfrak{X}(M)$. In general, we could have done this also for a generic smooth vector bundle???
	\end{Obs}
	\section{The Christoffel symbols}
	\chapter{Principal Bundles}
	\section{Generalizing fiber bundles}
	\begin{Def}\label{Def_6.1}
		Let $G$ be a Lie group and $M$ a manifold. We define a \textbf{right-action} of $G$ on $M$ as a map $\mu:G\times M\rightarrow M$ such that:
		\begin{itemize}
			\item $\mu(g,x)=x\cdot g$ is smooth (a Lie group is a manifold);
			\item $x\cdot e=x$ for any $x\in M$, where $e$ is the neutral element of $G$;
			\item $x\cdot (gh)=(x\cdot g)\cdot h$.
		\end{itemize}
		The left action is defined similarly.
	\end{Def}
	\begin{Def}\label{Def_6.2}
		We call \textbf{stabilizer} of $x\in M$ under $G$ Lie group the subgroup:
		$$Stab(x)=\{g\in G|x\cdot g=x\}$$
		We call the \textbf{orbit} of $x$ the set:
		$$O(x)=xG=\{x\cdot g|g\in G\}$$
	\end{Def}
	The fact that $Stab(x)$ is a subgroup is trivially verified. Also, the stabilizer is a Lie subgroup but we are not going to prove this result since it is well known and not necessary for the general understanding.
	\begin{Def}\label{Def_6.3}
		A manifold $M$ with a lei group $G$ (left) right-acting on it is called a \textbf{G-manifold}. We also say that a map $f:M\rightarrow N$ between $G-manifolds$ is equivariant if $f(x\cdot g)=f(x)\cdot g$.
	\end{Def}
	\begin{Def}\label{Def_6.4}
		Let $(E,M,\pi,G)$ be a fiber bundle, where $G$ is a Lie group and $M$ and $E$ are manifolds. Let $\pi$ be smooth and the trivializations diffeomorphic; let $G$ be acting on $E$ such that:
		\begin{itemize}
			\item $G$ acts on $E$ freely: freely: $Stab(E)=\{e\in G\}$;
			\item let $U\in M$ be a local trivializing set and $\phi:\pi^{-1}(U)\rightarrow U\times M$  the corresponding trivialization, then $\phi$ is equivariant: $\phi(x,h)\cdot g=\phi(x,hg)$
		\end{itemize} 
		Then we say that this fiber bundle is a \textbf{principal $G$-bundle}.
	\end{Def}
		 Let us now make some examples of $G$-bundles to better understand the structure.
		 \begin{Ex}[\textbf{The product bundle}]
		 	Consider the product bundle as usual, but let $E$ and $M$ be manifolds and $G$ be a Lie group. Then $E=M\times G$. We can give this the fiber bundle structure in the usual way: we first of all define a projection:
		 	$$\pi:M\times G\rightarrow M\hbox{ like }\pi(x,g)=x$$
		 	This map is clearly continuous and surjective. Also, it can be shown to be smooth through the use of charts (it is just a matter of substituting coordinates, since both $M$ and $G$ have the structure of a manifold). The local trivializations will just be the identity map, which is clearly diffeomorphic. Now, we need to ask for a $G$ action of $G$ on $M\times G$. But the action is abvious:
		 	$$(x,g)\cdot h=(x,gh)$$
		 	Also, the identity map clearly preserves this action in an equivariant way.
		 \end{Ex}
		 \begin{Ex}
		 	Let $G$ be a Lie group and $H$ be a closed subgroup of $G$.
		 \end{Ex}
	Now we start to see some properties which will come in handy in a while.
	\begin{Prop}
		Let $(E,M,\pi,G)$ be a principal $G-$bundle, then $G$ acts transitively on each fiber. Also, for any group $G$, any right-equivariant map is a left translation.
	\end{Prop}
	\begin{proof}
		$G$ acts transitively on $\{p\} \times G$, obviously. Looking at the fiber diffeomorphism $\phi:\pi^{-1}\rightarrow U\times G$, it can be reduced to $\phi_p:E_p\rightarrow \{p\}\times G$, which is equivariant. Thus, on each fiber, $G$ acts transitively.\\
		\\
		As for the second proposition, it is just an easy resulto of group theory: if $f(gh)=f(g)h$, setting $g=e$ we get: $f(h)=f(e)h=l_{f(e)}h$.
	\end{proof}
	\section{Transition functions}
	Suppose to have a principal $G$ bundle $(E,M,\pi,G)$ and two trivializing open sets $U_1,U_2$ with the corresponding trivialization maps $\phi_{1/2}:\pi^{-1}(U_{1/2})\rightarrow U_{1/2}\times G$. Suppose that $U_1\cap U_2\neq\emptyset$. Then we can look at the composition between one map and the other:
	$$\phi_1\circ\phi_2^{-1}:U_{1}\cap U_2\times G\rightarrow U_{1}\cap U_2\times G$$
	This map is a composition of two diffeomorphisms, so it is a diffeomorphism. Also, for each point $p\in M$, the fiber reduces to an equivariant map from $G$ to $G$. By the previous lemma, the equivariant map is a left translation!!! Thus:
	$$\phi_1\circ\phi_2^{-1}(p,g)=(p,h)=(p,g_{12}(p)h)$$
	\begin{Obs}
		The \textbf{transition functions} $g_{ij}$ are smooth since $\phi$ are diffeomorphisms. Note however that $g_{ij}$ are not necessarily diffeomorphisms since the two manifolds $M,G$ have in general not the same dimension??????
	\end{Obs}
	Now we look at some properties of the transition functions:
	\begin{Prop}
		Let $\{U_\alpha\}$ be a trivializing open cover for a principal $G-bundle$. Then the transition functions $g_{\alpha\beta}$ satisfy the following properties:
		\begin{itemize}
			\item $g_{\alpha\beta}g_{\beta\gamma}=g_{\alpha\gamma}$;
			\item $g_{\alpha\beta}=g_{\beta\alpha}^{-1}$;
			\item $g_{\alpha\alpha}=e$.
		\end{itemize}
	\end{Prop}
	\begin{proof}
		Suppose we have 3 open sets that intersect: $U_\alpha\cap U_\beta\cap U_\gamma\neq\emptyset$. Then we will have 3 trivializing maps:
		$$\phi_\alpha:\pi^{-1}(U_\alpha)\rightarrow U_\alpha\times G;
		\hspace{6pt}
		\phi_\beta:\pi^{-1}(U_\beta)\rightarrow U_\beta\times G;
		\hspace{6pt}
		\phi_\gamma:\pi^{-1}(U_\gamma)\rightarrow U_\gamma\times G
		$$
		Now we look at the composition of those diffeomorphisms:
		$$\phi_\alpha\circ\phi^{-1}_\beta(p,h)=(p,g_{\alpha\beta}(p)h)$$
		$$\phi_\beta\circ\phi^{-1}_\gamma(p,h)=(p,g_{\beta\gamma}(p)h)$$
		Obviously:
		$$\phi_\alpha\circ\phi^{-1}_\gamma=\phi_\alpha\circ\phi^{-1}_\beta\circ \phi_\beta\circ\phi^{-1}_\gamma$$
		so that we get:
		$$(p,g_{\alpha\gamma}(p)h)=(\phi_\alpha\circ\phi^{-1}_\beta)\circ (\phi_\beta\circ\phi^{-1}_\gamma)(p,h)=\phi_\alpha\circ\phi^{-1}_\beta(p,g_{\beta\gamma}(p)h)=(p,g_{\alpha\beta}(p)g_{\beta\gamma}(p)h)$$
		This proves the first property.
		\\
		The second one just follows from the fact that $\phi$ are diffeomorphisms.\\
		As for the third one, we take the first and the second property and evalue:
		$$g_{\alpha\beta}g_{\beta\alpha}=g_{\alpha\alpha}=e$$
	\end{proof}
	\section{The frame bundle}
	\subsection{The manifold structure of the frame bundle}
	\begin{Def}\label{Def_6.5}
		Consider a vector space $V$ of finite dimension. We define $Fr(V)$ as the set made of all ordered bases of $V$. This means that $Fr(V)$ contains all of the possible ways of choosing a base for $V$.
	\end{Def}
	\begin{Obs}
		There is a natural action of the general linear group on $Fr(V)$. The idea is the following: since $Fr(V)$ contains all of the possible basis of $V$, we get that, having choosen a base, we can re-order the elements of it and obtain the same base. We can also linearly combine those elements to obtain another base. Thus, all of the elements of $Fr(V)$ will be reached by matrices in $GL(n,\mathbb{R})$, where $n$ is the dimension of the vector space.\\
		\\
		In particular, identifying an element of $Fr(V)$ with a row vector $\vec{e}=(e_1,...,e_n)$, we have a right-action of the Lie group $GL(n,\mathbb{R})$ of the form:
		$$\vec{e}A=(e_1,...,e_n)\begin{pmatrix}
			a_{11}&a_{12}&...&a_{1n}\\
			a_{21}&a_{12}&...&a_{2n}\\
			.     &.     &.   &     \\
			.     &.     &.   &     \\
			a_{n1}&.     &...&a_{nn}\\
		\end{pmatrix}$$
		We also note that the stabilizer of any $\vec{e}\in Fr(V)$ is just $\mathbb{I}_n\in GL(n,\mathbb{R})$, so that the group acts freely on $Fr(V)$. Moreover, the orbit of any element is the whole space: 
		$$O(\vec{e}\in Fr(V))=Fr(V)$$
		This last result is obvious: given any two basis I can always fine one as a linear combination of the other and so the two will be related by a matrix. Moreover, this matrix will be invertible since the correspoinding mao has to be a diffeomorphism (we are sending a base into a base).\\
		Now, by the orbit-stabilzer theorem, having fixed an element $\vec{e}\in Fr(V)$, we have a bijection:
		$$\phi_{\vec{e}}:{GL(n,\mathbb{R})\over Stab(\vec{e})}=GL(n,\mathbb{R})\rightarrow O(\vec{e})=Fr(V);\hbox{ such that } \phi_{\vec{e}}([A])=\vec{e}A$$
		The idea now is to put a manifold structure on $Fr(V)$ such that the above map becomes a diffeomorphism. Note also that the stabilizer of any element is just the identity group. Thus, not only the image, but also the domain of $\phi_{\vec{e}}$ will be independent of the choiche of $\vec{e}$.\\
		\\
		If $\vec{e'}$ is another element of $Fr(V)$, then there will exist $A\in GL(n\mathbb{R})$ such that $\vec{e'}=\vec{e}A$. Thus:
		$$\phi_{\vec{e'}}(g)=\vec{e'}g=\vec{e}Ag=(\phi_{\vec{e}}\circ l_A)(g)$$
		Since the left multiplication is a diffeomorphism, we can give to any $\phi_{\vec{e'}}$ the same diffeomorphism structure as $\phi_{\vec{e}}$.\\
		\\
		I DO NOT KNOW IF THIS IS TRUE\\
		The map $\phi_{\vec{e}}$ is an homeomorphism since it is a bijection and is obviously continuous. Now we simply endow $Fr(V)$ with charts. Suppose that $(U,\varphi)$ is a chart of $GL(n,\mathbb{R})$. Then $(\phi(U),\varphi\circ\phi^{-1}_{\vec{e}})$ will be our chart for $Fr(V)$. This is the composition of two homeomorphisms so it is an homeomorphism. The composition of two charts is smooth since the one on $GL(n\mathbb{R})$ is. This gives $Fr(V)$ the manifold structure we wanted.\\
		I DO NOT KNOW IF THIS WAS TRUE I MADE IT UP
		\end{Obs}
		\subsection{The frame bundle}
		Suppose to have a smooth vector bundle $(E,M,\pi,F)$. Our idea is now to associate to $M$ a smooth principal $GL(n,\mathbb{R})-bundle$. We first of all need a projection: define the Frame Bundle as the following space:
		$$Fr(E)=\bigsqcup_{p\in M}Fr(E_p)$$
		The disjoint union of the fibers at each point. There is a natural projection that we can use:
		$$\pi_G:Fr(E)\rightarrow M\hbox{ such that } \pi_G(\vec{e_p}\in Fr(E_p))=p$$
		Now we need to find local trivializations. The idea is quite simple but somewhat elaborate.\\
		\\
		Since we arer starting with a smooth vecotr bundle, the fiber has in each point the structure of a vector space of finite dimension $k$. Thus, we can construct for each $p$ the manifold $Fr(E_p=F_p)$ and then consider the disjoint union $Fr(E)$, which will be a manifold as well. Now, suppose to have a chart on $Fr(E)$ like $(U_{M},\psi:U_{Fr(E)}\rightarrow \mathbb{R}^n\times\mathbb{R}^k)$. Due to the existence of the projection $\pi_G$ we can induce a chart on $M$ by finding the corresponding open set on it $U_M$ and it's map $\phi:U_M\rightarrow \mathbb{R}^n$. We choose as trivializtion the same map we had choosen to trivialize the tangent bundle:
		$$\varphi\equiv(\phi^{-1}\circ \mathbb{I}_{\mathbb{R}^k})\circ \psi: U_{Fr(E)}\xrightarrow{\psi}\phi(U_M)\times\mathbb{R}^k\xrightarrow{\phi^{-1}\times\mathbb{I}_{\mathbb{R}^k}}U_m\times\mathbb{R}^k$$
		This is clearly a diffeomorphism.\\
		\\
		Note that at every point there is an identification of $Fr(E_p)$ with the general lienar group $GL(k,\mathbb{R})$. To make this bundle structure into a principal bundle we need to verify that $GL(k,\mathbb{R})$ acts freely on $Fr(E)$ and that the trivializations are equivariant under the action of the group.\\
		\\
		Now, the action of $GL(k,\mathbb{R})$ on the fiber is just given by the product of matrices and so it is free. In particular, $GL(k,\mathbb{R})$ acts on $Fr(E)$ like this:
		$$(p\in M,\vec{e_p}\in Fr(E_p))\longrightarrow(p,\vec{e_p}A)$$
		This action is evidently free. Moreover, considering the chart $\varphi$, it is equivariant under this action. Thus, we have a principal $GL(k,\mathbb{R})$-bundle $(Fr(E),M,\pi_G,GL(k,\mathbb{R}))$.\\
		\\
		\begin{Obs} [\textbf{An intuition for the frame bundle}]
			The frame bundle is nothing more than the union of all of the frame manifolds of the fibers. Basically, once we have a smooth fiber bundle on $M$, constructing the frame bundle means associating another bundle to $M$. This new bundle will be a principal bundle and at any point the fiber will be $Fr(F)$ the frame manifold of the fiber, which is 1 to 1 with $GL(k,\mathbb{R})$.
		\end{Obs}
		\subsection{Fundamental vector fields}
		Suppose to have a Lie group $G$ that right-acts smoothly on a manifold $M$. Then, consider any element $A\in\mathfrak{g}$ inside the Lie Algebra of $G$. Define:
		$$\bar{A}={d\over dt}\bigg|_{t=0}p\cdot e^{tA}\in T_pM$$
		This, at any point is called \textbf{fundamental vector field associated to $A$}.\\
		\begin{center}
			\includegraphics[scale=0.3]{fundamental_vector.pdf}
		\end{center}
		The idea is to take a point $p\in M$ and a curve passing through it like $c_p(t)=p\cdot e^{tA}$. This is a smooth curve and so it can be derived. The fundamental vector field in $p$ is the initial tangent vector of this curve (initial in the sense that for $t=0$ the curve goes through $p$).
		\begin{Prop}
			The fundamentale vector field associated to $A$ is always smooth.
		\end{Prop}
		\begin{proof}
			Non ne ho voglia ma è una cazzata di conti.
		\end{proof}
		There is an alternative mathematical construction for the same concept: define the map $j_p:G\rightarrow M$ like follows:
		$$j_p(g)=p\cdot g$$
		Its differential is defined as:
		$$dj_p(A\in\mathfrak{g})={d\over dt}\bigg|_{t=0}j_p(e^{tA})={d\over dt}\bigg|_{t=0}p\cdot e^{tA}$$
		\begin{Ex}
			Consider the flat Euclidean manifold $\mathbb{R}^2$ and the Lie group $U(1)$ of complex modulo 1 numbers. The Lie algebra of the group is $\mathfrak{u}(1)={i\theta;\theta\in\mathbb{R}}$ made up of all multiples of $i$. We set the following action of $U(1)$ on $\mathbb{R}^2$ given by: $e^{i\theta}(x,y)=(xcos(\theta)-ysin\theta,ycos(\theta)+xsin(\theta))$.
			Then, choosing the point $p=(1,1)\in\mathbb{R}^2$ and the generic element $e^{i\theta}\in U(1)$, we can compute the fundamental vector field in $p$. Consider the curve $c_p(t)=p\times e^{it\theta}$ and the fundamental vector field in $(1,1)$ is given by:
			$$\bar{\theta}={d\over dt}\bigg|_{t=0}j_p(e^{it\theta})$$
			To make things more explicit:
			$${d\over dt}\bigg|_{t=0}(1,1)\cdot (cos(t\theta))+sin(t\theta)=
			{d\over dt}\bigg|_{t=0}(cos(t\theta)-sin(t\theta),cos(t\theta)+sin(t\theta))=(-1,1)\theta$$
		\end{Ex}
		\begin{Prop}
			Let $G$ be a Lie group right-acting smoothly on a manifold $M$ and let $r_g(p)=p\cdot g$ the right translation. For $A\in\mathfrak{g}$, it's fundamental vector field $\bar{A}$ satisfies:
			$$dr_g(\bar{A})=\overline{Ad(g^{-1})(A)}$$
		\end{Prop}
		\begin{proof}
			Recall that the adjoint representation is given by the differential of the conjugation map:
			$$Ad(g)=dc_g:\mathfrak{g}\rightarrow\mathfrak{g};\hbox{ in particular }Ad(g)(A)=gAg^{-1}$$
			Now, consider $(r_g\circ j_p)(h)=p\cdot hg=p\cdot gg^{-1}hg=p\cdot g c_{g^{-1}}(h)=(j_{pg}\circ c_{g^{-1}})(h)$
			By the chain rule:
			$$dr_g(\bar{A}_p)=dr_gdj_p(A)=dj_{pg}dc_{g^{-1}}(A)=dj_{pg}(Ad(g^{-1})(A))=\overline{Ad(g^{-1})(A_{pg})}$$
		\end{proof}
		This means that the push-forward of the fundamental vector field under the right action "transofrms" under the adjoint representation. Let us make an example to better interiorize this:
		\begin{Ex}
			Consider again the space $\mathbb{R}^2$ acted on by $U(1)$ like the previous example. Then, suppose $A=i\theta$ and $g=e^{iB}$. We saw that the fundamental vector field associated to $(x,y)$ was:
			$$\bar{A}_{x,y}=(-\partial_y,\partial_x)\theta$$
			Now, the push-forward of this vector field under $g$ will be the fundamental vector field of:
			$$Ad(g^{-1})(A)=g^{-1}Ag=e^{-iB}Ae^{iB}$$
			However, $U(1)$ is a commutative group, so that $Ad(g)A=A$. Thus, the push forward of the fundamental vector field is the same fundamental vector field.
		\end{Ex}
		\subsection{The integral curves of the fundamental vector fields}
		\begin{Prop}
			The curve $c_p(t)=p\cdot e^{tA}$ is the integral curve of the fundamental vector field $\bar{A}$ passing through $p\in M$.
		\end{Prop}
		\begin{proof}
			It suffices to differentiate:
			$$c'_p(t)={d\over dt}\bigg|_{s=0}c_p(t+s)=
			{d\over dt}\bigg|_{s=0}p\cdot e^{tA}e^{sA}=\bar{A}_{pe^{tA}}=
			\bar{A}_{c_p(t)}$$
		\end{proof}
		\begin{Prop}
			The fundamental vector field $\bar{A}$ vanishes at $p$ if and only if $A$ is in the Lie algerba of $Stab(p)$.
		\end{Prop}
		\begin{proof}
			If $\bar{A}_p=0$ it means that the constant map $c_p(t)=p\in M$ is an integral curve of this vecotr field. By the previous proposition, we have the equivalence:
			$p=p\cdot e^{tA}$ from which we see that $A$ must be in the Lie algebra of $Stab(p)$ since $e^{tA}\in Stab(p)$.\\
			\\
			On the contrary, if $A\in Stab(p)$ then we ge that:
			$$\bar{A}_p={d\over dt}\bigg|_{t=0}p\cdot e^{tA}=
			{d\over dt}\bigg|_{t=0}p=0$$
		\end{proof}
		\begin{Obs}
			As a corollary of the previous statement we can say that $Ker(dj_p)$ at the identity is the stabilizer $Stab(p)$. This is pretty esay to see and pretty obvious to think about:
			$$Ker dj_p=\{A\in \mathfrak{g}|dj_p(A)=0\}=Stab(p)\hbox{ By the previous result}$$
		\end{Obs}
		\subsection{The vertical subbundle of the tangent bundle}
		Let $(E,M,\pi,G)$ be a principal $G$-bundle on a manifold $M$. By construction, the projection $\pi:E\rightarrow M$ is smooth, so it can be differentiated. The differential $d\pi_{(p,g)}:T_{(p,g)}E\rightarrow T_{\pi(p,g)}M$ is surjective since the projection is. We now define the \textbf{vertical tangent subspace} $\mathcal{V}_{(p,g)}\subset T_{(p,g)}E$ as the Kernel of the differential of the projection:
		$$\mathcal{V}_{(p,g)}\equiv Ker(d\pi_{(p,g)})$$
		\begin{Obs}
			Since, by the theorem of dimension, 
			$$dim \bigg(Ker(d\pi_{(p,g)})\bigg)+dim\bigg(Im(d\pi_{(p,g)})\bigg)=dim\bigg(T_{(p,g)}E\bigg)$$
			Moreover, the map $d\pi$ is surjective and so the image corresponds to the full space $T_{\pi(p,g)}M$, we get:
			$$dim \bigg(Ker(d\pi_{(p,g)})\bigg)=dim\mathcal{V}_{(p,g)}=dim\bigg(T_{(p,g)}E\bigg)-dim\bigg(T_{\pi_{(p,g)}}M\bigg)=dim G$$
		\end{Obs}
		\begin{Prop}
			For every element of $E$, the corresponding fundamental tangent vector is vertical.
		\end{Prop}
		\begin{proof}
			We just need to prove that the fundamental tangent vector is in the Kernel of the differential. Recall that the fundamental vecotr field can be seen as the differential of the map $j_{(p,g)}:G\rightarrow E$ defined as 
			$$j_{(p,g)}(h)=(p,g)\cdot h $$
			Taking now the composition with the projection:
			$$(\pi\circ j_{(p,g)})(h)=\pi((p,g)\cdot h)=\pi(p,g)$$
			This result is independent of the choice of $h\in G$, thus the composition $\pi\circ j_{(p,g)}$ is the constant map and hao it's differential the null map.\\
			Recalling that $\bar{A}_{(p,g)}=dj_{(p,g)}(A)$ we clearly see that:
			$$d\pi_{(p,g)}(\bar{A}_{(p,g)})=
			(d\pi_{(p,g)}\circ dj_{(p,g)})(A)=
			d(\pi\circ j_{(p,g)})(A)=0$$
		\end{proof}
		We can now generalize proposiotion REFERENZA for a principal $G$-bundle:
		\begin{Prop}
			Let $(E,M,\pi,G)$ be a principal $G-$bundle and $(p,g)$ a point in $E$. Then: 
			$$dj_{(p,g),e}:\mathfrak{g}\rightarrow \mathcal{V}_{(p,g)}$$ is an isomorphism.
		\end{Prop}
		\begin{proof}
			By proposition REFERENZA we know that the Kernel of the differential at the identity is the Stabilizer of $(p,g)$. However, for a principal $G-$bundle, the group acts freely on the manifold $E$ and so the stabilizer is trivial $Stab(p,g))=\{e\}$. Thus, $Ker dj_{(p,g),e}=0$ and the map is injective. By the prpevious propositon REFERENZA the image of the differential lies in the vertical subspace. Moreover, $dim\mathcal{V}_{(p,g)}=dim G$ and so the map must be an isomorphism of vector spaces.			
		\end{proof}
		\begin{Obs}
			As a consequence of the last result, all the vertical vectors are the fundamental vectors and all the fundamental vectors are the vertical vectors.\\
			Let $T^a$ be a base for the Lie algebra $\mathfrak{g}$, then, at any point, by the isomorphism in the previous result, we get that they also form a base of $\mathcal{V}_{(p,g)}$ (under the action of the diffeomorphism). This defines the \textbf{Vertical Subbundle} $$\mathcal{V}=\bigsqcup_{(p,g)\in E}\mathcal{V}_{(p,g)}$$
		\end{Obs}
		\begin{Obs}
			The name "vertical" comes from the definition $\mathcal{V}=Ker(d\pi)$. The basi idea is that, if we take any smooth vector field $X_E\in\mathfrak{X}(E)$, then the differential of the projection $d\pi:TE\rightarrow TM$ induces a new smooth vector field $X_M=d\pi X_E$ on the base space. The "vertical vector fields" are the ones who get annihilated when projected, since they belong to the Ker of $d\pi$. They are not to be though as vertical with respect to $E$, but to $M$.
		\end{Obs}
		\subsection{The horizontal subbundle}
		We have seen that on a principal $G$-bundle there is a well defined vertical subbundle $\mathcal{V}$ of the tangent bundle $TE$. We say thet there is an \textbf{horizontal distribution} $\mathcal{H}$ if $TE=\mathcal{H}\oplus\mathcal{V}$. This means that for every point $(p,g)\in TE$ we have $T_{(p,g)}E=\mathcal{H}_{(p,g)}\oplus\mathcal{V}_{(p,g)}$ with $\mathcal{H}_{(p,g)}\cap\mathcal{V}_{(p,g)}=0$.\\
		\\
		In general, there is no canonically defined horizontal distribution on a principal $G$-bundle.
		COSE CBE DEVO METTERE????
	\section{Connections on principal bundles}
		Let $(E,M,\pi,G)$ be a principal $G$-bundle with a vertical and horizontal distribution, so that $TE=\mathcal{H}\oplus\mathcal{V}$. Let $\nu_{(p,g)}:\mathcal{H}_{(p,g)}\oplus\mathcal{V}_{(p,g)}\rightarrow \mathcal{V}_{(p,g)}$ be the standard projection on the vertical subbundle. Obiouvly, if $X_{(p,g)}\in T_{(p,g)}E$ is a vector field on $E$ at a point $(p,g)$, we call $\nu_{(p,g)}(X_{(p,g)})$ its \textbf{vertical component}.
		\begin{Def}\label{Def_6.6}
			We call an \textbf{Ehresmann connection}, or \textbf{connection on a principal bundle} a $\mathfrak{g}$-valued 1-form $\omega:TE\rightarrow \mathfrak{g}$ that respects the following properties:
			\begin{itemize}
				\item Given any $A\in\mathfrak{g}$ and $(p,g)\in E$ we have $\omega_{(p,g)}(\bar{A}_{(p,g)})=A$;
				\item $(G-equivariance)$ for $g\in G$ we have $r^*_g\omega=Ad(g^{-1})\omega$;
				\item $\omega$ is $C^\infty$.
			\end{itemize}
		\end{Def}
		Now it is just a matter to find a map that satisfies those conditions. A good candidate is the following composition:
		$$\omega_{(p,g)}=dj_{(p,g)}^{-1}\circ \nu:T_{(p,g)}E\xrightarrow{\nu} \mathcal{V}_{(p,g)}\xrightarrow{dj_{(p,g)}}\mathfrak{g}$$
		\begin{Theo}
			If $\mathcal{H}$ is a smooth right-invariant horizontal distribution on a principal $G$-bundle $(E,M,\pi,G)$, then 1-form $\omega_{(p,g)}=dj_{(p,g)}^{-1}\circ \nu$ respects the following properties:
				\begin{itemize}
				\item Given any $A\in\mathfrak{g}$ and $(p,g)\in E$ we have $\omega_{(p,g)}(\bar{A}_{(p,g)})=A$;
				\item $(G-equivariance)$ for $g\in G$ we have $r^*_g\omega=Ad(g^{-1})\omega$;
				\item $\omega$ is $C^\infty$.
			\end{itemize}
		\end{Theo}
		\begin{proof}
			Consider any $A\in\mathfrak{g}$. Then its fundamental vector field is in the vertical sub-bundle and so the projection $\nu$ leaves it invariant:
			$$\omega_{(p,g)}(\bar{A}_{(p,g)})=(dj_{(p,g)}^{-1}\circ \nu)(\bar{A}_{(p,g)})=dj_{(p,g)}^{-1}(\bar{A}_{(p,g)})=
			dj_{(p,g)}^{-1}(dj_{(p,g)}(A))=A$$
			This proves the first property.\\
			\\
			As for the second property, we recall that in proposition PROPOSIZIONE we proved that for a fundamental vector field  we have:
			$$dr_g(\bar{A})=\overline{Ad(g^{-1})(A)}$$
			Now, we can directly prove this for fundamental vector fields, since they are the vertical ones and the horizontal ones would be annihilated by the projection $\nu$ contained inside $\omega$. \\
			Recalling that on principal bundles the pullback of a 1 form gives:
			$$r^*_h\circ \omega_{(p,g)}(\overline{X_{(p,g)\cdot h}})=\omega_{(p,g)\cdot h}(dr_h(\overline{X_{(p,g)}}))$$
			We need to prove that:
			$$r^*_h\circ \omega_{(p,g)}(\overline{X_{(p,g)}})=(Adh^{-1})\omega_{(p,g)}(\overline{X_{(p,g)}})$$
			If $X_{(p,g)}\in\mathcal{V}$ is vertical then it will be a fundamental vector field of an element $X\in\mathfrak{g}$ and so by proposition PROPOSIZIONE we get:
			$$\omega_{(p,g)\cdot h}(dr_h(\overline{X_{(p,g)}}))=\omega_{(p,g)\cdot h}\overline{(Ad(h^{-1})(X_{(p,g)\cdot h}))}$$
			Now by the previous property applied 2 times::
			$$\omega_{(p,g)\cdot h}\overline{(Ad(h^{-1})(X_{(p,g)\cdot h}))}=Ad(h^{-1})(X)=Ad(h^{-1})\omega_{(p,g)}(\overline{X_{(p,g)}})$$
			If instead $X_{(p,g)}$ is horizontal then by the invariance horizontal distribution the field $dr X$ will still be horizontal ad so it will be annihilated by $\omega$ since it contains the projection on the vertical subbundle. Thus proving:
			$$r^*_h\circ \omega_{(p,g)}(X_{(p,g)})=(Adh^{-1})\omega_{(p,g)}X_{(p,g)}=0$$
			As for the last property, we need to prove that $\omega$ is smooth in a neighbourhood of an arbitrary point $(p,g)\in E$. Since $E$ is a manifold, we choose a chart $(U_E,\phi_E)$ centered in a point $(p,g)\in E$. Let $\{T^i\}$ be a base of the Lie algebra $\mathfrak{g}$ and $\{\bar{T^i}\}$ the associated fundamental vectors. Those fundamental vectors are all smooth by proposition PROPOSIZIONE. By the previous isomorphism, those span the whole vertical sub-bundle at any point. Furthermore, since by hypothesis $\mathcal{H}$ is a smooth distribution  of horizontal vectors, there is a base of smooth vector fields $X^i$ in any point that spans the whole space. Thus, since $TE=\mathcal{H}\oplus\mathcal{V}$, we can expand any element of $TE$ as a linear combination:
			$$V=a_i \bar{T^i}+b_iX^i;\hbox{ with } V\in TE, a_i,b_i\in \mathcal{F}$$
			The coefficeints $a_i,b_i$ aare smooth by the previous observations. Now, applying $\omega$, we get that the horizontal components are annihilated:
			$$\omega(a_i \bar{T^i}+b_iX^i)=\omega(a_i \bar{T^i})=a_i T^i$$
		\end{proof}
		Basically an Ehresmann connection arises on a principal bundle if one can find an horizontal smooth right-invariant distribution on it. One simply uses it to construct the 1-form as above.\\
		\\
		There is another equivalent manifestation of a connection: if one can find a smooth $G$-equivariant 1-form that takes value on the lie algebra $\mathfrak{g}$ of $G$ such that $\omega(\bar{A})=A$, then one can construce the horizontal distribution. Naively, one defines $$\mathcal{H}_{(p,g)}=Ker(\omega_{(p,g)})$$
		We are going to see later how this second definition is completely equivalent to the first one. In order to achieve this, it is useful to see something about sequences.
		\subsection{Sequences of vector bundles}
		Consider a short exact sequence of vector bundles over a manifold $M$:
		$$0\rightarrow A\xrightarrow{i}B\xrightarrow{j}B\rightarrow C\rightarrow0$$
		We call \textbf{splitting} a map $k:C\rightarrow B$ such that $j\circ K=\mathbb{I}_C$.
		\begin{Theo}
			Let
			$$0\rightarrow A\xrightarrow{i}B\xrightarrow{j}B\rightarrow C\rightarrow0$$
			be a short exact sequence of vector bundles on a manifold $M$. Then there is a $1-1$ correspondence between the following sets:
			$$\{\hbox{subbundles } H\in B;B=i(A)\oplus H\}\longleftrightarrow\{splittings: k:C\rightarrow B\}$$
		\end{Theo}
		\begin{proof}
			Consider NON HO VOGLIA
		\end{proof}
	\subsection{Horizontal sub-bundle and connections}
	\begin{Prop}
		If $\mathcal{H}$ is a smooth right-invariant distribution on a principal $G$-bundle, then $\forall g\in G$, the differential of the right-action $dr_g:TE\rightarrow TE$ commutes with the horizontal and the vertical projection operators.
	\end{Prop}
	\begin{proof}
		The result is pretty obvious. Let $Hor:TE\rightarrow\mathcal{H}$ be the horizontal projection. Since both the horizontal and the vertical distributions are right-invariant, it means that for any $X_{(p,g)}\in TE$ we have $dr_h X_{(p,g)}=dr_h \nu(X_{(p,g)})+dr_hHor(X_{(p,g)})$ and each of the two terms is still vertical/horizontal, so that:
		$$dr_h X_{(p,g)}=dr_h \nu(X_{(p,g)})+dr_hHor(X_{(p,g)})=
		\nu dr_h(X_{(p,g)})+Hordr_h(X_{(p,g)})$$
	\end{proof}
	We are now going to show that given any 1-form $\omega$ with certain oproperties, we can always construct an horizontal distribution. The proof is pretty lenghty.
	\begin{Theo}
		Consider a principal $G$-bundle $(E,M,\pi,G)$ where we can find a smooth $\mathfrak{g}$-valued and $G$-equivariant 1-form on it, such that $\omega(\bar{A})=A$ for every fundamental vector field. Then $Ker(\omega)=\mathcal{H}$ is a smooth right-invariant horizontal distribution on $TE$.
	\end{Theo}
	\begin{proof}
		We need to prove 3 basic properties:
		\begin{itemize}
			\item for every point $(p,g)\in E$, the tangent space of this manifold decomposes as:
			$$TE=\mathcal{H}_{(p,g)}\oplus \mathcal{V}_{(p,g)}$$
			\item $\mathcal{H}_{(p,g)}$ is right-invariant: for every $h\in G$ we have $dr_h \mathcal{H}_{(p,g)}\subset \mathcal{H}_{(p,g)\cdot h}$;
			\item $\mathcal{H}_{(p,g)}$ is a smooth sub-bundle of $TE$.
		\end{itemize}
		There is a short exact sequence of vector spaces for every $(p,g)$:
		$$0\rightarrow \mathcal{H}_{(p,g)}\xrightarrow{\mathbb{I}???} T_{(p,g)}E\xrightarrow{\omega_{(p,g)}} \mathfrak{g}\rightarrow0$$
		Furthermore, we have a splitting of this exact sequence: 
		$$dj_{(p,g)}^{-1}:\mathfrak{g}\rightarrow \mathcal{V}_{(p,g)}$$
		Thus, by proposition PROPOSIZIONE, there is a sequence of isomorphisms:
		$$T_{(p,g)}E\simeq\mathfrak{g}\oplus\mathcal{H}\simeq\mathcal{V}\oplus\mathcal{H}$$
		This proves the first requirement.\\
		\\
		As for the second property, we just evalue with a simple calculation the right-equivariance. Let $X\in\mathcal{H}$:
		$$\omega_{(p,g)\cdot h}(dr_hX_{(p,g)})=Ad(h^{-1})\omega{(p,g)}X_{(p,g)}=0$$
		This means that $\omega$ acting on a right-transalted vector of $\mathcal{H}$ is 0, so that this vector is still in $\mathcal{H}$, thus proving the second property.\\
		\\
		As for the third property NON HO VOGLIA
		\end{proof}
		\subsection{The Horizontal Lift of a vector field}
		Suppose that $\mathcal{H}$ is an horizontal distribution on a principal bundle $(E,M,\pi,G)$. Let $X\in\mathfrak{X}(M)$ be a vector field on $M$. Then for every point $(p,g)\in E$, we have $\mathcal{V}_{(p,g)}=Ker(d\pi_{(p,g)})$ and so we have some induced isomorphisms:
		$${T_{(p,g)}E\over \mathcal{V}_{(p,g)}}\simeq \mathcal{H}_{(p,g)}\simeq T_{\pi(p,g)}M$$
		Thus, there is a unique horizontal vector $\mathcal{X}_{(p,g)}\in\mathcal{H}_{(p,g)}$ such that:
		$$d\pi(\mathcal{X}_{(p,g)})=X_{\pi(p,g)}\in T_{\pi(p,g)}M$$
		We call this unique vecotr the Horizontal Lift of $X$ to $E$.
		\begin{Obs}
			Basically, when we have a principal $G$-bundle with an horizontal distribution, we see from the previous series of isomorphisms that the tangent space of the base manifold can  be identified with the horizontal tangent space of the principal bundle.\\
			\\
			Note that in our construction we have not assumed smoothness nor right-invariance. We are now going to prove that if $\mathcal{H}$ is smooth and right-invariant so will be the lift. (This is, honestly, pretty useless as a concept since it fucking obvious, yet it is useful to learn how to calculate stuff).
		\end{Obs}
		\begin{Prop}
			Let $(E,M,\pi,G)$ be a principal bundle with a smooth right-invariant horizontal distribution $\mathcal{H}$. Then the horizontal lift $\mathcal{X}$ of any vector field $X\in\mathfrak{X}(M)$ is smooth and right-invariant as well.
		\end{Prop}
		\begin{proof}
			Suppose that $p\in M$ and $(p,g)\in\pi^{-1}(p)$. Then we must have by definition of lift: $d\pi(\mathcal{X}_{(p,g)})=X_p$. Suppose now to take any other point of $\pi^{-1}(x)$ like $(p,h)$. Then, we must have, by the cosntruction of the principal bundle, that $(p,h)=dr_{g'}(p,g)=(p,g)\cdot g'$. Recalling that $\pi\circ r_{g'}=\pi$ we have:
			$$d\pi(dr_{g'}\mathcal{X}_{(p,g)})=d\pi\circ dr_{g'}(\mathcal{X}_{(p,g)})=d\pi\mathcal{X}_{(p,g)}=d\pi\mathcal{X}_{(p,h)}$$
			By uniqueness of the horizontal lift, $\mathcal{X}_{(p,g)}=\mathcal{X}_{(p,h)}$ and this proves right-invariance.\\
			\\
			As for the smoothness, we simply choose a trivializing chart on $M$ like $(U,\phi)$ and construct a trivialization 
			$$\varphi_U:\pi^{-1}(U)\rightarrow U\times G$$
			Now we define the vector field:
			$$Z_{(p,g)}=(X_p,0)\in T_{(p,g)}(U\times G)$$
			Clearly $Z$ is a smooth vector field on $U\times G$. Given the projection $\eta:U\times G\rightarrow U$ we have:
			$$d\eta Z_{(p,g)}=X_p$$
			To see this, note that $d\eta:T(U\times G)\rightarrow TU$ and so we get: $d\eta(Z)\in TU$ is a vector field. In particular, by definition of differential, we must have for any $(p,g)\in U\times G$:
			$$d\eta_{(p,g)}(Z_{(p,g)})f=Z_{(p,g)}(f\circ \eta_{(p,g)})$$
			And since $\eta$ projects the points from $U\times G$ to $U$, we get:
			$$d\eta_{(p,g)}(Z_{(p,g)})=X_p$$
			This implies that
			We can also define a smooth vector field on $\pi^{-1}(U)$ like:
			$$Y=d\varphi^{-1}(Z):U\times G \rightarrow T_{\pi^{-1}(U)}E$$
			For the same reason as before, given that $\pi$ is a projection, we must have:
			$$d\pi(Y_{(p,g)})=X_{\pi(p,g)}$$
			We know that the projection $Hor(Y)$ is smooth by hypothesis, If now we decompose $Y_{(p,g)}=\nu(Y_{(p,g)})+Hor(Y_{(p,g)})$ and apply the differential of the projection, recalling that $\mathcal{V}=Ker(d\pi)$, we get:
			$$d\pi(Y_{(p,g)})=d\pi(Hor(Y_{(p,g)}))=Hor(d\pi(Y_{(p,g)}))=X_{\pi(p,g)}$$
			This means that $Hor(Y)$ is the lift of $X$. By uniqueness, $X$ is smooth on $M$.
		\end{proof}
		\chapter{Connections on vector bundles}
		\section{The Pullback bundle}
			Consider a vector bundle $(E,M,\pi,F)$ and $f:N\rightarrow M$ is a smooth map over manifolds. Then we can define a smooth vecgtor bundle, called pullback bundle $f^*E$ over $N$. In particular, the base space of this bundle will be $N$ and the total space of this bundle will be 
			$$f^*E=\{(x_N,x_E)\in N\times E:f(x_N)=\pi(x_E)\}$$
			Note that we can construct a commutative diagram by using the two projections:
			$$\pi_N:f^*E\rightarrow N; \hbox{ and } \pi_E:f^*E\rightarrow E$$
			Those act in the obvious way:
			$$\pi_N(x_N,x_E)=x_N; \hbox{ and } \pi_E(x_N,x_E)=x_E$$
			The commutative diagram those two fit in is the following:\\
			\begin{center}
				\begin{tikzcd}
					f^*E \arrow{r}{\pi_E} \arrow[swap]{d}{\pi_N} & E \arrow{d}{\pi}\\
					N \arrow{r}{f}& M 
				\end{tikzcd}
			\end{center}
			Now, we show that there is a vector bundle structure on $(f^*E,N,\pi_N,\mathbb{R}^D)$ where $D$ will be: $dim(N)+dim(F)$.
			\begin{Prop}
				Let $(E,M,\pi,F)$ be a product bundle with $E=M\times F$ and $f^*E:N\rightarrow M$ is a smooth map between manifolds, then $f^*E$ is homeomoprhic to the product bundle $N\times F$.
			\end{Prop}
			\begin{proof}
				$$f^*E=\{(x_N,(x_M,x_F))\in N\times (M\times F);f(x_N)=\pi(x_M,x_E)=x_M\}=$$
				$$=\{(x_N,(f(x_N),x_F))\in N\times (M\times F)\}$$ 
				is by definition the base space of the new vector bundle. Consider now the following map:
				$$\sigma:f^*E\rightarrow N\times F \hbox{ such that }
				\sigma(x_N,(f(x_N),x_F))=(x_N,x_F)$$
				is clearly invertible and continuous and with continuous inverse. So it is an homeomprphism.
			\end{proof}
			\begin{Theo}
				Let $(E,M,\pi,F)$ be a smooth vector bundle on $M$ and $f:N\rightarrow M$ a smooth map between manifolds. Then $(f^*E,N,\pi_N,F)$ is a smooth vector bundle with same fiber.
			\end{Theo}
			\begin{proof}
				Suppose to take a trivializign open set of $M$, called $U$. Then, there is a map $\varphi_U:\pi^{-1}(U)\rightarrow U\times F$ diffeomorphic. Thus, locally, $E$ is homeomorphic to the product bundle $U\times F$. By the previous proposition, there is an homeomorphism $f^*E\simeq N\times F$. It remains to show that the trivializing charts are diffeomorphic since the projection is already smooth by construction.\\
				If we take the map $\sigma$ of the previous proposition, this is clearly smooth, invertible and with smooth inverse.
			\end{proof}
			Now we make some examples of the pullback bundle:
			\begin{Ex} [Vectors along a curve]
				Consider a vector bundle $(E,M,\pi,F)$ and a smooth curve on $M$ like $\gamma:[a,b]\rightarrow M$. Then there is a pullbakc smooth bundle $(\gamma^*E,[a,b],\pi_\mathbb{R},F)$ over a subset of the real line. Thus, a section of the pullback bundle in this case is a map $s:[a,b]\rightarrow \gamma^*E$. In particular, it assigns to each $t\in[a,b]$ a couple $(t,x_E)$ so when composed with the proper projection it gives a section of the original bundle along the curve $\gamma$.\\
				\\
				To be more precise, if we consider the original bundle to be the tangen bundle, sections of the pullback bundle over the real line would give us smooth vector fields along the curve $\gamma(t)$.
			\end{Ex}
		\section{The parallel transportation on a vector bundle}
			We are now basically going to generalize the notion of covariant derivative for generic vecotr bundles. One could already have guessed that this was possible in the previous chapters: we have not used once any property of the tangent bundle.\\
			\\
			Suppose to have a vector fiber bundle $(E,M,\pi,F)$ and a connection $\nabla:\mathfrak{X}(M)\times \Gamma(E)\rightarrow\Gamma(E)$ on it. Consider now a smooth curve on $M$ like $\gamma:[a,b]\rightarrow M$. This is a smooth map between manifolds, so we can construct a pullback bundle:
			$$\gamma^*E=\{(t,x_E)\in [a,b]\times E:\gamma(t)=\pi(x_E)\}$$
			As we know, there is an induced smooth vector bundle structure on $(\gamma^*E,[a,b],\pi_\mathbb{R},F)$. The sections of the pullback bundle can be thought of smooth sections of $E$ along the line $\gamma(t)$.
			\begin{Theo}
				Let $(E,M,\pi,F)$ be a vector bundle and $\nabla:\mathfrak{X}(M)\times \Gamma(E)\rightarrow\Gamma(E)$ a connection on it. Consider any smooth curve $\gamma:[a,b]\rightarrow M$ on the manifold and consider the induced opullback bundle $\gamma^*E$ and it's space of smooth sections $\Gamma(\gamma^*E)$. Then there is a unique linear map called \textbf{Covariant Derivative} along $\gamma$
				$${D\over dt}:\Gamma(\gamma^*E)\rightarrow \Gamma(\gamma^*E)$$
				such that it respects the follwoing properties:
				\begin{itemize}
					\item for every smooth section $s\in \Gamma(\gamma^*E)$,
					then the covariant derivative is $\mathbb{R}$ linear;
					\item The covariant derivative respects the Leibniz rule: for any $f\in [a,b]\rightarrow M$ 
					$${D(fs)\over dt}={df\over dt}s+f{Ds\over dt}$$
					\item if $s$ is an induced section from a global section $S\in\Gamma(E)$ in the sense that $S(\gamma(t))=s(t)$ then:
					$${Ds\over dt}=\nabla_{\gamma'(t)}S$$  
				\end{itemize} 
			\end{Theo}
			\begin{proof}
				The proof is identical to the one we saw in theorem TEOREMA. The idea is to fix a framed open set $(U,e_i)$ and expand the section of $\Gamma(\gamma^*E)$ as:
				$$s(t)=s^i(t)e_{i,\gamma(t)}$$
				then repeat the same calculations, defining the covariant derivative as:
				$${Ds\over dt}={ds^i\over dt}e_{i,\gamma(t)}+s^i\nabla_{\gamma'(t)}e_{i}$$
			\end{proof}
			In this new generalized definition we simply have that $E$ has the role of $TM$. The idea is the same and it is still useful to imagine the sections of the vector bundle as smooth vector fields, even if they are not.
			\begin{Def}
				We say that a section $s\in\Gamma(\gamma^*E)$ is parallel along the cure $\gamma:[a,b]\rightarrow M$ if ${Ds\over dt}=0$ for every $t\in[a,b]$. If $s$ is parallel, we call $s(b)$ the parallel transport of $s(a)$ along the curve $\gamma(t)$.
			\end{Def}
			\begin{Theo}
				Suppose that $M$ is a manifold with a connection $\nabla:\mathfrak{X}(M)\times\Gamma(E)\rightarrow\Gamma(E)$. Then, given any curve $\gamma:[a,b]\rightarrow M$ and a section $s_a$ of $\Gamma(\gamma^*E)$ in $a$, there always exists a parallel section field $s$ along $\gamma$ such that $s(a)=s_a$. Furthermore, there is an isomorphism $\phi_{(a,b)}:E_{\gamma(a)}\rightarrow E_{\gamma(b)}$
			\end{Theo}
			\begin{proof}
				The proof is not hard
			\end{proof}
		\section{Horizontal vectors on the frame bundle}
			We are going to show that, given a smooth vector bundle $(E,M,\pi,F)$, a connection on $E$ automatically induces a notion of Horizontal vectors on the Frame bundle. In general, we have seen that the notion of Horizontal vector is not well defined for principal bundles. However, the sole existence of a connection on a vector bundle ensures a well behaving notion of horizontal vectors.
\end{document}
