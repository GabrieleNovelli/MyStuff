\documentclass[12pt,a4paper]{report}

\usepackage[italian]{babel}
\usepackage{newlfont}
\usepackage{color}
\usepackage{float}
\usepackage{frontespizio}
\usepackage{amsmath,amssymb}
\usepackage{amsthm}
\usepackage{geometry}
\usepackage{tikz}
\usepackage{biblatex}
\usepackage{csquotes}
\usepackage{pgfplots}
\usepackage{hyperref}
\usepackage{amssymb}
\usepackage{comment}

\hypersetup{
	colorlinks=true,
	linkcolor=blue,
	filecolor=magenta,      
	urlcolor=cyan,
	pdftitle={Overleaf Example},
	pdfpagemode=FullScreen,
}

\textwidth=450pt\oddsidemargin=0pt
\geometry{a4paper, top=3cm, bottom=3cm, left=3cm, right=3cm, % heightrounded, bindingoffset=5mm 
}
\theoremstyle{definition}
\newtheorem{Def}{Definizione}[chapter]

\theoremstyle{Theorem}
\newtheorem{Theo}[Def]{Teorema}
\newtheorem{Prop}[Def]{Proposizione}

\newtheorem{Lm}[Def]{Lemma}

\theoremstyle{definition}
\newtheorem{Ex}[Def]{Esempio}

\theoremstyle{definition}
\newtheorem{Lem}[Def]{Lemma:}

\theoremstyle{definition}
\newtheorem{Obs}[Def]{Osservazione:}

\begin{document}
	\begin{center}
		{{\Large{\textsc{Gabriele Novelli}}}} 
		\rule[0.1cm]{15.8cm}{0.1mm}
		\rule[0.5cm]{15.8cm}{0.6mm}
	\end{center}
\begin{center}\textcolor{black}{
		{\LARGE{\bf Credo di non averci capito nulla}}
}\end{center}
\tableofcontents
\newpage
\thispagestyle{empty}
\mbox{}
\newpage
\chapter*{Introduzione}
Questa rubrica è nata in un fresco giorno d'estate, dentro la mia cameretta, al riparo dai raggi del sole e dalle fastidiosissime zanzare; mentre godevo della lettura di un libricino.\\
Il mio intento sarà, per quanto lungo o breve esso possa rivelarsi, quello di commentare. Ciò che il buon lettore ritroverà dentro questa rubrica non sarà altro che una collezione di recensioni e pareri, probabilmente errati, su cose che mi capita di leggere.\\
Nonostante la mia propensione per il linguaggio matematico (e quindi la chiarezza, checché ne strillino gli ignoranti) e la mia totale incapacità per la comprensione del linguaggio musicale occidentale, questa impresa temo si rivelerà per me di estrema ed estenuante laboriosità. Temo, in particolar modo, che la principale delle difficoltà sorgerà dalla necessità (imposta da me stesso ora) di mantenere un registro costante e decoroso, che perduri per tutta l'opera. Ma bando alle ciance. Cominciamo.
\chapter{Fëdor Dostoevskij: Memorie del sottosuolo}
Questo libricino (un po' più di cento pagine) l'ho trovato davvero complesso. Non solo dal punto di vista sintattico, ma anche filosofico. Temo di non averci capito nulla. L'opera racconta una parte della storia di un uomo, a cui ci riferiremo come Il Protagonista, in quanto non ha nome. Tuttavia, il vero e proprio racconto è preceduto da una vera e propria premessa filosofica, estremamente elaborata e longeva, che prende il nome de: Il Sottosuolo.\\
In questa prima sezione il Protagonista si lascia trasportare da una sorta di cinica critica verso sè stesso e tutta la società che lo circonda. Devo confessare che non sono bene riuscito a cogliere il punto principe di tutto questo primo discorso. Sarò io stupido, ma mi è parso tutto così incasinato, ingarbugliato e senza conclusione, che mi riesce difficile districarmi in una precisa spiegazione. Il Protagonista parte accusando sè stesso, (m'è parso) delineando la propria persona come estremamente maliziosa ed irragionevole, a tratti malata. E qui, mi sento di lasciarmi andare alla prima vera e propria opinione personale rilevante: la caratterizzazione filosofica di sè e del mondo espressa dal Protagonista pare (credo volutamente?) esagerata. E questa esagerazione verrà mantenuta lungo tutto il corso dell'opera. L'esagerazione qua è intesa negativamente: ciò che è esagerato è l'irrazionalità e la malignità del personaggio. Mi viene difficile credere che possa esistere davvero qualcuno di così coscientemente  malato. Arriviamo ora alle vere e proprie critiche: viene delineata più e più volte, all'interno de Il Sottosuolo, una aperta battaglia al positivismo e all'ottimismo come filosofie. In particolare è molto utilizzata la metafora del 2$\times$2=4 e delle "tabelle matematiche":
\begin{center}
	\textit{Per conseguenza, queste leggi naturali basta scoprirle, che l'uomo non risponderà più delle sue azioni, e vivere gli sarà straordinariamente facile Tutte le azioni umane, va da sé, allora saranno calcolate secondo queste leggi, matematicamente, sul tipo d'una tavola di logaritmi fino a 108000, e inscritte nel calendario; o, meglio ancora, compariranno delle pubblicazioni benpensanti, sul tipo degli odierni dizionari enciclopedici, in cui tutto sarà enumerato e segnato in modo cosi preciso che nel mondo non ci saranno  né azioni né avventure.}
\end{center}
Tuttavia l'opposizione al 2$\times$2=4 non è solo filosoficamente perseguita, ma è anche accentuata da "lamentele" del Protagonista: \\
\begin{center}
\textit{
	"- Abbiate pazienza, - vi grideranno – rivoltarsi è impossibile; è come due per due fa quattro! La natura non vi consulta; non gliene importa nulla dei vostri desideri e se vi piacciano o non vi piacciano le sue leggi. Siete obbligato ad accettarla così com'è, e per conseguenza anche ad accettare tutti i suoi risultati. Una muraglia, per conseguenza, è una muraglia.. ecc. ecc.-
	Signore Iddio, ma
	che me ne importa delle leggi naturali e dell’aritmetica, quando per qualche
	ragione queste leggi e il due per due non mi piacciono? Si intende che questa
	muraglia non la sfonderò col capo, se davvero non avrò la forza di sfondarla,
	ma nemmeno l’accetterò."
}
\end{center}
Dalle mie precedenti letture di Fëdor, mi pare di capire che questo tema (della critica al positivismo), unito all'opposizione valori/relativismo, siano centrali nella filosofia di questo autore. E questo mi fa riflettere: da una parte abbiamo il trionfo della libertà individuale (2$\times$2=5), dall'altra, un'apparente negazione di questa (critica la relativismo). Il punto de Il Sottosuolo, tuttavia, credo sia più che altro quello di mostrare la profonda irrazionalità dell'uomo i tutte le sue azioni, a partire dalla guerra. E forse è proprio per questo che la figura de Il Protagonista è così tremendamente caricata di irrazionale stupidità e pathos vomitevole (cringe insomma). 
\begin{center}
	\textit{ Lo schifoso, basso desiderucolo ripagare l'offensore con lo stesso male forse gli pruderà dentro ancora più infamemente che nell'homme de la nature et de la vérité, perché l'homme de la nature et de la vérité, per la sua innata stupidità, considera la propria vendetta puramente e semplicemente come un atto di giustizia; ma il topo, per effetto dell'intensa coscienza, nega questa giustizia. Si giunge finalmente al fatto, all'atto stesso della vendetta. Il disgraziato topo, oltre a un'infamia iniziale, ha già fatto in tempo ad ammucchiare intorno a sé, sotto forma di interrogativi e di dubbi, tante altre infamie; al un interrogativo ha aggiunto tanti interrogativi insoluti, che per forza gli si raccoglie intorno una sorta di broda fatale, di fetida melma,
		costituita dai suoi dubbi e turbamenti, nonché, infine, dagli sputi che gli piovono addosso
		da parte degli uomini immediati e d'azione, che stanno solennemente all'intorno in veste di giudici e despoti e sghignazzano di gusto di lui. S'intende che non gli rimane altro che fare con la sua zampetta un gesto di rinuncia a tutto e, con un sorriso di ostentato disprezzo al quale esso stesso non crede, sgusciare ignominiosamente nel suo buco. Là, nel suo schifoso,
		fetido sottosuolo, il nostro topo offeso, maltrattato e deriso si sprofonda immediatamente in una fredda, velenosa e, soprattutto, eterna malignità.}
\end{center}
Ma arriviamo ora ad un punto centrale, che, a dire il vero, non mi è per nulla chiaro: che cos'è di preciso il Sottosuolo? Da quanto mi pare di aver capito, il Sottosuolo è una sorta di stato mentale o angolo della nostra psiche, estremamente remoto e nascosto, all'interno del quale il Protagonista si sente ingabbiato. Nel Sottosuolo siamo soli ed abbandonati a noi tessi e pare non esserci alcuna via d'uscita. Non sono ben riuscito a capire se il Sottosuolo è uno stato psicologico o una condizione di esistenza, ma mi pare entrambe le cose. Non mi è nemmeno chiaro se il Sottosuolo è qualcosa all'interno del quale alberga questa irrazionale maliziosità di cui Fëdor parla, o se ne è una sorta di conseguenza. Però, una differenza fondamentale tra l'uomo del Sottosuolo (il topo) e l'uomo d'azione (il suo opposto) è che il primo è incapace di accettare le leggi naturali (ma non in grado di superarle), mentre il secondo vi si arrende senza condizioni. In qualche strano modo, l'uomo del sottosuolo mi ha ricordato molto la figura del Nichilista Passivo di Nietzsche, ovvero un uomo che, sopraffatto dalla bestialità e dalla grandezza del relativismo, rimane come inebetito e vi si lascia cullare, crogiolandosi nella sua stessa sofferenza senza far nulla. Mi sembra tanto che l'uomo del sottosuolo, per l'appunto, non faccia nulla per migliorare la propria situazione (che è evidentemente negativa). Si badi bene che questa mia critica non vuol'essere una giustificazione al positivismo, ma una semplice analogia filosofica.\\
La seconda parte del libricino è invece intitolata: A proposito della neve bagnata. In essa vengono narrati una serie di ricordi della vita del Protagonista, i cui temi riprendono quelli espressi all'interno della sezione precedente. Secondo la mia personalissima opinione, questi episodi delineano la figura del Protagonista come a dir poco malata, estremamente egocentrica ed a tratti sado-masochista. Esso infatti presenta sè stesso come una sorta di inetto; un emarginato (per propria volontà si badi) dalla società, un poveraccio che svolge controvoglia una lavoro che odia e per cui è miseramente pagato. Il Protagonista non ha amici e questa mancanza di amicizia, mi pare avere una duplice causa: da un lato, la natura schiva e "cosciente" del Protagonista tendono all'autoisolazione, dall'altro non vi pare essere nessun individuo che lo tratti con un rispetto gratuito. 
\begin{center}
	\textit{Tutti i nostri impiegati d'ufficio, s'intende, li odiavo dal primo all'ultimo, e li disprezzavo tutti, e insieme pareva che li temessi}
\end{center}
 O ancora: 
 \begin{center}
 	\textit{Mi tormentava allora anche un'altra circostanza: precisamente, che nessuno mi somigliasse, e di non somigliare a nessuno.}
 \end{center}
E, caro lettore, mi è parso a tratti esilerante la lunga sfilza di giustificazioni che il Protagonista porta per alleggerire la propria condizione. La figura del Protagonista è altamente contraddittoria: da un lato esso pare ritenersi superiore agli altri, dall'altro:
\begin{center}
	\textit{Facevo perfino degli esperimenti: avrei sopportato lo sguardo almeno del tale su di me?}
\end{center}
Ed è proprio questa incapacità di sostenere lo sguardo altrui che lo porterà a disprezzare un enorme ufficiale, la cui unica colpa fu quella, durante una rissa, di spostare il Protagonista da parte senza considerarlo. Ma il disprezzo e la vergogna di sè, paiono mescolarsi così male con la superbia e l'egoismo del Protagonista, che l'unica "vendetta" che esso riesce a perpetrare nei confronti dell'ufficiale, è quella di urtarlo con una spalla camminando.\\
La cornice di pensieri che ci porta a questa misera vendettucola è l'analogo moderno de la "sega mentale" per eccellenza. Il Protagonista, per tutto il corso del libro, si lascia cullare (passivamente per l'appunto) nella propria immaginazione, abbandonandosi ai più inverosimili e disdicevoli pensieri (film mentali). E sono, da quanto mi sembra di aver capito, esattamente questi stessi che lo conducono ad agire in modo così sconsiderato ed irrazionalmente maligno.\\
Voglio ora spezzare una lancia a favore del Protagonista: è pur sempre vero che esso è profondamente circondato da cattiveria, da offese e da povertà. Prendendo come esempio la cena in onore di Zverkov: nessuno degli invitati si preoccupa di comunicare la posticipazione dell'appuntamento, lasciando così il povero Protagonista in imbarazzante attesa per un'ora intera.
\begin{center}
	\textit{- Sono arrivato alle cinque in punto, come mi era stato fissato ieri, - risposi ad alta voce e con un'irritazione che prometteva un non lontano scoppio.}
\end{center}

\end{document}
