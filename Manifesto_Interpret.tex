\documentclass[]{article}
\usepackage{amsmath,amssymb}
\usepackage{amsthm}
\usepackage{geometry}
\usepackage[autostyle]{csquotes}

\begin{document}
	\begin{flushright}
		\begin{tabular}{@{}l@{}}
			Nove, Dan
		\end{tabular}
	\end{flushright}
	\section{Il Problema dell'Interpretazione}
	Spesso si tende a dimenticare, studiando, che l'obbiettivo ultimo della fisica è la pura comprensione della realtà. In questo senso non esiste nulla di più reale della fisica, nonostante la sua meravigliosa componente astratta. Spesso si tende anche a dimenticare che le molteplici interpretazioni che la fisica offre della realtà non sono altro che, per l'appunto, interpretazioni.  
	\\
	Il problema che verrà sviscerato all'interno di questo documento può essere sintetizzato semplicemente nella questione: \enquote{E se la realtà non potesse essere descritta?}. In particolare, verrà totalmente abbandonata l'idea che la realtà possa essere in alcun modo descritta a livello fondamentale. Assumeremo, ragionevolmente, di poter solamente trovare interpretazioni più o meno soddisfacenti. Chiameremo questo pensiero Interpretazionismo e cercheremo di definirne i tratti pricipali.\\
	Prima di partire, è doveroso sottolineare che le considerazioni esposte all'interno di questo manifesto riguardano solamente la fisica. Non è escluso che tutti i risultati che si otterranno potranno essere estesi a scienze meno \enquote{pure} (pure inteso come astratte e riguardanti le strutture fondamentali del reale), ma ciò non sarà affrontato all'interno di queste pagine. 
	\subsection{Il Principio di Falsificazione}
	Vediamo innanzitutto, in modo semplicistico, come poter definire una teoria scientifica.
	Il meraviglioso principio di falsificazione di Popper è il punto di partenza: \enquote{Una teoria è scientifica solamente se può essere falsificata}. Questa definizione fornisce un metodo (scientifico) per rappresentare la realtà ad infinitum, ovvero elaborare teorie sempre più esatte. Una teoria fisica che si rispetti deve accordarsi non solo con le misure sperimentali, ma anche con le sue predecessore, almeno entro certe approssimazioni. Come esempio si consideri che la relatività generale riproduce localmente i risultati della relatività ristretta%, la quale riproduce, nel regime di basse velocità (rispetto a quella della luce), la Meccanica Classica, e così via%
	. Tuttavia, la prima ci permette di spiegare fenomeni che la seconda non vede; per questo diciamo che la relatività generale è più esatta della relatività ristretta. \\
	Ritornando al principio, noi crediamo che la vera magnificenza di questo risieda nella spinta evolutiva che esso conferisce ad ogni disciplina scientifica: ogni teoria può essere corroborata o smentita e dunque aggiornata, migliorata, in modo da approssimarsi sempre di più alla corretta comprensione di ciò che è reale. 
	Risulta tuttavia problematico utilizzare questo principio per asserire che la realtà non possa essere descritta. In particolare, l'Interpretazionismo non consegue dalla falsificabilità, ma dalla limitatezza intrinseca umana; ovvero dalla necessità insormontabile per ogni misura sperimentale di passare attraverso il filtro dei nostri sensi.  
	
	%La seconda si ricollega al nostro quesito iniziale: essendo ogni teoria in linea di principio falsificabile, non si avrà mai la certezza di aver trovato una interpretazione che si accordi esattamente con la descrizione della realtà. Ma attenzione, vi è una sostanziale differenza con l'asserire che essa non possa essere descritta!% 
	\subsection{L'Interpretazionismo fisico}
	
	Ogni teoria fisica si basa su di un preciso apparato matematico, estremamente rigoroso. Attraverso quest'ultimo, vengono concretizzati degli assiomi fisici totalmente arbitrari, che permettono di costruire la teoria. Noi crediamo che questi assiomi si suddividano in due categorie, che chiameremo postulati e principi.\\
	\\
	\textbf{Def:}\\
	Chiamiamo Postulato una qualsiasi assunzione sull'essenza fondamentale della natura, formalizzata attraverso una rigorosa rappresentazione matematica.
	\\
	\\
	I postulati corrispondono all'interpretazione della realtà su cui la teoria è costruita e per loro natura non sono falsificabili.
	\\\\
	\textbf{Def:}\\
	Chiamiamo Principio una qualsiasi legge su cui si basa la teoria in esame, che deve poter essere corroborata dalle misure sperimentali.
	\\
	\\
	Rifacendoci al pensiero di Leibniz, scelta una determinata teoria fisica, i postulati corrispondono sostanzialmente alle verità di ragione mentre i principi alle verità di fatto. Con questo non intendiamo asserire che i postulati corrispondano a tautologie, ma sottolineare l'impossibilità di falsificarli. Essi incorniciano la realtà, sorreggono la teoria. I principi fisici potranno e dovranno invece essere sottoposti a falsificazione.\\
	\\
	Al fine di comprendere appieno le definizioni precedenti, consideriamo un semplice esempio: la meccanica (classica) newtoniana.\\
	Questa teoria usufruisce principalmente di due sovrastrutture matematiche chiamate calcolo vettoriale ed analisi matematica. Molto brevemente, l'analisi matematica si occupa di descrivere il comportamento di relazioni (funzioni) tra insiemi di numeri reali; mentre il calcolo vettoriale si occupa di descrivere il comportamento dei cosiddetti vettori. Formalmente, un vettore è un elemento astratto di uno spazio vettoriale, ovvero un insieme su cui sono definite certe operazioni e proprietà. Per fare un esempio, $\mathbb{R}$ con la somma che conosciamo tutti è uno spazio vettoriale. L'apparato matematico ci fornisce l'impalcatura su cui muoversi: lo spazio che ci circonda è riprodotto attraverso sistemi di riferimento costruiti con 3 assi cartesiani ortogonali tra loro. L'insieme dei punti dello spazio costruiti in questo modo è uno spazio vettoriale. Dunque potete immaginare i vettori nello spazio come punti o frecce (coppie di punti). \\
	I principi su cui si basa la meccanica newtoniana sono essenzialmente 3. Tuttavia, è necessario prima di tutto assumere la presenza delle forze. Assumiamo che un'interazione tra due corpi si manifesti sotto forma di una forza, ovvero un vettore (una freccia nello spazio (tangente)) a cui possiamo associare delle precise quantità, come la direzione e l'intensità.		
	Possiamo allora elencare alcuni dei postulati su cui si basa la teoria in esame:
	\\\\
	\textbf{Postulato:}\\
	L'universo è in 3 dimensioni, ovvero è rappresentabile come $\mathbb{R}^3$.
	\\\\
	\textbf{Postulato:}\\
	Le interazioni tra corpi sono causate da forze, le quali si manifestano attraverso accelerazioni. Le forze sono rappresentabili come elementi di $\mathbb{R}^3$ e sono tali per cui, data $\vec{a}$ accelerazione del corpo, la forza causa di questa stessa accelerazione sarà $\vec{F}=(f_1(\vec{a}),f_2(\vec{a}),f_3(\vec{a}))$ dove $f_i$ sono funzioni scalari di $\vec{a}$.
	\\\\
	\`E naturale chiedersi perché il secondo postulato sia tanto lungo. Risponderemo a questo quesito in seguito. Notiamo che il primo postulato si traduce nella rappresentazione dello spazio attraverso gli assi cartesiani, mentre il secondo è formalizzato matematicamente attraverso il calcolo vettoriale e l'analisi. Tuttavia, è necessario ricordarsi che i postulati costituiscono l'interpretazione della realtà su cui costruiamo la teoria. La loro natura è totalmente arbitraria, convenzionale. Nulla ci assicura, ad esempio, che lo spazio in cui viviamo sia, matematicamente, quello Euclideo. Non ci soffermeremo a discutere i particolari di questa affermazione, ma il lettore interessato è invitato a consultare l'opera - La Scienza e l'Ipotesi - di Poincaré, di cui ora citiamo un breve passaggio: \enquote{Si vuol dire che, per selezione naturale, il nostro intelletto si è adattato alle condizioni del mondo esterno, e che ha adottato la geometria più conveniente per la specie o, in altri termini, la più comoda. Tutto questo è pienamente conforme alle nostre conclusioni: la geometria non è vera, è conveniente}.\\
	Ritornando ora alla meccanica newtoniana, vediamone i principi:\\
	Il primo principio di Newton recita: 
	\\\\
	\textbf{Principio:}\\
	Se la somma delle forze agenti su di un corpo è nulla allora esso ha accelerazione nulla.
	\\\\ 
	Sostanzialmente, se questa condizione è verificata, o il corpo è fermo o si muove a velocità costante. Il secondo principio recita:
	\\\\
	\textbf{Principio:}\\
	La somma delle forze agenti su di un corpo è proporzionale all'accelerazione del corpo stesso, tramite la massa (inerziale), che può essere tradotta come $\sum\vec{F}=m\vec{a}$.
	\\\\
	Il terzo principio recita: \\\\
	\textbf{Principio:}\\
	Se un corpo C esercita una forza su di un corpo H allora il corpo H esercita una forza uguale e contraria su C.
	\\\\
	A partire da questi principi totalmente arbitrari non sarebbe solo possibile ricostruire la dinamica di un corpo rigido conoscendo un numero sufficiente di parametri, ma si potrebbe anche prevedere la sua futura evoluzione. Idealmente, secondo la meccanica newtoniana, possedendo un numero sufficientemente elevato di informazioni sarebbe possibile ricostruire il tragitto di una particella dal momento del big bang fino ad oggi. Oggi sappiamo che questa teoria si accorda con i risultati sperimentali solo entro certe approssimazioni.\\
	In particolare, alcuni problemi iniziano a manifestarsi quando i corpi analizzati si muovono \enquote{troppo velocemente}. Sostanzialmente, pare sperimentalmente che la velocità massima per un corpo sia $c$, la velocità della luce. Questo distrugge completamente la meccanica newtoniana, rendendola una teoria non corroborata universalmente.\\
	Da questo esempio è chiaro come i principi di una teoria siano sottoponibili a falsificazione: falsificare la teoria corrisponde a falsificarne i principi. Questa operazione non è chiaramente possibile con i postulati, in quanto essi, concernendo l'essenza stessa della natura, sfuggono alle nostre misure sperimentali: falsificare i principi fisici non falsifica necessariamente l'interpretazione della natura che la teoria presuppone. Conseguentemente, essendo le nostre supposizioni circa l'essenza della realtà non falsificabili, non possiamo avere la certezza che esse siano effettive descrizioni. \enquote{Conoscere l'altezza dell'albero maestro non basta a stabilire l'età del capitano} -Jules Henri Poincaré.  
	\\
	In sostanza uno dei problemi della meccanica newtoniana è che essa postula l'esistenza di forze, veri e propri fantasmi che si manifestano come motori del movimento dei corpi. Tuttavia, noi misuriamo solamente il movimento e le interazioni tra questi. Noi diciamo che le forze \enquote{esistono} e sono la causa degli spostamenti semplicemente perché la teoria funziona. Nulla ci garantisce che muovendo la matita con il dito vi siano forze rappresentate da frecce che si manifestino. Esse non sono altro che un puro costrutto fisico-matematico non falsificabile dai nostri esperimenti: fisico in quanto concerne l'essenza della realtà, matematico in quanto rappresentabile matematicamente. \`E importante sottolineare che nonostante sia matematicamente possibile calcolare il verso, la direzione e l'intensità di una serie di forze (e quindi definirle in tutto e per tutto), questo non cambia la non falsificabilità della loro essenza.\\
	\\
	Dunque, paradossalmente, ogni teoria fisica è costituita da una parte di assunzioni \textit{non} \enquote{scientifiche} in senso stretto, ovvero impossibili da verificare sperimentalmente: ciò che possiamo falsificare non è l'impalcatura di postulati su cui si basa la nostra teoria, bensì i risultati a cui i principi conducono, in quanto la prima rimarrà sempre costretta sotto il velo di Maya della falsificazione.\\
	Il nodo centrale di questo problema è che ogni teoria fisica è inseparabile dalle effettive misure sperimentali, poiché esse sono l'unico processo che ci permette di mettere in atto il meccanismo di falsificazione. Una misura non è altro che un'osservazione di un fenomeno naturale (fisico) attraverso il filtro dei nostri sensi, i quali sono estremamente limitati. Di conseguenza, tanto più complesse risulteranno essere le effettive misure, tanto più astratta sarà la teoria. \\
	\\
	Consideriamo un secondo esempio: l'elettrone. In meccanica classica, pensiamo l'elettrone come un corpo puntiforme, dotato di massa e carica elettrica. Tuttavia, questa interpretazione fallisce nel giustificare alcuni comportamenti della particella in esame. Un esempio è dato dall'esperimento della doppia fenditura: molto brevemente, lanciando elettroni contro una lastra munita di fessure estremamente sottili, essi daranno vita a fenomeni di tipo ondulatorio (esperimento alla Young). La meccanica quantistica nasce per risolvere problemi di questo tipo, associando ad ogni particella una funzione d'onda. In questa nuova ottica gli elettroni non sono più solamente corpi puntiformi, ma enti in grado di presentare proprietà sia corpuscolari che ondulatorie.\\
	Tuttavia, è necessario ricordare che non ci è dato \enquote{vedere} attraverso le misure sperimentali una particella. Tutto ciò che misuriamo sono le interazioni che questa ha con la nostra strumentazione. Dunque, il fatto che una interpretazione appaia più precisa di un'altra, come nel caso della meccanica quantistica rispetto a quella classica, non ci assicura che l'interpretazione che la seconda teoria fornisce della realtà sia effettivamente una descrizione.    \\\\ 
	%Si consideri ora un esempio lievemente più complicato: lo spin. In meccanica classica esiste una grandezza, chiamata momento angolare, che ci permette di manipolare i corpi in rotazione. In generale, il lettore inesperto può banalizzare il momento angolare come una quantità che indica "quanto il corpo sta ruotando".\\In meccanica quantistica ad ogni particella è associata una proprietà fisica chiamata spin. Lo spin degli elettroni è (all'incirca) la causa a cui associamo le interazioni che questi presentano con un campo magnetico esterno. Le equazioni che si ritrovano nello studio di queste interazioni presentano una struttura sorprendentemente simile a quelle del momento angolare. Per questo tendiamo ad immaginare, classicamente, un elettrone dotato di spin come una piccola sfera in rotazione su sé stessa. Tuttavia non ci è dato "vedere" attraverso degli strumenti un elettrone roteare. Tutto ciò che misuriamo sono le interazioni che esso ha con i nostri rivelatori. Classicamente, questa interpretazione dello spin è estremamente funzionale, tuttavia, per le ragioni di cui sopra, non è altro che, per l'appunto, un'interpretazione. \\\\%
	La mira dell'Interpretazionismo non è screditare l'approccio semplicistico e diretto della fisica (in quanto spesso e volentieri è quello più corretto), ma di ricordare a noi stessi di non intrappolarci nell'idea che la realtà sia l'esatta copia di teorie umane, \textit{troppo umane}. Il rischio che si corre è quello di dimenticare la sostanziale differenza tra interpretazione e descrizione della realtà, perdendoci nei meandri della teoria.\\
	\\
	Consideriamo infine teoria dei campi quantizzati. Questa meravigliosa teoria ci permette di analizzare le interazioni tra le particelle che compongono l'universo. L'ente fondamentale è il campo, un'entità fisica che permea tutto lo spazio tempo, come fosse un mare. L'esistenza del campo non è falsificabile, in quanto sfugge alle nostre misure, che si limitano alle banali interazioni tra quelle che supponiamo essere eccitazioni del campo. Ancora una volta, la sola misurazione delle interazioni non ci assicura che la nostra interpretazione corrisponda alla realtà. \\
	\\
	In conclusione, generalizzando, il punto chiave della fisica è l'assunzione di verità indimostrabili che definiscono essa stessa. L'arbitrarietà di questi postulati rende intangibile la certezza che essi siano effettivamente descrittivi, in senso stretto, della realtà. Seguendo questo ragionamento noi diciamo che \enquote{ciò che è reale non può essere descritto} riferendoci alla totale soggettività e non-falsificabilità degli assunti con cui tentiamo per l'appunto di descrivere la natura. Ovviamente, nulla esclude che esistano degli assiomi \enquote{corretti}, ma questa loro accezione di correttezza rimane (per ora) semplicemente indimostrabile e, dunque, priva di alcun valore. La realtà è interpretabile, ma non descrivibile.
	
	\subsubsection{Specchietto sulla meccanica classica}
	Abbiamo precedentemente enunciato i principi ed alcuni dei postulati della meccanica classica. In questa piccola digressione giustificheremo le nostre scelte. \\
	Una più semplice formulazione del primo postulato potrebbe essere:\\\\
	\textbf{Postulato (Insufficiente):}\\
	Le interazioni tra corpi sono causate da forze, rappresentabili come elementi di $\mathbb{R}^3$.
	\\\\
	Tuttavia la quantità di informazioni presente in questo assunto rende problematica la falsificazione del secondo principio: $\sum\vec{F}=m\vec{a}$. Il problema è che è necessario esplicitare la relazione tra forza ed accelerazione, in quanto noi siamo solo in grado di misurare esplicitamente la seconda, ma non la prima. Asserire la presenza di forze come causa delle interazioni non è sufficiente. Se non ammettessimo che esiste	una relazione tra l'accelerazione di un corpo e la forza a cui esso è sottoposto, non potremmo effettivamente falsificare l'assunzione $\sum\vec{F}=m\vec{a}$, poiché non vi sarebbe correlazione tra le due grandezze.\\
	La difficoltà nella stesura dei principi e postulati della meccanica classica risiede nel fatto che questi non concernono proprietà emergenti del sistema fisico: non vi è metodo di misura diretta per le forze, è sempre necessario analizzare i movimenti dei corpi rigidi nello spazio. Risulta allora problematico falsificare una relazione matematica tra forza ed accelerazione in assenza di un modo alternativo per valutare la forza. Allo stesso tempo, è facile rendersi conto di quanto risulterebbe più semplice, a livello concettuale, falsificare il secondo principio della termodinamica: basta calcolare l'entropia.\\
	Consideriamo invece la nostra formulazione:\\\\
	\textbf{Postulato (Sufficiente):}\\
	Le interazioni tra corpi sono causate da forze, le quali si manifestano attraverso accelerazioni. Le forze sono rappresentabili come elementi di $\mathbb{R}^3$ e sono tali per cui, data $\vec{a}$ accelerazione del corpo la forza causa di questa stessa accelerazione sarà $\vec{F}=(f_1(\vec{a}),f_2(\vec{a}),f_3(\vec{a}))$ dove $f_i$ sono funzioni scalari di $\vec{a}$.
	\\\\
	\' E evidente che, attraverso la formalizzazione più estesa, segue necessariamente che $\vec{F}=m\vec{a}+\vec{q}$, ovvero $\vec{F}$ è una funzione lineare di $\vec{a}$: se così non fosse le leggi di composizione non si accorderebbero con i risultati sperimentali. Per di più, $\vec{q}=\vec{0}$ poiché se così non fosse sarebbe possibile esercitare un'unica forza diversa da $\vec{0}$ su di un corpo, ed esso rimarrebbe fermo. Ciò è in evidente contrasto con i nostri principi.\\
	\\
	Notiamo infine che la seconda versione del postulato ci assicura una definizione della massa (inerziale): essa non è altro che il coefficiente di proporzionalità tra la forza e l'accelerazione.
	\\\\
	In conclusione, è chiara la ragione per cui la prima stesura del postulato risulta insufficiente: omettendo una relazione tra forza e accelerazione, risulta impossibile falsificare il secondo principio di Newton.
	\\\\
	Ignoriamo completamente la problematicità nel definire gli osservatori inerziali. 
	\section{La Causa}
	
	Da dove proviene questa insormontabile necessità di postulare? Quali sono le ragioni per cui siamo costretti ad adoperare assunzioni affinché le nostre teorie fisiche funzionino? \\
	Per trovare una risposta a questi interrogativi è necessario analizzare la relazione tra fisica teorica e sperimentale. Il fine di una qualsiasi teoria è quello di accordarsi il più precisamente possibile con le effettive misure. Queste, tuttavia, sono il risultato di interazioni tra enti fisici. Il filtro irremovibile dei nostri sensi ci costringe nella posizione di spettatori dei soli effetti che gli enti fondamentali della natura producono. Una teoria fisica tenta, assumendo una certa interpretazione della realtà, di risalire alla forma matematica delle interazioni a partire dalle misure. In questo senso possiamo dire che la fisica è la ricerca di una causa a partire da un effetto. Ed è esattamente questa operazione che ci forza a postulare.\\
	Immaginatevi d'essere un medico e di somministrare un medicinale ad un paziente. Il medicinale produrrà un certo effetto, misurabile sperimentalmente. Sarà ragionevole assumere che le modificazioni sulla salute del paziente saranno da imputare al medicinale, in quanto il mutamento delle prime è comparso in stretta relazione al secondo. Per cui, in questo primo caso saremo certi (con un ragionevole margine di errore, per i più pignoli), che la causa prima delle misure sarà il medicinale somministrato.\\
	Ora immaginatevi di poter conoscere solamente gli effetti e le modificazioni che il paziente presenta; ma di non sapere se esso è un uomo o un altro animale, di non conoscere il suo completo stato di salute, nè la sua anatomia, di non sapere nè se ad esso sono stati somministrati medicinali nè, in caso affermativo, quanto tempo addietro è avvenuta la somministrazione, e così via. Come indicare la causa prima del suo stato? Ancora di più: Supponete di non poter toccare il paziente, ma di poterlo solamente guardare attraverso un cannocchiale, da molto molto lontano. Supponete d'essere come un uomo delle caverne e di non aver le nozioni di ossa, di globulo rosso, di medicina e così via.
	Sareste in grado di indicare la causa prima dello stato del paziente?\\
	Immaginate ancora di trovarvi all'interno di una stanza ammobiliata e d'osservar le sedie, il tavolo, le cassettiere e tutto il resto muoversi a piacimento, come fossero infestate. Sareste in grado di indicare la causa di questi eventi? \\
	Queste comiche situazioni esemplificano lo stato del fisico di fronte alla natura. L'impossibilità umana di trascendere i propri sensi nella realizzazione di misure sperimentali, unita alla necessità di trovarne la causa, costringe ad inventare spiegazioni ragionevoli (ma non falsificabili) attraverso l'atto di postulare. \\
	Più chiaramente: noi crediamo che ogni interpretazione segua come conseguenza necessaria dalla ricerca della causa, dal maldestro tentativo di risalirvi a partire dai soli effetti, filtrati dai nostri sensi.\\
	\\
	La comprensione della profonda limitatezza che le misure sperimentali ereditano dalla nostra natura umana è fondamentale. Immaginate di voler \textit{descrivere} il moto di un corpo rigido in un laboratorio. Saremmo indotti a pensare che ciò sia facilmente realizzabile attraverso un sistema di orologi e metri, così da mappare alla perfezione la posizione, in relazione al tempo, del nostro corpo. Tuttavia, ciò che noi osserviamo, attraverso l'uso dei nostri occhi, è solamente l'effettivo movimento del corpo e non, come nel caso della meccanica newtoniana, il manifestarsi delle forze. \\
	Considerate ora di voler osservare al microscopio atomico un determinato materiale. Il microscopio vi restituirà un'immagine in cui sarà possibile per voi distinguere con precisione l'estensione dei singoli atomi. Tuttavia, questo strumento fornisce solamente una mappatura delle sue interazioni elettromagnetiche con i singoli. \`E chiaro dunque che  le particelle rimarranno invisibili, mascherate dalle interazioni.\\
	Noi crediamo che ogni tipo di misura si traduca inevitabilmente ad una pura interazione e che dunque fallisca nel cogliere l'essenza fondamentale della natura.\\
	\\ 
	Noi stimiamo vitale comprendere come l'impossibilità di descrivere la natura non si traduca in una negazione di essa. Proposizioni come \enquote{gli elettroni non esistono} sono ovviamente prive di senso, poichè le nostre misure indicano che innegabilmente qualche cosa c'è. Noi affermiamo che ciò che è non possa essere descritto, nonostante possa venir avvertito.
	\subsection{Specchietto matematico}
	Matematicizzando i risultati del paragrago precedente, considerando $E$ come l'insieme degli effetti e $C$ come quello delle cause, la relazione $f:C\rightarrow E$ che sussiste tra essi non è necessariamente iniettiva. Secondo il pensiero Interpretazionista, il fisico cerca di invertire questa relazione a partire da $E$, e postulare equivale ad escludere certi punti dall'insieme $C$.
 
	\section{Le interazioni}
	La nostra capacità di studiare i fenomeni naturali nasce dal fatto che, per l'appunto, questi sono fenomeni, ovvero interazioni. Si è scritto che ogni misura sperimentale corrisponde ad un'interazione. Segue allora come necessario requisito che la natura interagisca con sè stessa (con i nostri strumenti di misura in particolare). Supponete, per gioco, che esista un ente fondamentale della natura che non interagisca con il resto di essa, nè con sè stesso: una particella fantasma. Ebbene ne conseguirà che sarà impossibile per noi accorgerci di questo oggetto, interpretarne la natura e studiarne i comportamenti, poiché non vi sarebbe alcun modo di misurarli. Ponendoci pressoché in linea con il pensiero espresso da Rovelli in Helgoland: all'interno di una determinata interpretazione \enquote{le caratteristiche di un oggetto sono il modo in cui esso agisce sugli altri oggetti. L'oggetto non è che un insieme di interazioni su altri oggetti}. Sostanzialmente un qualsiasi ente che non interagisca con il sistema fisico in esame, all'interno della teoria assunta, non ha alcuna proprietà: è equivalente a qualcosa che non esiste. \\
	Supponiamo ora di possedere una teoria del tutto, ovvero che fornisca un'interpretazione e una serie di principi in completo accordo con la totalità dei fenomeni naturali. Noi crediamo che questa teoria, seppur perfettamente corroborata, non descriverebbe la realtà, ma ne fornirebbe solamente un'interpretazione. Questo poiché, oltre alla non falsificabilità dell'interpretazione, vi potrebbe essere una parte di natura che semplicemente non interagisce. Essa necessariamente sfuggirebbe a qualsiasi interpretazione, costretta sotto un inviolabile plumbeo velo di Maya. Tuttavia, non avrebbe alcun senso considerare come esistente questa parte fantasma. Il punto è che, supponendo che essa per natura si sottragga a qualsiasi tipo di interazione, le stiamo automaticamente conferendo una accezione divina. Se anche possedessimo i sensi di Dio, non saremmo comunque in grado di rilevare questa parte di realtà, poichè per definizione essa andrebbe oltre i sensi stessi.\\
    Inoltre, a causa dei limiti intrinseci dei nostri apparati sperimentali, sarebbe impossibile ottenere una descrizione della realtà anche soltanto considerando ciò che può interagire con essi.\\ 
	\\
	\enquote{If you can't plot it, it doesn't exist} -R.C.
	\section{Il problema del Vero}
	Se assumiamo come postulato che la realtà possa solo essere interpretata ma non descritta in alcun modo a livello fondamentale, stiamo implicitamente distruggendo la nozione di verità. \`E dunque richiesta  una nuova formulazione di questo concetto, la quale deve non solo accordarsi al pensiero Interpretazionista, ma anche ridursi con le giuste approssimazioni alla verità intesa come universale.
	Chiarifichiamo meglio quest'ultimo concetto con la seguente definizione:\\\\
	\textbf{Def:}\\
	Una proposizione è detta Universalmente Vera se è in accordo con la descrizione della realtà.\\
	\\
	Il punto principale dell'Interpretazionismo è l'impossibilità di formulare qualsiasi proposizione Universalmente Vera, come conseguenza della non-descrivibilità della realtà. Tuttavia, per ragioni logiche, assumiamo che una descrizione fondamentale di essa esista, anche se intoccabile. Intuitivamente, la nostra assunzione è ragionevole dato che la realtà ovviamente esiste ed è unica e dunque in qualche modo deve \enquote{essere}. Rifiutiamo insomma, senza preoccuparcene troppo, la gabbia del cogito di Cartesio. Per giustificare la supposizione circa l'unità della natura citiamo ancora una volta Poincaré: \enquote{se le diverse parti dell'universo non fossero come gli organi di uno stesso corpo, non agirebbero le une sulle altre e si ignorerebbero a vicenda}.\\
	Preoccupiamoci ora di definire la nozione di Verità Interpretazionista. La nostra definizione sarà semplicemente:\\
	\\
	\textbf{Def:}\\
	Una proposizione è detta Vera se è corroborata all'interno dell'apparato fisico a cui appartiene ed attraverso cui si tenta di falsificarla.\\
	\\
	Ad una prima occhiata, questa formulazione pare tutt'altro che banale. Al fine di comprenderne la stesura, consideriamo alcune asserzioni e procediamo ad analizzarle servendoci della nostra attuale \enquote{conoscenza} della natura.\\
	La proposizione \enquote{Il cielo è blu} è ovviamente vera, dato che attraverso l'uso unico dei nostri occhi è facile da corroborare. Ciononostante, il grado di verità di questa semplice asserzione è strettamente legato all'apparato fisico con cui conduciamo la misura: i nostri occhi. \`E risaputo infatti che la quantità di radiazione che riceviamo dal cielo si estende al di là del solo spettro visibile (senza considerare la continua pioggia di particelle che ci investono costantemente). Dunque, la proposizione sopra riportata risulterebbe ovviamente falsa se analizzata attraverso strumenti e teorie più precise. Tuttavia, noi diciamo che il cielo è blu perché implicitamente assumiamo il solo uso della nostra vista. Il punto di questa analisi esageratamente puntigliosa non è solamente mostrare come l'apparato fisico influenzi la nozione del vero, ma anche renderci consapevoli delle innumerevoli e ragionevoli approssimazioni che tutti noi operiamo.\\
	\\
	Consideriamo ora un esempio più specifico: la proposizione \enquote{gli atomi sono piccole sfere} è vera, classicamente, ma estremamente falsa in fisica quantistica. Tuttavia, dipendentemente dall'esperimento che conduciamo, siamo portati ad interpretare la realtà in un modo o nell'altro e, basandoci sull'interpretazione che scegliamo, la precedente proposizione può esser considerata o vera o falsa. Se ad esempio volessimo calcolare il libero cammino medio di un gas di molecole sufficientemente semplici, potremmo trattarle come sfere ed il risultato che otterremmo sarebbe valido entro certi regimi di approssimazione sorprendentemente larghi.\\
	Ritorniamo ora alla definizione di Verità Interpretazionista: la proposizione \enquote{gli atomi sono piccole sfere} risulta particolarmente utile per comprendere la necessità di includere l'appartenenza ad un certo apparato fisico nella definizione. Infatti, questa stessa affermazione sarebbe priva di senso in teoria dei campi quantizzati, dove l'ente fondamentale della teoria è il campo e le particelle sono viste come sole eccitazioni di esso. Come onde, in un mare che si estende per tutto lo spazio-tempo.\\
	Gli scorsi esempi aiutano a comprendere come la Verità Interpretazionista (alla quale ci riferiremo come verità se non altrimenti specificato) sia una semplice conseguenza delle nostre interpretazioni della natura. Ciò non vuol ovviamente significare che la verità differisca sempre da quella universale, ma la risposta a questo problema è semplicemente inarrivabile e di conseguenza non ha alcun senso perseguirla.\\
	\\
	Sapendo che la madre delle persone senza intelletto è in costante dolce attesa, è ovvia la necessità di chiarificare il precedente paragrafo con ulteriori esempi. Proposizioni come \enquote{la terra è piatta} sono false (non vere), in quanto sono in disaccordo con qualsiasi teoria fisica corroborata. Proposizioni con un grado tanto esteso di falsità possono essere (e devono essere in qualsiasi ragionamento che non sia astratto al massimo grado) trattate come universalmente false (o vere).
	Il punto fondamentale è che non basta una teoria fisica nuova per rendere una qualsiasi allucinazione veritiera: è necessario che la teoria stessa sia corroborata, ovvero sia sottoposta a falsificazione e riproduca con sufficiente precisione i risultati misurati. Allo stesso modo, non è lecito asserire che una proposizione corroborata sia falsa appigliandosi all'argomentazione che potrebbe esistere una teoria che la falsifica, se questa non esiste già. Questo punto è di fondamentale importanza.		
	Si invita caldamente chiunque non abbia chiaro il significato dell'ultimo paragrafo a rileggere le definizioni di verità.\\
	Tutti coloro che interpretano questo manifesto come una giustificazione di una qualsiasi teoria pseudo-scientifica, che non abbia basi logico-matematiche e fisiche corroborate da innumerevoli esperimenti, non hanno compreso nulla di ciò che è riportato all'interno di questo scritto.
	\section{Approssimazioni e previsioni}
	Nel corso di queste pagine v'è stata sostanzialmente una totale demolizione del concetto di verità, un annichilimento della nostra presunzione riguardo la natura. Tuttavia, questi astrusi ragionamenti non hanno come mira quella di instillare il dubbio, nè di criticare puntigliosamente ogni affermazione ovvia. Al contrario, questo manifesto vuol sia essere un modo di ricordarci quanto, nonostante la nostra incapacità di trascendere i cinque sensi, siamo arrivati lontano; sia essere un mezzo che impedisca di fossilizzarci su una sola interpretazione. Una qualsiasi teoria che fornisca risultati soddisfacenti in determinati regimi non è da rigettare una volta falsificata. La relatività generale spiega in modo eccellente fenomeni gravitazionali su larga scala, ma non si accorda con teoria dei campi quantizzati nello studio delle interazioni tra particelle. Allo stesso modo, la meccanica newtoniana è perfettamente adatta per analizzare problemi meccanici che coinvolgono corpi macroscopici, ma non è in accordo con la quantistica.\\
	Non avrebbe alcun senso applicare una relatività ristretta per studiare il moto non relativistico di una palla da biliardo, perché la meccanica newtoniana è già sufficientemente precisa nel trattare fenomeni come questo. Approssimare, semplificare, quindi scegliere un'interpretazione al posto di un'altra, è spesso e volentieri non solo ragionevole, ma necessario.\\
	\\
	In conclusione a questo manifesto, noi vogliamo ricordare come nonostante tutte le teorie fisiche non siano descrittive, esse sono pur sempre in grado di far previsioni sul comportamento della natura.
	Si può dire che una delle caratteristiche fondamentali di una teoria fisica è la sua capacità di prevedere gli esiti di un esperimento. Viene dunque naturale chiedersi come un'interpretazione possa, pur non conformandosi alla descrizione fondamentale della realtà, fare previsioni su di essa. Il punto chiave è che le ipotesi sui risultati di un esperimento riguardano gli esiti delle interazioni tra gli enti fondamentali della natura. Una previsione è associata ad una misura che, come mostrato in precedenza, è inevitabilmente filtrata dai nostri sensi.
	Un qualsiasi esperimento X verrà condotto presupponendo una teoria (e quindi un'interpretazione della realtà) Y, la quale sarà in grado di fare previsioni sulle misure sperimentali.\\
 Noi facciamo previsioni sulle interazioni, non sull'essenza della natura.\\
	Per fare un esempio: un pirata che volesse sparare una palla di cannone sarà in grado di calcolare, attraverso la meccanica classica, l'inclinazione corretta affinché il suo proiettile segua, per merito della supposta forza di gravità, la traiettoria che lo porterà al bersaglio. Dunque l'interpretazione della forza di gravità intesa come vettore, permetterà al pirata di affondare le navi nemiche.   
	\\
	\\
	Allora per costruire il circuito di un qualsiasi apparecchio elettronico si utilizzeranno le leggi dell'elettromagnetismo, ed esso funzionerà. Per descrivere il moto di un satellite si useranno la relatività e la meccanica, e la traiettoria del satellite si accorderà con quella prevista. Per progettare un aereo si farà uso di fluidodinamica, e l'aereo volerà.  
	L'impossibilità di descrivere la realtà non impedisce di fare previsioni su di essa.
	\section{Paradossi}
	\subsection{Il Vero}
	Come possiamo dire che sia falso che la realtà non può essere descritta, se definiamo il vero in relazione alla teoria fisica?\\
	Il paradosso è risolto poichè noi assumiamo, in modo del tutto ragionevole ma arbitrario, che la realtà non possa essere descritta.
	\subsection{La Falsificazione}
	Il principio di falsificazione asserisce che una teoria deve essere falsificabile. Tuttavia si è dimostrato nel corso del manifesto che ogni teoria fisica assume postulati non falsificabili. Questa è una evidente contraddizione.\\
	Si può risolvere questo paradosso asserendo che la falsificazione concerne solamente le effettive misure sperimentali, ovvero una teoria è scientifica se, eccettuata la sovrastruttura matematico-fisica dei postulati, è falsificabile.
	\subsection{L'Unità}
	Abbiamo supposto che la realtà sia unica, tuttavia potremmo supporre che essa sia molteplice, ma che tutte le parti eccetto per una non interagiscano tra loro.
	Il paradosso si risolve constatando che, poiché una parte sola della natura interagisce con sé stessa, le nostre interpretazioni si rivolgono solamente ad essa, che dunque appare unica.
	\subsection{Il Principio di Minima Azione}
	Possiamo considerare il principio di minima azione come uno dei pilastri fondamentali della fisica. Molto brevemente, banalizzando, esso asserisce che ogni processo fisico avviene minimizzando una specifica quantità, chiamata azione. Non ci soffermeremo sulla natura di questa grandezza, bensì sulla problematicità dell'applicazione del principio.\\
	La dinamica di ogni sistema fisico è contenuta all'interno di specifiche equazioni, chiamate equazioni del moto. Queste ultime sono determinabili attraverso la minimiazzazione dell'azione del sistema in esame. Sorprendentemente, questa strategia meccanica viene applicata in numerosissimi campi della fisica e pare ritrovare in ognuno di essi una forte validità. Proprio per questo risulta naturale lasciarsi cullare dal suddetto principio ed innalzarlo a legge universale.\\
	Tuttavia spesso e volentieri le equazioni del moto sono già conosciute in anticipo ed è l'azione ad essere costuruita in modo da accordarsi ad esse. Sostanzialmente è molto gradito ai fisici contemporanei far discendere le relazioni chiave di un sistema fisico da una determinata azione, quasi dimenticandosi di come essa venga costruita ad hoc.\\
	\\
	Il problema che rappresenta per noi il principio di minima azione è che esso si pone, per certi versi, a metà tra un principio ed un postulato. La sua natura ed il suo nome paiono appartenere alla prima categoria, ma il modo in cui viene applicato ne rende impossibile la falsificazione.
	\\\\
	Consideriamo un elettrone ed analizziamolo in meccanica classica. Esso viene trattato come un corpo carico puntiforme e dotato di massa. 
	Ricaviamo le sue equazioni del moto classiche da una determinata azione. Tuttavia, non è possibile trovarne una che conduca a delle soluzioni quantistiche. Si proverà allora a descrivere l'elettrone con la meccanica quantistica, associandovi una funzione d'onda e così via fino a teoria dei campi. In tutte queste teorie, ritroveremo azioni sempre diverse.\\
	\\
	Veniamo ora al nòcciolo del problema: la minima azione è falsificabile? Da una parte si potrà argomentare che si è appena eseguito un processo di falsificazione; in quanto abbiamo verificato che il nostro principio conduce alle equazioni attese, a prescindere dalla loro correttezza.
	Tuttavia si potrebbe sostenere che la falsificazione del principio non è avvenuta poiché, in sostanza, cambiando teoria abbiamo cambiato l'azione e lasciato invariato il principio. Ciò che possiamo falsificare sono le equazioni del moto ma non il fatto che esse possano essere ricavate dalla minimizzazione di un'azione, in quanto possiamo sempre procedere a ritroso e trovarne una che si accordi con ciò che è atteso.\\ 
	\\
	Ma allora dove risiede la differenza tra minima azione ed altri principi, come ad esempio la conservazione dell'energia?\\
	Supponiamo un sistema isolato. Termodinamicamente, asseriamo che l'energia di questo deve rimanere costante. Questo è un principio in quanto è idealmente verificabile sperimentalmente attraverso il calcolo dell'energia. La minima azione invece pare sfuggire a questa verifica, data la totale arbitrarietà dell'azione, scelta apposta per \enquote{far tornare i conti}.\\
	\\
	Sostanzialmente, la minima azione è falsificabile in quanto si riduce ad una computazione, ma non falsificabile in quanto è sempre possibile trovare un'azione che la renda corretta.
	
	\bigskip 
	\begin{flushright}
		Nove, Dan
	\end{flushright}
\end{document}
