\documentclass[]{article}
\usepackage{amsmath,amssymb}
\usepackage{amsthm}
\usepackage{geometry}

\begin{document}
	\section{Il Problema dell'Interpretazione}
	Spesso si tende a dimenticare, studiando, che l'obbiettivo ultimo della fisica è la pura comprensione della realtà. In questo senso non esiste nulla di più reale della fisica, nonostante la sua meravigliosa componente astratta. Spesso si tende anche a dimenticare che le molteplici interpretazioni che la Fisica offre della realtà non sono altro che, per l'appunto, interpretazioni.  
	\\
	Il problema che verrà sviscerato all'interno di questo documento può essere sintetizzato semplicemente nella questione: "E se la realtà non potesse essere descritta?". In particolare, verrà totalmente abbandonata l'idea che la realtà possa essere in alcun modo descritta a livello fondamentale. Assumeremo, ragionevolmente e con le giuste motivazioni, di poter solamente trovare interpretazioni più o meno soddisfacenti di essa. Chiameremo dunque questo pensiero Interpretazionismo e cercheremo di definirne i tratti pricipali.\\
		Prima di partire, è doveroso sottolineare che le considerazioni esposte all'interno di questo manifesto riguardano solamente la fisica. Non è escluso che tutti i risultati che si otterranno potranno essere estesi a scienze meno "pure" (pure inteso come astratte e riguardanti le strutture fondamentali del reale), ma ciò non sarà affrontato all'interno di queste pagine. 
		\subsection{Il Principio di Falsificazione}
	Vediamo innanzitutto, in modo semplicistico, come poter definire una teoria scientifica.
	Il meraviglioso Principio di Falsificazione di Popper è il punto di partenza: "Una teoria è scientifica solamente se può essere falsificata". Questa definizione fornisce un metodo (scientifico) per descrivere la realtà ad infinitum, ovvero elaborare teorie sempre più "esatte" nella loro descrizione. Una teoria fisica che si rispetti deve infatti accordarsi sia con le effettive misure sperimentali, che con le sue predecessore, almeno entro certe approssimazioni. Come esempio si consideri che la Relatività Generale riproduce localmente i risultati della Relatività Ristretta%, la quale riproduce, nel regime di basse velocità (rispetto a quella della luce), la Meccanica Classica, e così via%
	. Tuttavia, la prima ci permette di spiegare fenomeni che la seconda non vede; per questo diciamo che la Relatività Generale è più esatta della Relatività Ristretta. \\
	Ritornando al principio, noi crediamo che la vera magnificenza di questo risieda nella spinta evolutiva che esso conferisce ad ogni disciplina scientifica: ogni teoria può essere corroborata o smentita e dunque aggiornata, migliorata, in modo da approssimarsi sempre di più alla corretta comprensione di ciò che è reale. 
	Risulta tuttavia problematico utilizzare questo principio per asserire che la realtà non possa essere descritta. In particolare, l'Interpretazionismo non consegue dalla falsificabilità, ma dalla limitatezza intrinseca umana; ovvero dalla necessita insormontabile per ogni misura sperimentale di passare attraverso il filtro dei nostri sensi.  
	
	%La seconda si ricollega al nostro quesito iniziale: essendo ogni teoria in linea di principio falsificabile, non si avrà mai la certezza di aver trovato una interpretazione che si accordi esattamente con la descrizione della realtà. Ma attenzione, vi è una sostanziale differenza con l'asserire che essa non possa essere descritta!% 
	\subsection{L'Interpretazionismo fisico}

	Ogni teoria fisica si basa su di un preciso apparato matematico, estremamente rigoroso ed infallibile. Al di sopra di questo apparato, tramite degli assunti completamente arbitrari, viene costruita una teoria fisica. Ciò che è importante interiorizzare è la profonda separazione tra questi due fattori: ad oggi non è possibile interpretare la realtà a partire dalla sola matematica; sono sempre necessari principi totalmente ingiustificati la cui validità è assunta universalmente.\\
	Per quanto riguarda gli assunti di una teoria fisica, noi crediamo che essi si suddividano in due categorie, che chiameremo postulati e principi.\\
	\\
	\textbf{Def:}\\
	Chiamiamo Postulato un'assunzione sull'essenza fondamentale della natura, formalizzata attraverso una rigorosa rappresentazione matematica. I postulati corrispondono all'interpretazione della realtà su cui la teoria è costruita.
	\\
	\\
	\textbf{Def:}\\
	Chiamiamo Principio una qualsiasi assunzione su cui si basa la teoria in esame, che deve poter venir corroborato dalle misure sperimentali.
	\\
	\\
	Al fine di comprendere totalmente le definizioni precedenti, consideriamo un semplice esempio: la Meccanica (Classica) Newtoniana.\\
	Questa teoria usufruisce principalmente di due sovrastrutture matematiche chiamate Calcolo Vettoriale ed Analisi Matematica. Molto brevemente, l'Analisi Matematica si occupa di descrivere il comportamento di relazioni (funzioni) tra insiemi di numeri Reali; mentre il Calcolo Vettoriale si occupa di descrivere il comportamento dei cosiddetti vettori. Formalmente, un vettore è un elemento astratto di uno Spazio Vettoriale, ovvero un insieme su cui sono definite certe operazioni e proprietà. Per fare un esempio, $\mathbb{R}$ con la somma che conosciamo tutti è uno spazio vettoriale. L'apparato matematico ci fornisce l'impalcatura su cui muoversi: lo spazio che ci circonda è riprodotto attraverso sistemi di riferimento costruiti con 3 assi cartesiani ortogonali tra loro. L'insieme dei punti dello spazio costruiti in questo modo è uno spazio vettoriale. Dunque potete immaginare i vettori nello spazio come punti o frecce (coppie di punti). \\
	I principi su cui si basa la meccanica newtoniana sono essenzialmente 3. Tuttavia, è necessario prima di tutto assumere la presenza delle forze. Assumiamo che un'interazione tra due corpi si manifesti sotto forma di una forza, ovvero un vettore (una freccia nello spazio (tangente)) a cui possiamo associare delle precise quantità, come la direzione e l'intensità.		
	Possiamo facilmente elencare alcuni dei postulati su cui si basa la teoria in esame:
	\\\\
	\textbf{Postulato:}\\
	Le interazioni tra corpi si manifestano attraverso forze.
	\\\\
	\textbf{Postulato:}\\
	L'universo è in 3 dimensioni.
	\\\\
	Il primo postulato è formalizzabile matematicamente attraverso il calcolo vettoriale, mentre il secondo si traduce nella rappresentazione dello spazio attraverso gli assi cartesiani.
	Ritornando ora ai principi:\\
	Il primo principio di Newton recita: 
	\\\\
	\textbf{Principio:}\\
	Se la somma delle forze agenti su di un corpo è nulla allora esso ha accelerazione nulla.
	\\\\ 
	Sostanzialmente, se questa condizione è verificata, o il corpo è fermo o si muove a velocità costante. Il secondo principio recita:
	\\\\
	\textbf{Principio:}\\
	La somma delle forze agenti su di un corpo è proporzionale all'accelerazione del corpo stesso, tramite la massa (inerziale), che può essere tradotta come $\sum\vec{F}=m\vec{a}$.
	\\\\
	Il terzo principio recita: \\\\
	\textbf{Principio:}\\
	Se un corpo C esercita una forza su di un corpo H allora il corpo H esercita una forza uguale e contraria su C".
	\\\\
	A partire da questi principi totalmente arbitrari idealmente non solo si potrebbe ricostruire la dinamica di un corpo rigido conoscendo un numero sufficiente di parametri, ma si potrebbe anche prevedere la sua futura evoluzione. Idealmente, secondo la meccanica newtoniana, conoscendo un numero sufficientemente elevato di informazioni sarebbe possibile ricostruire il tragitto di una particella dal momento del big bang fino ad oggi. Oggi sappiamo che questa teoria si accorda con i risultati sperimentali solo entro certe approssimazioni.\\
	In particolare, i problemi (alcuni) iniziano a manifestarsi quando i corpi analizzati si muovono "troppo velocemente". Sostanzialmente, pare sperimentalmente che la velocità massima per un corpo sia $c$, la velocità della luce. Questo distrugge completamente la meccanica newtoniana, rendendola semplicemente una teoria non corroborata universalmente.\\
	Da questo esempio è chiaro come i principi di una teoria siano sottoponibili a falsificazione: falsificare la teoria corrisponde a falsificarne i principi. Questa operazione non è chiaramente possibile con i postulati, in quanto essi, concernendo l'essenza stessa della natura, sfuggono alle nostre misure sperimentali: falsificare i principi fisici non falsifica necessariamente l'interpretazione della natura che la teoria presuppone. Conseguentemente, essendo le nostre supposizioni circa l'essenza della realtà non falsificabili, non possiamo avere la certezza che esse siano effettive descrizioni.  
	\\
	In sostanza uno dei problemi della meccanica newtoniana è che essa postula l'esistenza di forze, veri e propri fantasmi che si manifestano come motori del movimento dei corpi. Tuttavia, noi misuriamo solamente il movimento e le interazioni tra questi. Noi diciamo che le forze "esistono" e sono la causa degli spostamenti semplicemente perché la teoria funziona. Nulla ci garantisce che spostando la matita con il dito vi siano forze rappresentate da frecce che si manifestino. Esse non sono altro che un puro costrutto fisico-matematico non falsificabile dai nostri esperimenti. Fisico in quanto concerne l'essenza della realtà, matematico in quanto rappresentabile matematicamente. E' importante sottolineare che nonostante sia matematicamente possibile calcolare il verso, la direzione e l'intensità di una serie di forze (e quindi definirle in tutto e per tutto), questo non cambia la non falsificabilità della loro essenza.\\
	Dunque, paradossalmente, ogni teoria fisica è costituita da una parte di assunzioni non "scientifiche" in senso stretto, ovvero impossibili da verificare sperimentalmente. Ciò che conta non è l'impalcatura su cui si basa la nostra teoria, bensì i risultati a cui essa conduce, in quanto la prima rimarrà sempre costretta sotto il velo di Maya della falsificazione.\\
	Il nodo centrale di questo problema è che ogni teoria fisica è inseparabile dalle effettive misure sperimentali, in quanto esse sono l'unico processo che ci permette di mettere in atto il meccanismo di falsificazione. Una misura non è altro che un'osservazione di un fenomeno naturale (fisico) attraverso il filtro dei nostri sensi; i quali sono estremamente limitati. Di conseguenza, tanto più complesse risultano essere le effettive misure, tanto più astratta sarà la teoria. \\
	\\
	Si consideri ora un esempio lievemente più complicato: lo spin.
	In meccanica classica esiste una grandezza, chiamata momento angolare, che ci permette di manipolare i corpi in rotazione. In generale, il lettore inesperto può banalizzare il momento angolare come una quantità che indica "quanto il corpo sta ruotando".\\
	In meccanica quantistica ad ogni particella è associata una proprietà fisica chiamata spin. Lo spin degli elettroni è (all'incirca) la causa a cui associamo le interazioni che questi presentano con un campo magnetico esterno. Le equazioni che si ritrovano nello studio di queste interazioni presentano una struttura sorprendentemente simile a quelle del momento angolare. Per questo tendiamo ad immaginare, classicamente, un elettrone dotato di spin come una piccola sfera in rotazione su sé stessa. Tuttavia non ci è dato "vedere" attraverso degli strumenti un elettrone roteare. Tutto ciò che misuriamo sono le interazioni che esso ha con i nostri rivelatori. Classicamente, questa interpretazione dello spin è estremamente funzionale, tuttavia, per le ragioni di cui sopra, non è altro che, per l'appunto, un'interpretazione. \\
	\\
	La mira dell'Interpretazionismo non è screditare l'approccio semplicistico e diretto della fisica, in quanto spesso e volentieri è quello più corretto, ma di ricordare a noi stessi di non intrappolarci nell'idea che la realtà sia l'esatta copia di teorie umane, troppo umane.  Il rischio che si corre è quello di dimenticare la sostanziale differenza tra interpretazione e descrizione della realtà, perdendoci nei meandri della teoria.\\
	\\
	Consideriamo infine Teoria dei Campi Quantizzati. Questa meravigliosa teoria ci permette di analizzare le interazioni tra le particelle che compongono l'universo. L'ente fondamentale è il campo, un'entità fisica che permea tutto lo spazio tempo, come fosse un mare. L'esistenza del campo non è falsificabile, in quanto sfugge alle nostre misure, che si limitano alle banali interazioni tra quelle che supponiamo essere eccitazioni del campo. Ancora una volta, la sola misurazione di queste stesse interazioni non ci assicura che la nostra interpretazione corrisponda alla realtà. \\
	\\
	In conclusione, il punto chiave della fisica è l'assunzione di verità indimostrabili che definiscono essa stessa. L'arbitrarietà di questi postulati rende intangibile la certezza che essi siano effettivamente descrittivi, in senso stretto, della realtà. Seguendo questo ragionamento noi diciamo che "ciò che è reale non può essere descritto" riferendoci alla totale soggettività e non-falsificabilità degli assunti con cui tentiamo per l'appunto di descrivere la natura. Ovviamente, nulla esclude che esistano degli assiomi "corretti", ma questa loro accezione di correttezza rimane (per ora) semplicemente indimostrabile e, dunque, priva di alcun valore. La realtà è interpretabile, ma non descrivibile.
	
	\section{La Causa}
	Da dove proviene questa insormontabile necessità di postulare? Quali sono le ragioni per cui siamo costretti ad inventare assunzioni affinché le nostre teorie fisiche funzionino? \\
	Per trovare una risposta a questi interrogativi è necessario analizzare la relazione tra fisica teorica e sperimentale. Il fine di una qualsiasi teoria è quello di accordarsi il più precisamente possibile con le effettive misure. Queste, tuttavia, sono il risultato di interazioni tra enti fisici. Il filtro irremovibile dei nostri sensi ci costringe nella posizione di spettatori dei soli effetti che gli enti fondamentali della natura producono. Una teoria fisica tenta, assumendo una certa interpretazione della realtà, di risalire alla forma matematica delle interazioni a partire dalle misure. In questo senso possiamo dire che la fisica è la ricerca di una causa a partire da un effetto. Ed è esattamente questa operazione che ci forza a postulare.\\
	Immaginatevi d'essere un medico e di somministrare un medicinale ad un paziente. Il medicinale produrrà un certo effetto, misurabile sperimentalmente. Sarà ragionevole assumere che le modificazioni sulla salute del paziente saranno da imputare al medicinale, in quanto il mutamento delle prime è comparso in stretta relazione al secondo. Per cui, in questo primo caso saremo certi (con un ragionevole margine di errore, per i più pignoli), che la causa prima delle misure sarà il medicinale somministrato.\\
	Ora immaginatevi di poter conoscere solamente gli effetti e le modificazioni che il paziente presenta; ma di non sapere se esso è un uomo o un altro animale, di non conoscere il suo completo stato di salute, nè la sua anatomia, di non sapere nè se ad esso sono stati somministrati medicinali nè, in caso affermativo, quanto tempo addietro è avvenuta la somministrazione, e così via. Come indicare la causa prima del suo stato? Ancora di più: Supponete di non poter toccare il paziente, ma di poterlo solamente guardare attraverso un cannocchiale, da molto molto lontano. Supponete d'essere come un uomo delle caverne e di non aver le nozioni di ossa, di globulo rosso, di medicina e così via.
	Sareste in grado di indicare la causa prima dello stato del paziente?\\
	Questa comica situazione esemplifica lo stato del fisico di fronte alla natura.  L'impossibilità umana di trascendere i propri sensi nella realizzazione di misure sperimentali, unita alla necessità di trovare la causa di queste, costringe ad inventare spiegazioni ragionevoli (ma non falsificabili) attraverso l'atto di postulare. Noi crediamo che questo stesso atto segua come conseguenza necessaria dalla ricerca della causa. Più precisamente, dal maldestro tentativo di risalire alla causa a partire dai soli effetti, filtrati dai nostri sensi.\\
	\\
	Noi stimiamo fondamentale la comprensione della profonda limitatezza che le misure sperimentali ereditano dalla nostra natura umana. Immaginate di voler descrivere il moto di un corpo rigido in un laboratorio. Ciò è facilmente realizzabile attraverso un sistema di orologi e metri, così da mappare alla perfezione la posizione, in relazione al tempo, del nostro corpo. Tuttavia, ciò che noi osserviamo, attraverso l'uso ei nostri occhi, è solamente l'effettivo movimento del corpo e non, come nel caso della meccanica newtoniana, il manifestarsi delle forze. \\
	Considerate ora di voler osservare al microscopio atomico un determinato materiale. Il microscopio vi restituirà un'immagine in cui sarà possibile per voi distinguere con precisione l'estensione dei singoli atomi. Ciononostante è di vitale importanza accorgersi che ciò che il microscopio mappa sono solamente le interazioni elettromagnetiche con i singoli.\\
	Noi crediamo che ogni tipo di misura si traduca inevitabilmente ad una pura interazione e dunque fallisca nel cogliere l'essenza fondamentale della natura.\\
	\\ 
	Noi stimiamo vitale comprendere come l'impossibilità di descrivere la natura non si traduca in una negazione di essa. Proposizioni come "gli elettroni non esistono" sono ovviamente prive di senso, poichè le nostre misure indicano che effettivamente qualche cosa c'è, innegabilmente. Noi affermiamo che ciò che è, nonostante possa essere avvertito, non può venir descritto. Tuttavia, il problema del vero verrà trattato nella prossima sezione.
	\subsection{Specchietto matematico}
	Matematicizzando i risultati del paragrago precedente, considerando $E$ come l'insieme degli effetti e $C$ come quello delle cause, la relazione $f:C\rightarrow E$ che sussiste tra essi non è necessariamente iniettiva. Secondo il pensiero Interpretazionista, il fisico cerca di invertire questa relazione a partire da $E$, e postulare equivale ad escludere certi punti dall'insieme $C$.
	\section{Il problema del Vero}
	Se assumiamo come postulato che la realtà possa solo essere interpretata ma non descritta in alcun modo, per lo meno a livello fondamentale, stiamo implicitamente distruggendo la nozione di Verità. E' dunque richiesta  una nuova formulazione di questo concetto, la quale deve non solo accordarsi al pensiero Interpretazionista, ma anche ridursi con le giuste approssimazioni alla verità intesa come universale.
	Chiarifichiamo meglio quest'ultimo concetto con la seguente definizione:\\\\
	\textbf{Def:}\\
	Una proposizione è detta Universalmente Vera se è in accordo con la descrizione della realtà.\\
	\\
	Il punto principale dell'Interpretazionismo è l'impossibilità di formulare qualsiasi proposizione Universalmente Vera, come conseguenza della non-descrivibilità della realtà. Tuttavia, per ragioni logiche, assumiamo che una descrizione fondamentale di essa esista, anche se intoccabile. Intuitivamente, la nostra assunzione è ragionevole dato che la realtà ovviamente esiste e dunque in qualche modo deve "essere". \\
	Preoccupiamoci ora di definire la nozione di Verità Interpretazionista. La nostra definizione sarà semplicemente:\\
	\\
	\textbf{Def:}\\
	Una proposizione è detta Vera se è corroborata all'interno dell'apparato fisico a cui appartiene ed attraverso cui si tenta di falsificarla.\\
	\\
	Ad una prima occhiata, questa formulazione pare tutt'altro che banale. Al fine di comprenderne la stesura, consideriamo alcune asserzioni e procediamo ad analizzarle servendoci della nostra attuale "conoscenza" della natura.\\
	La proposizione "Il cielo è blu" è ovviamente vera, dato che, attraverso l'uso unico dei nostri occhi, è facile da corroborare. Ciononostante, il grado di verità di questa semplice asserzione è strettamente legato all'apparato fisico con cui conduciamo la misura: i nostri occhi. E' risaputo infatti che la quantità di radiazione che riceviamo dal cielo si estende al di là del solo spettro visibile (senza considerare la continua pioggia di particelle che ci investono costantemente). Dunque, la proposizione sopra riportata risulterebbe ovviamente falsa se analizzata attraverso strumenti e teorie più precise. Tuttavia, noi diciamo che il cielo è blu perché implicitamente assumiamo il solo uso della nostra vista. Il punto di questa analisi esageratamente puntigliosa non è solamente mostrare come l'apparato fisico influenzi la nozione del vero, ma anche renderci consapevoli delle innumerevoli e ragionevoli approssimazioni che tutti noi operiamo.\\
	\\
	Consideriamo ora un esempio più specifico: la proposizione "gli atomi sono piccole sfere" è vera, classicamente, ma estremamente falsa in fisica quantistica. Tuttavia, dipendentemente dall'esperimento che conduciamo, siamo forzati ad interpretare la realtà in un modo o nell'altro e, basandoci sull'interpretazione che scegliamo, la precedente proposizione può esser considerata o vera o falsa. Se ad esempio volessimo calcolare il libero cammino medio di un gas di molecole arbitrarie, potremmo trattarle come sfere ed il risultato che otterremmo sarebbe valido entro certi regimi di approssimazione sorprendentemente larghi.\\
	Ritorniamo ora alla definizione di Verità Interpretazionista: la proposizione "gli atomi sono piccole sfere" risulta particolarmente utile per comprendere la necessità di includere l'appartenenza ad un certo apparato fisico. Infatti, questa stessa affermazione sarebbe semplicemente priva di senso in Quantum Field Theory, dove l'ente fondamentale della teoria è il campo e le particelle sono viste come sole eccitazioni di esso. Come onde, in un mare che si estende per tutto lo spazio-tempo.\\
	Gli scorsi esempi aiutano a comprendere come la Verità Interpretazionista (alla quale ci riferiremo come Verità se non altrimenti specificato) sia una semplice conseguenza delle nostre interpretazioni della natura. Ciò non vuol ovviamente significare che la verità differisca sempre da quella universale, ma la risposta a questo problema è semplicemente inarrivabile e di conseguenza non ha alcun senso perseguirla.\\
	\\
	Sapendo che la madre delle persone senza intelletto è in costante dolce attesa, è ovvia la necessità di chiarificare il precedente paragrafo con ulteriori esempi. Proposizioni come "la terra è piatta" sono false (non vere), in quanto falsificate da qualsiasi teoria fisica corroborata. Proposizioni con un grado tanto esteso di falsità possono essere (e devono essere in qualsiasi ragionamento che non sia astratto al massimo grado) trattate come universalmente false (o vere).
	Il punto fondamentale è che non basta una teoria fisica nuova per rendere una qualsiasi allucinazione veritiera: è necessario che la teoria stessa sia corroborata, ovvero sia sottoposta a falsificazione e riproduca con sufficiente precisione i risultati misurati. Allo stesso modo, non è lecito asserire che una proposizione corroborata sia falsa appigliandosi all'argomentazione che potrebbe esistere una teoria che la falsifica, se questa non esiste già. Questo punto è di fondamentale importanza.		
	Si invita caldamente chiunque non abbia compreso a fondo il significato dell'ultimo paragrafo a rileggere le definizioni di Verità.\\
	Chiunque veda questo manifesto come una giustificazione di una qualsiasi teoria pseudo-scientifica, che non abbia basi logico-matematiche e fisiche corroborate da innumerevoli esperimenti, non ha compreso nulla di ciò che è riportato all'interno di questo, nè della nostra filosofia.
	\section{La semplicità}
	Nel corso di queste pagine v'è stata sostanzialmente una totale demolizione del concetto di verità universale, un annichilimento della nostra presunzione riguardo la natura. Tuttavia, questi astrusi ragionamenti non hanno come mira quella di instillare il dubbio, nè di criticare puntigliosamente ogni affermazione ovvia. Al contrario, questo manifesto vuol sia essere un modo di ricordarci quanto, nonostante la nostra incapacità di trascendere i cinque sensi, siamo arrivati lontano, sia essere un mezzo che impedisca di fossilizzarci su una sola interpretazione. Una qualsiasi teoria che fornisca risultati soddisfacenti in determinati regimi non è da rigettare una volta falsificata. La Relatività Generale descrive (interpreta) in modo eccellente fenomeni gravitazionali su larga scala, ma non si accorda con Teoria dei Campi Quantizzati nello studio delle interazioni tra particelle. Allo stesso modo, meccanica newtoniana è perfettamente adatta per analizzare problemi meccanici che coinvolgono corpi macroscopici, ma non è in accordo con la quantistica.\\
	Non avrebbe alcun senso applicare una relatività ristretta per studiare il moto non relativistico di una palla da biliardo. Questo perché la meccanica newtoniana è già sufficientemente precisa nel trattare fenomeni come questo. Approssimare, semplificare, quindi scegliere un'interpretazione invece che un'altra, è spesso e volentieri non solo ragionevole, ma necessario.
	\section{Paradossi}
	\subsection{Il Vero}
	Come possiamo dire che sia falso che la realtà non può essere descritta, se definiamo il vero in relazione alla teoria fisica?\\
	Noi assumiamo, in modo del tutto ragionevole ma arbitrario, che la realtà non possa essere descritta.
	\subsection{La falsificazione}
	Il principio di falsificazione asserisce che una teoria deve essere falsificabile, tuttavia, si è dimostrato nel corso del manifesto che ogni teoria fisica assume principi non falsificabili. Questa è una evidente contraddizione.\\
	Si può risolvere questo paradosso asserendo che la falsificazione concerne solamente le effettive misure sperimentali, ovvero una teoria è scientifica se, eccettuata la sovrastruttura matematico-fisica, è falsificabile.
\end{document}
