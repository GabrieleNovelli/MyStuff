\documentclass[12pt,a4paper]{report}

\usepackage[english]{babel}
\usepackage{newlfont}
\usepackage{color}
\usepackage{multicol}
\usepackage{float}
\usepackage{frontespizio}
\usepackage{amsmath,amssymb}
\usepackage{amsthm}
\usepackage{geometry}
\usepackage{tikz}
\usepackage{biblatex}
\usepackage{csquotes}
\usepackage{pgfplots}
\usepackage{hyperref}
\usepackage{amssymb}
\usepackage{comment}
\usepackage[compat=1.0.0]{tikz-feynman}
\usepackage{tikz-cd}
\usepackage{mathtools}
\usepackage{braket}
\usepackage{pxfonts}
%\usepackage{lmodern}%THIS MUST GO AFTER PXFONTS

\hypersetup{
	colorlinks=true,
	linkcolor=blue,
	filecolor=magenta,      
	urlcolor=cyan,
	pdftitle={Overleaf Example},
	pdfpagemode=FullScreen,
}

\textwidth=450pt\oddsidemargin=0pt
\geometry{a4paper, top=3cm, bottom=3cm, left=3cm, right=3cm, % heightrounded, bindingoffset=5mm 
}
\theoremstyle{definition}
\newtheorem{Def}{Definition}[chapter]

\theoremstyle{Theorem}
\newtheorem{Theo}[Def]{Theorem}
\newtheorem{Prop}[Def]{Proposition}

\newtheorem{Lm}[Def]{Lemma}

\theoremstyle{break}
\newtheorem{Ex}[Def]{Example}
\newtheorem{Exe}[Def]{Exercise}

\theoremstyle{definition}
\newtheorem{Cor}[Def]{Corollary}
\newtheorem{Obs}[Def]{Observation}

\begin{document}
	\tableofcontents
	\chapter*{Preamble}
	\chapter{The necessary algebra}
	In this chapter we go through some algebraic constructions which will be at the base of all of the future calculations. Nevertheless, it will be assumed a basic strong knowledge of linear algebra.
	\section{The tensor product}
	In this section we will rigorously define the notion of tensor product. There are many ways to define tensors and we will try to give a strong general definition. However, for practical reasons, we will restrict ourselves to tensor products of vector spaces. It is useful to keep in mind that the tensor product is rigorously defined over modules. However, there is no reason to employ this generalization in our analysis since there is no field (that i know of) in which it is used.
	\begin{Def}\label{def_1.1.1}
		Let $V,W$ be real vector spaces. We define the \textit{free real module} of $V\times W$ as the set:
		$$R\braket{V^*\times W^*}=\{f:V\times W\rightarrow\mathbb{R}\hbox{ vanishing for all but a finite number of} (v,w)\}$$  
	\end{Def}
	Taking $(v^*_1,w^*_1)$ as the map defined like:
	$$(v^*_1,w^*_1)(v_2,w_2)=\delta_{v_1,v_2}\delta_{w_1,w_2}$$
	Any element of this new set is a linear combination:
	$$\lambda^i(v_i^*,w_i^*)\in R\braket{V^*\times W^*}$$
	This is clearly a vector space. The idea is now pretty straightforward. We want to be able to construct a product which keeps as fundamental property the multilinearity. Consider the following:
	$$I=span\bigg\{\lambda(v,w)-(\lambda v,w),\lambda(v,w)-(v,\lambda w),(v+v',w)-(v,w)-(v',w),$$$$,(v,w+w')-(v,w)-(v,w')\bigg\}$$
	This is clearly a subspace of the above vector space $R\braket{V^*\times W^*}$, spanned by the above elements. Thus, we can take the quotient:
	\begin{Def}\label{def_1.1.2}
		We define the \textit{tensor product} of $V^*\times W^*$ as the space:
		$$V^*\otimes W^*=R\braket{V^*\times W^*}/I$$
	\end{Def} 
	\begin{Obs}\label{obs_1.1.1}
		Clearly, sicne we took a quotient with a subspace, the reuslt is still a vector space. In particular, elements of $V^*\otimes W^*$ are equivalence classes of elements, we use the following identification:
		$$v^*\otimes w^*=[(v^*,w^*)]$$
		Since a dual space is a vector space and the double dual is canonically isomorphic to the starting space, we clearly generalize the above construction also to regular bector spaces.\\
		It is clear that by construction the tensor product is multilinear (as an equivalence class):
		$$\lambda v\otimes w=\lambda (v\otimes w)= v\otimes \lambda w$$
		$$(v+v')\otimes w=v\otimes w+v'\otimes w$$
		$$v\otimes (w+w')=v\otimes w+v\otimes w'$$
	\end{Obs}
	\begin{Prop}\label{prop_1.1.1}[\textbf{Universal property of tensor product}]
		Let $V,W,Z$ be real finite dimensional vector spaces. Let $f:V\times W\rightarrow Z$ be a multilinear map. Then there is a unique linear map $\tilde{f}:V\otimes W\rightarrow Z$ such that the following diagram commutes:
		\\
		\begin{center}
			\begin{tikzcd}
				V\times W\arrow{rr}{f}\arrow{dd}{\pi}&&Z\\\\
				V\otimes W \arrow{uurr}{\tilde{f}}&&
			\end{tikzcd}
		\end{center}
	\end{Prop}
	\begin{proof}
		Define the map $F:R\braket{V\times W}\rightarrow Z$ as $F((v,w))=f(v,w)$. Clearly, since $f$ is bilinear by hypothesis, $I\subset Ker(F)$. This means that the map $F$ induces a map:
		$$\tilde{f}:R\braket{V\times W}/I\rightarrow Z\hbox{ like }\tilde{f}(v\otimes w)=f(v,w)$$
		This proves the existence. As for the uniqueness, suppose there is another map $g$ which satisfies the same properties as $\tilde{f}$. By definition $V\otimes W$ is spanned by all elements like $v\otimes w$, and since for all of those the maps agree $g(v\otimes w)=f(v\otimes w)=\tilde{f}(v\otimes w)$, they are indeed the same map. 
	\end{proof}
	\begin{Theo}\label{theo_1.1.1}
		$$V^*\otimes W^*\simeq Bil(V\times W\rightarrow \mathbb{R})$$
		This is canonical.
	\end{Theo}
	\begin{proof}
		We make use of the universal property. Let $f(v^*,w^*)(v',w')=v^*(v')w^*(w')$ be a map. This is a mapping like:
		$$f:V^*\times W^*\rightarrow Bil(V\times W)$$ 
		Then from \ref{prop_1.1.1} there is a unique mapping:
		$$\tilde{f}:V^*\otimes W^*\rightarrow Bil(V\times W)$$
		This is defined as: $f(v^*\otimes w^*)=f(v^*,w^*)$. Choose any bases for $V^*,W^*$ like $v^i,w^i$. Now it is only a matter to find an inverse of $\tilde{f}$. Consider:
		$$g:Bil(V\times W)\rightarrow V^*\otimes W^*\hbox{ like }g(B)=B(v_i,w_j)v^i\otimes w^j$$
		Let now $\rho$ be any element of $V\otimes W$. The composition is:
		$$(g\circ \tilde{f})(\rho)=\tilde{f}(\rho)(v_i,w_j)v^i\otimes w^j$$
		Recall that $\tilde{f}$ is linear, thus:
		$$(g\circ \tilde{f})(\rho)=\rho_{km}\tilde{f}(v^k\otimes w^m)(v_i,w_j)v^i\otimes w^j$$
		Moreover, $\tilde{f}\circ \pi=f$ from \ref{prop_1.1.1}:
		$$(g\circ \tilde{f})(\rho)=\rho_{km}f(v^k,w^m)(v_i,w_j)v^i\otimes w^j$$
		By definition of $f$, we have:
		$$(g\circ \tilde{f})(\rho)=\rho_{km}\delta^k_i\delta^m_jv^i\otimes w^j=\rho_{ij}v^i\otimes w^j=\rho$$
		Thus, $g\circ \tilde{f}=\mathbb{I}$ is the identity. One can also easily prove the contrary:
		$$(\tilde{f}\circ g)(B)=\tilde{f}(B(v_i,w_j)v^i\otimes w^j)=B(v_i,w_j)\tilde{f}(v^i\otimes w^j)=$$
		$$(\tilde{f}\circ g)(B)(v',w')=B(v_i,w_j)f(v^i,w^j)(v'^kv_k,w'^lw_l)=B(v_i,w_j)v'^kw'^l\delta_i^k\delta_j^l=$$$$=B(v_i,w_j)v'^iw'^j=B(v',w')$$
		By arbitrarity of the base this is canonical.
	\end{proof}
	\begin{Obs}\label{obs_1.1.2}
		Since the above isomorphism is canonical it makes sense to think about bilinear maps as tensor products. 
	\end{Obs}
	Now we go through some properties of the tensor product, which are very technical but will help us develop the right tools for our future calculations.
	\begin{Prop}\label{prop_1.1.2}
		There is a unique canonical isomorphism: 
		$$V\otimes W\simeq W\otimes V$$
	\end{Prop}
	\begin{proof}
		This is just a consequence of the universal property. Let $f:V\times W\rightarrow V\otimes W$ be a bilinear map like:
		$$f(v,w)=w\otimes v$$
		Then, by the universal property, there is a unique linear mapping:
		$$\tilde{f}:V\otimes W\rightarrow W\otimes V\hbox{ like }\tilde{f}(v\otimes w)=w\otimes v$$
		Similarly, one can construct a map like: 
		$$\tilde{g}:W\otimes V\rightarrow V\otimes W\hbox{ like }g(w\otimes v)=v\otimes w$$
		This descends from the linear map $g(w,v)=v\otimes w$. Clearly, those maps are one the inverse of the other on both sides. Moreover, they are inverses on decomposable elements so they are unique inverses.
	\end{proof}
	\begin{Prop}\label{prop_1.1.3}
		The tensor product is associative.
	\end{Prop}
	\begin{proof}
		Clearly, since $V\otimes W$ is by construction a vector space, we can take its tensor product with $Z$ another vector space. This will define new equivalence classes by the same procedure as above, like: $(v\otimes w)\otimes z$. The aim is to prove $(V\otimes W)\otimes Z=V\otimes (W\otimes Z)$. This is a consequence of the universal property as well.\\
		\\
		Let $X$ be a vector space. Consider the trilinear mapping:
		$$f:V\times W\times Z\rightarrow X$$
		From the \ref{prop_1.1.1} universal property, there is a bilinear map: 
		$$\tilde{f}:V\times (W\otimes Z)\rightarrow X\hbox{ like }\tilde{f}(v,w\otimes z)=f(v,w,z)$$
		Since $\tilde{f}$ is bilinear, there is another linear map, from the universal property, like $\tilde{\tilde f}:V\otimes(W\otimes Z)=f(v,w,z)$. Choosing now $X=(V\otimes W)\otimes Z$ and $f(v,w,z)=(v\otimes w)\otimes z$ (this is clearly trilinear) we have:
		$$\tilde{\tilde f}(v\otimes (w\otimes z))=(v\otimes w)\otimes z$$
		One can apply the same reasoning by commuting the order in which the tensor product is obtained. This will give a linear map:
		$$\tilde{\tilde g}((v\otimes w)\otimes z)=v\otimes (w\otimes z)$$
		Those are clear inverses.
	\end{proof}
	\begin{Def}\label{def_1.1.3}
		We call $T^k_lV$ the space of multilinear maps 
		$$f:V\times...\times V\times V^*\times...\times V^*\rightarrow\mathbb{R}$$
		We call $\bigotimes^k V$ the space of $k$ times the tensor product of $V$ with itself:
		$$\bigotimes^k V:=V\otimes V\otimes...\otimes V\hbox{ k times}$$
	\end{Def}
	\begin{Cor}\label{cor_1.1.1}
		$$T^k_lV=\bigotimes^k V^*\otimes \bigotimes^l V$$
	\end{Cor}
	This follows immediately from the associativity and the isomorphism \ref{theo_1.1.1}.
	\begin{Prop}\label{prop_1.1.4}
		Let $V,W$ be vector spaces and $\{e_i\}$ a base for it. Then, $e_i,f_j$ bases for them. Then, a base for $V\otimes W$ is:
		$$\big\{e_i\otimes w_j\big\}$$
	\end{Prop}
	\begin{proof}
		We use the isomorphism with multilinear maps. If $f\in Bil(V\times W)$, then it can be expressed as:
		$$f=f_{ij} e^i\otimes f^j$$
		This automatically implies that the above set spans all of the elements. To show linear independence, we see that:
		$$f(e_m,f_l)=f_{ml}$$
		So the 0 can be obrained only by taking all constant equal to 0.
	\end{proof}
	\begin{Cor}\label{cor_1.1.2}
		$$\{e^{i_1}\otimes...e^{i_k}\otimes e_{j_1}\otimes...\otimes e_{j_l}\}\hbox{ is a base for }T^k_lV$$
	\end{Cor}
	\section{The exterior algebra}
	In this section we take yet another step forward in the analysis on the tensor product. A posteriori, we will be mainly interested in antisymmetric tensors. This means we need a precise and rigorous way to define them. We will thus introduce the wedge product of vector spaces. This is unfortunately defined through the use of another quotient. \\
	\\
	Let $\bigotimes^k V$ be the $k$-tensor product space of $V$. Consider the ideal $I^k(V)$ spanned by all elements $v^1\otimes...\otimes v^k$ that have at least two equal vectors in the product.
	\begin{Def}\label{def_1.2.1}
		We call \textit{exterior $k$-power} of $V\otimes W$ the quotient:
		$$\bigwedge^k V=\bigotimes^kV/I^k(V)$$	
		We index elements of $\bigwedge^k V$ as:
		$$v^1\wedge...\wedge v^k=[v^1\otimes...\otimes v^k]$$
	\end{Def}
	\begin{Obs}\label{obs_1.2.1}
		Clearly the exterior power is a vector space since it is the quotient of a vector space with one vector subspace. This translates in linearity of the equivalence classes.
	\end{Obs}
	\begin{Obs}\label{obs_1.2.2}
		Clearly, by construction $$v^1\wedge...\wedge v^i\wedge v^{i+1}\wedge...\wedge v^k=-v^1\wedge...\wedge v^{i+1}\wedge v^{i}\wedge...\wedge v^k$$ is antysimmetric. This follows from the fact that:
		$$[0]=0=v^1\wedge...\wedge (v^i+v^{i+1})\wedge (v^{i+1}+v^{i})\wedge...\wedge v^k=$$
		$$=v^1\wedge...\wedge v^i\wedge v^{i+1}\wedge...\wedge v^k+v^1\wedge...\wedge v^{i+1}\wedge v^{i}\wedge...\wedge v^k+0+0$$
		The last two 0 are elements that after the linear expansion contain two equal vectors.
		Clearly, the above operations follow from the linearity of the tensor product.
		\\
		Given the above result it is not hard to see that if $\sigma$ is a permutation of the vectors (or the indices), then:
		$$v^1\wedge...\wedge v^k=sgn(\sigma)v^{\sigma(1)}\wedge...\wedge v^{\sigma(k)}$$
	\end{Obs}
	\begin{Prop}\label{prop_1.2.1}
		The wedge product is associative.
	\end{Prop}
	\begin{proof}
		This is clear from the fact that the tensor product is associative and equivalence classes are linear.
	\end{proof}
	\begin{Prop}\label{prop_1.2.2}[\textbf{Universal property for the wedge product}]
		Let $f:V^k\rightarrow Z$ any alternating multilinear map on a vector space. There is a unique alternating linear map $\tilde{f}:\bigwedge^kV\rightarrow Z$ such that the following diagram commutes:
		\begin{center}
			\begin{tikzcd}
				\bigwedge^k V \arrow{rrdd}{\tilde{f}}&&\\\\
				V^k\arrow{rr}{f}\arrow{uu}{\wedge}&&Z
			\end{tikzcd}
		\end{center}
		Where $\wedge(v_1,...,v_k)=v_1\wedge ...\wedge v_k$ is the projection.
	\end{Prop}
	\begin{proof}
		Since $f$ is multilinear, by \ref{prop_1.1.1} the universal property of the tensor product, there is a linear map $h:\bigotimes^k V\rightarrow Z$ such that:
		$$h(v_1\otimes...\otimes v_k)=f(v_1,...,v_k)$$
		By the alternating property, $I^k(V)$ is in the kernel of this map. Thus, there exists a map:
		$$\tilde{f}:\bigwedge^k V\rightarrow Z$$
		which satisfies the above properties. This proves the existence. The uniqueness follows from the uniqueness of $h$.
	\end{proof}
	We now present an extremely important result regarding the space of alternating multilinear maps. In particular, we denote with $Alt^k(V,\mathbb{R})$ the space of all multilinear maps from $V^k=V\times V\times...\times V$ k-times to $\mathbb{R}$.
	$$Alt^k(V,\mathbb{R})=\big\{f:V^k\rightarrow \mathbb{R} \hbox{ multilinear }\big\}$$
	\begin{Theo}\label{theo_1.2.1}
		There is a canonical isomorphism:
		$$Alt^k(V,\mathbb{R})\equiv \bigwedge^k V$$
	\end{Theo}
	\begin{proof}
		This is just a consequence of the previous result on the tensor product \ref{cor_1.1.1}. Since any multilinear map can be seen as an element of $\bigotimes^k V$ and any alternating multilinear map can be obtained by antisymmetrization, the equivalence is obvious.
	\end{proof}
	Lastly, like we did for the tensor product, we would like to find a basis for this new vector space which we constructed.
	\begin{Theo}\label{theo_1.2.2}
		If $\{e_i\}$ is a basis for $V$, then $\{e_{i_1}\wedge...\wedge e_{i_k}\hbox{ with }i_1<i_2<...<i_k\}$ is a base for $\bigwedge^k V$.
	\end{Theo}
	\begin{proof}
		This follows immediately from proposition REFERENZA due to the quotient.
	\end{proof}
	\begin{Obs}\label{obs_1.2.3}
		It is easy to see that if $k>n=dim(V)$, then $\bigwedge^kV=0$. This is due to the fact that this space must by definition contain some repeated elements.
	\end{Obs}
	\chapter{Smooth manifolds}
		\section{Topological manifolds}
			In this section we will define the concept of topological manifolds and look at some examples and properties of them. The main idea is to find a way of describing spaces which are not necessaarily $\mathbb{R}^n$. To achieve this, we simply look at spaces which locally "look like" $\mathbb{R}^n$. This is the basic construction of a topological manifold. However, we will see that smooth manifolds take a step further. In what follows we will assume a basic knowledge of topology i.e. the notions of open and closed sets, closure, continuity and homeomorphisms.
			\begin{Def}
				Let $(X,\tau)$ be a topological space. We call \textit{basis} for $X$ a subcollection $\beta\subset\tau$ of sets $\{U_i\}$ such that:
				\begin{itemize}
					\item $X$ is covered by the union of all the elements in $\beta$;
					\item For any $x\in X$ and $U\in \tau$ containing $x$, there is a set $U_i\in \beta$ such that:
					$$p\in U_i\subset U$$ 
				\end{itemize}
			\end{Def}
			\begin{Prop}
				A collection $\beta$ of subsets of  $X$ is a basis for some topology of $X$ if and only if:
				\begin{itemize}
					\item the union of all the elements of $\beta$ covers $X$;
					\item for any $x\in X$ and $U,V\in \beta$ with $x\in U\cap V$, there is $W\in \beta$ with $x\in W\subset U\cap V$.
				\end{itemize}
			\end{Prop}
			\begin{proof}
				PROVA
			\end{proof}
			Now we start to enunciate the actual important topologicla properties we want our spaces to have:
			\begin{Def}
				A topological space $(X,\tau)$ is called:
				\begin{itemize}
					\item T2 or \textit{Hausdorff} if for any two distinct points $x,y\in X$, $x\neq y$, we can find two open sets $U,V\in \tau$, such that: $x\in U;y\in V$ and $U\cap V=\emptyset$;
					\item \textit{second countable} of it has a countable basis;
					\item \textit{locally euclidean} if for any open set $U$ we can find an homeomorphism $\phi:U\rightarrow \mathbb{R}^n$. The couple $(U,\phi)$ is called \textit{chart}.	
				\end{itemize}
			\end{Def}
			\begin{Def}
				A topological space which is also T2, second countable and locally euclidean is called \textit{topological manifold}.
			\end{Def}
			\begin{Ex}
				We now make some examples of topological manifolds.
				\begin{itemize}
					\item $\mathbb{R}$ is a topological manifold. Clearly, $\mathbb{R}$ is locally euclidean and $T_2$, and we can find a countable basis by choosing the open balls in $\mathbb{Q}$ like:
					$$B_{r,s}=\big\{]r,s[\hbox{ with }r,s\in\mathbb{Q}\big\}$$
					\item Since the cross product of two T2 and second countable spaces has the same topological properties, the space $\mathbb{R}^n$ is trivially a topological manifold.
					\item The sphere $S^n$ is a topological manifold. It is clearly T2 and second countable since it is inside $\mathbb{R}^n$. Moreover, it can be covered by the following charts:
					$$U^{\pm}_i=\big\{\vec{x}\in S^n\big| \pm x_i>0\big\}$$
					$$\phi_i^\pm:U^\pm_i\rightarrow \mathbb{R}^n\hbox{ like }\phi^\pm_i(\vec{x})=(x_1,...,x_{i-1},x_{i+1},...,x_{n+1})$$
					Those have some clear inverses:
					$$(\phi^\pm_i)^{-1}(x_1,...,x_{n+1})=(x_1,...,x_{i-1},\pm\sqrt{1-||x^2||},x_{i+1},...,x_{n+1})$$
				\end{itemize}
			\end{Ex}
			We conclude this section with a very useful proposition, which will come in handy later.
			\begin{Prop}
				If $\pi:M\rightarrow N$ is a quotient map, $M$ is a second countable topological space and $N$ is locally euclidean. Then $N$ is also second countable.
			\end{Prop}
			\begin{proof}
				a
			\end{proof}
		\section{Smooth structures}
		In this section we will define the notion of smooth manifold. This will be dependent on the notion of smooth structure. The basic idea is to be able to map diffeomorphically our spaces into $\mathbb{R}^n$.
		\begin{Def}
			Let $M$ be a topological manifold and $(U,\phi)$, $(V,\psi)$ two charts on $M$ such that $U\cap V\neq \emptyset$. We say that the two charts are \textit{compatible} if the composition $\psi\circ \psi^{-1}:\psi(U\cap V)\longrightarrow \phi(U\cap V)$ is a diffeomorphism.
		\end{Def}
		\begin{Def}
			Let $M$ be a topological manifold. We define an \textit{Atlas} for $M$ as a collection of charts $\mathcal{A}=\{(U_i,\phi_i)\}$ that are smoothly compatible and that cover $M$.
		\end{Def}
		\begin{Def}
			We define the \textit{smooth structure} of a topological manifold as the equivalence class of atlases $[\mathcal{A}]$, under the relation:
			$$\mathcal{A}_1\simeq \mathcal{A}_2\iff \mathcal{A}_1\cup \mathcal{A}_2 \hbox{ is still an atlas for }M$$
		\end{Def}
		The above relation is clearly an equivalence relation. In practice, it means that any composition of the charts of $\mathcal{A}_1$ and $\mathcal{A}_2$ is a diffeomorphism.
		\begin{Theo}
			Every atlas is contained in a maximal atlas. Moreover, two atlases have the same maximal atlas if and only if their union is still an atlas.
		\end{Theo} 
		\begin{proof}
			The first one is kind of obvious. Consider any atlas $\mathcal{A}$ and define $\mathcal{A}^{max}$ as the set of all charts compatible with $\mathcal{A}$. Then if $\mathcal{A}^{max}$ is an atlas it is also maximal. Consider tow charts in $\mathcal{A}^{max}$ like $(U_1,\phi_1),(U_2,\phi_2)$. We can look at:
			$$\phi_1\circ \phi_2^{-1}$$
			Since $\mathcal{A}$ is an atlas, there will be a chart $(V,\varphi)$ such that $U_1\cap U_2\cap V\neq \emptyset$. So we can write:
			$$\phi_1\circ \varphi^{-1}\circ \varphi\circ \phi_2^{-1}$$
			This is clearly smooth, since by construction $\phi_i$ are compatible with $\mathcal{A}$.\\
			\\
			As for the second point, if two atlas have the same maximal atlas then their union is by definition still an atlas. Instead if the unioon of two atlases $\mathcal{A}_1,\mathcal{A}_2$ is an atlas, consider $\mathcal{A}^{max}_1,\mathcal{A}^{max}_2$ their maximal atlases. Take any two charts $(U_1,\varphi_1)\in\mathcal{A}_1$ and $(U_2,\varphi_2)\in\mathcal{A}_2$. Then the following map is smooth:
			$$\varphi_1\circ \varphi_2^{-1}$$
			Now, taking an overlapping chart inside $\mathcal{A}_1^{max}$ like $(V_1,\psi_1)$, we have:
			$$\varphi_1\circ \varphi_2^{-1}=\varphi_1\circ \psi_1^{-1}\circ \psi_1\circ \varphi_2^{-1}$$
			By smoothness of the left hand side, we have that $\psi_1\circ \varphi_2^{-1}$ is smooth. This means $\mathcal{A}_1^{max}\subset \mathcal{A}_2^{max}$. One can easily prove the opposite inclusion. This concludes the proof.
		\end{proof}
		\begin{Def}
			A topological manifold with a smooth structure is called a \textit{smooth manifold}.
		\end{Def}
		\begin{Ex}
			We now make some examples of smooth manifolds.
			\begin{itemize}
				\item $\mathbb{R}^n$ is a topologicla manifold, and a smooth structure can be found with the following atlas made of only one chart:
				$$\mathcal{A}=\{(\mathbb{R}^n,\mathbb{I}_n)\}$$
				\item Let $M$ be a smooth manifold, then any open subset $U$ of $M$ is also a smooth manifold with the atlas $\mathcal{A}\big|_U$ reduced to $U$. This is clear since the topological properties are passed down to open subsets.
				\item Consider the general linear group $GL(n,\mathbb{R})$. This is defined as:
				$$GL(n,\mathbb{R})=\big\{A\in M_{n\times n}\hbox{ such that }det(A)\neq 0\big\}$$
				It is then clear that $GL(n,\mathbb{R})$ is the preimage of the map:
				$$det:\mathbb{R}^{n^2}\longrightarrow \mathbb{R}$$ 
				This map is continuous and so, since $\mathbb{R}-\{0\}$ is open, $det(\mathbb{R}-\{0\})$ is open as well. This makes $GL(n,\mathbb{R})$ is an open subset of $\mathbb{R}^{n^2}$. This implies that it is a smooth manifold.
				\item $S^n$ is a smooth manifold. We have already seen in REFERENZA that this is a topological manifold. We just have to show that the composition of two charts is smooth. This is easy: let $\phi_1^\pm,\phi_2\pm$ be two such charts (without loss of generality we can fix two indices). Consider:
				$$\phi_1^{\pm}\circ(\phi_2^{\pm})^{-1}(x_1,x_3,...,x_{n+1})=$$$$=\phi^\pm_1(x_1,\pm\sqrt{1-||x^2||},x_3,...,x_{n+1})=(\pm\sqrt{1-||x^2||},x_3,...,x_{n+1})$$
				This is clearly smooth since taking the derivation of the square root we get:
				$${d\over dx^i}(\pm\sqrt{1-||x^2||})={\pm x^i\over\sqrt{1-||x^2||}}$$ 
				This is smooth since in our domain the root is well defined.
				\item Any finite dimensional vector space is a smooth manifold. From topology, we know that any two norms defined on a finite dimensional vecotr space are equivalent in defining the same topology. In particular, those define a T2 and second countable topology (REFERENZA). Fix a base for $V$, like $\{e_i\}$. Then, we can find an isomorphism:
				$$f:V\rightarrow \mathbb{R}^n\hbox{ like }f(v^ie_i)=(v^1,...,v^n)$$
				This is an homeomorphism REFERENZA. Considering an other base $f_j$, we can find a chang eof basis matrix $M^j_i$ so that our construction does not depend on the choice of the base.
			\end{itemize}
		\end{Ex}
		\begin{Def}
			Let $M$ be a smooth manifold and $f:M\rightarrow\mathbb{R}$ a function. We say that $f$ is \textit{smooth} if for any chart $(U,\phi)$ in the maximal atlas $\mathcal{A}^{max}$ we have that
			$f\circ \phi^{-1}$ is smooth.
		\end{Def}
		\begin{Obs}
			Of course, any atlas is good enough. This is easy to check. If $\mathcal{A}^{max}$ is a maximal atlas and $A\in[\mathcal{A}^{max}]$, then, considering any $(U,\phi)\in \mathcal{A}$ chart, we have:
			$$f\circ \phi^{-1}=f\circ \psi^{-1}\circ \psi\circ \phi^{-1}$$
			where $(V,\psi)$ is a chart in $\mathcal{A}^{max}$ overlapping with our domain. This makes $f$ smooth also in $\mathcal{A}$.
		\end{Obs}
		Lastly, we make a useful statement which will come in handy in a bit:
		\begin{Lm}
			Any smooth manifold $M$ has a countable basis made of coordinate charts.
		\end{Lm}
		\begin{proof}
			Let $\{U_i\}$ be an atlas for $M$. Consider a countable basis of open sets $\{u_j\}$ of $M$. Take $p\in M$ and $U_i$ containing $p$. Then, there is an element $u_i$ of the basis such that $p\in u_{i,p}\subset U_{i}$ by definition of basis. The collection of the sets $u_{j,p}$ constructed like this, without duplicates, is a subcollection of the basis and so it is countable. Moreover, it is a collection of charts and covers $M$. To prove it is a base, consider $u\subset M$ and $p\in U$. Then there is $U_i\subset U$ containing $p$. Thus, there is also, by construction, $u_{i,p}\subset U$ containing $p$.
		\end{proof}
		\section{The partition of Unity}
			In this section we construct the partition of unity. This is an analytical tool which will allow us to find some interesting results later on. This is very technical and it is a topic for the interested reader.
			\begin{Def}
				 Let $f:M\rightarrow \mathbb{R}$ be a smooth map. We call \textit{support of $f$} the set:
				 $$supp(f)=\{p\in M\hbox{ such that } f(p)\neq 0\}$$
			\end{Def}
			\begin{Def}
				We call \textit{bump function at $p\in\mathbb{R}^n$ supported in $U\subset \mathbb{R}^n$} any function $\rho:\mathbb{R}^n\rightarrow \mathbb{R}$ such that:
				\begin{itemize}
					\item $\rho=1$ in $V\subset U$ with $p\in V$;
					\item $supp(\rho)\subset U$.
				\end{itemize}
			\end{Def}
			FIGURA\\
			\\
			\begin{Ex}
				It is not hard to construct a bump function. Consider the following.
				$$f(x)=\begin{cases}
					e^{-{1\over x}} \hbox{ for }x>0\\
					0 \hbox{ otherwise}
				\end{cases}$$
				This is not a bump function as one can see from REFERENZA.
				\begin{center}
					\begin{tikzpicture}
						\begin{axis}[xmin=-1.2, xmax=2, ymin=-0.7, ymax=1.2,
							xtick = {-1,0,1}, ytick = { 1},
							scale=0.4, restrict y to domain=-1.5:1.2,
							axis x line=center, axis y line= center,
							samples=40]
							
							
							\addplot[black, samples=100, smooth, domain=-1.2:2, thick]
							plot (\x, {exp{-1/\x}});
						\end{axis}
					\end{tikzpicture}
				\end{center}
				Starting from this nice function (smooth but not analytical!) we construct the following:
				$$r(x)={f(a-x)\over f(a-x)+f(x-b)}$$
				where $a,b$ are positive real numbers with $a>b$.
				It is clear that $r(x)$ is a function supported in $]-a,a[$ and it is equal to $1$ in $]b,-b[$ at 0. One can then easily move this function with some translations.\\
				We then clearly see that the function:
				$$\rho(\vec{x})=r(|\vec{x}|)$$ 
				is indeed a bump function.
				\begin{center}
					\begin{tikzpicture}
						\begin{axis}[xmin=-2, xmax=2, ymin=-0.7, ymax=1.2,
							xtick = {-1,0,1}, ytick = { 1},
							scale=0.4, restrict y to domain=-1.5:1.2,
							axis x line=center, axis y line= center,
							samples=40]
							
							
							\addplot[black, smooth, thick, domain=-1:1]
							{1};
							
							\addplot[black, smooth, thick, domain=1.001:1.999]
							{exp(-1/(2 - x)) / (exp(-1/(2 - x)) + exp(1/(1 - x)))};
							
							\addplot[black, smooth, thick, domain=-1.001:-1.999]
							{exp(-1/(2 + x)) / (exp(-1/(2 + x)) + exp(+1/(1 + x)))};
							
						\end{axis}
					\end{tikzpicture}
				\end{center}
			\end{Ex}
			One can easily generalize the definition of bump function to smooth manifolds in the following way:
			\begin{Def}
				A function $\rho:M\rightarrow \mathbb{R}$ is called \textit{bump function  at $p\in M$ supported in $U\subset M$} if there is a chart $(U,\phi)$ such that, in coordinates, $\rho\circ \phi^{-1}$ is a bump function.
			\end{Def}
			\begin{Prop}
				Let $f_U:U\rightarrow\mathbb{R}$ be a smooth map. Then there always is a global $f:M\rightarrow \mathbb{R}$ smooth, such that $f=f_U$ in a possibly smaller $V\subset U$.
			\end{Prop}
			\begin{proof}
				The proof is very easy since it makes use of the bump function (provided it exists). Take $\rho$ a bump function at $p\in U$ supported in $U$ and define for $q\in M$:
				$$f(q)=\begin{cases}
					f_U(q)\rho(q)\hbox{ for }q\in U\\
					0 \hbox{ otherwise}
				\end{cases}$$
			\end{proof}
			\begin{Def}
				Let $\mathfrak{U}=\{U_i\}$ be a collection of sets of a topological space. This is called \textit{locally finite} if for any $p$ point there is a $U$ containing $p$ which intersects at most finitely many $U_i$. 
			\end{Def}
			\begin{Def}
				Let $\mathfrak{U}=\{U_i\}$ be covering of a smooth manifold $M$. A \textit{partition of unity subordinate to $\mathfrak{U}$} is a family of smooth functions $\{\phi_i:U_i\rightarrow\mathbb{R}\}$ such that:
				\begin{itemize}
					\item $\phi_i(p)\in[0,1]$ for all indices;
					\item $supp(\phi_i)\subset U_i$ for all indices;
					\item the set $\{supp(\phi_i)\}$ is locally finite;
					\item $\sum \phi_i(p)=1$ for all $p\in M$.
				\end{itemize}
			\end{Def}
			\begin{Obs}
				Note that the last sum in the above definition is always finite since the set of supports is always locally finite.
			\end{Obs}
			\begin{Theo}
				A smooth manifold always has a partition of unity.
			\end{Theo}
			\section{The tangent space}
			In this section we will introduce an extremely important construction, which will follow us through the whole book: the tangent space. We will give the precise notions on vector and vector fields for smooth amnifolds, and we will start to see how we can force the structure of $\mathbb{R}^n$ on our smooth manifolds.
			\begin{Def}
				Let $p\in M$. Consider all pairs $(f,U)$, with $f:U\subset M\rightarrow \mathbb{R}$ smooth and $p\in U$. We say that $(f,U)\sim (g,V)$ if there is a $W\subset U\cap V$ such that $p\in W$ and $f\big|_W=g\big|_W$.
				We call the \textit{germ of $f$ at $p$} the equivalence class $[(f,U)]$. We index the set of all germs at a point with:
				$$C^\infty_p(M)$$
			\end{Def}
			\begin{Obs}
				The above relation is clealry an equivalence relation (see it for yourself if you dont believe me).
			\end{Obs}
			\begin{Def}
				We call \textit{derivation at $p\in M$} any smooth, linear map $D_p:C_p^\infty(M)\rightarrow \mathbb{R}$ which respects the Leibniz rule i.e.:
				$$D_p(fg)=D_p(f)g(p)+f(p)D_p(g)$$
				The set of all derivation is called \textit{tangent space at $p$} and it is indexed with $$T_pM$$
			\end{Def}
			\begin{Obs}
				The space of derivation is clealry a vector space. I am not going to prove this because it is a matter of looking at 8 simple properties, and I would get extremely bored.
			\end{Obs}
			\begin{Obs}
				From the definition of derivation some clear properties immediately follow:
				\begin{itemize}
					\item If $f$ is constant then $D_p(f)=0$
					\\\\
					To see this simply apply Leibniz: suppose $f=1$ w.l.o.g. and see:
					$$D_p(f)=D_p(f^2)=2D_p(f)\implies 0$$
					\item  If $f(p)=0=g(p)$ then:
					$$D_p(fg)=0$$
					
					This is clear form the definition of Leibniz.
				\end{itemize}
			\end{Obs}
			Since $\mathbb{R}^n$ is a smooth manifold as well, our definitions and properties apply to it. The idea now it to show that the tangent space at any point of $\mathbb{R}^n$ has dimension $n$.
			\begin{Prop}
				Let $\vec{x}\in\mathbb{R}^n$. There is an isomorphism $\mathbb{R}^n\simeq T_{\vec{x}}\mathbb{R}^n$.
			\end{Prop}
			\begin{proof}
				Consider a vector $\vec{v}\in\mathbb{R}^n$. We define the following map:
				$$\vec{v}\rightarrow D_{\vec{v}}=v^i\partial_i$$
				This map is injective since $D_{\vec{v}}(x^j)=v^j$ and so in order to obtain 0 we would need $\vec{v}=\vec{0}$. As for surjectivity, consider any smooth map $f$ and its Taylor expansion around a point:
				$$f(\vec{z})=f(\vec{x})+\partial_i f(\vec{x})(z^i-x^i)+o(\vec{x})$$
				In particular, the residue will vanish in $\vec{x}$. Taking the derivation at $\vec{x}$ of this map, we see that the only non vanishing term is:
				$$D_{\vec{v}}f=\partial_i f(\vec{x})D_{\vec{v}}(z^i-x^i)$$
				Defining $v^i=D_{\vec{v}}(z^i-x^i)$ we have surjectivity.
			\end{proof}
			\begin{Cor}
				Any tangent vector to $\mathbb{R}^n$ is of the form $X_p=X^i(p)\partial_i\big|_p$.
			\end{Cor}
			The idea now is to somehow push the naive structure found on $\mathbb{R}^n$ onto our smooth manifolds. The key to do this is the so called push-forward.
			\begin{Def}
				Let $M,N$ be smooth manifolds and $F:M\rightarrow N$ be a smooth map. For any $p\in M$ we can define the \textit{push-forward at $p$ with $F$} as the map:
				$$F_{*,p}:T_pM\rightarrow T_{F(p)}N\hbox{ like }F_{\star,p}(X_p)f=X_p(f\circ F)$$
			\end{Def}
			Clearly, the definition of push-forward is pointwise. However, there are situations in which the point does not really matter, or it is not really necessary to specify it.
			\begin{Prop}
				The following properties hold for the push-forward:
				\begin{itemize}
					\item $(G\circ F)_*=G_*\circ F_*$ satisfies the chain rule.
					\item If $F$ is a diffeomorphism, then $F_*$ is an isomorphism.
				\end{itemize}
				\begin{proof}
					The first point is clear and just a matter of applying the definition. As for the second point: clearly, if $F$ is a diffeomorphism, there is an inverse $F^{-1}$ which is smooth as well. This will generate another push-forward:
					$$F^{-1}_*(Y_{F(p)})g=Y_{F(p)}(g\circ F^{-1})$$
					Now, clealry, by the above property, $F_*\circ F_*^{-1}$ is the push-forward of the identity; which is the identity.
				\end{proof}
			\end{Prop}
			Since $\mathbb{R}^n$ is a smooth manifold, we can use the push-forward of diffeomorphism (charts) to automatically inherit the tangent space straucture through the isomorphism.
			\begin{Obs}
				Let $r^i$ be the standard coordinate of $\mathbb{R}^n$ and let $(U,\phi)$ be a chart on $M$. We define:
				$${\partial\over \partial x^i}=\phi_*^{-1}\bigg({\partial\over \partial r^i}\bigg)$$
				Since the chart is a diffeomoeprhism it sends a base into a base.
			\end{Obs}
			\begin{Cor}
				$\big\{{\partial\over \partial x^i}\big\}$ is a base for $T_pM$.
			\end{Cor}
			\begin{Cor}
				$T_pM$ has dimension $n=dim(M)$.
			\end{Cor}
			\begin{Exe}
				Let $F:M\rightarrow N$ be a diffeomorphism between manifolds. Let $(U,\phi)$ be a chart on $M$ and $(V,\psi)$ be a chart on $N$. What is the base induced on $N$ through $F?$.\\
				\\
				Take any point $p\in M$ and take the base of the tangent space induced from the chart on $M$:
				$${\partial \over \partial x^i}=\phi_*^{-1}\big({\partial\over \partial r^i}\big)$$
				Where $r^i$ are the coordinates of $\mathbb{R}^n$. The pullback of this base through $F$ is given by:
				$$F_*\big({\partial \over \partial x^i}\big)f={\partial \over \partial x^i}(f\circ F)$$
				We now do the sandwich technique:
				$${\partial \over \partial x^i}(f\circ F)={\partial \over \partial x^i}(f\circ\psi^{-1}\circ \psi \circ F)={\partial \over \partial r^i}(f\circ\psi^{-1}\circ \psi \circ F\circ \phi^{-1})$$
				This is now just a derivative in $\mathbb{R}^k$. Let us call $z^a$ the induced coordinates on $\mathbb{R}^n$ through $\psi\circ F\circ\phi^{-1}$. From the chain rule we get:
				$${\partial \over \partial x^i}(f\circ F)={\partial (\psi\circ F^{-1}\circ \phi^{-1})^a\over \partial r^i}{\partial (f\circ \psi^{-1})\over \partial z^a}$$
			\end{Exe}
			\begin{Obs}
				We can apply the results found in the last exercise to a generic change of coordinate. In our case the map $F$ would just be the identity from $M$ to itself and the two charts would be on the same manifold (overlapping). In this case:
				$$\psi\circ\phi^{-1}(x^i)=y^j(x^i)$$
				So that $y^j$ would be the coordinate induced from $\psi$ onto $M$. This means that, applying the definition of pullback:
				$${\partial (f\circ \psi^{-1})\over \partial z^j}={\partial\over \partial y^j}f$$
				While the first term as well beccomes:
				$${\partial (\psi\circ \phi^{-1})^a\over \partial r^i}={\partial y^j\over \partial x^i}$$
				Thus, we get:
				$${\partial \over \partial x^i}={\partial y^j\over \partial x^i}{\partial\over \partial y^j}$$
			\end{Obs}
			\begin{Obs}
				The above result immediately implies that the components of a vector change in the opposite way: if $X_p=a^i(p){\partial\over \partial x^i}\bigg|_p$ at a certain point in a certain coordinate frame, then in another frame:
				$$X_p=b^j(p){\partial\over \partial y^j}\bigg|_p$$
				And so:
				$$b^j(p)={\partial y^j\over \partial x^i}a^i(p)$$
			\end{Obs}
			\begin{Exe}
				\begin{itemize}
					\item Consider the following map:
					$$F:\mathbb{R}^2\rightarrow \mathbb{R}\hbox{ like }F(x,y)=x^3+xy+y^3+1$$
					Find the points in which $F_*$ is injective and or surjective.
					\\
					\\
					The definition of push-forward is:
					$$F_*(X)f=X(f\circ F)$$
					This means that we can write in coordinates:
					$$F_*(X)f=X^i\partial_i(f\circ F)=X^i\partial_i F^j\tilde{\partial}_jf$$
					Where $\tilde{\partial}_j$ are the coordinates induced by $F$. Thus, we write:
					$$F_*=\partial_iF^j\tilde{\partial}_j$$
					In our case we have:
					$$F_*=\begin{pmatrix}
						\partial_x F(x,y) & \partial_y F(x,y)
					\end{pmatrix}=\begin{pmatrix}
					3x^2+y & 3y^2+x
					\end{pmatrix}$$
					This map is clearly never injective (from linear algebra) and it is surjective everywhere except in $(0,0),(-1/3,-1/3)$.
					\item Consider the following maps:
					$$F:\mathbb{R}^2\rightarrow \mathbb{R}^2\hbox{ like }F(x,y)=(x^2-2y,4x^3y^2)$$
					$$G:\mathbb{R}^2\rightarrow \mathbb{R}^3\hbox{ like }G(x,y)=(x^2y+y^2,x-2y^3,ye^x)$$
					Compute their push-forward. What is the image of $G(0,0)$?.\\
					\\
					As we showed before, computing the push-forward is equal to computing the Jacobian.
					$$F_*=\begin{pmatrix}
						\partial_x F_x & \partial_y F_x\\
						\partial_x F_y & \partial_y F_y
					\end{pmatrix}=\begin{pmatrix}
					2x & -2\\
					12x^2y & 8x^3y
					\end{pmatrix}$$
					$$G_*=\begin{pmatrix}
						\partial_x G_x & \partial_y G_x \\
						\partial_x G_y & \partial_y G_y\\
						\partial_x G_z & \partial_y G_z\\
					\end{pmatrix}=\begin{pmatrix}
						2xy & x^2+2y\\
						1 & -6y^2\\
						ye^x& e^x
					\end{pmatrix}$$
					Lastly, we have:
					$$G_*\big|_{(0,0)}=\begin{pmatrix}
						0 & 0\\
						1 & 0\\
						0& 1
					\end{pmatrix}$$
				\end{itemize}
			\end{Exe}
			\section{Submanifolds}
			In this sections we dive into the notion of submanifolds. This is a really necessary step in the whole architecture of differential geometry, as it will provide some results which will be used later on. The idea is to find a criterion to identify when a particular subset of a manifold is a submanifold (it behaves nicely topologically and is still smooth).
			\begin{Def}
				Let $M$ be a manifold of simension $n$. We call \textit{regular submanifold of dimension $k$} of $M$ a subset $S\subset M$ such that for all $p\in S$ there is a chart $(U,\phi)$ of $M$ such that $U\cap S$ is defined by the vanishing of the last $n-k$ coordinates.
			\end{Def}
			\begin{Prop}
				A regular submanifold of dimension $k$ is a manifold of dimension $k$.
			\end{Prop}
			\begin{proof}
				This is almost immediate. Let $S\subset M$ be our immersed submanifold. Then we can endow it with the subspace topology, which ensures that $S$ is both T2 and second countable. Moreover, considering two charts of $M$ like $(U,\phi)$ and $(V,\psi)$, we have some induced charts:
				$$(U\cap S,\phi)\hbox{ and }(V\cap S,\psi)$$
				Those are compatible:
				$$\phi\big|_S(p)=(x^1,...,x^k,0,...,0)\hbox{ and }\psi\big|_S(p)=(y^1,...,y^k,0,...,0)$$
				Those are compatible on $M$ and so they will also be on the reduction on $S$.
			\end{proof}
			\begin{Def}
				A smooth map $F:M\rightarrow N$ between manifolds is said to be an \textit{immersion} at $p$ if it's push-forward at $p$ is injective, a \textit{submersion} at $p$ if it is srujective.
			\end{Def}
			Clearly, if a map is an immersion/submersion at all points, we just call it immersion/submersion.
			\begin{Ex}
				Here are two clear examples:
				\begin{itemize}
					\item The inclusion map $\pi:\mathbb{R}^k\rightarrow \mathbb{R}^n$ with $k<n$ and:
					$$\pi(x^1,...,x^k)=(x^1,...,x^k,0,...,0)$$
					is clearly an immersion.
					\item The classical projection $\pi:\mathbb{R}^n\rightarrow \mathbb{R}^k$ with $k<n$ and:
					$$\pi(x^1,...,x^n)=(x^1,...,x^k)$$
					is a submersion.
				\end{itemize}
			\end{Ex}
			Be aware that sometimes an immersion is indexed with the hook:
			$$i:N\hookrightarrow M$$
			\begin{Def}
				Let $f:N\rightarrow M$ be a map between two sets. We call \textit{level set} for some $p\in M$ the set:
				$$f^{-1}(p)=\big\{q\in N\hbox{ such that }f(q)=p\big\}$$
			\end{Def}
			\begin{Def}
				Let $f:N\rightarrow M$ a smooth map between manifolds and $p\in M$. We call $f^{-1}(p)$ a \textit{regular level set} if $f_{*,q}$ is surjective i.e. if $f$ is a submersion in all the level set. 
			\end{Def}
			We now state an important result, whicb will allow us to identify when a level set of a smooth map defines a submanifold.
			\begin{Theo}[\textbf{Regular level set}]
				Let $f:N\rightarrow M$ be a smooth map between manifolds and let $f^{-1}(c)$ be a non empty regular level set. Then
	 			$f^{-1}(c)$ is a submanifold of $N$ of dimension $dim(N)-dim(M)$.			
			\end{Theo}
			\begin{proof}
				Consider a chart of $M$ like $(V,\psi)$ containing $c\in M$. Then, $f^{-1}(V)$ contains $f^{-1}(c)$. Also, susuppose to choose the chart such that $\psi(c)=\vec{0}$. This implies that $(\psi\circ f)^{-1}(0)$ is a regular level set.\\
				Now choose another chart, in $N$, like $(U,\phi=x^i)$, containing $p\in f^{-1}(c)$ and contained in $f^{-1}(c)$. Since by hypothesis we have a regular level set, the Jacobian matrix
				$$\partial F\over \partial x$$
				has maximal rank inside $f^{-1}(V)$. \\
				The idea is now to claim that the couple:
				$$(U,F^1,...,F^{dim(M)},x^{dim(M)+1},...,x^{dim(N)})$$
				is a chart, comaptible with the previous atlas. For simplicity, we call $dim(M)=m, dim(N)=n$. If we look at the Jacobian matrix of this new chart candidate, we have:
				$$J=\begin{pmatrix}
					{\partial F^i\over \partial x^j} & {\partial F^i\over \partial x^\alpha}\\
					{\partial x^\beta\over \partial x^j} & {\partial x^\beta\over \partial x^\alpha}
				\end{pmatrix}$$
				Where the indices $i,j$ go from $1$ to $m$, while $\alpha,\beta$ go from $m+1$ to $n$. This is of the form:
				$$\begin{pmatrix}
					{\partial F^i\over \partial x^j} & {\partial F^i\over \partial x^\alpha}\\
					0 & \mathbb{I}
				\end{pmatrix}$$
				Since the determinant of the up-left submatric is non 0 by the above arguments:
				$$det\big({\partial F^i\over \partial x^j}\big)\neq 0$$
				it means that the whole determinant is not 0. Thus, there is a neighbourhood of $p\in f^{-1}(c)$ in which the map:
				$$(F^1,...,F^{dim(M)},x^{dim(M)+1},...,x^{dim(N)})$$
				is a diffeomorphism in the previous coordinates. This concludes the proof.
			\end{proof}
			Before concluding, a couple of words on immersed submanifolds.
			\begin{Def}
				Let $S\subset M$ be the subset of a smooth manifold. We call $S$ an immersed submanifold if it can be given aa topology and a smooth structure such that $S$ becomes a smooth manifold and the inclusion $i:S\hookrightarrow M$ a smooth immersion.
			\end{Def}
			\begin{Obs}
				We have seen that for regular submanifolds, the subset $S\subset M$ was automatically given the subset topology. For immersed submanifold this hypothesis no longer holds. The topology one gives to $S$ is constructed so that the inclusion becomes a smooth immersion. This makes it so that the topology often differs from the subset topology.
			\end{Obs}
			\begin{Theo}
				Let $f:N\rightarrow M$ be a smooth map between manifolds. If $f$ is an injective immersion, then $F(N)$ can be given a unique topology and a smooth structure that make it an immersed submanifold of $M$.
			\end{Theo}
			\begin{proof}
				The biggest problem is the topology. In general, we do not want the subset topology. We declare a set $V\subset F(N)$ to be open if $f^{-1}(V)$ is open in $N$. The charts we take have the form:
				$$(f(U),\phi\circ f^{-1})$$
				Where $(U,\phi)$ is a chart on $N$. Those charts are compatible by construction so they make a smooth atlas. The topological properties are inherited from the fact that with this topology the restriction $f\big|_{f(N)}$ is an homeomorphism. To conclude the proof, we need to show that the inclusion is a smooth immersion:
				$$i:f(N)\xrightarrow{f^{-1}}N\rightarrow M$$
				The first map is a diffeomorphism and the second is a smooth immersion. Thus the result is a smooth immersion.
			\end{proof}
			\begin{Ex}
				We now make some examples of immersed submanifolds.
				\begin{itemize}
					\item 
				\end{itemize}
			\end{Ex}
	\chapter{Fiber bundles}
		\section{Fiber bundles}
		\section{Vector bundles}
			In this section we will introduce the notion of vector bundles. Intuitively, those are fiber bundles in which the fiber is a vector space.
			\begin{Def}
				Let $E,M$ be manifolds and $\pi:E\rightarrow M$ smooth and surjective. We define \textit{rank $k$ vector bundle} the quadruple $(E,M,\pi,\mathbb{R}^k)$ if the following conditions are met:
				\begin{itemize}
					\item for any $p\in M$, $\pi^{-1}(p)\equiv E_p$ is a vector space of dimension $n$;
					\item for any $p\in M$ there is an open set $U\subset M$ containing $p$ and a diffeomorphic map 
					$$\phi:\pi^{-1}(U)\longrightarrow U\times \mathbb{R}^k$$
					called \textit{local trivialization} such that:
					\begin{itemize}
						\item[1)] at any point $\phi_p=\phi\big|_{\pi^{-1}(p)}$ is a linear isomorphism;
						\item[2)] the following diagram commutes:\\\\
						\begin{center}
							\begin{tikzcd}
								E\arrow{dd}{\pi}\arrow{rr}{\phi}&&U\times \mathbb{R}^k\arrow{ddll}{Pr_1}\\\\
								M&&
							\end{tikzcd}
						\end{center}
						Where $Pr_1(p,v)=p$ is the standard projection. 
					\end{itemize}
				\end{itemize}
				As usual, we call $E$ the \textit{total space} of the bundle, $M$ the  \textit{base space} and $E_p$ the fiber.
			\end{Def}
			The idea is the same one used for fiber bundles: a vector bundle is the information of a manifold that locally (in general only locally) "looks like" a product space. 
			\begin{Def}
				If $E$ is diffeomorphic to $M\times \mathbb{R}^k$ we say that the bundle is \textit{trivial}.
			\end{Def}
			\begin{Ex}
				We make a couple of examples of some vector bundles.
				\begin{itemize}
				\item The \textit{trivial} bundle is $(M\times V,M,\pi,\mathbb{R}^n)$, where $V$ is a $n$-dimensional vector space and $\pi:M\times V\rightarrow M$ is the standard projection. This is clearly a fiber bundle as in REFERENZA and the local trivializations are just the identity map. It is clear that those reduce to linear isomorphisms.
				\end{itemize}
			\end{Ex}
	\section{Sections and transition functions}
		In this section we will briefly look at the notion of section of a vector bundle; and at the behaviour of transition functions. The idea is pretty simple, since to any point of the base space we associate a vector space, changing the trivialization means acting on the fiber with a linear map.
		\begin{Prop}
			Let $(E,M,\pi,\mathbb{R}^k)$ be a vector bundle. Let $\phi_{1,2}:\pi^{-1}(U_{1,2})\rightarrow U_{1,2}\times \mathbb{R}^k$ be two overlapping local trivialzizations. Then there exists a smooth map $g_{12}:U_1\cap U_2\rightarrow GL(n,\mathbb{R})$ such that:
			$$\phi_1\circ \phi_2^{-1}(p,\vec{v})=(p,g(p)\cdot \vec{v})$$
		\end{Prop}
		\begin{proof}
			By defining property, we have:
			$$Pr_1\circ \phi_1=\pi=Pr_1\circ \phi_2$$
			This implies, by invertibility:
			$$Pr_1=Pr_1\circ \phi_1\circ \phi_2^{-1}$$
			Automatically, we have:
			$$\phi_1\circ \phi_2^{-1}(p,\vec{v})=(p,f(p,\vec{v}))$$
			However, by construction, fixing $p\in M$ we have a linear isomorphism. This means that:
			$$f(p,\vec{v})=g(p)\cdot \vec{v}\hbox{ where }g(p)\in GL(k,\mathbb{R})$$
			Since the maps used in the composition are smooth, we also have that $g(p)$ is smooth. To better see this, choose a basis $\{e_i\}$ for $E_p$. Then there is a dual basis $\{e^i\}$ so that:
			$$f(p,\vec{v})=g^a_b(p)v^be_a\implies g^a_b(p)=e^a(f(p,e_b))$$
		\end{proof}
		\begin{Def}
			Let the notation be as above. We call the map $g:U\rightarrow GL(n,\mathbb{R})$ \textit{transition function}.
		\end{Def}
		Clearly, $g_{ij}$ has the following properties:
		\begin{Prop}
			Let the notation be as above, then the transition functions are such that:
			\begin{itemize}
				\item $g_{ij}^{-1}=g_{ji}$;
				\item $g_{ik}\circ g_{kj}=$
			\end{itemize}
		\end{Prop} 
		\begin{Def}
			We call \textit{local section of the bundle $E$} a map $s:U\subset M\rightarrow E$ such that: 
			$$\pi\circ s=\mathbb{I}$$
			We call the space of smooth sections on a bundle $\Gamma(E)$.
		\end{Def}
		The notion of section is extremely important and thus i feel like it is necessary to have a sort of physical interpretation for it. The idea is to "cut" the bundle. A section provides an element of the fiber to any point on its domain. Thus, it sorts of "cuts trajectories" in the bundle.
		\begin{Def}
			We call a \textit{local frame} for a bundle $E$ a set of sections $\{s_i:U\subset M\rightarrow E\}$ that form a base for $E_p$ at any $p\in M$.
		\end{Def}
		\begin{Prop}
			A vector bundle is trivial if and only if it has a smooth global frame.
		\end{Prop}
		\begin{proof}
			Suppose that the vector bundle is trivial i.e. there is a diffeomorphism $\psi:E\rightarrow M\times \mathbb{R}^k$. Then, picking any base $\{r_i\}$ for $\mathbb{R}^k$, we define the frame:
			$$s_i(p)=\psi^{-1}(p,r_i)$$
			Now, suppose to have a global frame $\{s_i\}$. This means that any $e\in E$ can be written like: $e=e^i(p)s_i(p)$ at any point. Then we can construct a global trivialization:
			$$\psi:E\rightarrow M\times \mathbb{R}^k$$
			$$\psi(e)=(\pi(e),e^1,...,e^k)$$
			This has a clear inverse and clearly reduces to an isomorphism on the fibers. The smoothness is free since $\pi$ is smooth by assumption.
		\end{proof}
		\section{The tangent bunlde}
			This section is devoted to the study of the most important example of vector bundle: the tangent bundle. We will go through its definition and we will see how we can describe vectors by the means of it.
			\begin{Def}
				We define the\textit{ tangent bundle} as the set:
				$$TM=\bigcup_{p\in M}T_pM$$
			\end{Def}
			The idea is simple, to glue together all of the possible tangent spaces. However, as of now, this is just a set.
			\begin{Theo}
				Let $M$ be a smooth manifold of dimension $n$. The tangent bundle $TM$ is a manifold of dimension $2n$ and describes a vector bundle on $M$ with fiber $\mathbb{R}^n$. 
			\end{Theo}
			\begin{proof}
				The proof will develop as follows: first of all we will show that $TM$ is a smooth manifold i.e. prove that it has a nice topology and a smooth structure; then we will look at the trivializations to show that it is a bundle.\\
				\\
				Let $(U,\phi)$ be a chart on $M$. This induces a base for $T_pM$ at any $p\in U$ like $\{\partial_i\}$. Thus, we can define the following map:
				$$\Phi:\pi_{TM}^{-1}(U)\rightarrow \mathbb{R}^{2n}$$
				$$\Phi(v_p)=(\phi(p),v^i(p)\partial_i\big|_p)$$
				This map has a clear inverse:
				$$\Phi^{-1}(\vec{x},\vec{v})=v_p\hbox{ such that }v_p=v^i(p)\partial_i\big|_p\hbox{ and }p=\phi^{-1}(\vec{x})$$
				We define the topology of $TM$ in such a way that this map becomes a homeomorphism: we call a set $A\subset TU$ open in $TU$ iff $\Phi(A)$ is open in $\mathbb{R}^{2n}$.\\
				\\
				The next step is not really strictly necessary but helps with the formality.\\
				\\
				The aim is now to biuld a base for this induced topology.
				Let $\mathcal{A}=\{(U_i,\phi_i)\}$ be an atlas for $M$ and define the following set:
				$$\mathcal{C}=\big\{A\subset TM\hbox{ such that }A \hbox{ is open in }TU_i\big\}$$
				\begin{Lm}
					If $A_{1,2}$ are open in $TU_{1,2}$, then $A_1\cap A_2$ are open in $T(U_1\cap U,2)$.
				\end{Lm}
				\begin{proof}
					This is easy since clearly, due to $A_{1,2}\subset TU_{1,2}$, $A_1\cap A_2\subset T(U_1\cap U_2)$ and, from the definition of subset topology, $A_{1,2}\cap T(U_1\cap U_2)$ are open. Automatically:
					$$A_1\cap A_2\cap T(U_1\cap U_2)=[A_1\cap T(U_1\cap U_2)]\cap [A_2\cap T(U_1\cap U_2)]$$
					Those are intersections of open sets and so are open.
				\end{proof}
				From REFERENZA, $\mathcal{C}$ is a basis for some topology since:
				\begin{itemize}
					\item All elements of $\mathcal{C}$ cover $TM$ by construction;
					\item for any $v_p\in A_1\cap A_2$ there is an open $A$ such that $v_p\in A\subset A_1\cap A_2$.
				\end{itemize}
				The second quality naturally descends from the lemma.\\
				\\
				Consider now a countable basis (from REFERENZA it exists) of charts for $M$ like $\{U_i\}$. Every one of those will define $TU_i$ which, being homeomorphic to $\mathbb{R}^{2n}$, is second countable. Since there is a countable number of $U_i$, we have a second countable basis for $TM$.\\
				\\
				The T2 quality immediately follows from the homeomorphism witb $\mathbb{R}^{2n}$.\\
				\\
				As for the trivializations, consider the map:
				$$\psi=(\phi^{-1}\times \mathbb{I})\circ \Phi:\pi_{TM}^{-1}(U)\rightarrow U\times \mathbb{R}^n$$
				This clearly reduces to a vector space isomorphism on the fibers. Moreover, it is clearly diffeomorphic. 
			\end{proof}
			\begin{Cor}
				Any smooth manifold $M$ of dimension $n$ has a rank $n$ vector bundle:
				$$(TM,M,\pi_{TM},\mathbb{R}^n)$$
			\end{Cor}
			\begin{center}
				$$
				\begin{tikzcd}
					&&U\times\mathbb{R}^n&&\\\\
					TM\arrow{dd}{\pi_{TM}}\arrow{rr}{}&&TU\arrow{dd}{\pi_{TM}}\arrow{rr}{\Phi}\arrow{uu}{\psi}&&\mathbb{R}^{2n}\\\\
					M\arrow{rr}{}&&U\arrow{rr}{\phi}&&\mathbb{R}^n
				\end{tikzcd}
				$$
			\end{center}
			Clearly, in the above diagram, the arrows without name are inclusions.
			\begin{Def}
				We define a \textit{vector field} as a section of the tangent bundle $\Gamma(TM)$ and we identify this space with $\mathfrak{X}(M)$.
			\end{Def}
		\section{The cotangent bundle}
		In this section we define yet another important object: the cotangent bundle. This can be thought of as the dual of the tangent bundle. We are going to see some properties of this object and study covectors or differential 1-forms. In some sense, the 1-forms will be the actual protagonist of our journey.
		\begin{Obs}
			At any point $p\in M$ we have a vector space $T_pM$ of dimension $n$. This implies the existence of a dual space $T_pM^* $ at any point.
		\end{Obs}
		\begin{Def}
			We define the \textit{co-tangent} bundle as the set 
			$$TM^*=\bigcup_{p\in M}T_pM^*$$
		\end{Def}
		\begin{Theo}
			The tangent bundle is a smooth manifold and defines a rank $n=dim (M)$ smooth vector bundle on $M$.
		\end{Theo}
		\begin{proof}
			The proof is identical to the one used in REFERENZA for the tangent bundle.
		\end{proof}
		\begin{Obs}
			Note that once we choose a base for the tangent base $\{{\partial \over \partial x^i}\}$ i.e. a chart $(U,\phi)$, we automatically have an isomorphism between $T_pM$ and $T_pM^*$ at each point in the set $U$. This implies the definition of a dual base: $dx^i$ such that:
			$$dx^i({\partial\over \partial x^j})=\delta^i_j$$
			Note that in general there is no canonical isomorphism between $T_pM$ and $T_pM^*$. In order for have a canonical identification we would need a product... we'll see what we can do.
		\end{Obs}
		\begin{center}
			$$
			\begin{tikzcd}
				&&U\times\mathbb{R}^n&&\\\\
				TM\arrow{dd}{\pi_{TM}}\arrow{rr}{}&&TU\arrow{dd}{\pi_{TM}}\arrow{rr}{\Phi}\arrow{uu}{\psi}&&\mathbb{R}^{2n}\\\\
				M\arrow{rr}{}&&U\arrow{rr}{\phi}&&\mathbb{R}^n\\\\
				TM^*\arrow{rr}{}\arrow{uu}{\pi_{TM^*}}&&TU\arrow{rr}{\Phi^*}\arrow{dd}{\psi^*}\arrow{uu}{\pi_{TM^*}}&&\mathbb{R}^{2n}\\\\
				&&U\times \mathbb{R}^n&&
				\end{tikzcd}
				$$
			\end{center}
		Clearly, in the above diagram, the dual trivializations and charts are indexed with the (*) symbol. Also, the projection $\pi_{TM^*}:TM^*\rightarrow M$ acts in the naive way $\pi_{TM^*}(\omega_p)=p$
		\begin{Def}
			We define a \textit{covector field} of a \textit{differential 1-form} as a section of the co-tangent bundle:
			$$\omega\in \Gamma(TM^*)$$
		\end{Def}
		\begin{Obs}
			Clearly, in any coordinate frame, we have the following decomposition:
			$$\omega=\omega_i dx^i$$
			where, being smooth, $\omega_i:M\rightarrow \mathbb{R}$ are smooth maps.\\
			Also, due to the action of dual vectors on vectors, given a differential 1-form  $\omega$ and a vector field $X$, we have the following smooth map:
			$$\omega(X):M\rightarrow \mathbb{R}\hbox{ such that }\omega(X)(p)=\omega_pX_p$$
		\end{Obs}
		\begin{Exe}
			Prove that the above mapping
			$$\omega(X):M\rightarrow \mathbb{R}\hbox{ such that }\omega(X)(p)=\omega_pX_p$$
			is smooth.\\
			\\
			This is really easy and amounts to choosing one single chart $(U,\phi)$. This induces some cooridnates on $U\subset M$ like $x^i$, on $TU\subset TM$ like $\partial_i$ and on $TU^*\subset TM^*$ like $dx^i$. At this point, we have:
			$$\omega_pX_p=\omega_i(p)dx_p^i(X^j(p)\partial_j\big|_p)=\omega_i(p)X^i(p)$$
			Done.
		\end{Exe}
		\begin{Def}
			Let $f:M\rightarrow \mathbb{R}$ be a smooth map. We define its \textit{differential} as a differential 1-form defined like:
			$$df(X)_p=X_pf$$
		\end{Def}
		\begin{Obs}
			Clearly, in coordinates:
			$$df=\partial_i fdx^i$$
			This is smooth as well.
		\end{Obs}
		\begin{Obs}
			Note that the differential amounts to taking the gradient. Recall also that the push-forward corresponded to calculating the Jacobian of the function. This justifies an abuse of notation for the pushforward of maps like $F:M\rightarrow N$. In coordinates sometimes it is written:
			$$F_*\equiv dF$$
			Basically in coordinates taking the push-forward is the same thing as taking the gradient of all of the layers of the function. 
		\end{Obs}
		\begin{Def}
			Let $F:M\rightarrow N$ be a smooth manp between manifolds. We define the \textit{pull-back through $F$} as a map:
			$$F^*:T_pN^*\rightarrow T_pM^*$$
			such that for any 1-form at a point $\omega_{F(p)}\in T_{F(p)}N$, we have that $F^*\omega_{F(p)}$ is a 1-form at $p$:
			$$(F^*\omega_{F(p)})(X_p)=\omega_{F(p)}(F_*X_p)$$ 
		\end{Def}
		\section{Curves in a manifold}
		In this section we briefly go through the notion of curve in a smooth manifold. We will see that to any point in a curve we can associate a tangent vector. There is an obvious physicsl interpretation to this: the tangent vector is the velocity.
		\begin{Def}
			We define a smooth curve in a manifold as a smooth map $\gamma:I\subset\mathbb{R}\rightarrow M$.
		\end{Def}
		\begin{Obs}
			Let $\gamma:]-\varepsilon,\varepsilon[\rightarrow M$ be a smooth curve and $p=\gamma(0)$ a point. Let $(U,\phi)$ be a chart around $p$. Then, for any map $f:M\rightarrow \mathbb{R}$ smooth, we can evalue how it behaves when restricted to the curve:
			$${d\over dt}(f\circ \gamma)=\gamma_*(\partial_t)f={d\over dt}(f\circ \phi^{-1}\circ \phi\circ \gamma)=$$
			$$={d\over dt}(\phi\circ\gamma)^i\partial_i\big|_{\gamma(t)}f$$
			We define the \textit{tangent vector to $\gamma$ in $p$} the following:
			$${d\over dt}\bigg|_0(\phi\circ \gamma)^i\partial_i=\hat{\gamma}(t)$$
		\end{Obs}
		\begin{Prop}
			Let $p\in M$ be a point and $X_p\in\mathfrak{X}(M)$. Then $X_p$ is a tangent vector to some curve in $p$.
		\end{Prop}
		\begin{proof}
			Consider a chart $(U,\phi)$ around $p$. We want to find a curve $\gamma$ such that:
			$${d\over dt}\bigg|_0(\phi\circ \gamma)=X_p$$
			This is easy: take $\phi\circ \gamma(t)=(tX^1,...,tX^n)$.
			Done.
		\end{proof}
		\begin{Def}
			We call $\gamma:I\rightarrow M$ an \textit{integral curve} of the vector field $X$ if $\hat{\gamma}(t)=X_{\gamma(t)}$.
		\end{Def}
		\begin{Prop}
			Given a vector field $X$ and a point $p\in M$, there is always a unique integral curve of $X$ around $p$.
		\end{Prop}
		\begin{proof}
			This is easy. Finding an integral curve means finding, in a coordinate chart, a curve $\phi\circ \gamma (t)=(c^1(t),...,c^n(t))$ such that:
			$$\begin{cases}
				{d\over dt}(\phi\circ \gamma)^i\partial_i=X^i\partial_i\\
				\gamma(0)=p
			\end{cases}$$
			This is a Cauchy problem and it is known from calculus that there is a unique solution in a neighbourhood of $\phi(p)$.
		\end{proof}
		\section{The pullback bundle for vector bundles}
		\section{Tensors}
		In this section we define the notion of tensor and tensor field on a smooth manifold and look at some of their properties.
		\begin{Obs}
			We have seen that $TM=\bigcup_{p\in M}T_pM$ and $TM^*=\bigcup_{p\in M}T_pM^*$ are vector bundles on $M$. Since at any point $T_pM$ is a vector space, we can define the tensor product $\bigotimes^k T_pM$ and equivalently for its dual. This automatically implies that at any point, havind selected a chart $(U,\phi)$ for $M$, there are some induced basis like $\{\partial_{i_1}\otimes...\otimes \partial_{i_k}\}$ and $\{dx^{i_1}\otimes...\otimes dx^{i_k}\}$. By the same identical pricedure used for the tangent bundle, we can define the following vector bundles:
			$$TM_k=\bigsqcup_{p\in M} \bigotimes^k T_pM$$
			$$ T^*M^k=\bigsqcup_{p\in M} \bigotimes^k T_pM^*$$
			$$TM^l_k=\bigsqcup_{p\in M} \bigotimes^k T_pM\bigotimes^l T_pM^*$$
			$$\bigwedge^k M=\bigsqcup_{p\in M} \bigwedge^k T_pM$$
		\end{Obs}
		\begin{Def}
			We call $(k,l)$ tensor a section of $TM^k_l$. This is a map $T:M\rightarrow TM^l_k$
		\end{Def}
		In a given coordinate chart we have:
		$$T=T_{i_1,...,i_l}^{j_1,...,j_k}dx^{i_1}\otimes...\otimes  dx^{i_l}\otimes \partial_{j_1}\otimes...\otimes \partial_{j_k}$$
		We can also extend the notion of pull-back in the obvious way.
		\begin{Def}
			Let $F:N\rightarrow M$ be a smooth map between manifolds. We define the \textit{pull-back} of a $k$-form $\omega\in\Omega^k(N)$ as:
			$$F^*_p\omega (X^1,...,X^k)=\omega_{F(p)}(F_*X^1,...,F_*X^k)$$
		\end{Def}
		The pull-back has some important properties.
		\begin{Prop}
			The following properties hold:
			\begin{itemize}
				\item[(1)] $F^*$ is $\mathbb{R}$-linear;
				\item[(2)] $F^*(f\omega)=(f\circ F)F^*\omega$;
				\item[(3)] $(F\circ G)^*=G^*\circ F^*$;
				\item[(4)] $F^*(\omega\otimes \rho)=F^*\omega\otimes F^*\rho$.
			\end{itemize}
		\end{Prop}
		\begin{proof}
			The proof is just a straightforward calculation.
			\begin{itemize}
				\item[(1)] Obvious.
				\item[(2)]  $F^*(f\omega)(X_1,...,X_k)=(f\omega)_F(p)(F_*X_1,...,F_*X_k)=(f\circ F(p))\omega_{F(p)}(F_*X_1,...,F_*X_k)$.
				\item[(3)] $$(F\circ G)^*\omega(X_1,...,X_k)=\omega(F_*\circ G_*(X_1),...,F_*\circ G_*(X_k))=$$$$=F^*\omega(G_*(X_1),..., G_*(X_k))=G^*\circ F^*\omega(X_1,...,X_k)$$
				\item[(4)] Obvious.
			\end{itemize}
		\end{proof}
		\begin{Def}
			We denote the space of smooth sections of the bundle $\bigwedge^k M$ like:
			$$\Omega^k(M):=\Gamma(\bigwedge^k M)$$
		\end{Def}
		\begin{Exe}
			\begin{itemize}
				\item Consider two vector fields $X,Y\in\mathfrak{X}(M)$. Prove that their commutator is still a vector field on $M$.
				\\
				\\
				This is easy in local coordinates:
				$$X=X^i\partial_i\hspace{20 pt}Y=Y^j\partial_j$$
				$$[X,Y]=X^i\partial_iY^j\partial_j-Y^j\partial_jX^i\partial_i+X^iY^j\partial_i\partial_j-Y^jX^i\partial_j\partial_i$$
				Due to the fact that partial derivatives commute, by renaming the indices we have:
				$$[X,Y]=(X^j\partial_jY^i-Y^j\partial_jX^i)\partial_i$$
				\item Consider the map $f:\mathbb{R}^3\rightarrow \mathbb{R}^2$ like $f(x,y,z)=(x-y,y^2+z)$. Consider then the form:
				$\omega=(2xy+x^2+1)dx+(x^2-y)dy$. Compute the pullback $f^*\omega$.\\
				\\
				By definition:
				$$f^*\omega=(\omega_i\circ f)f^*dx^i$$
				This means:
				$$\omega_x\circ f(x,y,z)=\omega_x(x-y,y^2+z)=2(x-y)(y^2+z)+(x-y)^2+1=$$
				$$=2xy^2+2xz-2y^3-2yz+x^2+y^2-2xy+1=$$
				$$=x^2+2xy(y-1)+2xz-2yz+y^2(1-2y)+1$$
				$$\omega_y\circ f(x,y,z)=\omega_y(x-y,y^2+z)=(x-y)^2-y^2-z=$$
				$$=x^2-2xy-z$$
				As for the basis elements, we have:
				$$F^*dx^i(\tilde{\partial_j})=dx^i(F_*\tilde{\partial_j})$$
				$$F_*\tilde{\partial_j}(g)=\tilde{\partial_j}(g\circ F)={\partial g\over\partial x^i}\tilde{\partial_j} F^i$$
				In turn this implies:
				$$F^*dx^i(\tilde{\partial_j})=dx^i(\tilde{\partial_j}F^k\partial_k)=\tilde{\partial_j}F^i$$
				Finally, we note that:
				$$d(x^i\circ F)(\tilde{\partial_j})=\tilde{\partial_j}(x^i\circ F)=\tilde{\partial_j}F^i$$
				Thus, it makes sense to write:
				$$d(x^i\circ F)=F^*dx^i$$
				Going back to our calculation, we have:
				$$f^*dx=d(x-y)=dx-dy\hspace{20 pt} f^*dy=2ydy+dz$$
				One then puts everything together.
			\end{itemize}
		\end{Exe}
		\section{The Frobenius theorem}
		In this section we study foliations and distributions. The final aim is to prove the Frobenius theorem, an extremely important result in smooth geometry.
		\begin{Def}
			We call \textit{distribution} $D$ on a manifold $M$ a smooth assignment $p\mapsto D_p\subset T_pM$ such that: for any $p$ one can find $U$ open containing $p$ such that for all $q\in U$ there is a collection of vector fields $\{X_i(q)\}$ that span $D_q$.
		\end{Def}
		Clearly, a distribution defines a subset of the tangent bundle $TM$.
		\begin{Def}
			A distribution is called regular of a \textit{constant rank} if the dimension of all spaces is the same. 
		\end{Def}
		\begin{Prop}
			A constant rank distribution defines a sub-bundle of the tangent bundle.
		\end{Prop}
		\begin{proof}
			This is clear from theorem REFERENZA. (L.TU 20.4 Geometry).
		\end{proof}
		\begin{Def}
			We call a distribution \textit{involutive} if $[D,D]\subset D$ i.e. it is closed under the Lie braket.
		\end{Def}
		It is useful to see some easy examples of distributions.
		\begin{Ex}
			\begin{itemize}
				\item Consider $\mathbb{R}^n$ with the usual smooth structure. We can easily see that $\{{\partial\over \partial x^1},...,{\partial\over \partial x^k}\}$ with $k\leq n$ is a constant rank distribution.
				\item Let $X$ be a nowhere vanishing vector field on $M$. Then, at any point $p\in M$ we have a rank 1 distribution defined as $p\mapsto X_p$.
			\end{itemize}
		\end{Ex}
		\begin{Def}
			We say that a distribution $D\subset TM$ is \textit{completely integrable} if for any $p\in M$ there is a chart $(U,\phi)$ of $M$ such that:
			\begin{itemize}
				\item $\phi(U)$ is a cube in $\mathbb{R}^n$;
				\item $D$ is spanned by the first $k$ coordinate vector fields $\{{\partial\over \partial x^1},...,{\partial\over \partial x^k}\}$.
			\end{itemize}
			A chart with those qualities is called \textit{flat} for $D$.
 		\end{Def}
 		\begin{Obs}
 			Clearly, in any flat chart, we have some submanifolds, defined by keeping the corresponding coordinates fixed.
 		\end{Obs}
 		\begin{Theo}[\textbf{Frobenius}]
 			A distribution is involutive if and only if it is completely integrable.
 		\end{Theo}
 		\begin{proof}
 			The proof is hard. We divide it into 2 blocks:
 			\begin{itemize}
 				\item Suppose $D$ is an involutive distribution. We first prove that the distribution admits a base of commuting vectors. Let $(U,\phi)$ be any chart around any point and let $\{\partial_i\}$ be the induced base for the tangent space. Let $Y_1,...,Y_k$ be a base for $D$ in this open set, then:
 				$$Y_i=a^j_i\partial_j$$
 				where the indices $j$ go from $0$ to $n$, while $i$ from $0$ to $k$.\\
 				Since the vectors $Y_i$ are by assumption linearly independent, the submatrix $a^i_j$ is invertible in the indices $i,j\in[1,k]$. Thus, setting $X_i=(a^{-1})^j_i Y_j$ we get new basis of vectors, which are of the form:
 				$$X_i=\partial_i+(a^{-1})^j_i a^l_j \partial_l$$
 				where in the last sum we have $l>k$.
 				Clearly, since the distribution is involutive, we must have:
 				$$[X_i,X_j]\in D$$
 				However, by construction, $[X_i,X_j]\in span\{\partial_{k+1},...,\partial_n\}$. This implies that $X$ is the desired base of commuting vectors.\\
 				\\
 				We are thus allowed to choose a chart $(U,\phi)$ in which the base for the distribution is made of commuting vevtors. In such base, those vector field are pushed forward to base coordinates:
 				$$\phi_*(X_i)=\partial_i$$
 				This proves the first part.
 				\item Suppose that the distribution $D$ is completely integrable. Take a point $p\in U\subset M$ and a chart around it $(U,\phi)$. Since the distribution is integrable, we assume that $\phi$ is a flat chart. This means that the tangent distribution space at any point $D_p$ is spanned by the first $k$-coordinates, which commute:
 				$$[\partial_i,\partial_j]=0$$
 				Since $\phi$ is a diffeomorphism, we have:
 				$$[\phi^{-1}_*\partial_i,\phi^{-1}_*\partial_j]=\phi_*[\partial_i,\partial_j]=0$$
 				This proves that the distribution is involutive.
 			\end{itemize}
 		\end{proof}
 		\begin{Def}
 			Let $M$ be a smooth manifold. We call \textit{$k$-dimensional foliation} $\mathcal{F}$ a family of $k$-dimensional submanifolds of $M$ such that:
 			\begin{itemize}
 				\item the elements of $\mathcal{F}$ are disjoint, connected, non empty, immersed $k$-dimensional submanifolds;
 				\item for any $p\in M$ there is a chart $(U,\phi)$ such that $\phi(U)$ is a cube in $\mathbb{R}^n$ and each of the elements of $\mathcal{F}$ intersects $U$ in either the empty set or a slice defined by keeping $n-k$ cooridnate constant: $x_{k+1}=Const,...x^n=Const$.
 			\end{itemize}
 			Such a chart is called \textit{flat} for $\mathcal{F}$. The elements of the foliation are called \textit{leaves}.
 		\end{Def}
 		\begin{Ex}
 			\begin{itemize}
 				\item The collection of hyperplanes $x_{k+1}=Const,...x^n=Const$ is a foliation of $\mathbb{R}^n-\{0\}$.
 				\item The family of spheres of increasing radius $S^{n-1}_r$ is a foliation of $\mathbb{R}^n$. This is pretty easy to prove: first of all, the spheres do not intersect, are all connected, non empty and are all immersed. As for the flat chart problem, take any point $\vec{x}\in \mathbb{R}^n$ and a ball around it of radius $\varepsilon$. Clearly, some of the spheres will intersect this ball and, in spherical coordinates, the radius coordinate is kept constant.
 			\end{itemize}
 		\end{Ex}
 		\begin{Obs}
 			It is not hard to see that given a completely integrable distribution we automatically have a foliation. In fact, each flat chart defines a family of submanifolds (locally). Those do not intersect since each one correspond to keeping fixed some coordinate of the base manifold $M$. The connectedness is found by taking the maximally connected generated submanifold. In other words:\\
 			\\
 			Suppose to take a point $p$ in $M$ and a flat chart $(U,\phi)$ for the distribution $D$. Then, i can define inside $U$ the submanifold corresponding to $x^{k+1}=0,...,x^n=0$. Taking another overlapping chart $(V,\psi)$ i can do the same if I choose it to be flat. By repeating the same argument for all the connected component of $p$, I get a connected submanifold.
 		\end{Obs}
 		As a corollary of Frobenius theorem we state:
 		\begin{Cor}
 			$D$ is an involutive distribution of constant rank if and only if it defines a foliation.
 		\end{Cor}
	\chapter{Lie theory}
	In this chapter we discuss Lie theory. We will go through the basic definitions and properties of Lie groups, Lie algebras and some representation theory. All of those are at the core of theoretical physics, as they represent the main instruments of particle physics and are at the core of almost all other theories.
	\section{Lie groups}
	In this section we treat Lie groups. This is one of the most important notions in differential geometry as it is the core object in which we will encapsulate symmetries. The idea behind Lie groups is not difficult at all: it is a group which is also a smooth manifold. There are many ways to see and interpret this. One of the most common intuitive thoughts of a Lie group is that of a group in which is possible to move smoothly accross all group elements. Even though this interpretation is to some satisfactory it is mathematically incorrect and incredibly imprecise. My personal suggestion is to see Lie groups without fantasy: it is just a group, which is also a manifold i.e it has nice topological properties and a smooth structure. It is worth mentioning that Lie groups are "a special kind of manifold". We are in fact going to see that the group structure automatically makes it so that the tangent bundle is "parallelizable" i.e. it is trivial. Before we begin, please be aware that the theory behind Lie group is incredibly big and much more complicated that it actually may look like from the following pages. We are just going through a brief introductory chapter which will be more than sufficient to develop our physical analysis.
	\begin{Def}
		A \textit{Lie group} $G$ is a group endowed with a manifold structure, such that:
		\begin{itemize}
			\item the multiplication operation $\mu:G\times G\rightarrow G$ like $\mu(g,h)=g\cdot h$ is smooth;
			\item the inversion operation $i:G\rightarrow G$ like $i(g)=g^{-1}$ is also smooth
		\end{itemize}
	\end{Def}
	\begin{Def}
		A \textit{Lie subgroup} $H$ of a Lie group $G$ is an immersed submanifold of $G$ which is still a group under the induced operations from $G$.
	\end{Def}
	\begin{Ex}
		Let us make some examples of Lie groups.
		\begin{itemize}
			\item The circle $S^1$ is a Lie group. This is clearly a smooth manifold from REFERENZA and it is endowed with the following operation:
			$$\mu(e^{i\theta},e^{i\omega})=e^{i(\theta+\omega)}$$
			This is clearly a smooth operation and it has a clear smooth inverse:
			$$i(e^{i\theta})=e^{-i\theta}$$
			\item The group $GL(n,\mathbb{R})$ was proved in REFERENZA to be a smooth manifold. This has an obvious multiplication operation: the row-columns multiplication:
			$$\mu(A,B)=A\cdot B$$
			This operation is smooth in coordinates. To convince ourselves of this, note that the entries of the resulting matrices are:
			$$(AB)^i_j=A^i_kB^k_j$$
			This is a polynomial, thus it is smooth.\\
			As for the inverse, every matrix in $GL(n,\mathbb{R})$ is invertible, so that this multiplication allows for a well defined operation:
			$$i:A\mapsto A^{-1}$$
			This is smooth as well since the entries of the inverse matrix are a polynomial of the entries of the first one. This can be easily seen by applying Gauss method.
			\\
			This proves that the real general linear group is a Lie group.
			\item Consider the special linear group 
			$$SL(n,\mathbb{R})=\big\{A\in GL(n,\mathbb{R})\hbox{ such that } det(A)=1\big\}$$
			This is clearly a subgroup of $GL(n,\mathbb{R})$. From REFERENZA we have that it is an immersed submanifold. Now it remains to show that the group operations behave well.
			\\
			\\
			Consider the multiplication. Clearly, the determinant
		\end{itemize}
	\end{Ex}
	\begin{Def}
		Given a Lie group $G$, we define the \textit{left} and \textit{right mutliplication by} $g\in G$ as:
		$$\ell_g:G\rightarrow G\hbox{ like }\ell_g(h)=g\cdot h$$
		$$r_g:G\rightarrow G\hbox{ like }r_g(h)=h\cdot gh$$
		We also define the \textit{conjugation} as the combination:
		$$c_g=\ell_g\circ r_{g^{-1}}$$
	\end{Def}
	\begin{Obs}
		Clearly: 
		$$Ad_g=\ell_g\circ r_{g^{-1}}=r_{g^{-1}}\circ \ell_{g}$$
	\end{Obs}
	\begin{Exe}
		Prove that $d\ell_g$ is an isomorphism.
		\\\\
		This is easy: it is sufficient to prove that $\ell_g$ is a diffeomorphism.\\
		\\
		By definition:
		$$\ell_g(h)=g\cdot h=\mu(g,h)$$
		This is smooth by definition. Then, we have an inverse, which is smooth as well:
		$$\ell_{g^{-1}}=g^{-1}\cdot h$$
		Those are clear inverses of one another:
		$$\ell_g\circ \ell_{g^{-1}}(h)=g\cdot g^{-1}\cdot h=h$$
		This proves the claim from REFERENZA. The same holds for $r_g$.  
	\end{Exe}
	\section{Lie algebras}
	\chapter{Representation theory}
	\chapter{Principal bundles}
	\chapter{Ehresmann geometries}
	\chapter{Cohomology}
	\section{De Rham Cohomology}
	In this section we give the definition of the De Rham Cohomology and prove some important results concerning this theory.
	\begin{Def}
		Let $M$ be a smooth manifold. We define the \textit{De Rham Cohomology of degree $p$} in $M$ as the quotient set:
		$$H_{dR}^p(M)={Ker(d:\Omega^p(M)\rightarrow \Omega^{p+1}(M))\over Im(d:\Omega^{p-1}(M)\rightarrow \Omega^{p}(M))}={\hbox{exact forms }\over \hbox{closed forms}}$$ 
	\end{Def}
	Elements of $H_{dR}^p(M)$ are equivalence classes, whoose elements differ by a closed form.
	\begin{Obs}
		$H_{dR}^p(M)$ is clearly a vector space under addition of forms. Moreover, since $\Omega^p(M)=0$ for $p>dim(M)$ we have that:
		$$H^p_{dR}(M)=0\hbox{ for }p<0,p>dim(M)$$
	\end{Obs}
	Now we look at some properties of the De Rham Cohomology.
	\begin{Prop}
		If $F.N\rightarrow M$ is a smooth map between manifolds, there is a linear map $F^*:H^p_{dR}(N)\rightarrow H^p_{dR}(N)$.
	\end{Prop}
	\begin{proof}
		The pullback $F^*$ of a form commutes with the exterior derivative, so that:
		\begin{itemize}
			\item if $\omega$ is closed then $dF^*\omega=F^*d\omega=0$ is closed;
			\item if $\omega$ is exact then $F^*\omega=F^*d\eta=dF^*\eta$ is exact.
		\end{itemize}
		By abuse of notation, define:
		$$F^*:H^p_{dR}(N)\rightarrow H^p_{dR}(N) \hbox{ like }F^*[\omega]=[F^*\omega]$$
		This is clearly well defined.
	\end{proof}
	There is an immediate important consequence to this result:
	\begin{Theo}
		If two manifolds are diffeomorphic then their De Rahm Cohomologies are isomorphic.
	\end{Theo}
	\begin{Prop}
		If $M$ is a connected smooth manifold then the $H^0_{dR}(M)$ is one dimensional and equal to the space of constant functions.
	\end{Prop}
	\begin{proof}
		Clearly, by definition $H^0_{dR}(M)=Ker(d:\Omega^0(M)\rightarrow\Omega^1(M))$ which is the space of constant functions $df=0$.
	\end{proof}
	There is another extremely important result:
	\begin{Theo}
		The De Rahm Cohomology spaces are topological invariants.
	\end{Theo}
	This can be seen as a corollary of the following proposition:
	\begin{Prop}
		The De Rahm Cohomology spaces are homotopy invariants.
	\end{Prop}
	\begin{proof}
		Let $F:M\rightarrow N$ be an homotopy with inverse $F^{-1}$. By the Whitney approximation theorem, there exist $\tilde{F}:M\rightarrow N,\tilde{F}^{-1}N\rightarrow M$ smooth and homotopic to $F,F^{-1}$. By REFERENZA, the theorem is proved.
	\end{proof}
	\section{The De Rham Theorem}	
	\chapter{Characteristic classes}
	\section{Characteristic classes of Vector Bundles}
	In this section we will look at the construction of topological invariants for vector bundles.
	\begin{Def}
		Let $V$ be a $n$-dimensional vector space. We call \textit{polynomial} of degree $k$ on $V$ any element $f\in Sym^k(V^*)$.
	\end{Def}
	\begin{Def}
		If $\mathfrak{g}$ is the Lie algebra of $G$ and $f:\mathfrak{g}\rightarrow \mathbb{R}$ is a polynomial of degree $k$, then $f$ is said to be $\rho$-invariant if
		$$f(\rho(g)X)=f(X)$$
		for all $g\in G$ and $X\in \mathfrak{g}$.
	\end{Def}
	\begin{Theo}[Chern-Weyl]
		Let $(E,M,\pi,\mathbb{R}^r)$ be a vector bundle, $\nabla$ a connection and $\Omega$ its curvature. If $f$ is an $Ad$-invariant homogeneous polynomial on $\mathfrak{gl}(n,\mathbb{R})$ then:
		\begin{itemize}
			\item[i)] $f(\Omega)$ is closed;
			\item[ii)] The cohomology class $[f(\Omega)]$ is independent of the connection $\nabla$.
		\end{itemize}
	\end{Theo}
	\begin{proof}
		It is sufficient to prove this for the trace polynomial, as it can be proven that the trac epolynomial generates every other polynomial in $\mathfrak{gl}(n,\mathbb{R})$.\\
		\begin{itemize}
			\item[i)] Clearly $dtr(\Omega)=tr(d\Omega)=0$ by symmetry.
			\item[ii)] Consider two connections $\nabla,\nabla'$ and define:
			$$\nabla_t=\nabla+t(\nabla'-\nabla )=\nabla+t\xi\hbox{ with } t\in[0,1]$$
			Clearly, the connection matrics follow the same rule:
			$$\omega_t=\omega+t(\omega'-\omega)=\omega+t\omega_\xi$$
			Let $\Omega_t$ be the curvature of $\nabla_t$. Now, consider the following calculations:
			$${d\over dt}Tr(\Omega_t,...,\Omega_t)=kTr({d\over dt}\Omega_t,...,\Omega_t)$$
			Now, $\Omega_t=d\omega_t+\omega_t\wedge\omega_t$. This immediately implies:
			$${d\over dt}\Omega_t=d\dot{\omega}_t+\dot{\omega}_t\wedge \omega_t+\omega_t\wedge\dot{\omega}_t$$
			Now, feeding this into the trace polynomial, we find:
			$$Tr({d\over dt}\Omega_t,...,\Omega_t)=Tr(d\dot{\omega}_t+\dot{\omega}_t\wedge \omega_t+\omega_t\wedge\dot{\omega}_t ,\Omega_t...,\Omega_t)=$$
			$$=Tr(d\dot{\omega}_t\wedge \Omega_t^{k-1}+\dot{\omega}_t\wedge \omega_t\wedge \Omega_t^{k-1}+\omega_t\wedge\dot{\omega}_t\wedge \Omega_t^{k-1})=$$
			$$Tr(d\dot{\omega}_t\wedge \Omega_t^{k-1}+\dot{\omega}_t\wedge \omega_t\wedge \Omega_t^{k-1}-\dot{\omega}_t\wedge \Omega_t^{k-1}\wedge \omega_t)=Tr(d(\dot{\omega}\wedge \Omega^{k-1}))=$$
			$$dTr(\dot{\omega}\wedge \Omega^{k-1})$$
			Finally, by integrating:
			$$\int_0^1dt kdTr(\dot{\omega}\wedge \Omega^{k-1})=\int_0^1dt {d\over dt}Tr(\Omega^k)=Tr(\Omega^{k'})-Tr(\Omega^{k})=d\tau$$
		\end{itemize}
	\end{proof}
	\subsection{The Chern Class}
	In this subsection we will construct the Chern classes for vector bundles.
	\begin{Def}
		Let $E$ be a vector bundle with structure group $GL(r,\mathbb{C})$ and a curvature $\Omega$. Then we defne the \textit{total Chern Class} as:
		$$c(\Omega)=det\bigg(\mathbb{I}-{i\over 2\pi}\Omega\bigg)$$ 
	\end{Def}
	\begin{Obs}
		One can show that all invariant polynomials on $\mathfrak{gl}(r,\mathbb{C})$ are generated by the coefficients $f_k$ in the expansion:
		$$det(\lambda\mathbb{I}-X)=\sum_{k=0}^r f_k(X)\lambda^{r-k}$$
		This implies that all invariant polynomials of $\Omega$ are generated by the elements of the Chern class expansion.
	\end{Obs}
	\begin{Obs}
		Since $\Omega$ is a $2$-form, $c(\Omega)$ is the sum of even degrees term:
		$$c(\Omega)=1+c_1(\Omega)+c_2(\Omega)+...$$
		where $$c_n(\Omega)\in\Omega^{2n}(M)$$
		is called the \textit{$n^th$ Chern Class}.\\
		Moreover, since each element is an invariant polynomial, $c_j(\Omega)$ correspond to an element $[\Gamma_j]\in H^{2j}(M)$. It immediately follows that for $j>dim(M)$ and $2j>n$ the Chern class vanishes 
		$$c_{j>m/2}(\Omega)=0,\hspace{20 pt}c_{j>n}(\Omega)=0$$
		Lastly, the finishing term is $c_{j=m}(\Omega)=det({i\over 2\pi}\Omega)$.
	\end{Obs}
	\begin{Obs}
		It is possible to prove that for the Lie algebra $\mathfrak{su}(n)$, all invariant polynomials are generated by the Chern Polynomial. However, from the requirement $Tr(X)=0$, we have $c_1(F)=0$.
	\end{Obs}
	\begin{Prop}
		If $E\oplus F$ is a sum of complex vector bundles with structure groups $GL(r_{1,2},\mathbb{C})$ and $c(E),c(F)$ are the total Chern classes of them, then:
		$$c(E\oplus F)=c(E)\wedge c(F)$$
	\end{Prop}
	\begin{proof}
		This is a clear consequence of the fact that if $\Omega_E,\Omega_F$ are the curvatures of the bundles, then $$\Omega_{E\oplus F}=\begin{pmatrix}
		\Omega_E && 0\\
		0&& \Omega_F
		\end{pmatrix}$$
		Clearly, the determinant function splits accoardingly.
	\end{proof}
	\subsection{Chern Characters}
	\begin{Def}
		We define the \textit{total Chern character} as:
		$$ch(\Omega)=Tr\bigg({i\over 2\pi}\Omega\bigg)=\sum_j {1\over j!}Tr\bigg({i\over 2\pi}\Omega\bigg)^j$$
		We call $ch_j(\Omega)={1\over j!}Tr\bigg({i\over 2\pi}\Omega\bigg)^j$ the $j^{th}$ \textit{Chern character}.
	\end{Def}
	The Chern characters behave in a nicer way with respect to the splitting principle. Consider the following:
	\begin{Prop}
		Let $E,F$ be vector bundles over $M$ with structure group $GL(n,\mathbb{C})$. Then:
		\begin{itemize}
			\item $ch(E\oplus F)=ch(E)\oplus ch(F)$;
			\item $ch(E\otimes F)=ch(E)\otimes ch(F)$.
		\end{itemize}
	\end{Prop}
	\begin{proof}
		Recall that by definition:
		$$ch(\Omega)=\sum_j {1\over j!}Tr\bigg({i\over 2\pi}\Omega\bigg)^j$$
		This immediately proves the first result:
		$$Tr(A\oplus B)^j=Tr(A^j)+Tr(B^j)$$
		As for the second equality instead, if $A=B\otimes C=B\otimes \mathbb{I}+\mathbb{I}\otimes C$ then:
		$$Tr(B\otimes \mathbb{I}+\mathbb{I}\otimes C)^j=\sum_{m=1}^j {j \choose m}Tr(B^m)Tr(C^{j-m})$$
		So that we find:
		$$ch(B\otimes C)=\sum_j {1\over j!}{i\over 2\pi}\sum_{m=1}^j {j \choose m}Tr(B^m)Tr(C^{j-m})={i\over 2\pi}\sum_j {1\over j!}Tr(B^m)\sum_{m}{1\over m!}Tr(C^{m})$$
	\end{proof}
	\subsection{The Pontrjagin Class}
		\begin{Def}
			Let $E$ be a vecotr bundle over $M$. We define the \textit{Pontrjagin class} of $E$ as:
			$$p(E)=det\bigg(\mathbb{I}+{1\over 2\pi}\Omega\bigg)$$
		\end{Def}
		\begin{Prop}
			If $f$ is an invariant polynomial on $\mathfrak{gl}(n,\mathbb{R})$, then $[f(\Omega)]$ is 0 in $H^{2k}(M)$.
		\end{Prop}
		\begin{proof}
			Put a Riemannian metric on $M$ and consider a curvature compatible with the metric. THis is skew-symmetric. Since $f$ is a linear combination of $Tr$ of odd degree, the final result is 0.
		\end{proof}
		\begin{Obs}
			By the previous result REFERENZA, since $\Omega$ is a $2$-form, we will have only the even-degree terms in the expansion:
			$$det\bigg(\mathbb{I}+{1\over 2\pi}\Omega\bigg)=1+f_2({1\over 2\pi}\Omega)+f_4({1\over 2\pi}\Omega)+...$$
			Clearly, $f_n)({1\over 2\pi}\Omega)\in\Omega^{4n}(M)$.
		\end{Obs}
		\begin{Def}
			We define the $k^{th}$ Pontjagin class as:
			$$p_k(\Omega)=[f_{2k}({1\over 2\pi}\Omega)]\in H^{4k}(M)$$
		\end{Def}
		\begin{Prop}
			If $E$ is a vector bundle on $M$ and $E_\mathbb{C}=E\otimes \mathbb{C}$ is the complexified bundle, then there is a correspondence:
			$$p_k(E)=(-)^kc_{2k}(E_\mathbb{C})$$
		\end{Prop}
		\begin{Prop}
			If $\Omega$ is the curvature of $E$, then clearly this induces a curvature on $E_\mathbb{C}$ like:
			$$\Omega_\mathbb{C}=\Omega\otimes\mathbb{I}_\mathbb{C}$$
			Now, a skewsymmetric matrix can be diagonalized over the complex numbers, and its eigenvalues will come in complex pairs $\pm ix_j$, so that:
			$$det(\mathbb{I}+iA)=det\begin{pmatrix}
				1+x_1 && 0 && ... &&... && ...\\
				0 && 1-x_1 && 0 && ... && ...  \\
				0 && 0 && 1+x_2 && 0 &&...  \\
				0 && 0 && 0 && 1-x_2 &&...  \\
				0 && 0 && 0 && 0 &&... 
			\end{pmatrix}=\prod (1-x_i)^2=$$
			while
			$$det(\mathbb{I}+A)=\prod (1+x_i)^2$$
			This proves the Proposition.
		\end{Prop}
		As a corollary of this statement:
		\begin{Prop}
			$$P(E\oplus F)=p(E)\wedge p(F)$$
		\end{Prop}
		\section{Characteristic classes of Principal Bundles}
		In this section we will look at the construction of topological invariants for principal bundles.
		\begin{Def}
			Let $V$ be a $n$-dimensional vector space. We call \textit{polynomial} of degree $k$ on $V$ any element $f\in Sym^k(V^*)$.
		\end{Def}
		\begin{Def}
			If $\mathfrak{g}$ is the Lie algebra of $G$ and $f:\mathfrak{g}\rightarrow \mathbb{R}$ is a polynomial of degree $k$, then $f$ is said to be $\rho$-invariant if
			$$f(\rho(g)X)=f(X)$$
			for all $g\in G$ and $X\in \mathfrak{g}$.
		\end{Def}
		We are interested in construction $Ad$-invariant polynomials, starting from the curvature form.
		\begin{Theo}[Chern-Weyl]
			Let $(P,M,\pi,G)$ be a principal bundle, $A$ a connection and $F$ its curvature. If $f$ is an $Ad$-invariant polynomial then:
			\begin{itemize}
				\item[i)] $f(F)$ is basic i.e. $f(F)=\pi^*\Gamma$;
				\item[ii)] $d\Gamma=0$ is closed;
				\item[iii)] The cohomology class $[\Gamma]$ is independent of the connection $A$.
			\end{itemize}
		\end{Theo}
		\begin{proof}
			We proceed with order:
			\begin{itemize}
				\item[i)] We need to show right-invariance and horizontality. In general, $$f(F)=a_IF^{i_1}\wedge ...\wedge F^{i_k}$$
				By horizontality of the curvature form, $f(F)$ is also horizontal. Moreover, by taking the right action:
				$$r_g^*f(F)=a_IAd(g^{-1})F^{i_1}\wedge ...\wedge Ad(g^{-1})F^{i_k}=f(Ad(g^{-1})F)=f(F)$$
				This proves that $f(F)$ is basic.
				\item[ii)]
				$$df(F)=d\pi^*\Gamma=\pi^*d\Gamma=0$$
				This is a consequence of $a_I$ being constant and $Df(F)=df(F)$.
				\item [iii)]
				Consider a curve that interpoles any two connections, like:
				$$A_t=A+t(A'-A)=A+t\alpha; \hbox{ with }t\in[0,1] $$
				Then, if $F_t=D_tA_t$ we can evalue:
				$${d\over dt} F_t={d\over dt}(dA_t+{1\over 2}[A_t,A_t]+t[\alpha,\alpha])=d\alpha+{1\over 2}[A_t,\alpha]+antisymm=D_t\alpha+antisymm$$
				Now, feeding this into the symmetric polynomial $f$, the last term dies off and so we find:
				$$D_tf(\alpha,F_t,...,F_t)=f(D_t\alpha,F_t,...,F_t)=f({d\over dt}F_t,F_t,...,F_t)$$
				Moreover, since $D_t\alpha=d\alpha+{1\over 2}[A_t,\alpha]$ and the second term is antisymmetric, the polynomial kills it so that:
				$$f(D_t\alpha,F_t,...,F_t)=f(d\alpha,F_t,...,F_t)=df(\alpha,F_t,...,F_t)$$
				Where the last equality used the fact that the covariant derivative of $f$ is equal to the exterior derivative of $f$, and $F_t$ is covariantly closed.
				By multilinearity of the polynomial, we can write:
				$$df(\alpha,F_t,...,F_t)=k{d\over dt}f(F_t,F_t,...,F_t)$$
				Finally, integrating:
				$$k\int_0^1 dt {d\over dt}f(F_t,F_t,...,F_t)=f(F)-f(F')=d\bigg(\int_0^1 dt f(\alpha,F_t,...,F_t)\bigg)$$
				This completes the proof.
			\end{itemize}
		\end{proof}
		\begin{Obs}
			The last theorem basically tells us that if we have a principal bundle and we choose any connection, and thus any curvature, we can find a unique cohomology class $[\Gamma]$ on the base manifold. In particular, $\Gamma\in\Omega^k(M,\mathbb{R})$. In fact, since $f:\mathfrak{g}^k\rightarrow \mathbb{R}$ is a polynomial with real values, by feeding to it the curvature we obtain a form of degree $2k$ which has values in $\mathbb{R}$. The cohomology class of $\Gamma$ is called \textit{characteristic class} of $P$ associated to $f$.
		\end{Obs}
		\begin{Prop}
			If two principal bundles are isomorphic then they have same characteristic class.
		\end{Prop}
		\begin{proof}
			Let $P,P'$ be two principal bundles over $M$, suche that there is a bundle isomorphism:
			$$\chi:P\rightarrow P'\hbox{ such that }\pi=\pi'\circ \chi$$
			Suppose you have a basic form $\omega'$ on $P'$. Then $\omega'=\pi'^*\Gamma$ and $\phi^*\omega=\phi^*\pi'^*\Gamma=(\pi'\circ \phi)^*\Gamma=\pi^*\Gamma$. So the basic form on $P'$ gets pulledback isomorphically to a basic form on $P$ corresponding to the same cohomology class.
		\end{proof}
		\begin{Obs}
			This analysis provides a tool for understanding if two principal bundles are not isomorphic: we check if the characteristic classes differ. 
		\end{Obs}
		\subsection{The Chern Class}
			In this subsection we will construct the Chern classes for a particular class of principal bundles.\\
			\\
			\begin{Def}
				Let $P$ be a principal $GL(r,\mathbb{C})$ bundle with a curvature $F$. Then we defne the \textit{total Chern Class} as:
				$$c(F)=det\bigg(\mathbb{I}-{i\over 2\pi}F\bigg)$$ 
			\end{Def}
			\begin{Obs}
				One can show that all invariant polynomials on $\mathfrak{gl}(r,\mathbb{C})$ are generated by the coefficients $f_k$ in the expansion:
				$$det(\lambda\mathbb{I}-X)=\sum_{k=0}^r f_k(X)\lambda^{r-k}$$
				This implies that all invariant polynomials of $F$ are generated by the elements of the Chern class expansion.
			\end{Obs}
			\begin{Obs}
				Since $F$ is a $2$-form, $c(F)$ is the sum of even degrees term:
				$$c(F)=1+c_1(F)+c_2(F)+...$$
				where $$c_n(F)\in\Omega^{2n}(P,\mathbb{R})\simeq\Omega^{2n}(M,\mathbb{R})$$
				is a basic form called the \textit{$n^th$ Chern Class}.\\
				Moreover, since each element is an invariant polynomial, $c_j(F)$ correspond to an element $[\Gamma_j]\in H^{2j}(M)$. It immediately follows that for $j>dim(M)$ and $2j>n$ the Chern class vanishes 
				$$c_{j>m/2}(F)=0,\hspace{20 pt}c_{j>n}(F)=0$$
				Lastly, the finishing term is $c_{j=m}(F)=det({i\over 2\pi}F)$.
			\end{Obs}
			\begin{Obs}
				It is possible to prove that for the Lie algebra $\mathfrak{su}(n)$, all invariant polynomials are generated by the Chern Polynomial. However, from the requirement $Tr(X)=0$, we have $c_1(F)=0$.
			\end{Obs}
			\begin{Ex}
				Consider the Hopf bundle $(S^3,S^2,\pi,U(1))$. This is a non trivial bundle. We immediately know that:
				$$c_0(F)=1,c_1(F)=0,c_{j>2}(F)=0$$
			\end{Ex}
			The Chern classes of a principal bundle correspond to the Chern classes of the adjoint bundle, since $F\rightarrow F_M\in\Omega^2(M,Ad(P))$ by the musical isomorphism REFERENZA.\\
			This immediately implies that all of the other properties of the Chern Classes exposed in REFERENZA also hold.
			\subsection{The Pontrjagin Class}
		\chapter{The Cartan Construction}
		In this chapter we will introduce the Cartan construction regarding principal bundles. We will see how to describe General Relativity in an alternative way. The construction made in this chapter will merely be an extension of architectures which we have already seen. 
		\section{Klein Geometries}
		\begin{Def}
			Let $G$ be a Lie group and $H$ be a closed Lie subgroup of $G$. We call the pair $(G,H)$ a \textit{Klein geometry} if $G/H$ is connected.
		\end{Def}
		ESEMPI
		\section{The Cartan connection}
		\begin{Def}
			Let $(G,H)$ be a Klein geometry and $(P,M,\pi,H)$ be a principal bundle. We call $\omega\in\Omega^1(P,\mathfrak{g})$ a \textit{Cartan connection} if:
			\begin{itemize}
				\item for all $p\in P$, $\omega_p:T_pP\rightarrow \mathfrak{g}$ is an isomorphism;
				\item $r_h^*\omega=Ad(h^{-1})\omega$ for all $h\in H$;
				\item $\omega(\overline{X})=X$ for all $X\in \mathfrak{h}$;
				\item $\omega$ is smooth.
			\end{itemize}
			We call the couple $[(P,M,\pi,H),\omega]$ a \textit{Cartan Geometry.}
		\end{Def}
		$$\begin{tikzcd}
			& G \arrow{d}{i}\\
			P \arrow{d}{\pi} & H \arrow{l}{\mu} \arrow{d}\\
			M & G/H
		\end{tikzcd}$$
		\begin{Obs}
			Clearly, $\omega$ is very similar to an Ehresmann connection on $\mathfrak{h}$. However, there are some slight differences. First of all, $\omega$ takes values in all of $\mathfrak{g}=Lie(G)$. Second of all, it is required that at any point $p\in P$, the Cartan form provides an isomorphism between the tangent space of $P$ and $Lie(G)$.
		\end{Obs}
		\begin{Def}
			If $G$ is a Lie group, $H\subset G$ a closed Lie subgroup, we say that $G/H$ is a \textit{reductive homogeneous space} if there is a decomposition $\mathfrak{g}=\mathfrak{h}\oplus\mathfrak{m}$ and:
			$$Ad(H)\mathfrak{m}\subset \mathfrak{m}$$
			The definition automatically extends to Klein geometries. 
		\end{Def}
		\begin{Obs}
			In general, the complement $\mathfrak{m}$ is not unique. However, any complement is isomorphic to $\mathfrak{g/h}$ as a vector space.
		\end{Obs}
		\begin{Obs}
			In general $\mathfrak{m}$ does not need to be a Lie algebra at all! In particular, the additional generators of $\mathfrak{m}$ are not required to be closed under the commutator.
		\end{Obs}
		\begin{Def}
			We define the \textit{Cartan curvature of a Cartan connection} as:
			$$\Omega=d\omega+[\omega\wedge \omega]$$
		\end{Def}
		\begin{Ex}
			We saw that, given a Lie group $G$ and a closed Lie subgroup $H$ of it, there is a principal $H$ bundle like:
			$$(G,G/H,\pi,H); \hbox{ where }\pi(g)=[g]$$
			If we take the right Maurer-Cartan form $\theta\in\Omega^1(G,\mathfrak{g})$ as acting like:
			$$\theta_g(X_g)=dr_{g^{-1}}X_g$$
			we see that this is a Cartan connection. In particular, $\theta$ is clearly smooth, by REFERENZA it transforms with the adjoint, it is clearly an isomorphism (since $dr$ is a diffeomorphism at all points) and lastly:
			$\theta_g(dj_{pg}(X))=dr_{g^{-1}}dj_{pg}(X)=X$. Clearly, by REFERENZA, the curvature of the Maurer-Cartan connection is 0.
		\end{Ex}
	\begin{Obs}
		Consider a reductive Cartan Geometry. Then, since $\mathfrak{g/h}$ is $Ad(H)$ invariant, we can split the Cartan connection into two pieces:
		$$\omega_p(v)=A_p(v)+e_p(v)$$
		where $A\in\Omega^1(P,\mathfrak{h}),e\in\Omega^1(P,\mathfrak{g/h})$. We will call $A$ the \textit{Cartan form} and $e$ the \textit{solder form}. In particular, this splitting is only possible due to the invariance of $\mathfrak{g/h}$. Namely, it follows from $\omega_p(v)\in\mathfrak{g}=\mathfrak{h}\oplus\mathfrak{g/h}$ and $Ad(h^{-1})\omega_p(v)\in \mathfrak{h}\oplus Ad(h^{-1})\mathfrak{g/h}=\mathfrak{h}\oplus \mathfrak{g/h}$.\\ 
		To be more precise:
		\begin{itemize}
			\item $A$ is an Ehresmann connection on $P$.\\
			\\
			This follows immediately from the property $\omega(\overline{X})=X$ and the fact that $A$ takes values in $\mathfrak{h}$;
			\item $e$ is smooth, right-equivariant and horizontal.\\
			\\
			The horizontality follows from the fact that $A$ is an Ehresmann connection. Then, we must have that for any $X\in\mathfrak{h}$, $e(\overline{X})=0$. The right equivariance is obvious by reducibility.
		\end{itemize}
		There is one last important property of $e$: it descends to an isomorphism on $TM$. We are now going to show this.
	\end{Obs}
	\begin{Obs}
		just like for principal connections REFERENZA, the principal bundle structure of any Cartan Geometry uniquely idenitfies a vertical distribution:
		$$\mathcal{V}=Ker(d\pi)$$
		In the  Ehressmann case, the choice of a connection implies the choice of a distribution. However, the Cartan connection amounts to something different: at any point we have an isomorphism:
		$$T_pP\xlongrightarrow{\omega_p}\mathfrak{g}\hbox{ and }\mathfrak{h}\simeq\mathcal{V}_p$$ by REFERENZA, the choice of a connection does not specify an horizontal distribution since, being an isomorphism, $Ker(\omega)$ is trivial. If instead the geometry is reductive, then $Ker(A)$ is the horizontal distribution.
	\end{Obs}
	\begin{Def}
		Let $[(P,M,\pi,H),\omega]$ be a Cartan gometry on a reductive Kelin Geometry $(G,H)$. Then, we define the \textit{soldering bundle} as the associated vector bundle:
		$$P\times_{H}\mathfrak{m}$$
	\end{Def}
	$$\begin{tikzcd}
		&& G \arrow{d}{i}\\
		P\times_H \mathfrak{m}\arrow{dr}{\pi_\mathfrak{m}}&P \arrow{d}{\pi} & H \arrow{l}{\mu} \arrow{d}\\
		& M & G/H
	\end{tikzcd}$$
	This is a well defined associated bundle since we have a representation $$\rho(h)=Ad(h)|_\mathfrak{m}$$
	from REFERENZA, we get the following vector bundle:
	$$(P\times_{H}\mathfrak{m},M,\pi_{\mathfrak{m}},\mathfrak{m})$$
	This is an associated bundle and $e$ is clearly a tensorial form under $\rho$. By the Musical Isomorphism REFERENZA we can send:
	$$e\rightarrow e_M\in\Omega^1(M,P\times_{H}\mathfrak{m})$$
	We call $e_M$ the \textit{Coframe field}.
	\begin{Prop}
		For a reductive Cartan geometry $[(P,M,\pi,H),\omega]$ there is an isomorphsim:
		$$TM\simeq P\times_{H}\mathfrak{m}$$
	\end{Prop}
	\begin{proof}
		Since $\omega$ is an isomorphism at any point and the space is reductive, we can split $\omega$ into two isomorphisms:
		$$A_p:\mathcal{V}_p\rightarrow \mathfrak{h}\hspace{ 20 pt }e_p:\mathcal{H}_p\rightarrow \mathfrak{m}$$
		This implies, by the Musical Isomorphism REFERENZA that there is another isomorphism at any point:
		$$e_{M}:TM\rightarrow P\times_H\mathfrak{m}$$
		Thus $TM\simeq P\times_{H}\mathfrak{m}$.
	\end{proof}
	$$\begin{tikzcd}
		&& G \arrow{d}{i}\\
		P\times_H \mathfrak{m}\arrow{dr}{\pi_\mathfrak{m}}&P \arrow{d}{\pi} & H \arrow{l}{\mu} \arrow{d}\\
		TM \arrow{u}{e_M} \arrow{r}{\pi_{TM}}& M & G/H
	\end{tikzcd}$$
	\section{Torsion and metric compatibility}
	In this section we will analyze the concept of torsion in a reductive Cartan geometry. The idea is to split the curvature tensor toghether with the Cartan connection. This will give rise to different tensorial quantities. Moreover, we will see that the solder form will enable us to pull-down the metric from the structure group to the base manifold. 
	\begin{Obs}
		For a reductive homogeneous space $(\mathfrak{g},\mathfrak{h})$, a Cartan connection splits:
		$$\omega=A+e$$
		This splitting influences the curvature, which splits as well.
		$$\Omega=dA+de+{1\over 2}[A+e,A+e]=dA+{1\over 2}[A,A]+de+[A,e]+{1\over 2}[e,e]$$
		This leads us to define:
		$$F_A=dA+{1\over 2}[A,A]+{1\over 2}[e,e]_{\mathfrak{h}}\hspace{20 pt}F_e=de+[A,e]+{1\over 2}[e,e]_{\mathfrak{m}}$$
		Where we have decomposed $[e,e]$ into two parts, taking values in $\mathfrak{h}$ and $\mathfrak{m}$ respectfully.
	\end{Obs}
	\begin{Def}
		Let the notation be as above. Then we call $R=F_A-{1\over 2}[e,e]_\mathfrak{h}$ the \textit{curvature }and $T=F_e-{1\over 2}[e,e]_\mathfrak{m}$ the \textit{torsion}. 
	\end{Def}
	We will later see that the framework of Cartan gometries can be used to reconstruct General Relativity. For this purpose, it is necessary to include a metric analysis of this setting.
	\begin{Theo}
		Let $[(P,M,\pi,H),\omega]$ be a Cartan Geometry modelled on a reductive Klein geometry $(G,H)$. Then, if $\mathfrak{m}$ has a $G$-invariant metric $g_0$, the solder form $e$ induces a metric on $M$. 
	\end{Theo}
	\begin{proof}
		By definition $g_0:\mathfrak{m}\times \mathfrak{m}\rightarrow \mathbb{R}$ is a $G$-invariant product. Let $s:M\rightarrow P$ be a local gauge and $e:TP\rightarrow \mathfrak{m}$ be the solder form. Then we can pullback it like:
		$$s^*e:TM\rightarrow \mathfrak{m}$$
		This means that we can define the following scalar product on $M$:
		$$g\in\Gamma(TM\otimes TM)\hbox{ like }g(X,Y)=g_0(s^*e(X),s^*e(Y))$$
		This is clearly a scalar product and, since $g_0$ is $Ad$-invariant by hypothesis, it is completely independent from the choice of the section.
	\end{proof}
	\section{The Covariant Derivative}
	In this section we will define the notion of covariant derivative for a Cartan geometry. This definition does not require, in general, the underlying Klein geometry to be reductive. However, in an Ehresmann geometry, the covariant derivative is defined from the Horizontal distribution. The lack of a canonical horizontal distribution in the Cartan case will force us to define the covariant derivative in an alternative way.
	\begin{Def}
		Let $[(P,M,\pi,H),\omega]$ be a Cartan geometry and $\rho:H\rightarrow Gl(V)$ any representation. We define the \textit{full exterior covariant derivative} as the map:
		$$D_\omega^\rho:\Omega^k(P,V)\rightarrow \Omega^{k+1}(P,V)\hspace{20 pt}D_\omega^\rho=d+d\rho(\omega)\wedge$$
	\end{Def}
	This is clearly a linear operator.
	Notice that this definition resembles the one of the usual Ehresmann covariant derivative. However, we cannot use the same definition since in a general Cartan geometry selecting a connection does not correpsond to a choice for the horizontal distribution.
	\begin{Prop}[Bianchi Identity]
		Let the notation be as above, then:
		$$d\Omega=[\Omega,\omega]$$
	\end{Prop}
	\begin{proof}
		The proof is the same as REFERENZA.
	\end{proof}
	\begin{Obs}
		As a corollary of the previous proposition, we have that in the general non reductive case:
		$$D_\omega^{Ad}\Omega=0$$
		Where we have selected the adjoint representation.
	\end{Obs}
	With a generic Ehressmann connection, the choice of a connection automatically implies the choice of an horizontal distribution. This is in general not true for a Cartan connection, since it's kernel is trivial (it is an isomorphism). However, in the reductive case, the Cartan connection splits into a soldering form plus an Ehresmann connection.
	\begin{Obs}
		Let $[(P,M,\pi,H),\omega]$ be a Cartan geometry on a reductive Klein geometry $(G,H)$ and $\rho:H\rightarrow Gl(V)$ any representation. Since $\omega=A+e$ and we have an horizontal distribution $\mathcal{H}=Ker(A)$, we automatically get an \textit{exterior covariant derivative} like in the Ehresmann case:
		$$D_A\eta=(d\eta)^h$$
		This clearly inherits all of the properties in REFERENZA since $A$ is an Ehressmann connection. Thus, in the reductive case, we have 2 different covariant derivatives i.e. the full exterior covariant derivative and the regular covariant derivative. \\
		\\
		It is known from REFERENZA that for tensorial forms:
		$$D_A=d+d\rho(A)\wedge$$
		Thus, by linearity of the representation and of the wedge product:
		$$D_\omega^{\rho}=D_A+d\rho(e)\wedge$$
		In particular, in the case of the adjoint representation:
		$$D_\omega^{Ad}=D_A+[e,\cdot]$$
		This means that the two derivatives differ by a \textit{torsion term} which however does not breaks horizontality.
	\end{Obs}
	\begin{Prop}
		Let $[(P,M,\pi,H),\omega]$ be a Cartan geometry on a reductive Klein geometry $(G,H)$. Then:
		$$D_\omega^\rho:\Omega^k_{\rho}(P,V)\rightarrow \Omega^{k+1}_{\rho}(P,V)$$
	\end{Prop}
	\begin{proof}
		Clearly $D_\omega^\rho$ preserves the right equivariance. The problem is the horizontality. Let $\Omega\in\Omega^k_\rho(P,V)$, $p\in P$ and $v_1,...,v_{k+1}\in T_pP$. Then 
		$$D_\omega^\rho\Omega_p(v_1,...,v_{k+1})=(d\Omega)_p(v_1,...,v_{k+1})+{1\over k!}\sum_\sigma sgn(\sigma)d\rho(\omega_p(v_{\sigma(1)}))\Omega_p(v_{\sigma(2)},...,v_{\sigma(k+1)})$$
		By linearity, we can decompose any vector into the sum of horizontal and vertical vectors. The proof is literally the same as REFERENZA case 2.
	\end{proof}
	\begin{Obs}
		In the general non reductive case we cannot speak or horizontality since we do not have a canonical horizontal distribution. However, if the geometry is reductive, we can check if the curvature forms are horizontal. 
	\end{Obs}
	\begin{Prop}
		Let $[(P,M,\pi,H),\omega]$ be a Cartan geometry on a reductive Klein geometry $(G,H)$. Then $\Omega, F_A,F_e$ are horizontal.
	\end{Prop}
	\begin{proof}
		Sicne $D$ maps tensorial forms into tensorial forms then: $D\omega$ is horizontal. Moreover:
		$$F_A=D_AA+{1\over 2}[e,e]_\mathfrak{h}\hbox{ and }F_e=D_Ae+{1\over 2}[e,e]_\mathfrak{m}$$
		Since the commutator between two horizontal forms is equivariant (clearly), then those are both equivariant.
	\end{proof}
	\begin{Prop}
		It is clear that:
		$$D_AR=0\hspace{20 pt}D_AT=[R,e]$$
	\end{Prop}
	\begin{proof}
		The first equality is trivial from REFERENZA. As for the second, it is a straightforward calculation:
		$$D_\omega \Omega=0=d\Omega+[\omega,\Omega]$$
		Substituting: $\Omega=R+T+{1\over 2}[e,e]$ we get:
		$$D_\omega \Omega=dR+dT+[de,e]+[A,R]+[A,T]+{1\over 2}[A,[e,e]]+[e,R]+[e,T]+{1\over 2}[e,[e,e]]=$$
		$$=D_AR+D_AT+[de,e]+[e,R]+[e,de+[A,e]]+{1\over 2}[A,[e,e]]$$
		Where in the last line we have used: ${1\over 2}[A,[e,e]]=0$. Moreover, from REFERENZA:
		$$[A,[e,e]]=2[[A,e],e]$$
		so that we get:
		$$D_\omega\Omega=0=D_AR+D_AT+[e,R]$$ 
	\end{proof}
	\section{The Universal covariant derivative}
	AAAAAAAAAAAA
	\section{Gauge transformations in Cartan geometries}
	In this section we will use previous result to introduce the notion of gauge transofmration on a Cartan geometry. We will see that the implications are very similar to the gauge transformations discussed for principal bundles.
	\begin{Obs}
		A Cartan geometry is constructed on a principal bundle. Thus, it is natural to define gauge transformations like we did for arbitrary principal bundles REFERENZA: as automorphisms from $P$ to $H$.
	\end{Obs}
	Recall that by REFERENZA we have the following results:
	\begin{itemize}
		\item $$\mathcal{G}(P)\simeq C^\infty(P,H)^H \hbox{ canonically }$$
		\item Given a section $s:U\rightarrow P$ of the principal bundle, we have:
		$$C^\infty(P|_U,H)^H\simeq C^\infty(U,H)$$
	\end{itemize}
	It is also clear that for any section $s:U\rightarrow P$ we can pullback the Cartan connection and curvature like:
	$$s^*\omega:TU\rightarrow \mathfrak{g}\hspace{20 pt} s^*\Omega:TU\otimes TU\rightarrow \mathfrak{g}$$
		\begin{Theo}[\textbf{Passive interpretation of diffeomorphisms}]
		Let $(P,M,\pi,H)$ be a principal $H$-bundle and $\omega$ a Cartan connection on it. Let $s_1,s_2:U\rightarrow P$ be local gauges and $\omega_{i}=s_i^*\omega$ the pulled-back connections on the manifold. Then 
		$$\omega_i=\hbox{Ad}(g_{ji}^{-1})\omega_j+\mu_{ji}$$
		Where $g_{ji}$ is the transition function between the local trivializations $s_i,s_j$ while $\mu_{ji}=g_{ji}^*\theta$ with $\theta$ the Maurer-Cartan form.
	\end{Theo}
	\begin{proof}
		By construction, $\omega_i=s_i^*\omega$ so that for any vector field $X\in \Gamma(U)$, we have:
		$$s^*\omega(X)=\omega(ds(X))$$
		The two sections induce trivializations by proposition REFERENZA. Furthermore, by observation REFERENZA, we know that the relation between the two sections at any point $x\in U$ is:
		$$s_i(x)=s_j(x)\cdot g_{ji}(x)=\mu(s_j(x),g_{ji}(x))$$
		By taking the differential:
		$$ds_{i,x}(X_x)=d\mu_{(s_{j}(x), g_{ji}(x))}(ds_{j,x}(X_x),dg_{ji,x}(X_x))$$
		Now, the mapping $g_{ji}:U\rightarrow H$ has as differential $dg_{ji,x}:T_xM\rightarrow T_{g_{ji}(x)}H$, so that it takes $X_x$ into a tangent vector to $H$ at $g_{ji}(x)$. Thus, since for every $h\in H$ the left action is a diffeomorphism, there exists an element of the Lie algebra, which we will call $T$, such that 
		$$d\ell_{g_{ji}(x)}(T)=dg_{ji,x}(X_x)$$
		This in turn implies:
		$$T=d\ell_{g^{-1}_{ji}(x)}\circ dg_{ji,x}(X_x)$$
		Moreover, applying the result obtained in proposition REFERENZA we find:
		$$d\mu_{(s_{j}(x), g_{ji}(x))}(ds_{j,x}(X_x),dg_{ji,x}(X_x))=d\mu_{(s_{j}(x), g_{ji}(x))}(ds_{j,x}(X_x),d\ell_{g_{ji}(x)}(T))=$$
		$$=dr_{g_{ji}(x)}\circ ds_{j,x}(X_x)+\overline{T}_{s_{j}(x)\cdot g_{ji}(x)}$$
		Finally, feeding this to $A$, we get:
		$$\omega_{s_i(x)}(ds_i(X_x))=\omega_{s_i(x)}(dr_{g_{ji}(x)}\circ ds_{j,x}(X_x)+\overline{T}_{s_{j}(x)\cdot g_{ji}(x)})=$$$$=r^*_{g_{ji}(x)}\omega_{{s_i(x)}}(ds_{j,x}(X_x))+T$$
		Where in the last line we have applied the defining properties of the connection $1$-forms. Substituting back the expression for $T$ we get:
		$$r^*_{g_{ji}(x)}\omega_{{s_i(x)}}(ds_{j,x}(X_x))+d\ell_{g^{-1}_{ji}(x)}\circ dg_{ji,x}(X_x)=r^*_{g_{ji}(x)}\omega_{{s_i(x)}}(ds_{j,x}(X_x))+(g^*_{ji}\theta)_x(X_x)$$
		By once again applying the defining properties of the connection:
		$$\omega_i=\hbox{Ad}(g_{ji}^{-1})\omega_j+\mu_{ji}$$
		This completes the proof.
	\end{proof}
	\begin{Obs}[\textbf{Passive interpretation of diffeomorphisms}]
		Recall that the curvature of a connection is defined as:
		$$\Omega=d\omega+{1\over 2}[\omega,\omega]$$
		Knowing the transformation rule for the pullback connection, we can find the analogue for the curvature: let $s_{i,j}:U\rightarrow P$ be two local gauges, then: $\Omega_{i,j}=s_{i,j}^*\Omega$ and we get, by applying the results found in proposition REFERENZA:
		$$s^*\Omega=ds^*\omega+{1\over 2}[s^*\omega,s^*\omega]$$
		By substituting the transformation rules for the connection one gets:
		$$s_i^*\Omega=\Omega_i=d(\hbox{Ad}(g_{ji}^{-1})\omega_j)+d\mu_{ji}+{1\over 2}[\hbox{Ad}(g_{ji}^{-1})\omega_j+\mu_{ji},\hbox{Ad}(g_{ji}^{-1})\omega_j+\mu_{ji}]=$$
		$$=d(r^*_{g_{ji}}\omega_j)+{1\over 2}[r^*_{g_{ji}}\omega_j,r^*_{g_{ji}}\omega_j]+d\mu_{ji}+{1\over 2}[\mu_{ji},\mu_{ji}]$$
		Using again proposition REFERENZA and expressing $\mu_{ji}=g_{ji}^*\theta$ we find:
		$$s_i^*\Omega=\Omega_i=r^*_{g_{ji}}\Omega_j+g_{ji}^*(d\theta+{1\over 2}[\theta,\theta])$$
		Finally, by example REFERENZA, the last term is 0 since it is the curvature induced by the Maurer-Cartan form. From the $H$-equivariance of the curvature found in theorem REFERENZA, we thus get:
		$$\Omega_i=\hbox{Ad}(g_{ji}^{-1})\Omega_j$$
		Clearly, since in the reductive case the subspaces $\mathfrak{h},\mathfrak{m}$ are $Ad(H)$-invariant, both $F_A,F_e$ transform like the total curvature $\Omega$.
	\end{Obs}
	\begin{Prop}
		Let $(P,M,\pi,H)$ be a principal $H$-bundle and $H$ an abelian Lie group. Then, given a Cartan connection $\omega$ on $P$, the pullback of its curvature $\Omega$ is independent of the choice of the local gauge.
	\end{Prop}
	\begin{proof}
		If $H$ is abelian, $\Omega$ is gauge invariant. This means that for any change of local section, the pullback of $\Omega$ remains invariant and so $\Omega$ is defined globally as a closed 2-form on $M$: $\Omega\in \Omega^2(M,\mathfrak{g})$.
	\end{proof}
	Finally, we would like to understand how the curvature transforms from the point of view of the associated bundle. Recall that proposition \ref{Mus_Iso} gave us the following isomorphism:
	$$\Omega^k_\rho(P,V)\simeq \Omega^k(M,E)$$
	The curvature of a connection is, as proved in proposition REFERENZA, $Ad$-equivariant and thus belongs to $\Omega^2_{Ad}(P,\mathfrak{g})$. This implies that once we have a connection on $P$ principal bundle, we can construct a $2$-form on the associated bundle $\Omega_M$.
	\begin{Prop} [\textbf{Active interpretation of diffeomorphisms}]
		Let $\Omega\in \Omega^2_{Ad}(P,\mathfrak{g})$ be a curvature form on a principal bundle and $\Omega_M\in\Omega^2(M,P\times_{H}\mathfrak{m})$ be the corresponding $2$-form with values on the associated bundle. The transformation $\phi\in\mathcal{G}(P)$ on $\Omega_M$ is induced by the one of $\Omega$ and is:
		$$\Omega_M\rightarrow \phi^{-1}\cdot \Omega_M$$
	\end{Prop}
	\begin{proof}
		Let $s:U\rightarrow P$ be a local gauge. By observation REFERENZA, the transformation $\phi\in\mathcal{G}$ on $\Omega$ is:
		$$s^*\Omega\rightarrow \hbox{Ad}(g^{-1})s^*\Omega$$
		where $g\in C^\infty(U,H)$ is the smooth map corresponding to $\phi$ in the isomorphism of proposition REFERENZA. The form $\Omega_M$ is constructed through theorem REFERENZAa as follows. Let $x\in M,X_x,X_y\in T_xM$, $p\in P_x$ and $\tilde{X}_p,\tilde{Y}_p\in T_pP$ their horizontal lifts. We have:
		$$\Omega_{M,x}(X_x,Y_x)=[p,\Omega_p(\tilde{X}_p,\tilde{Y}_p)]$$
		In REFERENZA we proved that this association is independent of the choice of the horizontal lift and of the point in the fiber. Thus, if $p=s(x)$ and $\tilde{X}_p,=ds(X_x);\tilde{Y}_p=ds(Y_x)$, we have:
		$$\Omega_{M,x}(X_x,Y_x)=[s(x),\Omega_{s(x)}(ds(X_x),ds(Y_x)]$$
		Now, applying the gauge transformation $\phi$ we get:
		$$[p,\hbox{Ad}(g^{-1})\Omega_{s(x)}(ds(X_x),ds(Y_x))]\sim [p\cdot \sigma^{-1}_\phi(p),\Omega_{s(x)}(ds(X_x),ds(Y_x))]=$$
		$$=[\phi^{-1}(p),\Omega_p(\tilde{X}_p,\tilde{Y}_p)]$$
		But this is exactly $\phi^{-1}$ on $Ad(P)$ (see proposition REFERENZA). Thus, we get:
		$$\Omega_M\rightarrow \phi^{-1}\cdot \Omega_M$$
	\end{proof}
	\begin{Prop} [\textbf{Active interpretation of diffeomorphisms}]
		Let $\omega\in (P,\mathfrak{g})$ be a Cartan connection on a principal bundle. The transformation $f\in\mathcal{G}(P)$ on $\omega$ is:
		$$\omega\rightarrow Ad(\sigma_f^{-1}) \omega+\sigma_f^*\theta$$
		where $\theta$.
	\end{Prop}
	\begin{proof}
		Let $f:P\rightarrow P$ be a gauge transformation and $\sigma_f:P\rightarrow G$ be the associated map from REFERENZA. Then for $v_p\in T_pP$ we have:
		$$f^*\omega_p(v_p)=\omega(df(v_p))$$
		Recall that $f(p)=\mu(p,\sigma_f(p))$ where $\mu$ is the action of the group $H$. Thus, by REFERENZA: 
		$$df_p(v)=d\mu(v,d\sigma_f(v))$$
		$$f^*\omega(v)=\omega(df(v))=Ad(\sigma_f^{-1})\omega+\sigma_f^*\theta$$
	\end{proof}
	\begin{Obs}
		For a reductive Cartan geometry, we have a splitting:
		$$\omega=A+e$$
		Since $A$ is an Ehressmann connection, it will transform like $\omega$, following REFERENZA. This implies that the transformation rule for the soldering form $e$ is:
		$$e\longrightarrow Ad(g^{-1})e\hbox{ \textbf{[Active] }}$$
		so it transforms just like the curvature (indeed it is a tensorial form). By setting two local gauges, we get:
		$$s_i^*e= Ad(g_{ji}^{-1})s_j^*e\hbox{ \textbf{[Passive] }}$$
	\end{Obs}
	\section{Parallel transport and Holonomy}
	The notion of parallel transport in a Cartan Geometry is different from the one on general principal bundles when treated with the Ehresmann framework. The main difference is the fact that a priori a Cartan Geometry does not specify an horizontal distribution. This forces us to find a new analogue of the Ehresmann parallel transport.
	\begin{Prop}
		Let $(P,M,\pi,H)$ be a princopal bundle, $G$ a Lie group and $\phi: H\rightarrow G$ a Lie group homomorphism. Then there is a principal $G$ bundle:
		$$P\times_H G$$
	\end{Prop}
	\begin{proof}
		We define: $P\times_H G$ as the set of points under the relation:
		$$(p,g)\sim(p\cdot h,\phi(h^{-1})g)$$
		This set is a smooth manifold (see REFERENZA). Consider the projection:
		$$\pi_G: P\times_H G\rightarrow M\hbox{ as }\pi_G([p,g])=\pi(p)$$
		This is clearly smooth and indepenent of the representatives since $P$ is a principal bundle on $M$. Local trivializations are defined as follows: let $\varphi:\pi^{-1}(U)\rightarrow U\times H$ be a trivialization for $P$. Then we define:
		$$\varphi_G^{-1}(x,g)=[\varphi^{-1}(x,e_H),g]$$
		where $e_H$ is the neutral element of $H$. There is a clear inverse: 
		$$\varphi_G([p,g])=(x,\phi(h)g)$$
		where $\varphi(p)=(x,h)$.\\
		Lastly, the natural action of $G$ on $P\times_H G$ is the following:
		$$[p,g]\cdot g'=[p,gg']$$
		This action is clearly free and equivariant with the trivializations. This proves the claim.
	\end{proof}
	\begin{Obs}
		Given a Cartan Geometry $[(P,M,\pi,H),\omega]$ modelled on a Klein geometry $(G,H)$, we can construct two different principal bundles:
		$$Q=P\times_H G, \hspace{20 pt}P\times_H G/H$$
		Those are defined starting from the inclusion map $i:H\rightarrow G$ and the projection:
		$$\Pr\circ i:H\rightarrow G\rightarrow G/H$$
		Thus, given a principal $H$ bundle and a homogeneous space $G/H$, we have a sequence of principal bundles:
		$$\begin{tikzcd}
			P \arrow{dr}\arrow{r}{i}& P\times_H \arrow{d} G\arrow{r}{\Pr} &P\times_H G/H \arrow{dl} \\
			& M &
		\end{tikzcd}$$
	\end{Obs}
	\begin{Theo}
		Let $G$ be a Lie group, $H$ be a closed Lie subgroup of $G$ and let $G/H$ be connected. Let $P$ be a principal $H$-bundle and $Q$ a prinicpal $G$-bundle, both over $M$. If $dim(G)=dim(P)$ and $\phi:P\rightarrow  Q$ is a bundle map (REFERENZA), then there is a bijection:
		$$\begin{cases}
			\hbox{Ehresmann connections } A \hbox{ on } Q \\ \hbox{ such that } Ker(A)\cap d\phi(TP)=\emptyset
		\end{cases}\xlongrightarrow{\phi^*} \begin{cases}
		\hbox{Cartan connections } \omega \hbox{ on } P
		\end{cases}$$
	\end{Theo}
	\begin{proof}
		The proof is hard.\\
		\\
		Let $A$ be an Ehresmann connection on $Q$. Then, $\phi^*A\in\Omega^1(Q,\mathfrak{g})$. Now we prove that this is a Cartan connection. Clearly, since $Ker(A)\cap d\phi(TP)=\emptyset$, $Ker(\phi^*A)=0$. Since $dim(G)=dim(P)$, $\phi^*A$ is an isomorphism. Moreover, since $\phi$ is a bundle map, we have that, for any $X\in\mathfrak{h}$:
		$$\phi^*A(\overline{X})=A(d\phi(\overline{X}))=X$$
		This is a consequence of: $\pi_Q\circ\phi=\pi_P$, which implies that $Ker(d\pi_P)=Ker(d\pi_Q\circ d\phi)$. Basically, $\phi$ sends vertical vectors to vertical vectors. The $Ad$-equivariance follows from the properties of the bundle map:
		$$r_h^*\phi^*A=(\phi r_h)^*A=( r_h\phi)^*A=\phi^*r_h^*A=\phi^*Ad(h^{-1})A=Ad(h^{-1})phi^*A$$
		This proves that $\phi^*A$ is a Cartan connection. Now we need the prove the opposite correspondence.
		\\
		Let $\omega$ be a Cartan connection on $P$. We can extend this to a form on $P\times G$ like:
		$$\varpi_{p,g}=Ad(g^{-1})\pi^*_P\omega_p+\pi^*_G\theta_{G,g}$$
		Where $\pi_P:P\times G\rightarrow P$ and $\pi_G:P\times G\rightarrow G$ are the canonical projections and $\theta_G$ is the Maurer-Cartan form on $G$. Now we verify that this descends to an Ehresmann connection on $P\times_H G$ under the projection $\pi_H:P\times G\rightarrow P\times_H G$.
		\\\\ 
		First of all, note that, for $X\in\mathfrak{g}$:
		$$\varpi_{(p,g)}((0,X_g))=X$$
		Due to the presence of the Maurer-Cartan form.\\
		Our aim is to show that this form is basic on $P\times G$ under the $H$ action. If this is shown, then from REFERENZA, $\varpi$ descends to a well defined form on $P\times_H G$. Now, consider the following behaviour under the right multiplication:
		$$(\mathbb{I}\times r_m)^*\varpi_{(p,gm)}=Ad(m^{-1}g^{-1})\pi_P^*\omega+Ad(m^{-1})\pi_G^*\theta_{G}=Ad(m^{-1})\varpi$$
		This ensures the right equivariance under the pullback with $\pi_H^*$. Moreover, it can be shown that under the action:
		$$\alpha_h:P\times G\rightarrow P\times_H G$$
		$$(p,g)\rightarrow (p\cdot h, h^{-1}g)$$
		Consider:
		$$\alpha_h^*\varpi_{(p,g)}=\varpi_{(p\cdot h,h^{-1}g)}\circ \alpha^*_h=Ad(g^{-1}h)\pi_P^*({\omega\circ r_{h^{-1}}})+\pi_G^*(\theta_{G,h^{-1}g}\circ d\ell_{h^{-1}})$$
		The first term, by equivariance of the Cartan connection, spits out an adjoint $Ad(h^{-1})$. As for the second term:
		$$\theta_{G,h^{-1}g}=d\ell_g^{-1}\circ d\ell_{h}$$
		This proves that:
		$$\alpha_h^*\varpi=\varpi$$
		This proves that $\varpi$ is an invariant form under the action of $H$. The horizontality is trickier: let $X\in \mathfrak{h}$. Since $H$ acts on $P\times G$ like $\alpha_h$, the fundamental vector field of $X$ in $(p,g)$ is:
		$$\overline{X}_{(p,g)}={d\over dt}\bigg|_0 (p\cdot e^{tX},e^{-tX}g)$$ From REFERENZA, this amounts to taking two times the differential of the group action:
		$$(d\mu_{P,p}(0,X),d\mu_{G,g}(-X,0))=$$
		The firse term is the fundamental vector field on $P$ of $X$, while the second term is $-dr_gX$ so that:
		$$\overline{X}_{(p,g)}=(\omega^{-1}(X),-dr_g(X))$$
		Feeding this to $\varphi$ we find:
		$$\varpi_{(p,g)}(\overline{X}_{(p,g)})=Ad(g^{-1})X-Ad(g^{-1})X=0$$
		\\
		The smoothness is obvious.\\ This proves that $\varpi$ can be pulled back to an Ehresmann connection. In aprticular, there is $A\in\Omega^1(Q,\mathfrak{g})$ Ehresmann connection, such that:
		$$\pi^*_HA=\varpi$$ 
		Clearly, due to $\pi_H\circ \phi$ being constant, those maps are one inverse to the other.
	\end{proof}
	We are now ready to start our construction. Ideally, since we do not have an horizontal distribution, we can still lift our paths on the base manifold $M$ to the Lie group. 
	\begin{Def}
		Let $[(P,M,\pi,H),\omega]$ be a Cartan Gometry modelled on a Klein geometry $(G,H)$. Let $\gamma:[a,b]\rightarrow M$ any connected path. We call \textit{lift} of $\gamma$ any curve $\tilde{\gamma}$ such that:
		$$\pi\circ\tilde{\gamma}=\gamma$$
		We call \textit{$G$-development} of $\gamma$ any curve $\hat{\gamma}:[a,b]\rightarrow G$ such that $\hat{\gamma}^*\theta_G=\tilde{\gamma}^*\omega$, where $\theta_G$ is the Maurer-Cartan form on $G$.
	\end{Def}
	$$
	\begin{tikzcd}
		P\arrow{d}{\pi} \arrow{rr}{\theta^{-1}_G\circ \omega\circ d\tilde{\gamma}}& & G\\
		M & \mathbb{R}\arrow{l}{\gamma} \arrow{ul}[swap]{\tilde{\gamma}}\arrow{ur}{\hat{\gamma}}
	\end{tikzcd}
	$$
	\begin{Obs}
		Clearly, the lift of a curve is not unique since we have vertical vectors. Unlike the Ehresmann case, we do not a priori have an horizontal distribution and so we cannot define an horizontal lift.
	\end{Obs}
	\begin{Prop}
		Let the notation be as above.
		For any $g\in G$, there is a unique $G$-development of $\gamma$.
	\end{Prop}
	\begin{proof}
		This is obvious since $\theta$ is invertible. Clearly, if $\gamma.[a,b]\rightarrow M$ is a smooth curve, $\hat{\gamma}:[a,b]\rightarrow G$. Then:
		$$\hat{\gamma}\theta=\theta\circ d\hat{\gamma}$$
		This implies:
		$$d\hat{\gamma}=\theta^{-1}\circ \omega\circ d\tilde{\gamma}$$
		This is a differential equation and once we choose an initial condition $\hat{\gamma}(a)=g$ the solution in unique.
	\end{proof}
	We are now ready to give the definition of holonomy for a Cartan connection.
	\begin{Def}
		Let $[(P,M,\pi,H),\omega]$ be a Cartan Gometry modelled on a Klein geometry $(G,H)$. Let $\gamma:[a,b]\rightarrow M$ any connected closed path with $\gamma(a)=\gamma(b)=x$. Let $\tilde{\gamma}$ be any lift of $\gamma$ to $P$ with $\tilde{\gamma}(a)=p$ and let $\hat{\gamma}$ be the unique development such that $\hat{\gamma}(a)=e_G$ the neutral element of $G$. Then we call \textit{holonomy} of $\gamma$ in $p\in P$ the element $\hat{\gamma}(b)\cdot h\in G$, where $\tilde{\gamma}(a)=\tilde{\gamma}(b)\cdot h$.
		\\ The set of all holonomies at $p$ is called the \textit{holonomy group} $Hol_p^\omega(G)$.
	\end{Def}
	\begin{Obs}
		Our definition seems a little tricky. The idea is that in general the lift to $P$ might not be closed (clearly the holonomy of $P$ might be non trivial), however, since we have a second lifting operation from $\tilde{\gamma}$ to the $G$-development $\hat{\gamma}$, we wish to keep track of "how much is $\tilde{\gamma}$ not closed". If we were to consider only the lifted loops $\tilde{\gamma}(a)=\tilde{\gamma}(b)$ in $P$, then the holonomy would jusst be $\hat{\gamma}(b)$.
	\end{Obs}
	\begin{Prop}
		The holonomy is independent of the choice of the lift $\tilde{\gamma}$.
	\end{Prop}
	\begin{proof}
		Let $\tilde{\gamma}_2(t)=\tilde{\gamma}_1(t)\cdot h(t)$ be two lifts with $h(t)$ a smooth curve valued in $H$ starting at $h(a)=e_H$. Then:
		$$d\hat{\gamma}_2=\theta^{-1}\circ \omega (d(r_h\circ \tilde{\gamma}_1))=\theta^{-1}\circ Ad(h^{-1})\tilde{\gamma}_1^*\omega=dr_{h}\theta^{-1}\circ \tilde{\gamma}_1^*\omega$$
		Thus, once we fix the initial point:
		$$\hat{\gamma}_2(t)=\hat{\gamma}_1(t)\cdot h(t)$$
		Now, since $\tilde{\gamma}_1(b)=p\cdot g=\tilde{\gamma}_2(b)\cdot h(b)^{-1}$, the holonomy with respect to the second lift is:
		$$\hat{\gamma}_2(b)\cdot h(b)^{-1}g^{-1}=\hat{\gamma}_1(b)\cdot g^{-1}$$
	\end{proof}
	\begin{Theo}
		Let $[(P,M,\pi,H),\omega]$ be a Cartan geometry and $\varpi$ be the associated Ehresmann connection on $Q=P\times_H G$. Then the holonomy on $Q$ generated by $\varpi$ is the same as the one on $P$ generated by $\omega$.
	\end{Theo}
	\begin{proof}
		Consider a smooth loop $\gamma:[a,b]\rightarrow M$. Then there is a unique horizontal lift on $Q$:
		$$\dot{\gamma}:[a,b]\rightarrow Q \hbox{ with }\dot{\gamma}(a)=[p,e];\dot{\gamma}(b)=([p,g])$$
		In general $\dot{\gamma}=[\dot{\gamma}_1,\dot{\gamma}_2]$.
		Then the Ehresmann holonomy is $g^{-1}$. Consider now the following:
		$$\dot{\gamma}^*\varpi=0\implies-Ad(\dot{\gamma}_2^{-1})\dot{\gamma}_1^*\omega=\hat{\gamma}_2^*\theta_G$$
		We have seen in REFERENZA that $\varpi$ is invariant if acted upon with different representatives, so we can choose any 2. We choose them so that $\hat{\gamma}_1(a)=p,\hat{\gamma}_2(a)=e$. Then, by definition, the above equation is solved by a development $\hat{\gamma}_2$, which is unique once we choose the initial condition. This means that the development with initial condition $e_G$ is precisely $\hat{\gamma}_2$ and that the holonomy is still $g^{-1}$.
	\end{proof}
	\begin{Obs}
		Clearly, in the case of a reductive geometry, there is an induced Ehresmann connetcion. Thus, in this simpler case, we have 2 holonomies,  THOSE ARE THE SAME????
	\end{Obs}
	\section{Metric compatibility and tensors}
	In this section we will look at the application of the covariant differentiation induced from the Cartan connections on tensors.\\
	\\
	We first of all define what is a tensor in a Cartan geometry. The definition is analogous to the one given in the case of an Ehresmann geometry.
	\begin{Def}
		Let $[(P,M,\pi,H),\omega]$ be a reductive Cartan geometry modelled on a Klein geometry $(G,H)$. Let $\rho:H\rightarrow GL(V)$ be a representation. A \textit{$\rho$-tensor} $\phi$ is a section of the associated bundle $P\times_H V$ i.e.
		$$\phi:M\rightarrow P\times_H V$$ 
	\end{Def}
	\begin{Obs}
		From REFERENZA, if $\rho_{1,2}:H\rightarrow GL(V),GL(W)$ are two representations, there is an identification:
		$$(P\times_H V)\otimes (P\times_H W)\simeq P\times_H(V\otimes W)$$
		It goes by itself than this justifies the natural generality of the above definition for tensors of all types.
	\end{Obs}
		\begin{Obs}
		Suppose to have a product $\eta:\mathfrak{m}\times \mathfrak{m}\rightarrow \mathbb{R}$ on $\mathfrak{m}$. Since the goemetry is reducible, this gives rise to a gauge-independent metric on $M$:
		$$g(X,Y)=\braket{s^*e(X),s^*e(Y)}\hbox{ with }X,Y\in\Gamma(M)$$
		Moreover, by REFERENZA, the solder form $e$ defines a bundle isomorphism:
		$$TM\simeq P\times_H \mathfrak{m}$$
	\end{Obs}
	\begin{Obs}
		Given the reducibility of the geometry, the Cartan connection splits as exposed in REFERENZA:
		$$\omega=A+e$$
		$A$ is an Ehresmann connection. Thus, $A$ gives rise to an exterior covariant derivative on the principal bundle $P$, which maps tensorial forms into tensorial forms. Moreover, given any representation $\rho:H\rightarrow GL(V)$, it induces a covariant differentiation on $P\times_H V$. In particular, let $s:M\rightarrow P$ be a local gauge and $\Phi:M\rightarrow P\times_H V\hbox{ like }\Phi(x)=[s(x),\phi(x)]$ be a section of any associated bundle: 
		$$\nabla^A_X \Phi(x)=[s(x),d\phi(X)+d\rho(s^*A(X))\phi(x)]$$
		$$\nabla^A:\mathfrak{X}(M)\times\Gamma(P\times_H V)\rightarrow \Gamma(P\times_H V)$$
	\end{Obs}
	\begin{Cor}
		The covariant derivative sends tensors into tensors. 
	\end{Cor}
	\begin{Obs}
		The above corollary has a deeper meaning. Recall that under a gauge transformation a tensor behaves like follows:
		$$\Phi(x)=[s(x),\phi(x)]\longrightarrow[s(x),\rho(g^{-1})\phi(x)]$$
		Taking its covariant derivative, under a gauge transformation we have:
		$$\nabla_X^A\Phi(x)\longrightarrow[s(x),\rho(g^{-1})(d\phi(X)+d\rho(s^*A(X))\phi(x))]$$
	\end{Obs}
	\begin{Cor}
		Clearly, if one has a section of an associated bundle like:
		$$\Phi:M\rightarrow P\times_H(V_1\otimes V_2\otimes...\otimes V_n)$$
		then the covariant derivative splits accoardingly:
		$$\nabla^A_X \Phi=[s,d\phi(X)+d\rho(s^*A(X))\phi]=$$
		$$=[s,d\phi+\sum_i d\rho_1(s^*A(X))\phi^i_1\otimes \phi^i_2\otimes...\otimes \phi^i_n+...\phi^i_1\otimes \phi^i_2\otimes...\otimes d\rho_n(s^*A(X))\phi^i_n]$$
		Where
		$$\rho=\rho_1\otimes \mathbb{I}_2\otimes...\otimes\mathbb{I}_n+...+\mathbb{I}_1\otimes \mathbb{I}_2\otimes...\otimes\rho_n$$
		is the induced total representation on the field:
		$$\phi(x)=\sum_i \phi^i_1(x)\otimes...\otimes \phi^i_n(x)$$
	\end{Cor}
	\begin{Prop}
		The solder form of a reductive Cartan geometry in which $\mathfrak{m}$ is endowed with a product, induces a metric tensor on the bundle $P\times_H\mathfrak{m}$.
	\end{Prop}
	\begin{proof}
		Let $\eta:\mathfrak{m}\times\mathfrak{m}\rightarrow \mathbb{R}$ be the product on $\mathfrak{m}$. From REFERENZA, the solder form can be used to pullback this on the manifold, in particular, there is a metric tensor $g$ on $M$ like:
		$$g=e_M^*\eta$$ 
		This defines an isomorphism $$TM\xrightarrow{\hat{g}} TM^* \hbox{ like }\hat{g}_x(X_x)=g_x(X_x,\cdot)$$ while the product $\eta$ on $\mathfrak{m}$ defines another isomorphism:
		$$P\times_H \mathfrak{m}\xrightarrow{\hat{\eta}} P\times_H\mathfrak{m}^*\hbox{ like }\hat{\eta}([p,v])=[p,\eta(v,\cdot)]$$. From REFERENZA, the dual bundle is $P\times_H \mathfrak{m}^*\simeq (P\times_H\mathfrak{m})^*$, so that this sequence of isomorphisms defines the map $e_M^*$ in the naive way:
		$$
		\begin{tikzcd}
			TM\arrow{d}{\hat{g}}\arrow{r}{e_M} &P\times_H\mathfrak{m}\arrow{d}{\hat{\eta}}\\
			TM^*\arrow{r}{e_M^*}&P\times_H\mathfrak{m}^*
		\end{tikzcd}
		$$
		$$e_M^*:TM^*\rightarrow P\times_H\mathfrak{m}^* \hspace{20 pt} e_M^*=\hat{\eta}\circ e_M\circ \hat{g}^{-1}$$
		Since the metric tensor is an element of:
		$$g\in\Gamma(TM^*\otimes TM^*)\hspace{20 pt}g:M\rightarrow TM^*\otimes TM^*$$
		we can construct:
		$$\tilde{g}=(e_M^*\otimes e_M^*)\circ g:M\rightarrow (P\times_H \mathfrak{m}^*)\otimes (P\times_H \mathfrak{m}^*)$$
		Clearly:
		$$\tilde{g}=(\hat{\eta}\circ e_M\circ \hat{g}^{-1} )\otimes(\hat{\eta}\circ e_M\circ \hat{g}^{-1} )\circ g$$
		Consider the following. From linear algebra, it is known that, since $\eta$ is a pairing from $P\times_H \mathfrak{m}$ to $P\times_H \mathfrak{m}^*$, then:
		$$\tilde{g}(x)(a,b)=(\eta\otimes \eta)((\hat{\eta}^{-1}\otimes \hat{\eta}^{-1})\tilde{g}(x),(a,b))$$
		where $a,b\in P\times_H\mathfrak{m}$.
		Substituting $\tilde{g}$ into the expression we get:
		$$\tilde{g}(x)(a,b)=(\eta\otimes \eta)[\sum_i (e_M\circ \hat{g}^{-1}(g^i_{1,x}))\otimes (e_M\circ \hat{g}^{-1}(g^i_{2,x})),(a,b)]$$
		Where we have expanded $g_x=\sum_{i}g^i_{1,x}\otimes g^i_{2,x}$. This automatically implies:
		$$\tilde{g}(x)(a,b)=\sum_i \eta(e_M\circ \hat{g}^{-1}(g^i_{1,x}),a)\cdot \eta(e_M\circ \hat{g}^{-1}(g^i_{2,x}),b)$$
		Recall from REFERENZA that $g=e_M^*\eta$ so that:
		$$\tilde{g}(x)(a,b)=\sum_i g(\hat{g}^{-1}(g^i_{1,x}),e_M^{-1}(a))\cdot g(\hat{g}^{-1}(g^i_{2,x}),e_M^{-1}(b))$$
		But this is exactly the tensor decomposition in components of:
		$$(e_M^{-1})^*g(a,b)$$
		Thus, we finally get:
		$$\tilde{g}=(e_M^{-1})^*g$$
		The symmetry is inherited from $g$, as well as the non degeneracy ($e_M$ is an isomorphism).
		$$
		\begin{tikzcd}
			M \arrow{rr}{g} \arrow[bend left=25]{rrrr}{\tilde{g}} && TM^*\otimes TM^* \arrow{rr}{e_M^*\otimes e_M^*} \arrow{dd}{\hat{g}^{-1}\otimes \hat{g}^{-1}} && (P\times_H \mathfrak{m}^*)\otimes (P\times_H \mathfrak{m}^*)\\\\
			&& TM\otimes TM \arrow{rr}{e_M\otimes e_M} && (P\times_H \mathfrak{m})\otimes (P\times_H \mathfrak{m}) \arrow{uu}{\hat{\eta}\otimes \hat{\eta}}
		\end{tikzcd}
		$$
	\end{proof}
	It is also worth noticing that the solder form allows us to construct orthonormal frames.
	\begin{Prop}
		Let $[(P,M,\pi,H),\omega]$ be a reductive Cartan geometry modelled on a Klein geometry $(G,H)$. And let $\eta$ be an invariant product for $\mathfrak{m}$. Then for any local gauge $s:M\rightarrow P$ and for any base $\{i_a\}$ of $\mathfrak{m}$, there is an orthonormal frame $f:M\rightarrow TM$ such that:
		$$e^af_b=\delta^a_b\hbox{ with } s^*e=e^a\otimes i_a$$
	\end{Prop}
	\begin{proof}
		Let $\{i_a\}$ be a base for $\mathfrak{m}$. This allows to define a unique frame:
		$$\{f_a:M\rightarrow TM\} \hbox{ such that }s^*e(f_a)=i_a$$
		This means that $e^a(f_b)=\delta^a_b$. Consequentially:
		$$g(f_a,f_b)=\eta_{ab}$$
	\end{proof}
	\section{Reduction of the Structure group}
	\begin{Obs}
		We have seen in REFERENZA that once we have an Ehresmann connection on a principal bundle $P$, we can induce a covariant differentiation on associated bundles i.e. a way to covariantly differentiate fields. In a reductive Cartan geometry, the Cartan connection splits $\omega=A+e$ and the new Ehresmann connection can be used as a tool to construct such differentiations.
	\end{Obs}
	\begin{Obs}
		Recall that from REFERENZA, if $A$ is an Ehresmann connection, $s:U\rightarrow P$ is a local gauge and $\Phi:M\rightarrow E$ is a section of an associated bundle to $P$ gener<ted from a representation $\rho$ like $\Phi(x)=[s(x), \phi(x)]$, then there is a covariant derivative:
		$$\nabla^A_X\Phi(x)=[s(x),d\phi(X)+d\rho(s^*A(X))\phi(x)]$$
		where $X\in\mathfrak{X}(M)$ is a vector field.
	\end{Obs}
	\begin{Def}
		Let $(P,M,\pi,G)$ be a principal $G$ bundle and $H$ a Lie group $G$. Let $\phi:h\rightarrow G$ be a Lie group homomorphism. We define a \textit{reduction of the structure group of $P$ via $\phi$} as a couple made of another principal bundle $(P',M,\pi',H)$ and a smooth fiber preserving map $\varphi: P'\rightarrow P$ such that:
		$$\varphi(p\cdot h)=\varphi(p)\cdot \phi(h)$$
		For all $h\in H$
	\end{Def}
		Let $M$ be a smooth manifold of dimension $n$. Suppose to endow it with a Lorentzian metric tensor $g\in\Gamma(TM\otimes TM)$. Given the smooth manifold structure we can automatically look at 2 bundles:
		\begin{itemize}
			\item[1] The tangent bundle:
			$$(TM,M,\pi_{TM},\mathbb{R}^n)$$
			\item[2] The frame bundle:
			$$(Fr(M),M,\pi_{Fr},GL(n,\mathbb{R}))$$
		\end{itemize}
		Since we have a Lorentzian metric $g$, we can perform a \textit{reduction of the structure group}. We define a new set:
		$$Fr_O(M)=\{e\in Fr(M)|e^*g=\eta\}$$
		where $\eta$ is the minkowski metric with signature $(p,q)$. We now show that this defines a new principal bundle.
		\begin{Prop}
			$Fr_O(M)$ is a smooth manifold and defines a new principal $O(p,q)$ bundle.
		\end{Prop}
		\begin{proof}
			We divide the proof in steps:
			\begin{itemize}
				\item [1] We first prove that the new set is a principal bundle. We use the \textit{preimage theorem} REFERENZA. In particular, consider:
				$$f:FR(M)\rightarrow \mathcal{L}$$
				Where $\mathcal{L}$ is the vector space of symmetric matrices with signature $(p,q)$. Being a vector space, this is a manifold. The map behaves like:
				$$f(u)=u^*g$$
				and is clearly smooth. N
				ow, consider $f^{-1}(\eta)=Fr_O(M)$ by definition. If we prove that the differential:
				$df_e:T_eFr_O(M)\rightarrow T_{f(e)}\mathcal{L}$ is surjective for all $e\in f^{-1}(\eta)$ we are done.
				To do this, we define the following smooth curve in $Fr(M)$:
				$$\gamma(t)=(ue^{tA})^*g=e^{tA^T}u^*ge^{tA}$$
				This is clearly a curve in $Fr(M)$ with starting point $u^*g$. Selecting $u=e\in Fr_O(M)$ we find:
				$$df_e(A)={d\over dt}\bigg|_0 \gamma(t)=A^T\eta+1\eta A$$
				This is clearly surjective since if $S\in\mathcal{L}$ is a symmetric bilinear form, then, choosing $A={1\over 2}\eta S$ we get:
				$${1\over 2}[S\eta^2+\eta^2S]=S$$
				Where we have used the fact that $\eta^2=1$. This proves that $Fr_O(M)$ is a submanifold of $Fr(M)$.
				\item [2] Now we prove that this new manifold induces a principal bundle on $M$. Consider the following projection:
				$$\pi_{Fr_O}:Fr_O(M)\rightarrow M\hspace{20 pt}\pi_{Fr_O}(e)=\pi_{Fr}(e)$$
				This is clearly a smooth projection. As for the right action of the group $O(p,q)$ on the manifold, it is given clearly by:
				$$\mu(O,e)=eO\hbox{ since }(eO)^*g=O^T\eta O=\eta$$
				This action is obviously free since frames are ordered bases. Moreover, the action of the group changes the basis, but not the point at which the base is at, so $\pi_{Fr_O}\circ \mu=\pi_{Fr_O}$\\
				As for the local trivializations, consider any $U\subset M$ open. Then, there is a natural map:
				$$\phi_O:\pi_{Fr_O(M)}^{-1}(U)\rightarrow U\times O(p,q) \hbox{ like }\phi_O(e)=(\pi(e),O)$$
				where $O$ is the matrix that sends the canonical base of $\mathbb{R}^n$ into $e$. This map is evidently smooth and has an inverse, which is also smooth:
				$$\phi_O^{-1}(x,O)=i(x)\cdot O\hbox{ where } i \hbox{ is the caonical base of }\mathbb{R^n}$$
				$$\phi_O\circ \phi_O^{-1}(x,O)=\phi(i(x)\cdot O)=(\pi(i),O)$$
				To conclude the bundle structure, we prove the equivariance of the trivializations:
				$$\phi_O(e\cdot A)=(\pi(e),AO)=(\pi(e),A)\cdot O$$
				This concludes the proof.
			\end{itemize}
			\begin{Obs}
				The last result REFERENZA implies that given a manifold with a metric of signature $(p,q)$ we can always reudce its structure group to $O(p,q)$ from $GL(n,\mathbb{R})$. Geometrically, this corresponds to working only with the orthonormal frames.\\
				Note that if in addition the manifold is orientable we can further reduce the structure group to $SO(p,q)$. The proof of this result is literally the same as the one of last result.
			\end{Obs}
		\end{proof}
		Now we can proceed with the analysis. Since our manifold is endowed with a Lorentzian metric and is supposed to be orientable, we can consider the special orthonormal bundle:
		$$(Fr_{SO}(M),M,\pi_{Fr_{SO}},SO(p,q))$$
		Our aim is to find a Cartan connection on this principal bundle. Consider the following:
		\begin{Theo}
			Let $(M,g)$ be a metric Lorentzian manifold. Then, there is a canonical solder form on the frame bundle. This descends to a canonical solder form on the orhtonormal and special orthonormal frame bundles.
		\end{Theo}
		\begin{proof}
			Recall that the solder form is an element $\theta\in\Omega^1(Fr(M),\mathfrak{m})$ which at any point provides an isomorphism. Our model homogeneous space for the most general case is:
			$$G=GL(n,\mathbb{R})\ltimes\mathbb{R}^n;\hspace{20 pt}H=GL(n,\mathbb{R})$$
			This is a Klein Geometry since $G/H$ is connected. Since the group is a semidirect product, by REFERENZA, the adjoint action of $GL(n,\mathbb{R})$ on $\mathbb{R}^n$ is:
			$$Ad(h)v=hv$$
			From this it clearly follows that this homogeneous space is reductive. Thus, we wish for a map:
			$$\theta:TFr(M)\rightarrow \mathbb{R}^n$$
			which when pulled back gives an isomorphism. The idea is to mimic the Maurer-Cartan form:
			$$\theta_e(X_e)=e^{-1}\circ d\pi_{Fr}(X_e)$$
			Basically, a point of the frame bundle $e\in Fr(M)$ is by definition a linear isomorphism $e(x):\mathbb{R}^n\rightarrow T_xM$, so it is invertible and the inverse maps a vector in the tangent space into a decomposition in $\mathbb{R}^n$. \\
			By definition of the projection, if we pull it back with any section of the frame bundle $e:U\subset M\rightarrow Fr(M)$, we get a linear isomorphism:
			$$e^*\theta\in\Omega^1(M,\mathbb{R}^n)$$
			$$(e^*\theta)_x(X_x)=\theta_{e(x)}(de(X_x))=e^{-1}(x)(d\pi\circ de(X_x))=e^{-1}(x)(X)$$
			This is clearly an isomorphism. If now we reduce the structure group to $O(p,q)$ we have another model space:
			$$G=O(p,q)\ltimes \mathbb{R}^n$$
			This is still a Klein Gometry and is reductive. The solder form is still the same, but it has values in $Fr_O(M)$. The same exact thing holds for $SO(p,q)$.
		\end{proof}
		The only thing which is missing is an Ehresmann connection $A$. Once we have it we automatically have a Cartan connection:
		$$A=\omega+\theta$$
		Then we can calculate the total curvature and all of the other meaningful quantities. In particular, the curvature and the torsion will be:
		$$R=d\omega+{1\over 2}[\omega,\omega]\hspace{ 20 pt} T=d\theta+{1\over 2}[\theta,\omega]$$
		Another important condition is the metric compatibility. Recall that from REFERENZA there is only one connection $\omega$ which is compatible with the metric and gives $0$ torsion: the Levi-Civita one. We are now going to show the Levi-Civita case and then look at the torsion case.\\
		\\
		The Levi-Civita connection conditions are the following:
		$$d\theta+{1\over 2}[\theta,\omega]=0 \hbox{ (Torsion condition)}$$
		$$\nabla g=0$$
		The second condition can be rephrased as:
		$$\omega\in\Omega^1(Fr_{SO}(M),\mathfrak{so}(n))$$
		this latter condition is already satisfied in the special orthonormal frame bundle.\\
		To see the uniqueness of the Levi-Civita connection we recall that if $\omega_{1,2}$ are two Ehresmann connnections, then they differ by a tensorial form of type $Ad$ of degree 1. By plugging $\omega_1=\omega_2+\xi$ into the Torsion condition we see:
		$$d\theta+{1\over 2}[\theta,\omega_2]+{1\over 2}[\theta,\xi]=0=+{1\over 2}[\theta,\xi]$$
		From this analysis it follows:
		\begin{Cor}
			All of the torsionless connections $\omega$ differ by some $\chi\in\Omega^1_{Ad}(P,\mathfrak{h})$ such that
			$$[\chi,\theta]=0$$
		\end{Cor}
		Thus, there is no unique torsionless Ehresmann connections in general. However, a funny thing happens when we look at this in the associated bundle. See REFERENZA.
		Let now $e:U\subset M\rightarrow Fr_{SO}(M)$ be a local gauge. Then, in coordinates:
		$$e^*\theta=(e^j)^{-1}\otimes i_j\hbox{ and }e^*\omega=\omega^i_j\otimes T^j_i$$
		where $i^j$ are the generator of $\mathbb{R}^n$ and $T^j_i$ are the generators of $\mathfrak{so}(n)$.
		In fact, $\omega$ has values in the matrix Lie algebra $\mathfrak{so}(n)$, so it is a matrix. Note that this description is not coordinate dependent!!! Thus, one gets $n$ differential equations (from REFERENZA the pullback commutes with the exterior derivative):
		$$d(e^j)^{-1}\otimes i_j+(e^j)^{-1}\wedge \omega^i_k \otimes [i_j,T^i_k]=0$$
		\textbf{Important:}
		\\
		Note that we wrote the commutator. However, the action of the total Lie algebra on the subspace $\mathbb{R}^n=\mathfrak{m}$ is linear, due to the direct product definition! (see REFERENZA) This means that in reality we are doing:
		$$[(0,i_j),(T^k_i,0)]=-(0,T_i^k\cdot i_j)$$
		Thus, we find:
		$$d(e^j)^{-1}\otimes i_j+ \omega^i_k\wedge (e^j)^{-1} \otimes T_i^k\cdot i_j=0$$
		Now, there is a natural choice for a basis in $\mathfrak{gl}(n)$, namely:
		$$T^i_j\hbox{ such that }T^i_j\cdot i_k=\delta_k^i i_j$$
		This basis can be used to expand an element of $\mathfrak{so}(n)$ like:
		$$e^*\omega=\omega^i_j\otimes T^j_i$$
		However, by construction $\omega\in\mathfrak{so}(p,q)$ so that:
		$$e^*\omega^T\eta+\eta e^*\omega=0$$
		$$\omega^j_k\eta_{ji}=-\eta_{ik}\omega^k_j$$ 
		This means:
		$$\omega_{ik}=-\omega_{ki}$$
		$$d(e^j)^{-1}\otimes i_j+ \omega^i_k\wedge (e^j)^{-1} \otimes T_i^k\cdot i_j=0=$$
		$$d(e^j)^{-1}\otimes i_j+ \omega^i_j\wedge (e^j)^{-1} \otimes i_i=0$$
		Or also:
		$$[d(e^j)^{-1}+ \omega_i^j\wedge (e^i)^{-1} ]\otimes i_j=0$$
		Note that $\omega$ is skew-symmetric when contracted with the metric $\eta$.\\
		\\
		Now we get to Christoffel symbols. Since $\omega^i_j$ is a 1 form with values in $\mathbb{R}$ (do not get confused with the 2 index notation, it is just because we have expanded the generators with 2 indices), we can expand it in a local frame:
		$$\omega^i_j=\Gamma^i_{jk}(e^k)^{-1}$$
		We call $\Gamma^i_{jk}$ the \textit{Christoffel symbols}. Equation REFERENZA becomes:
		$$[de_j^{-1}+ \Gamma^i_{jk}(e^k)^{-1}\wedge e_i^{-1}]\otimes i^j=0$$
		\begin{Prop} [\textbf{Passive interpretation of gauge transformations}]
			Under a gauge transformation the Christoffel symbols transform like:
			$$\Gamma^\rho_{\mu\nu}=(g^\rho_i)^{-1}\Gamma^i_{jk}g^j_\mu (g^k_\nu)^{-1}+(g_m^\rho)^{-1}\partial_\nu g^m_{\mu}$$
		\end{Prop}
		\begin{proof}
			Recall that $\omega$ is an Ehresmann connection. Thus, under a gauge transformation, it changes as:
			$$e^*\omega\rightarrow Ad(g^{-1})e^*\omega+\mu=$$
			$$=Ad(g^{-1})\Gamma^i_{jk}(e^k)^{-1}\otimes T^j_i+\mu$$
			Recall that in our case the Lie group is a matrix Lie group. This implies that the transtion function $g$ will be a matrix. In components:
			$$\Gamma^\rho_{\mu k}(f^k)^{-1}=(g^\rho_i)^{-1}\Gamma^i_{jk}g^j_\mu (e^k)^{-1}+(g_m^\rho)^{-1}dg^m_{\mu}$$
			The gauge transformation correspond to the change in local section. This means that, calling $f$ the new section, $f^\nu=g^\nu_ke^k$ in matrix terms. Substituting everything we find:
			$$\Gamma^\rho_{\mu\nu}=(g^\rho_i)^{-1}\Gamma^i_{jk}g^j_\mu (g^k_\nu)^{-1}+(g_m^\rho)^{-1}\partial_\nu g^m_{\mu}$$
			Where we have intended $g$ as a smooth map from $U\subset M$ to $G$.
		\end{proof}
		\begin{Prop}
			The Christoffel symbols in any coordinate frame are symmetric in the lower indices if and only if the Torsion tensor in the same coordinates are also 0.
		\end{Prop}
		\begin{proof}
			The torsion tensor is:
			$$T=d\theta+{1\over 2}[\theta,\omega]$$
			We have seen that under a pullback with a local gauge we find:
			$$T^j\otimes i_j=[d(e^j)^{-1}+ \Gamma^j_{ik}(e^k)^{-1}\wedge (e^i)^{-1}]\otimes i_j$$
			Let $(U,x^i)$ be the coordinate frame in which $e$ is the canonical base. Then by REFERENZA:
			$$d(e^j)^{-1}(\partial_{x^i},\partial_{x^j})=\partial_{x^i}(e^j)^{-1}(\partial_{x^j})-\partial_{x^j}(e^j)^{-1}(\partial_{x^i})-(e^j)^{-1}[\partial_{x^i},\partial_{x^j}]$$
			Clearly, this is 0. As for the second term, fromm REFERENZA:
			$$(e^k)^{-1}\wedge (e^i)^{-1}(X,Y)=(e^k)^{-1}(X) (e^i)^{-1}(Y)-(e^k)^{-1}(Y) (e^i)^{-1}(X)$$
			This measn that, since $e$ is the frame associated with the coordinates:
			$$(e^k)^{-1}\wedge (e^m)^{-1}(\partial_{x^i},\partial_{x^j})=\delta^k_i\delta^m_j-\delta^k_j\delta^m_i$$
			By substituting everything into the torsion formula we get:
			$$e^*T(\partial_m,\partial_l)=[\Gamma^j_{lm}-\Gamma^j_{ml}]\otimes i_j$$
		\end{proof}
		\begin{Obs}
			Note that in general the torsion is not 0 as a form. In fact, from REFERENZA, if we consider a general local gauge which does not induce a coordinate frame, then:
			$$e^*T(e_i,e_k)=[\Gamma^j_{ik}-\Gamma^j_{ki}-(e^j)^{-1}[e_i,e_k]]\otimes i_j$$
		\end{Obs}
		\begin{Obs}
			Recall that from REFERENZA, the solder form provides an isomprhism:
			$$TM\simeq FR_{SO}(M)\times_{SO(p,q)} \mathbb{R}^n$$
			This quotient is took with respect to the adjoint action. 
			In our case, the total group is $ISO(p,q)=SO(p,q)\ltimes \mathbb{R}^n$, so the restriction of the adjoint action on $\mathbb{R}^n$ is from REFERENZA the standard action of $SO(p,q)$ on vectors of $\mathbb{R}^n$.\\
			Now, by REFERENZA, there is a covariant derivative on sections of this associated bundle: vector fields. Namely, if $X\in\mathfrak{X}(M)$ is  avector field, then it can be seen also as a section of $FR_{SO}(M)\times_{SO(p,q)} \mathbb{R}^n$.
			$$X(x)=[e(x),v(x)]$$
			From REFERENZA, once we fix an Ehresmann connection $\omega$, we have a covariant derivative on this associated bundle:
			$$\nabla_Y^\omega X(x)=[e(x),dv(Y)+e^*\omega(Y)v(x)]$$
			Where, from REFERENZA, the differential of the standard representation is just the inclusion:
			$$d\rho_{std}(a)=a$$
			This is the \textit{covariant derivative induced on the tangent bundle}.
		\end{Obs}
		\begin{Prop}
			There is a natural extension of the covariant derivative on general tensors.
		\end{Prop}
		\begin{proof}
			So far, we have defined a covariant derivative on the bundle $TM$. Since there is an a priori Minkowsky metric on $\mathbb{R}^{p,q}$, we can identify canonically $\mathbb{R}^n\simeq \mathbb{R}^*$. Through the solder form and REFERENZA we get a bundle isomorphism:
			$$TM^*\simeq FR_{SO}(M)\times_{SO(p,q)} \mathbb{R}^{p,q}$$
			Since the adjoint action is just $\rho^*(g)=\rho(g^{-1})$ from REFERENZA, by differentiating we get a $(-)$ sign:
			$$\nabla_Y^\omega X^*(x)=[e(x),dv^*(Y)-v(x)^*(e^*\omega(Y))]$$
			Since from REFERENZA any representation of Lie algebras naturally extends to the tensor product space as fwollows (by abuse of notation):
			$$d\rho_1\otimes d\rho_2=d\rho_1\otimes \mathbb{I}+\mathbb{I}\otimes d\rho_2$$
			we have a recipe to extend the exterior derivative to a general tensor. In particular, if:
			$$X\in\Gamma(\bigotimes^n TM\bigotimes^m TM^*)$$
			then we have a natural covariant derivative like:
			$$\nabla_Y^\omega X=[e(x),dv_x(Y)+d\rho_{nm}(e^*\omega(Y))v(x)]=$$$$=[e(x)+dv_x(x)+d\rho_1(e^*\omega(Y))(v_1(x))\otimes\mathbb{I}(v_2)...\otimes\mathbb{I}(v_{n+m})+...]$$
			Where we have called $v=v_1\otimes...\otimes v_n\otimes v_{n+1}...\otimes v_{n+m}$ and $d\rho_{nm}$ the full tensor representation.
		\end{proof}
		\begin{Prop}
			The Levi-Civita Christoffel symbols are:
			$$\Gamma^\rho_{\mu\nu}={1\over 2}g^{m\rho}(\partial_\mu g_{\nu\rho}+\partial_\nu g_{\mu\rho}-\partial_m g_{\mu\nu})$$
		\end{Prop}
		\begin{proof}
			The proof relies on the extension of the covariant derivativ eto 2 tensors. In partiuclar, the condition $\nabla g=0$ can be re-written as the derivaitve of a section of $TM^*\otimes TM^*$, like:
			$$\nabla_X^\omega g(Y,Z)=0=[e,dg(X)(Y,Z)-g(e^*\omega(X)Y,Z)-g(Y,e^*\omega(X)Z)]$$
			this true only for:
			$$dg(X)(Y,Z)=g(e^*\omega(X)Y,Z)+g(Y,e^*\omega(X)Z)$$ 
			This must hold for all vector fields. Since we have selected a frame $e$, we can choose $\{e_i\}$ as our vector fields. Then:
			$$dg_{\mu\nu}(X)=\omega_\mu^jg_{j\nu}+\omega_\nu^jg_{j\nu}$$
			Clearly, by substituting the expression for $\Gamma^i_{jk}$ we get:
			$$dg_{\mu\nu}=\Gamma^j_{\mu\rho}(e^\rho)^{-1}g_{j\nu}+\Gamma^j_{\nu\rho}(e^\rho)^{-1}g_{j\mu}$$
			Suppose that $e$ induces a coordinate frame on the manifold. Then:
			$$dg_{\mu\nu}=\partial_\rho g_{\mu\nu}(e^\rho)^{-1}$$
			$$\partial_\rho g_{\mu\nu}(e^\rho)^{-1}=\Gamma^j_{\mu\rho}(e^\rho)^{-1}g_{j\nu}+\Gamma^j_{\nu\rho}(e^\rho)^{-1}g_{j\mu}$$
			Now, if the connection is torsionless i.e. Levi-Civita, then the Christoffel symbols are symmetric in the exchange of the lowe indices.
			$$\partial_\rho g_{\mu\nu}-\partial_\mu g_{\rho\nu}-\partial_\nu g_{\mu\rho}=$$
			$$=\Gamma^j_{\mu\rho}g_{j\nu}+\Gamma^j_{\nu\rho}g_{j\mu}-
			\Gamma^j_{\rho\mu}g_{j\nu}-\Gamma^j_{\nu\mu}g_{j\rho}-
			\Gamma^j_{\mu\nu}g_{j\rho}-\Gamma^j_{\rho\nu}g_{j\mu}$$
			Now, if we group those up by symmetry, we find:
			$$2g_{j\rho}\Gamma^j_{\mu\nu}=\partial_\rho g_{\mu\nu}-\partial_\mu g_{\rho\nu}-\partial_\nu g_{\mu\rho}$$
			This completes the proof. 
		\end{proof}
		\begin{Obs}
			Basically, we have seen that if we have a manifold endowed with a Lorentzian metric, we have a canonical choice for the Cartan connection.
		\end{Obs}
	\section{Gravitation in Cartan Geometry}
	In this section we will look at General relativity as a gauge theory, through the scope of the Cartan construction.
	\\\\
	We define $M$ to be an $n$ dimensional spacetime. Our general symmetry group is:
	$$G=\begin{cases}
		SO(1,n) \hspace{10 pt}(\Lambda>0 \hbox{ De Sitter})\\
		ISO(1,n-1)\hspace{10 pt} (\Lambda=0 \hbox{ Minkowski})\\
		SO(2,n-1)\hspace{10 pt} (\Lambda<0 \hbox{ Anti-De Sitter})
	\end{cases}$$
	Our stabilizer is always $H=SO(1,n-1)$.
	\begin{Prop}
		Let the notation be as above, then $G/H$ is always a reductive homogeneous space.
	\end{Prop}
	\begin{proof}
		We divide the proof into 3 cases, each one for the respective symmetry group.
		In general:
		$$\mathfrak{so}(p,q)=\{X\in M_{(p+q)\times(p+q)}|X^T\eta+\eta X=0\}$$
		\begin{itemize}
			\item Case 1: $G=SO(1,n)$
			\\
			In this first case the Lie algebra $\mathfrak{g}=\mathfrak{so}(1,n)$ is:
			$$\mathfrak{so}(1,n)=\{X\in M_{(n+1)\times(n+1)}|X^T\eta+\eta X=0\}$$
			To see what this implies, consider
			$$X=\begin{pmatrix}
				M && \vec{v} \\
				\vec{w}^T && A
			\end{pmatrix}\hbox{ with }A\in\mathbb{R},M\in M_{n\times n}(\mathbb{R})$$
			Then the above condition is:
			$$\begin{pmatrix}
				M^T && \vec{w} \\
				\vec{v}^T && A
			\end{pmatrix}\begin{pmatrix}
			\eta_{1,n-1} && 0 \\
			0 && 1
			\end{pmatrix}=-\begin{pmatrix}
			\eta_{1,n-1} && 0 \\
			0 && 1
			\end{pmatrix}\begin{pmatrix}
			M && \vec{v} \\
			\vec{w}^T && A
			\end{pmatrix}$$
			Which becomes:
			$$\begin{pmatrix}
				M^T\eta_{1,n-1} && \vec{w} \\
				\vec{v}^T\eta_{1,n-1} && A
			\end{pmatrix}=\begin{pmatrix}
			-\eta_{1,n-1}M && -\eta_{1,n-1}\vec{v} \\
			-\vec{w}^T && -A
			\end{pmatrix}$$
			This clearly implies $M^T\eta_{1,n-1}=-\eta_{1,n-1}M$ i.e. $M\in \mathfrak{so}(1,n-1)$, $-\eta_{1,n-1}\vec{v}=\vec{w}$, $A=0$. Thus, we see there is a splitting:
			$$\mathfrak{so}(1,n)=\mathfrak{so}(1,n)\oplus \mathfrak{m}$$
			Where 
			$$\mathfrak{m}=\{X\in M_{(n+1)\times (n+1)}|X=\begin{pmatrix}
				0 && -\eta_{1,n-1}\vec{v} \\
				\vec{v}^T && 0
			\end{pmatrix}\}$$
			Now it remains to show that this space is reductive:
			$$[\mathfrak{so}(1,n-1),\mathfrak{m}]=\begin{pmatrix}
				M && 0 \\
				0 && 0
			\end{pmatrix}\begin{pmatrix}
				0 && -\eta_{1,n-1}\vec{v} \\
				\vec{v}^T && 0
			\end{pmatrix}-\begin{pmatrix}
			0 && -\eta_{1,n-1}\vec{v} \\
			\vec{v}^T && 0
			\end{pmatrix}\begin{pmatrix}
			M && 0 \\
			0 && 0
			\end{pmatrix}=$$
			$$=\begin{pmatrix}
				0 && -M\eta_{1,n-1}\vec{v} \\
				-\vec{v}^TM && 0
			\end{pmatrix}=\begin{pmatrix}
			0 && -\eta_{1,n-1}^TM^T\vec{v} \\
			\vec{v}^TM && 0
			\end{pmatrix}\in\mathfrak{m}$$
			\item Case 2: $G=SO(2,n-1)$\\
			The proof is very similar, the only change is in:
			$$\eta=\begin{pmatrix}
				-1 && 0\\
				0 && \eta_{1,n-1}
			\end{pmatrix}$$
			\item Case 3: $G=ISO(1,n-1)$
			\\
			The Poincarè group is given by:
			$$ISO(p,q)=\{\}$$
		\end{itemize}
	\end{proof}
	Let $G$ be one of the above simmetry groups. Consider a generic principal bundle $P$ on spacetime:
	$$(P,M,\pi,H)\hbox{ where }H=SO(1,n-1)$$
	\begin{Prop}Let the notation be as above.
		$\mathfrak{m}$ has always a Lorentzian inner product.
	\end{Prop}
	\begin{proof}
		\begin{itemize}
			\item Case 1: $G=SO(1,n)$\\
			$$\mathfrak{m}=\mathbb{R}^n$$
			Consider the Killing form on $\mathfrak{so}(p,q)$, this is an $Ad$-invariant product, even if in this case it is not negative definite. Clearly, on $\mathfrak{m}$ this is $Ad(\mathfrak{so}(1,n-1))$ invariant. We can check the restriction of this product on $\mathfrak{m}$ to reduce to the Minkowsky metric. In particular it is known from REFERENZA that for $\mathfrak{so}(p,q)$:
			$$B(X,Y)=(p+q-2)Tr(XY)$$
			This automatically imples that:
			$$B_\mathfrak{m}(X,Y)=(q-1)Tr\bigg(\begin{pmatrix}
				0 && -\eta_{1,n-1}\vec{v} \\
				\vec{v}^T && 0
			\end{pmatrix}\begin{pmatrix}
			0 && -\eta_{1,n-1}\vec{w} \\
			\vec{w}^T && 0
			\end{pmatrix}\bigg)=$$
			$$=(q-1)Tr\bigg(\begin{pmatrix}
				-\eta_{1,n-1}\vec{v}\vec{w}^T && 0 \\
				0 && -\vec{v}^T\eta_{1,n-1}\vec{w}
			\end{pmatrix}\bigg)=-2(q-1)\eta_{1,n-1}(\vec{v},\vec{w}^T)$$
			\item Case 2: $G=ISO(1,n-1)$
			By definition 
			$$ISO(p,q)=SO(p,q)\ltimes \mathbb{R}^{p,q}$$
			This is a group with the following operations:
			$$(X,v)\cdot (Y,w)=(XY,Xw+v)$$
			The commutator is:			$$[(X,v),(Y,w)]=([X,Y],Xw-Yv)$$
			This clearly implies that all of the generators of $\mathbb{R}^{p,q}$ commute. Thus, the Killing form is going to be degenerate on the subset $\mathfrak{m}=\mathbb{R}^{p,q}$
			Consider now the adjoint action of $\mathfrak{so}(p,q)$ on $\mathfrak{m}=\mathbb{R}^{p,q}$:
			$$Ad(X)v=-Xv$$
			This makes an irreducible representation. Consider then a generic symmetric bilinear form $B$ on $\mathfrak{m}=\mathbb{R}^{p,q}$. Suppose that $B$ is $Ad(SO(p,q))$ invariant. Then, since we have the Minkowski product on $\mathbb{R}^{p,q}$, there is a 1-1 correspondence between bilinear forms and endomorphisms:
			$$T=\eta^{-1}\circ B$$ 
			By Schur's Lemma (see REFERENZA), invariant map between $V$ and itself is a multiple of $\mathbb{I}$. This implies that any $B$ is a multiple of the Minkowsky form.
			\item Case 3: $G=S=(2,n-1)$
			For $\mathfrak{so}(2,n-1)$ it is the same as point 1 up to a proper rescaling.
		\end{itemize}
	\end{proof}
	Thus, in each case we have $\eta=\eta_{1,n-1}$ as a Lorentzian inner product. Consider any Cartan connection $\omega$ on this principal bundle. By REFERENZA, we have a reductive Cartan Geometry:
		$$[(P,M,\pi,H),\omega]$$
	Furthermore, by REFERENZA, the Cartan connection splits as:
	$$\omega=A+e$$
	Where 
	\begin{itemize}
		\item $A\in\Omega^1(P,\mathfrak{h})$ is an Ehresmann connection;
		\item $e\in\Omega^1(P,\mathfrak{m})$ is the solder form.
	\end{itemize}
	From REFERENZA, since this geometry is reducible, there is an horizontal distribution $\mathcal{H}=Ker(A)$.\\
	\begin{Obs}
		From REFERENZA, since the goemetry is reducible, the metric on $\mathfrak{m}$ gives rise to a gauge-independent metric on $M$:
		$$g(X,Y)=\braket{s^*e(X),s^*e(Y)}\hbox{ with }X,Y\in\Gamma(M)$$
		Moreover, by REFERENZA, the solder form $e$ defines a bundle isomorphism:
		$$TM\simeq P\times_H \mathfrak{m}$$
		This is done through REFERENZA and we call the associated form $e_M$ \textit{coframe field}.
		Let $s:U\subset M\rightarrow P$ be a local gauge. Then we can expand the solder form:
		$$e:TP\rightarrow \mathbb{R}^{p,q}$$
		$$s^*e:TM\rightarrow \mathbb{R}^{p,q}\hspace{20 pt}s^*e=e^a\otimes i_a$$
		Where $i^a$ generate $\mathbb{R}^{p,q}$. Plugging this into the relation above, we get:
		$$g=\Lambda e^ae^b\eta_{ab}$$
		where $\Lambda$ is the coupling re-scaling constant of the product $\eta$. Clearly, from REFERENZA, there is an induced bundle metric on $TM\simeq P\times_H \mathfrak{m}$, which gives rise to the following local relation (see REFERENZA):
		$$\braket{e_M,e_M}=\braket{s^*e,s^*e}$$
		The core idea is the following: once the solder form is known (or quivalently the coframe field), the metric tensor also is.
	\end{Obs}
	\begin{Obs}
		The spin connection $A$ is the Ehresmann-like part of the Cartan connection in our setting. It defines an exterior covariant derivative $D_A$ on the principal bundle and a covariant derivative on $TM$ and $TM^*$ through the above coframe field. 
	\end{Obs}
	\begin{Prop}
		The request that the spin connection $A$ is compatible with the metric tensor is realized by the sole fact of $A$ taking values in $\mathfrak{so}(1,n-1)$.
	\end{Prop}
	\begin{proof}
		The metric compatibility condition is 
		$$\nabla g=0$$
		From REFERENZA, the spin connection $A$ induces a covariant derivative on the tangent bundle $TM\simeq P\times_H \mathfrak{m}$. In particular, let $s:M\rightarrow P$ be a local gauge and $\Phi:M\rightarrow P\times_H \mathfrak{m}\hbox{ like }\Phi(x)=[s(x),\phi(x)]$ be a section of the associated bundle. The covariant derivative on $P\times_H G$ is:
		$$\nabla^A_X \Phi(x)=[s(x),d\phi(X)+d\rho(s^*A(X))\phi(x)]$$
		where $\rho$ is the adjoint action of $H$. Since the section $\Phi\in\Gamma(P\times_H G)$ corresponds to a section of $TM$, namely $\Phi_M=e_M^{-1}\circ\Phi:M\rightarrow TM$, we can define:
		$$\nabla_X^A\Phi_M=e_M^{-1}\circ \nabla^A_X \Phi(x)$$
		This is globally defined as an operator and it is gauge invarant due to the gauge covariance of $\nabla^A_X$. Of course, there is also a covariant derivative on $TM^*$, induced from:
		$$e_M^*:TM^*\rightarrow P\times_H \mathfrak{m}^*$$
		From REFERENZA:
		$$\tilde{g}=(e_M^{-1})^*g$$
		To compute the covariant derivative of the metric tensor, consider the following: if $\varphi$ is a section of $P\times_H \mathfrak{m}^*$ and $\varphi_M$ is the corresponding section of $TM^*$, then:
		$$\nabla_X^A\varphi_M=(e_M^*)^{-1}\circ \nabla_X^A\varphi$$
		From this we have the obvious generalization:
		$$\nabla_X^A g=((e_M^*)^{-1}\otimes(e_M^*)^{-1})\nabla_X^A \tilde{g}$$
		This means:
		$$\nabla_X^A \tilde{g_x}=\nabla_X^A[s(x),\sum_iv^i\otimes w^i]\hbox{ with }v^i,w^i\in \mathfrak{m}^*$$
		Since the action of the adjoint on $\mathfrak{m}^*$ is the conjugate of the fundamental representation, we get:
		$$\nabla_X^A \tilde{g_x}=[s(x),d\tilde{g}_x(X)-\sum_is^*A(X)v^i\otimes w^i+v^i\otimes s^*A(X)w^i]$$
		Where we have applied the decompostion of the representation used in REFERENZA. Now, clearly by construction of the dual bundle, the following is forced: $v^i=g^i_1(e_M^{-1})$ and $w^i=g^i_2(e_M^{-1})$ so that:
		$$\nabla_X^A g=d((e_M^{-1})^*g)(X)-\sum_i s^*A(X)g^i_1(e_M^{-1})\otimes g^i_2(e_M^{-1})+g^i_1(e_M^{-1})\otimes s^*A(X)g^i_2(e_M^{-1})$$ 
		Now, the compatibility condition amounts to $\nabla g=0$. This means that, taking out the pullback:
		$$dg(X,Y,Z)-s^*A\cdot g(Y,Z)=0$$
		Where we have substituted the covariant derivatives on the metric components with the full metric expansion. Since $g=e^*\eta$ by construction we have:
		$$dg(X,Y,Z)=g(s^*A(Y),Z)+g(Y,s^*A(Z))$$
		Choose from REFERENZA an orthonormal frame $\{f_a\}$. Then the compatibility condition at a point $p$ becomes:
		$$X_pg_p(f_a(p),f_b(p))=0=[(s^*A)_a^cf_c \eta_{cb}+(s^*A)_b^cf_c \eta_{ca}]_p$$
		This clearly implies the skew symmetry. Note that we get: $s^*A\in\mathfrak{so}(p,q)$
		However, the pullback of the connection does not change the fact that the connection takes values in the Lie algebra $\mathfrak{so}(p,q)$. This property comes out as a requirement in an orthonormal frame, yet it is more general.
	\end{proof}
	Now it is only a matter to find the correct action for this theory. The aim is to recollect the Einstein equations. However, we first need to endow our space $H$ with a product.
	\begin{Prop}
		There is a natural choice for an $Ad(H)$-invariant product on $\mathfrak{g}=\mathfrak{h}\oplus\mathfrak{m}$. That is:
		$$\braket{,}_\mathfrak{g}=B_\mathfrak{h}\circ\Pi_\mathfrak{h}+\eta\circ\Pi_\mathfrak{m}$$
		Where $\Pi$ are the projectors on the subspaces.
	\end{Prop}
	\begin{proof}
		We have seen that in all the cases of REFERENZA, $\mathfrak{m}$ is always equipped with a Lorentzian inner product. As for the somplement, it is $\mathfrak{so}(1,n-1)$, which is simple for $n>1$, but non-compact. The Cartan-Killing form is thus a non degenerate indefinite product. This completes the proof.
	\end{proof}
	\begin{Prop}
		The Einstein Lagrangian for an empty spacetime is:
		$$\mathcal{L}dV=$$
		Where $dV$ is the volume form.
	\end{Prop}
	\begin{proof}
		From REFERENZA, we can extend the product defined in REFERENZA to define a bundle metric on the adjoint bundle $P\times_H \mathfrak{g}$. This is possible since the product is $Ad(H)$-invariant. This implies the extension of the Hodge operator $\star$ on the bundle $P\times_H \mathfrak{g}$ like in REFERENZA. In particular, if $dV$ is the volume form of $M$ and $F_M\in\Omega^k(M,P\times_H \mathfrak{g})$, we have:
		$$F_M\wedge \star F_M=\braket{F_M,F_M}dV$$ 
		Consider now the curvature tensor $F$ of a given Cartan connection $\omega\in\Omega^1(P,\mathfrak{g})$. In our reductive case, $\omega=A+e$ splits in the well known way. Consequentially, $F$ also splits:
		$$F=R+T+{1\over 2}[e,e]$$
		Moreover, $F$ can be isomorphically mapped into $F_M\in\Omega^2(M,P\times_H\mathfrak{g})$ through the Musical Isomorphism REFERENZA. Thus, we can evaluate:
		$$\braket{F_M,F_M}=\braket{R_M,R_M}+\braket{T_M,T_M}+\braket{e_M\wedge e_M,e_M\wedge e_M}+$$$$+2\braket{R_M,T_M}+2\braket{R_M,e_M\wedge e_M}+2\braket{T_M,e_M\wedge e_M}$$
		However, due to the construction of the product, the mixed terms vanish:
		$$\braket{F_M,F_M}=\braket{R_M,R_M}+\braket{T_M,T_M}+\braket{e_M\wedge e_M,e_M\wedge e_M}+$$$$+2\braket{R_M,e_M\wedge e_M}+2\braket{T_M,e_M\wedge e_M}$$
		Take now a local section $s:M\rightarrow P$. From REFERENZA, we know that the above product can also be expressed as the product of the pulled back forms, so that:
		$$\braket{F_M,F_M}=\braket{s^*F,s^*F}$$
		By taking the pullback of those, we find:
		$$s^*F=R_a\otimes J^a+T^a\otimes i_a+e^a\wedge e^b\otimes [i_a,i_b]$$
		Now, choosing an orthonormal base for $\mathfrak{h}$ and an orthonormal base for $\mathfrak{m}$, we have:
		$$\braket{J^a,J^b}=\eta^{ab}\hbox{ and }\braket{i_a,i_b}=\eta_{ab}$$
		Moreover, in our 3 cases we have 
		$$[J^a,J^b]=i\varepsilon^{ab}_{\hspace{9pt}c}J^c$$
		$$[i_a,i_b]=\Lambda(-i\varepsilon_{abc})J^c$$
		Where $\Lambda=0,1,-1$ like in REFERENZA. Thus, we find:
		This automatically gives us:
		$$\braket{F_M,F_M}=\braket{R_a,R^a}+\braket{T^a,T_a}-\braket{(e^a\wedge e^b),(e^c\wedge e^d)}\Lambda^2\varepsilon_{abe}\varepsilon_{cdf}\eta^{ef}+$$
		$$-2i\Lambda \braket{R_a,e^b\wedge e^c}\varepsilon_{bce}\eta^{ae}-2\Lambda T^a e^b\wedge e^c\varepsilon_{bce}\times 0$$
		The last term vanishes for the same reasons as above. Contracting the indices:
		$$\braket{F_M,F_M}=\braket{R_a,R^a}+\braket{T^a,T_a}-\braket{(e^a\wedge e^b),(e^c\wedge e^d)}\Lambda^2\varepsilon_{abe}\varepsilon_{cdf}\eta^{ef}+$$
		$$-2i\Lambda \braket{R^e,e^b\wedge e^c}\varepsilon_{bce}$$
		Now, choosing as a coordinate base ${x^i}$ for $M$, one can show that:
		$$\star \alpha_Idx^I={\sqrt{|g|}\over (n-k)!k!}\alpha_{i_1...i_k}g^{i_1j_1}...g^{i_kj_k}\varepsilon_{j_1...j_k,i_{k+1}...i_n}dx^{i_{k+1}}\wedge...\wedge dx^{i_n}$$
		Automatically:
		$$\star R^a=\star{1\over 2} R^a_{\mu\nu}dx^\mu\wedge dx^\nu={\sqrt{|g|}\over 4}R^{a,\mu\nu}\varepsilon_{\mu\nu\alpha\beta}dx^\alpha\wedge dx^\beta$$
		The same holds for the torsion $T$ and for the remaining solder-depending terms. We find:
		$$\star T^a=\star{1\over 2} T^a_{\mu\nu}dx^\mu\wedge dx^\nu={\sqrt{|g|}\over 4}T^{a,\mu\nu}\varepsilon_{\mu\nu\alpha\beta}dx^\alpha\wedge dx^\beta$$
		$$\star e^a\wedge e^b=\star{1\over 2} e^a_{\mu} e^b_{\nu}dx^\mu\wedge dx^\nu={\sqrt{|g|}\over 4}e^{a,\mu} e^{b,\nu}\varepsilon_{\mu\nu\alpha\beta}dx^\alpha\wedge dx^\beta$$
		Putting everything together and recalling that for the volume form, from REFERENZA, we have $\varepsilon^{\rho\sigma\alpha\beta}{\omega\over \sqrt{|g|}}=dx^\rho\wedge dx^\sigma \wedge dx^\alpha\wedge dx^\beta$ we obtain:
		$$\braket{R^a,R_a}\omega={\sqrt{|g|}\over8}R_{a,\rho\sigma}R^{a,\mu\nu}\varepsilon_{\mu\nu\alpha\beta}dx^\rho\wedge dx^\sigma \wedge dx^\alpha\wedge dx^\beta={1\over8}R_{a,\rho\sigma}R^{a,\mu\nu}\varepsilon_{\mu\nu\alpha\beta}\varepsilon^{\rho\sigma\alpha\beta}\omega$$
		Moreover: $\varepsilon_{\mu\nu\alpha\beta}\varepsilon^{\rho\sigma\alpha\beta}=4(\delta^\rho_\mu \delta^\sigma_\nu-\delta^\rho_\nu \delta^\sigma_\mu)$ and so:
		$$\braket{R^a,R_a}\omega=R^a_{\mu\nu}R_a^{\mu\nu}\omega$$
		The same exact thing applies to the torsion term:
		$$\braket{T^a,T_a}\omega=T^a_{\mu\nu}T_a^{\mu\nu}\omega$$
		As for the coframe field-dependent terms:
		$$\braket{(e^a\wedge e^b),(e^c\wedge e^d)}\omega={\sqrt{|g|}\over 4}e^a_\rho e^b_\sigma e^{c,\mu} e^{d,\nu}\varepsilon_{\mu\nu\alpha\beta}dx^\rho\wedge dx^\sigma \wedge  dx^\alpha\wedge dx^\beta=$$
		$${1\over 4}e^a_\rho e^b_\sigma e^{c,\mu} e^{d,\nu}\varepsilon_{\mu\nu\alpha\beta}\varepsilon^{\rho\sigma\alpha\beta}\omega=(e^a_\mu e^b_\nu-e^a_\nu e^b_\mu) e^{c,\mu} e^{d,\nu}$$
		
		$$\braket{(e^a\wedge e^b),(e^c\wedge e^d)}\Lambda^2\varepsilon_{abe}\varepsilon_{cdf}\eta^{ef}=\Lambda^2(e^a_\mu e^b_\nu-e^a_\nu e^b_\mu) e^{c,\mu} e^{d,\nu}\varepsilon_{abe}\varepsilon_{cdf}\eta^{ef}$$
		To evalue this we need to go through a little bit of algebra. From REFERENZA, we have:
		$$\varepsilon_{abe}\varepsilon_{cdf}=\delta_{ac}(\delta_{bd}\delta_{ef}-\delta_{bf}\delta_{ed})-\delta_{ad}(\delta_{bc}\delta_{ef}-\delta_{bf}\delta_{ec})+\delta_{af}(\delta_{bc}\delta_{ed}-\delta_{bd}\delta_{ec})$$
		Now, contracting this with everything else we find:
		$$(e^a_\mu e^b_\nu-e^a_\nu e^b_\mu) e^{c,\mu} e^{d,\nu}\varepsilon_{abe}\varepsilon_{cdf}\eta^{ef}=2e^a_\mu e^b_\nu e^{c,\mu} e^{d,\nu}\varepsilon_{abe}\varepsilon_{cdf}\eta^{ef}=$$
		$$2\bigg\{
			e^a_\mu e_a^{\mu}(\eta^{ee}e^b_\nu e_b^{\nu}-\eta_{db}e^b_\nu e^{d,\nu})-e^a_\mu e_a^{\nu}(\eta^{ee}e^b_\nu e_b^{\mu}-\eta_{cb}e^b_\nu e^{c,\mu})+(e^a_\mu e_a^\nu e^b_\nu e^{b,\mu}-e^a_\mu e_a^\mu e^b_\nu e_b^\nu) \bigg\}=$$
		Since $\eta^{ee}$ is summed over, we get that it is 3, and so:
		$$=2e^a_\mu e^\mu_ae_\nu^b e^\nu_b-2e^a_\mu e^\nu_ae_\mu^b e^\nu_b$$
	\end{proof}
	\chapter{Jets}
	\section{First Jet bundle}
	\begin{Obs}
		Suppose to have a fiber bundle $(E,M,\pi,F)$ and a coordinate system for $U\subset M$ indexed like $x^i$. Locally, the fiber bundle can be expressed as $M\times F$. This means that, letting $v^a$ be some coordinates for $F$, we can define $x_E^i=x^i \circ \pi$ to be coordinates on $E$ constant on each fiber. We call the set $(x^i,v^a)$ an \textit{adapted coordinate system}.
	\end{Obs}
	\begin{Lm}
		Let $(E,M,\pi,F)$ be any fiber bundle and $\phi,\psi\in\Gamma(E)$ be two sections of the bundle $E$, defined in an open set around $p\in M$. If
		\begin{itemize}
			\item $\phi(p)=\psi(p)$ 
			\item For any coordinate systems $(x^i,u^a),(y^j,v^b)$ it holds:
			$${\partial(u^a\circ\phi) \over \partial{x^i}}\bigg|_p={\partial(u^a\circ\psi) \over \partial{x^i}}\bigg|_p$$
		\end{itemize} 
		Then it also holds:
		$${\partial(v^b\circ\phi) \over \partial{y^j}}\bigg|_p={\partial(v^b\circ\psi) \over \partial{y^j}}\bigg|_p$$
	\end{Lm}
	\begin{proof}
		From the chain rule REFERENZA we find:
		$${\partial(v^b\circ\phi) \over \partial{y^j}}\bigg|_p={\partial(v^b\circ\phi) \over \partial{x^i}}\bigg|_p{\partial x^i \over \partial{y^j}}\bigg|_p$$
		The first term can also be written as:
		$$d(v\circ \phi)_p(\partial_{x^i})=dv_{\phi(p)}\circ d\phi_p(\partial_{x^i})$$
		Clearly:
		$$d\phi_p={\partial \phi\over \partial{x^i}}\bigg|_p dx^i_p$$
		$$dv_{\phi(p)}={\partial v\over  \partial{x^i_E}}\bigg|_{\phi(p)} (dx^i_E)_{\phi(p)}+{\partial v\over \partial u^a}\bigg|_{\phi(p)} du^a_{\phi(p)}$$
		Putting everything together we get:
		$$d\phi_p(\partial_{x^j})=\partial_{x^j}\phi(p)$$
		AS for the second term, the first part can be simplified, since $\pi\circ \phi=\mathbb{I}$:
		$$dx^i_E=dx^i\circ d\pi\hbox{ so that }dx^i_E(d\phi)=dx^i$$
		$$dv_{\phi(p)}\partial_{x^j}\phi(p)=\partial_{x^j_E}v(\phi(p))+\partial_{u^a}v(\phi(p)) du^a_{\phi(p)}(\partial_{x^j})$$
		The second term is just the derivative of $u^a\circ \phi$, so that:
		$$dv_{\phi(p)}\partial_{x^j}\phi(p)=\partial_{x^j_E}v(\phi(p))+\partial_{u^a}v(\phi(p)) \partial_{x^j}(u^a\circ \phi)_p$$
		The result follows immediately in $p$.
	\end{proof}
	The previous lemma allows us to construct an equivalence relation between sectionsa of a fiber bundle. The general idea is: two sections are equivalent if they have the same derivatives. This should immediately ring a couples of bells: first of all, first order Taylor polynomials appear to be pointwise defined in this equivalence class. Moreover, seing sections as fields, LAGRANGIAN.
	\begin{Def}
		Let $(E,M,\pi,F)$ be any fiber bundle and $\phi,\psi\in\Gamma(E)$ be two sections of the bundle $E$, defined in an open set around $p\in M$. If
		\begin{itemize}
			\item $\phi(p)=\psi(p)$ 
			\item For any coordinate systems $(x^i,u^a),(y^j,v^b)$ it holds:
			$${\partial(u^a\circ\phi) \over \partial{x^i}}\bigg|_p={\partial(u^a\circ\psi) \over \partial{x^i}}\bigg|_p$$
		\end{itemize} 
		We say that the two sections are \textit{1-equivalent in $p$}. We call the equivalence class of $\phi\in\Gamma(E)$ at $p\in M$ as $j_p^1\phi$.
	\end{Def}
	Clearly, the above relation is an equivalence relation.
	\begin{Def}
		Let $(E,M,\pi,F)$ be any fiber bundle. We define the \textit{first order jet bundle of $E$ on $M$} as the set:
		$$J^1(E,M)=\{j^1_p\phi \hbox{ with }p\in M, \phi\in\Gamma(E)\}$$
	\end{Def}
	It is now bestowed upon us the infamous task to prove that this set is indeed a smooth manifold.
	\begin{Theo}
		The first order jet bundle is a smooth manifold.
	\end{Theo}
	To prove this theorem we need a little bit of work. The idea is the following:
	\begin{itemize}
		\item Prove that $J^1(E,M)$ has a smooth atlas;
		\item Prove that $J^1(E,M)$ is the total space of a bundle;
		\item Deduce from the previous properties the topologicla qualities of $J^1(E,M)$. 
	\end{itemize}
	As usual, the hardest part is the topology.
	\begin{Def}
		Let $J^1(E,M)$ be the first order Jet bundle. Then we define:
		\begin{itemize}
			\item The \textit{source projection}:
			$$
				\pi_{1}:J^1(E,M)\longrightarrow M $$
				$$
				\pi_1(j^1_p\phi)=p$$	
			\item The \textit{target projection}:
			$$
			\pi_{1,0}:J^1(E,M)\longrightarrow E $$
			$$
			\pi_{1,0}(j^1_p\phi)=\phi(p)$$		
		\end{itemize}
	\end{Def}
	\begin{Obs}
		Let $(E,M,\pi,F)$ be a fiber bundle and let $(U_E,x^i,u^a)$ be an adapted coordinate system on $U_E\subset E$. Then we have an induced coordinate system of $J^1(E,M)$ like:
		$$(U^1,x^i,u^a,u^a_i)$$
		where
		$$U^1=\{j^1_p\phi\in J^1(E,M)\hbox{ such that }\phi(p)\in U_E\} \hbox{ this is clearly open}$$
		$$x^i(j^1_p\phi)=x^i(p)\hspace{20 pt}u^a(j^1_p\phi)=u^a(\phi(p))\hspace{20 pt}u^a_i(j^1_p\phi)={\partial (u^a\circ \phi)\over \partial x^i}\bigg|_p$$
	\end{Obs}
	\begin{Prop}
		Given a first order jet bundle, the adapted coordinates make a $C^\infty$ atlas.
	\end{Prop}
	\begin{proof}
		Obvious.
	\end{proof}
	\begin{Prop}
		Let $E$ be a set, $M,F$ be manifolds. Let $\pi:E\rightarrow M$ be a map such that for any $p\in M$, $\pi^{-1}(p)$ is an $n$-dimensional manifold. Let then $\pi_1:M\times F\rightarrow M$ and $\pi_2:M\times F\rightarrow F$ be the standard projections. If for any $p\in M$ there is an open set $U\subset M$ and a bijection $\varphi:\pi^{-1}(U)\rightarrow U\times F$ such that:
		\begin{itemize}
			\item $\pi_1\circ \varphi=\pi\big|_{\pi^{-1}(U)}$;
			\item for any $q\in U$, $\pi_2\circ \varphi\big|_{\pi^{-1}(q)}:\pi^{-1}(q)\rightarrow F$ is a diffeomorphism.
		\end{itemize}
		Then $E$ can be given the structure of an $n$ dimensional manifold such that $(E,M,\pi,F)$ is a fiber bundle with $\varphi$ trivializations.
	\end{Prop}
	\begin{proof}
		As usual, the hardest part is the topological part. We will first prove that $E$ has a smooth structure. By hypothesis, let $\varphi_p\circ \varphi_q^{-1}:(U_p\cap U_q)\times F\rightarrow (U_p\cap U_q)\times F$ be the composition of two local trivializations. It is clear that, if $r\in (U_p\cap U_q)$:
		$$\varphi_p\circ \varphi_q^{-1}\bigg|_{r\times F}:F\longrightarrow F$$
		is a diffeomorphism from $F$ to itself. This means that composing this map with the projections we get:
		$$\pi_1\circ \varphi_p\circ \varphi_q^{-1}:(U_p\cap U_q)\times F\rightarrow (U_p\cap U_q)$$
		$$\pi_2\circ \varphi_p\circ \varphi_q^{-1}:(U_p\cap U_q)\times F\rightarrow F$$
		Which are both smooth. Thus, $\varphi_p\circ \varphi_q^{-1}$ is a diffeomorphism. This automatically implies that if we choose coordinates on $M,F$, we have a $C^\infty$ atlas on $E$. Now the topological properties:
		\begin{itemize}
			\item T2:\\
			\\
			Let $a,b\in E$. Suppose $\pi(a)=\pi(b)=p$. Then we can find a local trivializations:
			$$\varphi_p:\pi^{-1}(U)\rightarrow U\times F$$
			Since this is a diffeomorphism and $\pi_1\circ \varphi_p(a)=\pi_1\circ\varphi_p(b)$, it means tha, in order for the map to be invertible:
			$$\pi_2\circ \varphi_p(a)\neq \pi_2\circ \varphi_p(b)$$
			This means that there are two open sets $V_a,V_b\subset E$ separate in $E$.\\
			In instead $\pi(a)=p\neq \pi(b)=q$ then we can find two local trivializations:
			$$\varphi_p:\pi^{-1}(U_p)\rightarrow U_p\times F$$
			$$\varphi_q:\pi^{-1}(U_q)\rightarrow U_p\times F$$
			Those sets can then be choosen such that they do not intersecate.
			\item Second Countability:\\
			\\
			This follows immediately from the presence of local trivializations. In particular, since we can cover $M$ with a countable array of open sets $U_i$, we can find a countable array of open local trivializations such that they cover $E$.
		\end{itemize}
	\end{proof}
	\begin{proof}[proof of theorem REFERENZA\\\\]
		From the previous results, we just need to show that the projections $\pi_1,\pi_{1,0}$ are such that $\pi_1^{-1}(p),\pi^{-1}_{1,0}(e)$ are $n$-dimensional manifolds, and that we can find local trivializations for $E$ and $M$. By the regular level set theorem REFERENZA, it suffices to show that the projections are smooth surjective submersions.\\\\
		Let us look at $\pi_1$ clearly $\pi_1:J^1(E,M)\rightarrow M$ like $\pi_1(j^i_p\phi)=p$ is surjective. Moreover, by construction, choosing some adapted coordinates $x^i$ on $M$ and $(x^i,u^a,u^a_i)$ on $J^1(E,M)$, we find:
		$$x^i\circ \pi_1 \circ (x^i,u^a,u^a_i)^{-1}:\mathbb{R}^{m+n+mn}\rightarrow \mathbb{R}^n$$
		Where $dim(M)=n$ and $dim(E)=m+n$. The smoothness of this map follows from the fact that this is simply the projection of $\mathbb{R}^{m+n+mn}$ to $\mathbb{R}^n$. The same holds for $\pi_2$.\\
		\\
		Lastly, we need to find local trivializations. Consider any local trivialization $\varphi:E\rightarrow M\times F$, this is a diffeomorphism in some local patch. Let $Pr_F:M\times F\rightarrow F$ be the standard projection. We can define:
		$$\varphi_1:\pi_1^{-1}(U)\rightarrow U\times \mathbb{R}^{m+mn}$$
		$$\varphi_1(j^1_p\phi)=(p,[Pr_F\circ \varphi\circ\phi(p)],[d(Pr_F\circ \varphi\circ\phi)_p])$$
		Where $[Pr_F\circ \varphi\circ\phi(p)]$ is the equivalence class of sections having fiber value equal at $p$ and $[d(Pr_F\circ \varphi\circ\phi)_p]$ the equivalence class of differentials of sections being fiberwise equal at $p$. This is clearly a diffeomorphism. As for $\pi_2$, we have:
		$$\varphi_2:\pi_2^{-1}(U_E)\rightarrow U\times \mathbb{R}^{mn}$$
		$$\varphi_2(j_p^1\phi)=(\phi(p)=e,[d(Pr_F\circ \varphi\circ\phi)_p])$$
	\end{proof}
	\begin{Cor}
		Given any fiber bundle $(E,M,\pi,F)$, we have 2 vector bundles:
		$$(J^1(E,M),M,\pi_1,\mathbb{R}^{m+mn})\hbox{ and }(J^1(E,M),E,\pi_{1,0},\mathbb{R}^{mn})$$
		We have the following diagram:
		\begin{center}
			\begin{tikzcd}
				&&J^1(E,M)\arrow{dddll}{\pi_1} \arrow{dddrr}{\pi_{1,0}}&&\\\\\\
				M&& &&E\arrow{llll}{\pi}
			\end{tikzcd}
		\end{center}
	\end{Cor}
	\begin{Ex}
		Let $(\mathbb{R}\times F,\mathbb{R},\pi,F)$ be the trivial bundle on $\mathbb{R}$. A section of this bundle is any map $s:\mathbb{R}\rightarrow \mathbb{R}\times F$ such that $\pi\circ s=\mathbb{I}$.
		Then the first order jet bundle $J^1(\mathbb{R}\times F,\mathbb{R})$ is defined as:
		$$J^1(E,\mathbb{R})=\{j^1_xs\big|x\in\mathbb{R},s\in\Gamma(\mathbb{R}\times F)\}$$
		Now, since this bundle is trivial, we can find a global trivialization:
		$$\varphi\equiv \mathbb{I}_d:\mathbb{R}\times F\rightarrow \mathbb{R}\times F$$
		And so, if $s(x)=(x,f)$, the local trivializations for the source and target bundles are:
		$$\varphi_1(j^1_xs)=(x,[f],[df_x])$$
		$$\varphi_2(j^1_xs)=((x,f),[df_x])$$
	\end{Ex}
	\begin{Exe}
		Prove that the first order source and target jet bundles over a trivial bundle are always trivial.
	\end{Exe}
	\begin{proof}
		It suffices to find a global trivialization. Let $(E,M,\pi,F)$ be the trivial bundle. This means that $E=M\times F$. Pick a global section for this bundle:
		$$s:M\rightarrow M\times F\hbox{ like }s(p)=(p,f)$$
		Define the two global trivialziations as:
		$$\varphi_1(j^1_ps)=(x,[f(x)],[df_x])$$
		$$\varphi_2(j^1_ps)=((x,f(x)),[df_x])$$
		Those are clearly globally defined since the section is.
	\end{proof}
	\section{Jet prolongations and contact forms}
	\begin{Def}
		Let $(E,M,\pi,F)$ be any fiber bundle, $U\subset M$ an open set and $\phi\in\Gamma(\pi^{-1}(U))$ be a section. We call the \textit{first jet prolongation} of $\phi$ the section $j^1\phi\in\Gamma(J^1(E,M))$ such that:
		$$j^1\phi:U\rightarrow J^1(E,M)$$
		$$j^1\phi(p)=j^1_p\phi$$
	\end{Def}
	\begin{Obs}
		Clearly the first jet prolongation of any section is a section of $J^1(E,M)$ since:
		$$\pi_1\circ j^1\phi=\mathbb{I}$$
		Moreover, $\pi_{1,0}\circ j^1\phi(p)=\phi(p)$, or $\pi_{1,0}\circ j^1\phi=\phi$. This implies by definition:
		$$j^1(\pi_{1,0}\circ j^1\phi)=j^1\phi$$
	\end{Obs}
	\begin{Cor}
		If $\psi:U\subset M\rightarrow J^1(E,M)$ is a section, then there is $\phi:M\rightarrow E$ such that $\psi=j^1\phi$ if and only if $j^1(\pi_{1,0}\circ \phi)=\psi$.
	\end{Cor}
	\begin{Obs}
		We are now interested in finding aa coordinate representation for the first jet prolongation. Let $(x^i,u^a)$ be an adapted coordinate system. Then:
		$$u^a(j^1\phi(p))=u^a\circ \phi (p)$$
		$$u_i^a(j^1\phi(p))={\partial (u^a\circ \phi)\over \partial x^i}\bigg|_p$$
		Clearly, if $\psi:U\rightarrow J^1(E,M)$ is a section of the Jet bundle, then its fiber coordinates are:
		$$(u^a\circ \psi,\psi^a_i)$$
		In general, the second coordinates have nothing to do with the first ones. Instead for a Jet prolongation, there is a relation between the two:
		$$(u^a\circ \psi,{\partial (u^a\circ \psi)\over \partial x^i})$$
	\end{Obs}
	\begin{Obs}
		Let $(E,M,\pi,F)$ be a fiber bundle and let $J^1(E,M)$ be the first Jet bundle. We have a canonical target projection:
		$$\pi_{1,0}:J^1(E,M)\rightarrow E\hbox{ like } \pi_{1,0}(j^1_p\phi)=\phi(p)$$
		Of course there is the tangent bundle of the vector bundle $E$ like: $(TE,E,\pi_{TE},\mathbb{R}^k)$.
		The target projection is by definition continuous and thus can be used to define a pullback bundle (see REFERENZA):
		$$(\pi_{1,0}^*(TE),J^1(E,M),\pi_{JT},\mathbb{R}^{k})$$
		Recall that the set $\pi_{1,0}^*(TE)$ is defined as:
		$$\pi_{1,0}(TE)=\bigg\{(x_{TE},x_J)\in TE\times J^1(E,M)\bigg|\pi_{1,0}(x_j)=\pi_{TE}(x_E)\bigg\}$$
		In terms of diagrams:
		\begin{center}
			\begin{tikzcd}
				\pi_{1,0}^*(TE)\arrow{rr}{\pi_{JT}}&&J^1(E,M)\arrow{dd}{\pi_{1,0}}&&
				\\\\
				&&E&&TE\arrow{ll}{\pi_{TE}}
			\end{tikzcd}
		\end{center}
	\end{Obs}
	\begin{Def}
		Let $(E,M,\pi,F)$ be a fiber bundle and $X_p\in T_pM$, $p\in M$, let $\phi\in\Gamma(E)$ be a section around $p$. We define the \textit{holonomic lift of $X$ through $\phi$} as:
		$$(d\phi_p(X_p),j^1_p\phi)\in \pi_{1,0}^*(TE)$$
	\end{Def}
	\begin{Obs}
		In general, the holonomic lift of a vector is not dependent on the section, but of equivalence classes. In particular, the first term in the above couple only depends on the values of $\phi$ and its first derivatives at $p\in M$. Those informations are contained in $j^1_p\phi$.
	\end{Obs}
	\begin{Theo}
		There is a canonical decomposition of $\pi_{1,0}^*(TE)_{j^1_p\phi}$ like:
		$$\pi_{1,0}^*(TE)_{j^1_p\phi}=\pi_{1,0}^*(\mathcal{V}_E)_{j^1_p\phi}\oplus Holl(\phi)_p$$
		Where $Holl(\phi)_p$ is the collection of all holonomic lifts of all tangent vectors at $p$ through $\phi$. 
	\end{Theo}
	\begin{proof}
		Take a generic element of $\pi_{1,0}^*(TE)_{j^1_p\phi}$ like: 
		$$(\xi,j^1_p\phi)$$
		Clearly, by definitioin:
		$$(d\phi\circ d\pi(\xi),j^1_p\phi)\in Holl(\phi)_p$$
		Now, since $0=d\pi(\xi-d\phi\circ d\pi(\xi))$, we have that this vector is vertical in $TE$:
		$$(\xi-d\phi\circ d\pi(\xi),j^1_p\phi)\in \pi_{1,0}^*(\mathcal{V}_E)$$
		Now, consider the intersection:
		$$(\xi,j^1_p\phi)\in \pi_{1,0}^*(\mathcal{V}_E)_{j^1_p\phi}\cap Holl(\phi)_p$$
		Clearly, $d\pi(\xi)=0$ and $\xi=d\phi_p(X_p)$ for $X\in T_pM$. Autocamically, $X_p=d\pi\circ d\phi(X_p)=0$. This completes the proof.
	\end{proof}
	\begin{Obs}
		The last result implies that once we choose an equicalence class of sections $[j^1_p\phi]$ at a point $p$, there is an induced decomposition of the bundle $\pi_{1,0}(TE)_{j^1_p\phi}$. This is a canonical decomposition.
	\end{Obs}
	\begin{Exe}
		Let $X$ be a vector field in $M$ in a coordinate chart like:
		$$X_p=X^i_p\partial_{x_p^i}$$
		Find the coordinate representation of the holonomic lift in the adapted coordinates $(x^i,u^a)$.\\
		\\
		Take a section $\phi:U\rightarrow E$ with $p\in U\subset M$. The holonomic lift is by definition:
		$$(d\phi_p(X_p),j^1_p\phi)$$
		Computing the differential:
		$$d\phi_p(X_p)=X_p^id\phi_p(\partial_{x_p^i})$$
		Our induced coordinate system implies the following decomposition:
		$$d\phi={\partial (x^j\circ\phi)\over \partial x^i}dx^i\otimes {\partial \over \partial x^j}+{\partial (u^a\circ\phi) \over \partial u^a}du^a\otimes {\partial\over \partial u^a}$$
	    Applying this to the tangent vector basis and recalling that $x^i\circ \phi=x^i$, we find:
		$$d\phi\bigg({\partial\over \partial x^i}\bigg)={\partial \over \partial x^i}+{\partial (u^a\circ\phi) \over \partial x^i}{\partial\over \partial u^a}={\partial \over \partial x^i}+u^a_i(j^1\phi){\partial\over \partial u^a}$$
	\end{Exe}
	\begin{Def}
		An element $(\eta_p,j^1_p\phi)\in \pi_{1,0}(T^*E)$ is called \textit{contact cotangent form} if $\phi^*(\eta_p)=0$.
	\end{Def}
	THE DUAL IS THE SUM OF THE ANNIHILATORS\\\\
	\begin{Def}
		Any section $\mathcal{X}^h\in\Gamma(Holl(\phi)_p)$ is called \textit{total derivative}.
		This is a map:
		$$\mathcal{X}^h:J^1(E,M)\rightarrow Holl(\phi)_p$$
	\end{Def}
	Basically, we call total derivative a vector field a map that takes a vector in $J^1(E,M)$ and associates the holonomic lift.
	\begin{Ex}
		Prove that any vector field on $M$ corresponds to a total derivative.\\
		\\
		To see this, define:
		$$X^0_{j^1_p\phi}=d\phi_p(X_p)$$
		Then we have a clear total derivative:
		$$\mathcal{X}^h(j^1_p\phi)=(d\phi_p(X_p),j^1_p\phi)$$
	\end{Ex}
	\begin{Obs}
		Recall that if $\mathcal{X}^h$ is a total derivative, then in coordinates, since it is an holonomic lift, it is given by:
		$$\mathcal{X}^h(j^1\phi)={\partial\over \partial x^i}+u_i^a(j^1\phi){\partial\over \partial u^a}$$
		This means that a base for the vertical part is given by:
		$$\bigg\{{\partial\over\partial u^a_i}\bigg\}$$
	\end{Obs}
	\begin{Obs}
		A contact cotangent form is an element like: $(\eta_p,j^1_p\phi)\in \pi_{1,0}(T^*E)$ such that $\phi^*(\eta_p)=0$. Given the natural action of $\pi_{1,0}(T^*E)$ onto $\pi_{1,0}(TE)$ having fixed a base, we have:
		$$(\eta_p,j^1_p\phi)\cdot (\xi,j^1_p\phi)=\eta(\xi)$$
		This means that, since for an holonomic lift $\xi=d\phi(X)$, we have:
		$$\eta(\xi)=\eta(d\phi(\xi))=\phi^*\eta(\xi)=0$$
		The contact cotangent forms are the ones that annihilate the holonomic lifts i.e. the vertical directions. Knowing this, we can see that they have the following form:
		$$\eta_p=du^a-u^a_idx^i$$
	\end{Obs}
	\begin{Ex}
		Consider $J^1(E,M)$ as a bundle on $E$.
		Let $\nabla:\mathfrak{X}(E)\times \Gamma(J^1(E,M))\rightarrow \Gamma(J^1(E,M))$ be a connection on the first order jet bundle. Choosing a vector field $X\in\mathfrak{X}(E)$, we have a spin connection defined:
		$$\nabla(X):\Gamma(J^1(E,M))\rightarrow \Gamma(J^1(E,M))$$
		In particular, for any frame $\{n_i:U_E\subset E\rightarrow J^1(E,M)\}$ of sections of the first order jet bundle, we can write:
		$$\nabla(X)n_i=\omega^j_i(X)n_j$$
		Moreover, choosing a base for $E$ like $(x^i,u^a)$, we have that $X=X^i\partial_i+X^a\partial_a$,
	\end{Ex}
	\section{Second order Jets}
	The construction of second order jets is a straight up generalization of first order jets.
	\begin{Lm}
		Let $(E,M,\pi,F)$ be any fiber bundle and $\phi,\psi\in\Gamma(E)$ be two sections of the bundle $E$, defined in an open set around $p\in M$. If
		\begin{itemize}
			\item $\phi(p)=\psi(p)$ 
			\item For any coordinate systems $(x^i,u^a),(y^j,v^b)$ it holds:
			$${\partial(u^a\circ\phi) \over \partial{x^i}}\bigg|_p={\partial(u^a\circ\psi) \over \partial{x^i}}\bigg|_p$$
			$${\partial^2(u^a\circ\phi) \over \partial{x^i}\partial{x^j}}\bigg|_p={\partial^2(u^a\circ\psi) \over \partial{x^i}\partial{x^j}}\bigg|_p$$
		\end{itemize} 
		Then it also holds:
		$${\partial(v^b\circ\phi) \over \partial{y^j}}\bigg|_p={\partial(v^b\circ\psi) \over \partial{y^j}}\bigg|_p$$
		$${\partial^2(v^b\circ\phi) \over \partial{y^k}\partial{y^l}}\bigg|_p={\partial^2(v^b\circ\psi) \over \partial{y^k}\partial{y^l}}\bigg|_p$$
	\end{Lm}
	\begin{proof}
		The first assertion follows from REFERENZA. As for the second one, it is just a matter of calculations:
		$${\partial^2(v^b\circ\phi) \over \partial{y^k}\partial{y^l}}\bigg|_p={\partial\over \partial y^k}\bigg\{\bigg[{\partial v^b\over \partial{x^j}}\circ \phi+\bigg({\partial v^b\over \partial{u^a}}\circ \phi\bigg) {\partial(u^a\circ \phi)\over \partial{x^j}}\bigg]{\partial x^j\over  \partial y^l}\bigg\}=$$
		$$={\partial\over \partial x^i}\bigg\{\bigg[{\partial v^b\over \partial{x^j}}\circ \phi+\bigg({\partial v^b\over \partial{u^a}}\circ \phi\bigg) {\partial(u^a\circ \phi)\over \partial{x^j}}\bigg]{\partial x^j\over  \partial y^l}\bigg\}{\partial x^i\over \partial y^k}$$
		Since $\phi(p)=(x,u^a(x))$ we get:
		$${\partial\over \partial x^i}\bigg({\partial v^b\over \partial{x^j}}\circ \phi\bigg)=
		{\partial^2 v^b\over \partial{x^j}\partial x^i}+{\partial^2 v^b\over \partial{x^j}\partial u^a}{\partial u^a\over \partial x^i}$$
		$${\partial\over \partial x^i}\bigg({\partial v^b\over \partial{u^a}}\circ \phi\bigg)=
		{\partial^2 v^b\over \partial{u^a}\partial x^i}+{\partial^2 v^b\over \partial{u^a}\partial u^c}{\partial u^c\over \partial x^i}$$
		Since partial derivatives commute, we have:
		$${\partial^2 x^j\over \partial x^i\partial y^l}=0$$
		It immediately follows that:
		$${\partial^2(v^b\circ\phi) \over \partial{y^k}\partial{y^l}}={\partial\over \partial x^i}\bigg\{\bigg[{\partial^2 v^b\over \partial{x^j}\partial x^i}+{\partial^2 v^b\over \partial{x^j}\partial u^a}{\partial u^a\over \partial x^i}+\bigg({\partial^2 v^b\over \partial{u^a}\partial x^i}+{\partial^2 v^b\over \partial{u^a}\partial u^c}{\partial u^c\over \partial x^i}\bigg){\partial(u^a\circ \phi)\over \partial{x^j}}+$$$$+\bigg({\partial v^b\over \partial{u^a}}\circ \phi\bigg) {\partial^2(u^a\circ \phi)\over \partial{x^i}\partial{x^j}}\bigg]{\partial x^j\over  \partial y^l}\bigg\}{\partial x^i\over \partial y^k}$$
		Clearly, by substituting every $\phi$ with $\psi$ at $p$, the result follows.
	\end{proof}
	We proceed again with the usual construction:
	\begin{Def}
		Let $(E,M,\pi,F)$ be any fiber bundle and $\phi,\psi\in\Gamma(E)$ be two sections of the bundle $E$, defined in an open set around $p\in M$. If
		\begin{itemize}
			\item $\phi(p)=\psi(p)$ 
			\item For any coordinate systems $(x^i,u^a),(y^j,v^b)$ it holds:
			$${\partial(u^a\circ\phi) \over \partial{x^i}}\bigg|_p={\partial(u^a\circ\psi) \over \partial{x^i}}\bigg|_p$$
			$${\partial^2(u^a\circ\phi) \over \partial{x^i}\partial{x^j}}\bigg|_p={\partial^2(u^a\circ\psi) \over \partial{x^i}\partial{x^j}}\bigg|_p$$
		\end{itemize} 
		We say that the two sections are \textit{2-equivalent in $p$}. We call the equivalence class of $\phi\in\Gamma(E)$ at $p\in M$ as $j_p^2\phi$.
	\end{Def}
	Clearly, the above relation is an equivalence relation.
	\begin{Def}
		Let $(E,M,\pi,F)$ be any fiber bundle. We define the \textit{second order jet bundle of $E$ on $M$} as the set:
		$$J^2(E,M)=\{j^2_p\phi \hbox{ with }p\in M, \phi\in\Gamma(E)\}$$
	\end{Def}
	\begin{Theo}
		The second order jet bundle is a smooth manifold.
	\end{Theo}
	Just like we did for the first order jet bundle, we first of all need to show that we can find some smooth coordinates. Consider the adapted coordinates $(x^i,u^a)$ for $E$ in $U_E\subset E$. We define the following charts:
	$$\bigg(U^2,(x^i,u^a,u^a_i,u^a_{ij})\bigg)$$
	where: 
	$$U^2=\{j^2_p\phi\in J^2(E,M)\big|\phi(p)\in U_E\}$$
	Those are defined as follows:
	$$x^i(j^2_p\phi)=x^i(p)\hspace{10pt} u^a(j^2_p\phi)=u^a(\phi(p))\hspace{10pt}
	u^a_i(j^2_p\phi)=u^a_i(j^1_p\phi)$$
	AS for the last coordinate, it behaves as:
	$$u^a_{ij}(j^2_p\phi)={\partial^2 (u^a\circ \phi)\over \partial x^i \partial x^j}\bigg|_p$$
	\begin{Obs}
		Clearly, since partial derivatives commute, we have:
		$$u^a_{ij}=u^a_{ji}$$
	\end{Obs}
	\begin{Lm}
		The above coordinate system is smooth.
	\end{Lm}
	\begin{proof}
		This is obvious since the above lemmas REFERENZA prove that changing coordinates gives smooth transition functions.
	\end{proof}
	\begin{Obs}
		Just like in the first order case, we have some canonical projections:
		$$\pi_2:J^2(E,M)\longrightarrow M$$
		$$\pi_2(j^2_p\phi)=p$$
		This is called \textit{source projection.}
		$$\pi_{2,0}:J^2(E,M)\longrightarrow E$$
		$$\pi_{2,0}(j^2_p\phi)=\phi(p)$$
		This is called \textit{target projection.}
		$$\pi_{2,1}:J^2(E,M)\longrightarrow J^1(E,M)$$
		$$\pi_{2,1}(j^2_p\phi)=j^1_p\phi$$
		This is called \textit{1-jet projection.}
	\end{Obs}
	\begin{center}
		\begin{tikzcd}
			J^2(E,M)\arrow{dddd}{\pi_{2,1}}\arrow{ddrr}{\pi_{2,0}}\arrow{ddrrrr}{\pi_2}&&&&\\\\
			&&E\arrow{rr}{\pi}&&M\\\\
			J^1(E,M)\arrow{uurr}{\pi_{1,0}}\arrow{uurrrr}{\pi_1}&&&&
		\end{tikzcd}
	\end{center}
	To show that $J^2(E,M)$ can be given the structure of a smooth manifold, we need to use proposition REFERENZA as above. Thus:
	\begin{Lm}
		Let the notation be as above. The projections $\pi_2,\pi_{2,0},\pi_{2,1}$ are smooth and surjective.
	\end{Lm}
	\begin{proof}
		The smoothness is obvious for the source and the target. For the 1-jet projection instead, consider the coordinate system $(x^i,u^a)$ on $E$. This automatically induces a coordinate system both on $J^1(E,M)$ like $(x^i,u^a,u^a_i)$; and on $J^2(E,M)$ like $(x^i,u^a,u^a_i,u^a_{ij})$. This implies that:
		$$(x^i,u^a,u^a_i)\circ \pi_{2,1}\circ (x^i,u^a,u^a_i,u^a_{ij}):\mathbb{R}^{n+m+nm+{1\over 2}mn(m+1)}\rightarrow\mathbb{R}^{m+n+mn}$$
		is the standard projection which is smooth. The surjectivity is obvious.
	\end{proof}
	To complete the proof that $J^2(E,M)$ is a manifold we need local trivializations. We first of all construct them in coordinates. For the 1-jet projection we have:
	$$\varphi_{2,1}:\pi_{2,1}^{-1}(U^1)\longrightarrow U^1\times \mathbb{R}^{{1\over 2}mn(m+1)}$$
	$$\varphi_{2,1}(j^2_p\phi)=(j^1_p\phi,u^a_{ij}(j^2_p\phi))$$
	This is clearly smooth and invertible:
	$$(j^1_p\phi,z^a_{ij})\mapsto(j^2_p\phi)$$
	Basically we map the class $j^1_p\phi$ into the class $j^2_p\phi$ such that the second derivatives of $u^a\circ \phi$ are $z^a_{ij}$ at $p$. In particular, we define:
	$$u^a\circ\phi=u^a(\phi(p))+u^a_i(j^1_p\phi)(x^i-x^i(p))+{1\over 2}z_{ij}^a(x^i-x^i(p))(x^j-x^j(p))$$
	Clearly $z_{ij}^a=z^a_{ji}$ is symmetric. It is easy to see that this mapping respects all of the properties of proposition REFERENZA, as it is (in coordinates) merely a separation of the coordinates.\\
	\\
	This makes $J^2(E,M)$ a smooth manifold and a bundle on $J^1(E,M)$. 
	\begin{Theo}
		There exist the following vector bundles:
		$$(J^2(E,M),J^1(E,M),\pi_{2,1},\mathbb{R}^{{1\over 2}nm(m+1)})\hspace{10 pt}(J^2(E,M),E,\pi_{2,0},\mathbb{R}^{nm+{1\over 2}nm(m+1)})$$
		$$\hspace{10 pt}(J^2(E,M),M,\pi_{2},\mathbb{R}^{n+m+nm+{1\over 2}nm(m+1)})\hspace{10 pt}$$
	\end{Theo}
	\begin{proof}
		The firsst bundle was defined above. As for the other bundles, the local trivializations are the following: 
		\begin{itemize}
			\item $$\varphi_2:\pi_2^{-1}(U_E)\longrightarrow U_E\times \mathbb{R}^{mn+{1\over 2}(m+1)mn}$$
			$$\varphi_2(j^2_p\phi)=(\phi(p),u_i^a(\phi(p)),u_{ij}^a(\phi(p)))$$
			\item 
			$$\varphi_{2,0}:\pi_{2,0}^{-1}(U_E)\longrightarrow U\times \mathbb{R}^{n+m+mn+{1\over 2}(m+1)mn}$$
			$$\varphi_{2,0}(j^2_p\phi)=(x^i(p),u^a\circ \phi(p),u_i^a(\phi(p)),u_{ij}^a(\phi(p)))$$		
		\end{itemize}
		Checking that those are smooth and ivertible is literally done as above; you define the inverse through the Taylor expansion.
	\end{proof}
	\section{Lagrangian and Cartan form}
	\section{Higher order jets}
	In this section we will generalize the construction of the jet bundle. The idea is to recover all of the Taylor coefficients of the expansion of a section. To achieve the following analysis, we need to construct a new notation, which will allow us to compute derivatives in a clearer way.
	\subsection{Multi-Index notation}
	\begin{Def}
		We call \textit{multi-index} a $n$-tuple of natural numbers $I=(I(1),...,I(n))$. Each element of $I$ is indexed with $I(j)$ with $j\in [1,n]$. We define the \textit{length} of $I$ as $|I|=\sum_i I(i)$. Lastly, we define the \textit{factorial} of a multi-index as:
		$$I!=\sum_{i}I(i)!$$
	\end{Def}
	\begin{Obs}
		Clearly, a basis for the above multi-indices is:
		$$1_i(j)=\begin{cases}
			1 \hbox{ if } i=j\\
			0 \hbox{ otherwise}
		\end{cases}$$
	\end{Obs}
	\begin{Def}
		We define the following notation:
		$${\partial^{|I|}\over \partial x^I}=\prod_{i=1}^n\bigg({\partial \over \partial x^i}\bigg)^{I(i)}$$
	\end{Def}
	\begin{Exe}
		Prove that:
		$${\partial^{|I|}(fg)\over \partial x^I}=\sum_{J+K=I} {I!\over J!K!}{\partial^{|J|}f\over \partial x^J}{\partial^{|K|}g\over \partial x^K}$$
		Use induction.\\
		\\
		The result is trivial for $|I|=0$. Suppose now that this holds for $|I|=n$. Then, from the Leibniz rule, we have:
		$${\partial^{|I|+1}(fg)\over \partial x^{I+1_i}}=\sum_{J+K=I} {I!\over J!K!}\bigg[{\partial^{|J|+1}f\over \partial x^{J+1_i}}{\partial^{|K|}g\over \partial x^{K}}+{\partial^{|J|}f\over \partial x^{J}}{\partial^{|K|+1}g\over \partial x^{K+1_i}}\bigg]$$
		Now we rescale the above indices alternatively: 
		$$J\mapsto J-1_i\hbox{ and }K\mapsto K-1_i$$
		$${\partial^{|I|+1}(fg)\over \partial x^{I+1_i}}=\sum_{J+K=I+1_i} {I!\over (J-1_i)!K!}{\partial^{|J|}f\over \partial x^{J}}{\partial^{|K|}g\over \partial x^{K}}+\sum_{J+K=I+1_i}{I!\over J!(K-1_i)!}{\partial^{|J|}f\over \partial x^{J}}{\partial^{|K|}g\over \partial x^{K}}$$
		Clearly we also have the following:
		$${I!\over (J-1_i)!K!}+{I!\over J!(K-1_i)!}={I!\over (J-1_i)!(K-1_i)!}\bigg({1\over J(i)}+{1\over K(i)}\bigg)$$
		$$={I!\over (J-1_i)!(K-1_i)!}\bigg({J(i)+K(i)\over J(i)K(i)}\bigg)={I!(I(i)+1)\over (J-1_i)!(K-1_i)!(J(i)K(i))}=$$
		$$={(I+1_i)!\over J!K!}$$
		This concludes the exercise.
	\end{Exe}
	The following construction is a straight up generalization of the 1-jet bundle one.
	\begin{Lm}
		Let $(E,M,\pi,F)$ be a bundle and $\phi,\psi:U\rightarrow E$ be sections around $p\in U\subset M$. Then if we have two coordinate systems $(x^i,u^a),(y^j,v^b)$ and:
		\begin{itemize}
			\item $\phi(p)=\psi(p)$;
			\item for every multi-index $I$ with $1\leq |I|\leq k$
			$${\partial^{|I|} (u^a\circ \phi)\over \partial x^I}\bigg|_p={\partial^{|I|} (u^a\circ \psi)\over \partial x^I}\bigg|_p$$
		\end{itemize}
		Then it is also true that for every multi-index $J$ with $1\leq |J|\leq k$
		$${\partial^{|J|} (v^b\circ \phi)\over \partial y^J}\bigg|_p={\partial^{|J|} (v^a\circ \psi)\over \partial x^J}\bigg|_p$$
	\end{Lm}
	\begin{proof}
		We go by induction. If $|I|=0$ the proof is trivial. Now, suppose that for $|I|=n$ it holds:
		$${\partial^{|J|}(v^b\circ \phi)\over \partial y^J }=F^b_J\circ(x^k,{\partial^{|K|} (u^a\circ \phi)\over \partial x^K})$$
		with $0\leq |K|\leq |J|$ and $F^b_J$ does not depend on the choice of the section $\phi$. Then:
		$${\partial^{|J|+1}(v^b\circ \phi)\over \partial y^{J+1_j} }={\partial x^i\over \partial y^j}\bigg({\partial F^b_J\over \partial x^i}\circ\bigg(x^k,{\partial^{|K|} (u^a\circ \phi)\over \partial x^K}\bigg)+\sum_{|L|=0}^{|J|}{\partial^{|L|+1}(u^c\circ \phi)\over \partial x^{L+1_i}} F^{bL}_{Jc}\circ\bigg(x^k,{\partial^{|K|} (u^a\circ \phi)\over \partial x^K}\bigg)\bigg)$$
		Where $F^{bL}_{Jc}$ is the derivative of $F^b_J$ with respect to the coordinate ${\partial^{|L|}(u^c\circ \phi)\over \partial x^{L}}$. Clearly we can write:
		$${\partial^{|J|+1}(v^b\circ \phi)\over \partial y^{J+1_j} }=F^b_{J+1_i}\circ \bigg(x^k,{\partial^{|K|} (u^a\circ \phi)\over \partial x^K}\bigg)$$
		This completes the proof.
	\end{proof}
	\begin{Def}
		We define two local sections $\phi,\psi$ to be $k$-equivalent in $p$ if in any coordinate frame $(x^i,u^a)$:
		\begin{itemize}
			\item $\phi(p)=\psi(p)$;
			\item $${\partial^{|I|} (u^a\circ \phi)\over \partial x^I}\bigg|_p={\partial^{|I|} (u^a\circ \psi)\over \partial x^I}\bigg|_p$$
			for any multi-index $0\leq |I|\leq k$.
		\end{itemize}
		We call the equivalence class of such sections $j^k_p\phi.$
	\end{Def}
	\begin{Def}
		We define the \textit{$k^{th}$ order Jet bundle} as the set:
		$$J^k(E,M)=\bigg\{j^k_p\phi \hbox{ with } \phi:U\rightarrow E\bigg\}$$
	\end{Def}
	It is now once again bestowed on us the problem of showing that the $k^{th}$ order jet bundle is a smooth manifold. Just like we did before, we do this through the bundle construction.
	\begin{Obs}
		There are $k+1$ natural projections:
		\begin{itemize}
			\item[a] The \textit{source projection}:
			$$\pi_{k}:J^K(E,M)\longrightarrow M$$
			$$\pi_k(j^k_p\phi)=p$$
			\item[b] The \textit{target projection}:
			$$\pi_{k,0}:J^K(E,M)\longrightarrow E$$
			$$\pi_k(j^k_p\phi)=\phi(p)$$
			\item[c] The \textit{$l$-jet projection}:
			$$\pi_{k,l}:J^K(E,M)\longrightarrow J^l(E,M)$$
			$$\pi_k(j^k_p\phi)=j^l_p\phi$$
			With $l\leq k$.
		\end{itemize}
		The following commutative diagram is quite useful:
		\begin{center}
			\begin{tikzcd}
				&&&&&&J^k(E,M)\arrow{dd}{\pi_{k,k-1}}\arrow{ddddrrrrrr}{\pi_{k,0}}\arrow{ddddllllll}{\pi_k}&&&&&&\\\\
				&&&&&&J^{k-1}(E,M)\arrow{dd}{\pi_{k-1,k-2}}\arrow{ddrrrrrr}{\pi_{k-1,0}}\arrow{ddllllll}{\pi_{k-1}}&&&&&&\\\\
				M&&&&&&...&&&&&&E\\\\
				&&&&&&J^2(E,M)\arrow{dd}{\pi_{2,1}}\arrow{uurrrrrr}{\pi_{2,0}}\arrow{uullllll}{\pi_{2}}&&&&&&\\\\
				&&&&&&J^1(E,M)\arrow{uuuurrrrrr}{\pi_{1,0}}\arrow{uuuullllll}{\pi_{1}}&&&&&&\\\\
			\end{tikzcd}
		\end{center}
	\end{Obs}
	\begin{Obs}
		Once a coordinate system is selected on the initial bundle $E$, like $(U_E,(x^i,u^a))$, there is an induced coordinate system on $J^k(E,M)$ oike follows:
		$$U^k=\big\{j^k_p\phi\big|\phi(p)\in U_E\big\}$$
		$$(x^i,u^a,u^a_I)(j^k_p\phi)=\bigg(x^i(p),u^a(\phi(p)),{\partial^{|I|}(u^a\circ \phi)\over \partial x^I}\bigg|_p\bigg)$$
		Clearly, it is intended that such functions $u^a_I$ are a family of functions. In particular, if $m$ is the dimension of $F$ the fiber of the bundle $E\rightarrow M$ (it is the number of indices of $u^a$), there are:
		$$m\bigg({n+k\choose k}-1\bigg)$$
		such functions.
	\end{Obs}
	\begin{Prop}
		Let the notation be as above. Then the induced coordinate system is smooth.
	\end{Prop}
	\begin{proof}
		The proof is obvious from lemma REFERENZA.
	\end{proof}
	To show that the $k^{th}$ order jet bundle satisfies the topological properties of a manifold we go on to work with the projections.
	\begin{Prop}
		The map $\pi_{k,k-1}:J^k(E,M)\longrightarrow J^{k-1}(E,M)$ is a smooth surjective projection.
	\end{Prop}
	\begin{proof}
		The proof is easy. The idea is to show that if we compose this with the coordinate charts we get back the regular projection on $\mathbb{R}^h$.
		Let $(x^i,u^a)$ be a coordinate frame on $E$. This induces a coordinate frame both on $J^k$, called $u^k$ and on $J^{k-1}$, called $u^{k-1}$. Thus:
		$$u^{k-1}\circ \pi\circ (u^k)^{-1}(x^i,u^a,u^a_I)=(x^i,u^a,u^a_J)$$
		where $J$ is the multi-index we got from $I$ by removing all of the $k$-order derivatives.
	\end{proof}
	\begin{Cor}
		The $l$-jets projections are all smooth and surjective.
	\end{Cor}
	\begin{proof}
		This follows from the previous demonstration due to the fact that $\pi_{k,m}=\pi_{l,k}\circ \pi_{k,l}$ by construction.
	\end{proof}
	The last ingredients are the local trivializations. Those are obvious in terms of Taylor Polynomials.\\
	\\
	Considering $J^k(E,M)$ as a bundle on $J^{l}(E,M)$ we have the following trivialization:
	$$\varphi_{k,l}:\pi^{-1}_{k,l}(U^{l})\longrightarrow U^{l}\times\mathbb{R}^{N}$$
	where $N=n\sum_{r=l+1}^{k}{n+r-1\choose r}$. Define:
	$$\varphi_{k,l}(j^k_p\phi)=(j^l_p\phi,u^a_I(j^k_p\phi))$$
	This map is clearly smooth in coordinates as it is just the composition of the identity with the coordinate chart. As for the inverse, consider:
	$$(j^l_p\phi,z^a_I)\in U^{l}\times\mathbb{R}^{N}$$
	The last coordinates can be used to completely define the taylor expansion of the section $\phi$. In particular, define:
	$$\psi=u^a\circ \phi(p)+{\partial (u^a\circ \phi)\over \partial x^i}(x^i-x^i(p))+...$$
	Up to the last multi-index of order $k$. Now, clearly, the inverse map will take the derivatives term from $l+1$ to $k$ and name them $u^a_I(j^k_p\phi)$. This completely defines the inverse trivialization in a smooth way. This completes the proof.\\
	\\
	The other trivializations for the descending bundles are defined in the same way.
	\section{The general contact structure}
		\begin{Def}
			Let $(E,M,\pi,F)$ be a fiber bundle and $\phi\in\Gamma(E)$ a section around $p\in M$. We define the \textit{$k^th$-jet prolongation} of $\phi$ as:
			$$j^k\phi:U\rightarrow J^k(E,M)\hbox{ like }j^k\phi(p)=j^k_p\phi$$ 
		\end{Def}
		Clearly, the jet prolongation is a section since $\pi_k\circ j^k\phi=\mathbb{I}$ is the identity. This means that in local coordinates, the jet prolongation is given by:
		$$j^k\phi=(u^a\circ \phi,u^a_I\circ \phi)$$
		\begin{Obs}
			Consider the trivial bundle $M\times \mathbb{R}$ over $\mathbb{R}$. It is clear that sections of this bundle are in 1-1 with smooth maps:
			$$s_f(x)=(x,f(x))$$
			Thus, the $k^{th}$-jet prolongation of this section is:
			$$j^ks_f:U\rightarrow J^k(M\times \mathbb{R},M)\hbox{ like }j^ks_f(x)=j^k_xs$$
			This is the equivalence class of all functions (sections) that have same derivatives up to order $k$.
		\end{Obs}
		\begin{Def}
			Let $(E,M,\pi,F)$ be a fiber bundle and $\phi\in\Gamma(E)$ a section around $p\in M$ and $X_p\in T_pM$. We define the \textit{$k^{th}$-order holonomic lift of $X_p$ through $\phi$} as:
			$$(j^k\phi_*(X_p),j^{k+1}_p(\phi))\in\pi_{k+1,k}(TJ^k(E,M))$$ 
		\end{Def}
		This definition is a lot to take in. Let us dissect it. First of all, the $k^{th}$ holonomic lift is an element of the pullback bundle through $\pi$. In particular, from REFERENZA, we have the following ingredients for the construction:
		\begin{itemize}
			\item $(TJ^k(E,M),J^k(E,M),\pi_{J^k},\mathbb{R}^N)$ the tangent bundle of the $k^{th}$-order jet bundle, which is a vector bundle of dimension $N$;
			\item a map $\pi_{k+1,k}:J^{k+1}(E,M)\rightarrow J^k(E,M)$ which is the $k$-jet projection. 
		\end{itemize}
		This means there is a bundle:
		$$(\pi_{k+1,k}^*(TJ^k(E,M)),J^{k+1}(E,M),\pi_{\pi_{k+1,k}^*},\mathbb{R}^N)$$
		The fiber of this bundle is the same of the bundle on $J^k(E,M)$.\\
		\\
		By definition, an element of this new bundle is a couple:
		$$(X_{j^k_p\phi},j^{k+1}_p\phi)$$
		An element of $J^{k+1}(E,M)$ and an element of the bundle on $J^k(E,M)$; such that $\pi_{k+1,k}(j^{k+1}_p\phi)=\pi_{TJ^k}(X_{j^k_p\phi})$. This is the case since:
		$$\pi_{k+1,k}(j^{k+1}_p\phi)=j^k_p\phi$$
		Moreover, $X_{j^k_p\phi}$ is a vector at $j^k_p\phi$, so that:
		$$\pi_{TJ^k}(X_{j^k_p\phi})=j^k_p\phi$$
		This settles the definition.
		\begin{Obs}
			It is obvious (almost) that the holonomic lift does not depend on the representative selected $\phi$. If in fact we take another element of the class $j^{k+1}_p\phi$ like $\psi$ we would have the same informations about the derivatives of $\phi$, since they equal the ones of $\psi$.
		\end{Obs}
		This construction seems unnecessarily complicated. It would be, if it wasn't for the following theorem:
		\begin{Theo}
			There is a canonical decomposition of the above defined space $\pi_{k+1,k}^*(TJ^k(E,M))_{j^{k+1}_p\phi}$ at any point, like:
			$$\pi_{k+1,k}^*(TJ^k(E,M))_{j^{k+1}_p\phi}=\pi_{k+1,k}^*(\mathcal{V}_{TJ^k})_{j^{k+1}_p\phi}\oplus Holl_p^k(\phi)$$
			Where $Hol_p^k(\phi)$ is the collection of the $k^{th}$-holonomic lifts at $p$ through $\phi$ of $T_pM$.
		\end{Theo}
		\begin{proof}
			Take any element of $\pi_{k+1,k}^*(TJ^k(E,M))_{j^{k+1}_p\phi}$ like $(\chi_{j^{k}_p\phi},j^{k+1}_p\phi)$. Clearly, by definition, we have:
			$$d\pi_k:TJ^k(E,M)\rightarrow TM$$
			This means that, since $\chi_{j^{k+1}_p\phi}$ is in $T_{j^k_p\phi}J^k(E,M)$, we have that:
			$$d\pi_k(\chi_{j^{k}_p\phi}) \hbox{ is an element of }T_pM$$
			Clearly:
			$$j^k\phi_*\circ d\pi_k(\chi_{j^{k}_p\phi})\hbox{ is an holonomic lift}$$
			Thus, we write:
			$$(j^k\phi_*\circ d\pi_k(\chi_{j^{k}_p\phi}),j^{k+1}_p\phi)\in Holl_p^k(\phi)$$
			Furthermore, since $j^k\phi$ is a section of the bundle $J^k(E,M)$ onto $M$, we must have that it annihilates the projection:
			$$d\pi_k(j^k\phi_*\circ d\pi_k)(\chi_{j^{k}_p\phi})=d\pi_k(\chi_{j^{k}_p\phi})$$
			This automatically implies:
			$$d\pi_k(\chi_{j^{k}_p\phi}-j^k\phi_*\circ d\pi_k)(\chi_{j^{k}_p\phi})=0$$
			It follows that elements such as:
			$$(\chi_{j^{k}_p\phi}-j^k\phi_*\circ d\pi_k(\chi_{j^{k}_p\phi}),j^{k+1}_p\phi)$$
			belong to the space $\pi_{k+1,k}^*(\mathcal{V}_{TJ^k})_{j^{k+1}_p\phi}$. This is clear. Now we check the intersections: suppose to take an element which lives both in $\pi_{k+1,k}^*(\mathcal{V}_{TJ^k})_{j^{k+1}_p\phi}$ and in $Holl_p^k(\phi)$. Then we must have, from the belongings of the vertical space, that:
			$$d\pi_k(\chi_{j^{k}_p\phi})=0$$
			From the other space we inherit instead the fact that $\chi_{j^{k}_p\phi}=j^k\phi(X_p)$ is an holonomic lift. Due to the annihilation properites of sections we find:
			$$d\pi_k( \chi_{j^{k}_p\phi})=d\pi_k\circ j^k\phi(X_p)=X_p=0$$
			This completes the proof.
		\end{proof}
		\begin{Cor}
			There is a canonical splitting of the above bundle:
			$$\pi_{k+1,k}^*(TJ^k(E,M))=\pi_{k+1,k}^*(\mathcal{V}_{TJ^k})\oplus Holl^k$$
		\end{Cor}
		\begin{proof}
			The proof amounts to showing that the splitting is smooth. Consider some adapted coordinates on $E$ like $(x^i,u^a)$. Those induce as seen some coordinates in every possible jet. Now, we define the \textit{total derivative} vector fields:
			$$D_i:J^{k+1}(E,M)\rightarrow Holl^k$$
			$$D_i={\partial \over \partial x^i}+\sum_{|J|=0}^ku^a_{i,J}{\partial \over \partial u^a_J}$$
			where $$u^a_{i,I}(j^{k+1}_\phi)={\partial u^a_I\circ \phi\over \partial x^i}$$.
			This is clearly a base of the horizontal vector fields i.e. the holonomic lifts. In fact:
			$$j^k\phi_*\bigg({\partial\over \partial x^i}\bigg|_p\bigg)={\partial \over \partial x^i}\bigg|_{j^k_p\phi}+\sum_{|J|=0}^k{\partial u^a_I\circ \phi\over \partial x^i}\bigg|_p{\partial \over \partial u^a_J}\bigg|_{j^k_p\phi}$$
			This proves the $Holl^k$ horizontal distribution is smooth since we have found a smooth local base. 
		\end{proof}
		\begin{Def}
			An element $(\eta,j_p^{k+1}\phi)\in \pi_{k+1,k}^*(TJ^k(E,M)^*)$ is called \textit{contact cotangent vector} if:
			$$j^k\phi^*(\eta)=0$$
		\end{Def}
		What does this mean? Well, to put it simply, the contact vectors are the ones that annihilate holonomic lifts. In fact it is not hard to check:
		$$\eta(j^k\phi_*(X))=j^k\phi^*(\eta)(X)=0$$
		\begin{Exe}
			Show the form in coordinates of contact cotangent vectors.\\
			\\
			We simply apply the definition of the annihilation property, with the basis of the total derivatives found above:
			$$\eta=\eta_idx^i+\eta_a^Idu^a_I$$
			$$\eta(D_j)=0=\eta_i\delta^i_j+\sum_{|J|=0}^k\eta_a^I\delta_I^J\delta^a_b u^a_{j,J}=0$$
			This gives:
			$$\eta=\sum_{|I|=0}^k\eta_a^I(du^a_I- u^a_{i,I}dx^i)$$
		\end{Exe}
		\begin{Theo}
			There is a splitting of the dual bundle $\pi_{k+1,k}(T^*J^k(E,M))$ into:
			$$\pi_{k+1,k}(T^*J^k(E,M))=BLABLA ANNICHILATORI$$
		\end{Theo}
		\begin{proof}
			This is clear from the annihilators properties.
		\end{proof}
		\begin{Obs}
			Clearly, each vector field on $M$ corresponds to a total derivative: its holonomic lift.
		\end{Obs}
		\begin{Obs}
			Let $X\in\mathfrak{X}(M)$ be a vector field and $f\in C^\infty(J^k(E,M))$ be a smooth map. Then there is a natural derivation operator:
			$$d^k_X:C^\infty_p(J^k(E,M))\rightarrow\mathbb{R}$$
			$$d^k_X(f)(j^{k+1}_p\phi)=j^k\phi_*(X_p(f))$$
			This respects the Leibniz rule since the vector field do.
		\end{Obs}
		\begin{Def}
			We call \textit{generalized vector field} a section of the bundle $\pi_{k+1,k}^*(\mathcal{V}_{TJ^k})$ on $J^k(E,M)$.
		\end{Def}
		\begin{Obs}
			Clearly, due to the coordinate representation of the holonomic lift, a vertical vector field is expressed in cooridnates as:
			$$X=X^a{\partial\over \partial u^a}$$
		\end{Obs}
		\section{Lagrangians}
		In this section we finally define what is a lagrangian, using the language of jet bundles.
		\begin{Def}
			We call \textit{$k^{th}$ order Lagrangian density} a map:
			$$\mathcal{L}\in C^\infty(J^k(E,M))$$
			If $\Omega$ is the volume form of $M$, the full lagrangian is: BOH  
		\end{Def}
		\begin{Def}
			We call a section $\phi\in\Gamma(E)$ an extremal of the lagrangian density if:
			$${d\over dt}\bigg|_0\int_C(j^k(\psi^t\circ \phi))^*\mathcal{L}\Omega=0$$
			Where $C$ is a compact $m$-dimensional sumbmanifold of $M$, $\psi^t$ is the flow of a vertical vector field $X\in\mathcal{V}_E$ such that $X\big|_{\pi^{-1}\partial C}=0$.
		\end{Def}
		\begin{Obs}
			Recall that the flow $\psi^t$ of a vector field $X$, inside an open set $U$ is defined as the map:
			$$\psi^t:U\rightarrow M$$
			$${\partial\over \partial t}\psi^t(p)=X_{\psi^t(p)}$$
			In our case, $\psi^t$ is a flow for a vertical vector $X$ on $TE$.
		\end{Obs}
		\begin{Obs}
			The notation $j^k(\psi^t\circ \phi)$ is dangerous, since we have defined $j^k\phi:M\rightarrow J^k(E,M)$ not only to be a section, but to also take as an input a section. However, while $\phi$ is indeed a section of $E$ on $M$, $\psi:E\rightarrow E$, while $\psi^t:E\rightarrow E$ is a flux i.e. a diffeomorphism. Moreover, the field $X$ is vertical. This means that: $d\pi(X)=0$ so that:
			$${d\over dt}\pi(\psi^t)=0$$
			This implies that for any $t$, the projection is constant along the flux:
			$$\pi\circ\psi\circ \phi=\pi\circ\psi=\mathbb{I} $$
			So indeed the notation $j^k(\psi\circ\phi)$ is well defined.
		\end{Obs}
		\begin{Theo}
			The local section $\phi$ is an extremal for a lagrangian density for $\mathcal{L}$ if and only if:
			$$0=\int_C j^k\phi^* d^k_{X}\mathcal{L}\Omega$$
		\end{Theo}
		\begin{proof}
			The proof is one of the most magical result i have ever came accross in mathematics. It is truly beautiful and astonishing, as it doe snot make any sense whatsoever. Consider:
			$$\partial_t\big|_0(\mathcal{L}(j^k(\psi^t\circ \phi)))=dL(j^k(\psi^0\circ \phi))\partial_t(j^k(\psi^t\circ \phi))\big|_0$$
			This was just the definition of differential. Now, we have that $\psi^0=\mathbb{I}$ is the identity. This implies:
			$$\partial_t\big|_0(\mathcal{L}(j^k(\psi^t\circ \phi)))=\partial_t(j^k(\psi^t\circ \phi))\big|_0L(j^k(\phi))$$
			Repeat with the chain rule:
			$$\partial_t(j^k(\psi^t\circ \phi))\big|_0=\partial_t(\psi^t\circ \phi)\big|_0j^k(\phi)$$
			Now, $\partial_t(\psi^t\circ \phi)\big|_0=X$ is the vector tangent in $0$. This means:
			$$\partial_t\big|_0(\mathcal{L}(j^k(\psi^t\circ \phi)))=X(j^k\phi)(\mathcal{L}(j^k\phi))$$
			Now, $X$ is a vector field on $E$, so that: 
			$$X(j^k\phi)(\mathcal{L}(j^k\phi))=dj^k\phi(X)(\mathcal{L}(j^k\phi))=j^k\phi^*d_X^k\mathcal{L}$$
			This does not make any sense in terms of compositions of maps, naively.
			Now, taking outside the derivation, the only term that depends on $t$ is the flow. However, by hypothesis, the boundary term on $\partial C$ vanishes, so that:
			$$\int_C j^k\phi^* d^k_{X}\mathcal{L}\Omega=\int_C{\partial\over \partial t}\bigg|_0(j^k(\psi^t\circ \phi))^*\mathcal{L}\Omega={d\over dt}\bigg|_0\int_C(j^k(\psi^t\circ \phi))^*\mathcal{L}\Omega$$
		\end{proof}
		\section{Derivations}
		This small section is devoted to the definition of yet again some derivations on fiber bundles. We will study some of their properties and make some examples.
		\begin{Def}
			Let $(E,M,\pi,F)$ be any fiber bundle over $M$. We call \textit{derivation along $\pi$ of degree $n$} a map $D:\bigwedge M\rightarrow \bigwedge E$ such that:
			\begin{itemize}
				\item $D$ is $\mathbb{R}$-linear;
				\item if $\omega\in\bigwedge^k M$ then $D\omega\in \bigwedge^{n+k}E$;
				\item the following holds:
				$$D(\omega\wedge \theta)=D\omega\wedge \pi^*\theta+(-)^{n\cdot deg(\omega)}\pi^*\omega\wedge D\theta$$
			\end{itemize}
		\end{Def}
		\begin{Def}
			A derivation $D$ along $\pi$ is called \textit{of type $i_*$} if for any $f\in C^\infty(M)\simeq \bigwedge^o M$ we have $Df=0$.
		\end{Def}
		We now state a key result, which will actually be the only useful thing in this section.
		\begin{Prop}
			Let $\omega:E\rightarrow \bigwedge^n TE^*\otimes \mathfrak{X}(E)$ be a vector valued form and $(E,M,\pi,F)$ a fiber bundle. Then there is an induced derivation of type $i_*$ and degree $n-1$ along $\pi$ like follows:
			$$i_\omega\theta(X_1,...,X_{k+n-1})=$$
			$$=\sum_{\sigma\in S_{n,k-1}}sgn(\sigma)\theta(\omega(X_{\sigma(1)},...,X_{\sigma(n)}),d\pi \circ X_{\sigma(n+1)},...,d\pi \circ X_{\sigma(n+k-1)})$$
			Where $deg(\theta)=k$, $S_{n,k-1}$ is the subgroup of the permutation group $S_{n+k-1}$ such that:
		$$\sigma(1)<...<\sigma(n)\hbox{ and }\sigma(n+1)<...<\sigma(n+k-1)$$
		\end{Prop}
		\begin{proof}
			The fact that the above map is a derivation is obvious by construction: it is clearly $\mathbb{R}$-linear since forms are; it is clearly of degree $n-1$ and the Leibniz rule follows from the fact that we are contracting a form.
		\end{proof}
		\begin{Obs}
			It is possible to prove that every derivation of this kind is determined by a unique vector valued form $\omega$, but we do not care at all.
		\end{Obs}
		\section{The $S$ tensor}
		In this section we define a new kind of operation on jet bundles. Even though the construction will turn out to be quite excessive and intricate, it will be useful to obtain the Euler Lagrange equations. 
		\begin{Def}
			Let $\omega$ be a closed 1-form on $M$, $p\in M$, $j^k_p\phi\in J^k(E,M)$, $\xi_{j^{k-1}_p\phi}\in\mathcal{V}(\pi_{k-1})_{j^{k-1}_p\phi}$. In an adapted coordinate system $(x^i,u^a)$, if
			$$\xi=\sum_{|I|=0}^{k-1}\xi_I^a{\partial\over \partial u^a_I}\hbox{  and  }\omega=\omega_idx^i$$
			we define the \textit{vertical lift} of $\xi$ to $j^k_p\phi$ through $\omega$ as:
			$$\xi\circledvee_{j^k_p\phi}\omega:=\sum_{|J+K|=0}^{k-1}\sum_i{(J+K+1_i)!\over (J+1_i)!K!}\xi_K^a{\partial^{|J|} \omega_i\over \partial x^J}\bigg|_p{\partial\over \partial u^a_{J+K+1_i}}\bigg|_{j^k_p\phi}$$
		\end{Def}
		\begin{Ex}
			As an example, consider the $k=2$ case. It is no hard to see that:
			$$\xi=\sum_{|I|=0}^{1}\xi_I^a{\partial\over \partial u^a_I}$$
			$$\xi=\xi_j^a{\partial\over \partial u^a_j}$$
			$$\xi\circledvee_{j^2_p\phi}\omega=\sum_{|J+K|=0}^{1}\sum_i{(J+K+1_i)!\over (J+1_i)!K!}\xi_K^a{\partial^{|J|} \omega_i\over \partial x^J}\bigg|_p{\partial\over \partial u^a_{J+K+1_i}}\bigg|_{j^2_p\phi}$$
			Clearly, in this case either $K=J=0$, $K=1_j$ or $J=1_j$ for all possible indices $j$. In both cases, the other multi-index is 0.
			$$\xi\circledvee_{j^2_p\phi}\omega=\sum_{i,j}{(1_i+1_j)!\over 1_i!1_j!}\xi_j^a \omega_i\bigg|_p{\partial\over \partial u^a_{1_j+1_i}}\bigg|_{j^k_p\phi}+\sum_{i,j}{(1_j+1_i)!\over (1_j+1_i)!0!}\xi^a{\partial \omega_i\over \partial x^j}\bigg|_p{\partial\over \partial u^a_{1_j+1_i}}\bigg|_{j^2_p\phi}+\sum_i\xi^a\omega_i{\partial\over \partial u^a_{1_i}}=$$
			$$\xi\circledvee_{j^2_p\phi}\omega=\sum_{i,j}{2}\xi_j^a \omega_i\bigg|_p{\partial\over \partial u^a_{1_j+1_i}}\bigg|_{j^k_p\phi}+\sum_{i,j}\xi^a{\partial \omega_i\over \partial x^j}\bigg|_p{\partial\over \partial u^a_{1_j+1_i}}\bigg|_{j^2_p\phi}+\sum_i\xi^a\omega_i{\partial\over \partial u^a_{1_i}}$$
		\end{Ex}
		\begin{Obs}
			The vertical lift is clearly a vector field on the bundle $J^k(E,M)$ and it is vertical with respect to the projection $\pi_{k,0}$. Thus:
			$$\xi\circledvee \omega\in \Gamma(TJ^k(E,M))$$
		\end{Obs}
		\begin{Prop}
			The vertical lift is independent from the choice of the coordinate system.
		\end{Prop}
		\begin{proof}
			This quantity is clearly fully contracted, so its real value as a map will not depend on the coordinates selected. It is only a matter to check that the verticality is preserved. This is immediate since there will never be any term like $\partial\over \partial x^i$ alone.
		\end{proof}
		\begin{Def}
			Let $\omega$ be a closed 1-form on $M$. We define the \textit{vertical endomorphism of order $k$} as $S^k_\omega:TJ^k(E,M)\rightarrow TJ^k(E,M)$ such that for $\xi_{j^k_p\phi}\in T_{j^k_p\phi}J^k(E,M)$:
			$$S^k_\omega(\xi_{j^k_p\phi})=v_{j^k_p\phi}\circledvee \omega_p$$
			Where $v_{j^k_p\phi}$ is the vertical projection of $\xi$. IN coordinates:
			$$S^k_\omega=\sum_{|J+K|=0}^{k-1}\sum_i{(J+K+1_i)!\over (J+1_i)!K!}{\partial^{|J|} \omega_i\over \partial x^J}\bigg|_p(du^a_K-u^a_{K+1_j}dx^j)\otimes {\partial\over \partial u^a_{J+K+1_i}}\bigg|_{j^k_p\phi}$$
			One can also see this as:
			$S^k_\omega\in(Holl^k)^{\circ}\otimes \mathcal{V}(\pi_{k,0})$
		\end{Def}
		\begin{Obs}
			The annihilator of the holonomic lifts i.e. the cotangent vector space, is isomprhic to the dual of the vertical space.
		\end{Obs}
		\begin{Ex}
			We now make some examples to better understand the structure of this new map:
			\begin{itemize}
				\item Suppose to have a dimension 1 manifold $M$ and $\omega=dt$ to be a volume form. Then, for $k=1$ we must have:
				$$S^1_\omega=(du^a-u^adt)\otimes {\partial \over \partial u^a}$$
				\item for $k=1$ on any bundle it is clear that $S^1_\omega$ and the vertical lift do not depend on the space derivatives of $\omega$:
				$$S^1_\omega=\omega^i(du^a-u_j^adx^j)\otimes {\partial \over \partial u^a_i}$$
			\end{itemize}
		\end{Ex}
		\begin{Obs}
			Clearly, we can generalize the previous operator by asking it to be a map on the space of closed 1-forms. However, it will be more useful for us to restrict this operator to the space of smooth functions on $M$, since every exact form is closed but the contrary is not true. Define:
			$$S^k(\xi,\omega)=S^k_\omega(\xi)$$
			In coordinates
			$$S^k=\sum_{|J+K|=0}^{k-1}\sum_i{(J+K+1_i)!\over (J+1_i)!K!}(du^a_K-u^a_{K+1_j}dx^j)\otimes {\partial\over \partial u^a_{J+K+1_i}}\otimes{\partial^{|J|+1} \over \partial x^{J+1_i}}$$
			Clearly, this can also be seen as an element of:
			$$S^k\in (Holl^k)^\circ\otimes \mathcal{V}(\pi_{k,0})\otimes \pi_k^*(S^k(TM))$$
			\end{Obs}
		\begin{Def}
			Let $\sigma:E\rightarrow TJ^k(E,M)^*$ a section, then define 
			$$S\sigma=S^k(\cdot,\sigma,\cdot)$$
		\end{Def}
		\begin{Obs}
			It goes by itself that in coordinates:
			$$S\sigma=\sum_{|J+K|=0}^{k-1}\sum_i{(J+K+1_i)!\over (J+1_i)!K!}\sigma^{J+K+1_i}_a(du^a_K-u^a_{K+1_j}dx^j)\otimes{\partial^{|J|+1} \over \partial x^{J+1_i}}$$
		\end{Obs}
		\begin{Obs}
			For $k=1$ it is clear that 
			$$S^1\in \bigwedge_C^1J^1(E,M)\otimes \mathcal{V}(\pi_{1,0})\otimes \pi^*_1(TM)$$
			Clearly, $S\sigma\in \bigwedge_C^1J^1(E,M)\otimes \pi^*_1(TM)$ is a vector valued 1-form and can be used to define, from REFERENZA a derivation of type $i_*$ called $i_{S\sigma}$. In particular:
			$$S\sigma\big|_{k=1}=\sigma^{i}_a(du^a_K-u^a_{j}dx^j)\otimes{\partial \over \partial x^{i}}$$
		\end{Obs}
		\begin{Def}
			Let $\Omega$ be the volume form on $M$. We define the \textit{canonical vector valued $m$-form on $J^1$} through its action:
			$$S_\Omega\lrcorner \sigma=i_{S\sigma}\Omega$$
			Where $\sigma\in\bigwedge^1J^1(E,M)$.
		\end{Def}
		The above definition is of a map 
		$$S_\Omega: \bigwedge^{1}J^1(E,M)\rightarrow \bigwedge^{dim(M)+1}J^1(E,M)$$
		From REFERENZA it is also clear that:
		$$\sigma^{i}_a(du^a_K-u^a_{j}dx^j)\otimes{\partial \over \partial x^{i}} \lrcorner \Omega=\sigma^{i}_a(du^a_K-u^a_{j}dx^j)\wedge\bigg({\partial \over \partial x^{i}}\lrcorner \Omega\bigg)$$
		Thus:
		$$S_\Omega=(du^a_K-u^a_{j}dx^j)\wedge\bigg({\partial \over \partial x^{i}}\lrcorner \Omega\bigg)\otimes {\partial\over \partial u^a_i}$$
		This is clearly a vector valued $dim(M)$-form.
		\section{The Cartan Form}
		In this section we use the previous knowledge to find a way to extract the Euler-Lagrange equations from the lagrangian.
	\end{document}
